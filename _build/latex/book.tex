%% Generated by Sphinx.
\def\sphinxdocclass{report}
\documentclass[letterpaper,10pt,french]{sphinxmanual}
\ifdefined\pdfpxdimen
   \let\sphinxpxdimen\pdfpxdimen\else\newdimen\sphinxpxdimen
\fi \sphinxpxdimen=.75bp\relax

\PassOptionsToPackage{warn}{textcomp}
\usepackage[utf8]{inputenc}
\ifdefined\DeclareUnicodeCharacter
% support both utf8 and utf8x syntaxes
  \ifdefined\DeclareUnicodeCharacterAsOptional
    \def\sphinxDUC#1{\DeclareUnicodeCharacter{"#1}}
  \else
    \let\sphinxDUC\DeclareUnicodeCharacter
  \fi
  \sphinxDUC{00A0}{\nobreakspace}
  \sphinxDUC{2500}{\sphinxunichar{2500}}
  \sphinxDUC{2502}{\sphinxunichar{2502}}
  \sphinxDUC{2514}{\sphinxunichar{2514}}
  \sphinxDUC{251C}{\sphinxunichar{251C}}
  \sphinxDUC{2572}{\textbackslash}
\fi
\usepackage{cmap}
\usepackage[T1]{fontenc}
\usepackage{amsmath,amssymb,amstext}
\usepackage{babel}



\usepackage{times}
\expandafter\ifx\csname T@LGR\endcsname\relax
\else
% LGR was declared as font encoding
  \substitutefont{LGR}{\rmdefault}{cmr}
  \substitutefont{LGR}{\sfdefault}{cmss}
  \substitutefont{LGR}{\ttdefault}{cmtt}
\fi
\expandafter\ifx\csname T@X2\endcsname\relax
  \expandafter\ifx\csname T@T2A\endcsname\relax
  \else
  % T2A was declared as font encoding
    \substitutefont{T2A}{\rmdefault}{cmr}
    \substitutefont{T2A}{\sfdefault}{cmss}
    \substitutefont{T2A}{\ttdefault}{cmtt}
  \fi
\else
% X2 was declared as font encoding
  \substitutefont{X2}{\rmdefault}{cmr}
  \substitutefont{X2}{\sfdefault}{cmss}
  \substitutefont{X2}{\ttdefault}{cmtt}
\fi


\usepackage[Sonny]{fncychap}
\ChNameVar{\Large\normalfont\sffamily}
\ChTitleVar{\Large\normalfont\sffamily}
\usepackage[,numfigreset=1,mathnumfig]{sphinx}

\fvset{fontsize=\small}
\usepackage{geometry}


% Include hyperref last.
\usepackage{hyperref}
% Fix anchor placement for figures with captions.
\usepackage{hypcap}% it must be loaded after hyperref.
% Set up styles of URL: it should be placed after hyperref.
\urlstyle{same}


\usepackage{sphinxmessages}




\title{Cours Introduction a Python}
\date{mai 29, 2022}
\release{}
\author{Iaousse M\textquotesingle{}barek}
\newcommand{\sphinxlogo}{\vbox{}}
\renewcommand{\releasename}{}
\makeindex
\begin{document}

\ifdefined\shorthandoff
  \ifnum\catcode`\=\string=\active\shorthandoff{=}\fi
  \ifnum\catcode`\"=\active\shorthandoff{"}\fi
\fi

\pagestyle{empty}
\sphinxmaketitle
\pagestyle{plain}
\sphinxtableofcontents
\pagestyle{normal}
\phantomsection\label{\detokenize{intro::doc}}


\sphinxAtStartPar
Python est couramment utilisé, entre autres, pour \sphinxstylestrong{le développement des sites Web et des logiciels}, \sphinxstylestrong{l’automatisation des tâches}, \sphinxstylestrong{machine learning}, \sphinxstylestrong{data science}, \sphinxstylestrong{l’analyse des données} et \sphinxstylestrong{la visualisation des données}. Comme il est relativement facile à apprendre, Python a été adopté par de nombreux non\sphinxhyphen{}programmeurs tels que des comptables et des scientifiques, pour une variété de tâches quotidiennes, comme l’organisation des finances.

\sphinxAtStartPar
Le présent cours est une introduction à ce langage de programmation. Il va être utile à toute personne désirant creuser davantage dans python pour attaquer l’une des disciplines susmentionnées. Il traitera les axes suivants :
\begin{itemize}
\item {} 
\sphinxAtStartPar
Introduction ;

\item {} 
\sphinxAtStartPar
éléments de base ;

\item {} 
\sphinxAtStartPar
instructions conditionnelles et boucles ;

\item {} 
\sphinxAtStartPar
types de données composites (chaines des caractères, listes, tuples, dictionnaires, ensembles).

\end{itemize}


\chapter{Avant de commencer}
\label{\detokenize{content0:avant-de-commencer}}\label{\detokenize{content0::doc}}
\sphinxAtStartPar
Le présent chapitre introduit quelques informations préliminaires. Il contiendra :
\begin{itemize}
\item {} 
\sphinxAtStartPar
Liens utiles

\item {} 
\sphinxAtStartPar
Guide d’installation

\item {} 
\sphinxAtStartPar
interpréteur Python

\item {} 
\sphinxAtStartPar
IPython

\item {} 
\sphinxAtStartPar
Jupyter notebook

\item {} 
\sphinxAtStartPar
Google Colab

\end{itemize}


\section{Informations préliminaires}
\label{\detokenize{ch00:informations-preliminaires}}\label{\detokenize{ch00::doc}}

\subsection{Liens utiles :}
\label{\detokenize{ch00:liens-utiles}}\begin{itemize}
\item {} 
\sphinxAtStartPar
\sphinxhref{https://www.python.org/}{python.org}

\item {} 
\sphinxAtStartPar
\sphinxhref{https://www.python.org/downloads/windows/}{Télécharger python pour Windows}

\item {} 
\sphinxAtStartPar
\sphinxhref{https://www.python.org/downloads/macos/}{Télécharger Python pour max os}

\item {} 
\sphinxAtStartPar
\sphinxhref{https://www.anaconda.com/}{anaconda}

\item {} 
\sphinxAtStartPar
\sphinxhref{https://repo.anaconda.com/archive/Anaconda3-2022.05-Windows-x86\_64.exe}{Télécharger anaconda pour Windows}

\item {} 
\sphinxAtStartPar
\sphinxhref{https://code.visualstudio.com/Download}{Télécharger vscode}

\item {} 
\sphinxAtStartPar
\sphinxhref{https://atom.io/}{atom}

\item {} 
\sphinxAtStartPar
\sphinxhref{https://www.jetbrains.com/pycharm/}{pycharm}

\item {} 
\sphinxAtStartPar
\sphinxhref{https://research.google.com/colaboratory/}{Google colab}

\end{itemize}


\subsection{Guide d’installation :}
\label{\detokenize{ch00:guide-dinstallation}}\begin{itemize}
\item {} 
\sphinxAtStartPar
\sphinxhref{https://realpython.com/installing-python/}{python}

\item {} 
\sphinxAtStartPar
\sphinxhref{https://code.visualstudio.com/docs/setup/setup-overview}{vscode}

\item {} 
\sphinxAtStartPar
\sphinxhref{https://www.jetbrains.com/help/pycharm/installation-guide.html}{pycharm}

\item {} 
\sphinxAtStartPar
\sphinxhref{https://flight-manual.atom.io/getting-started/sections/installing-atom/}{atom}

\item {} 
\sphinxAtStartPar
\sphinxhref{https://docs.anaconda.com/anaconda/install/windows/}{anaconda}

\end{itemize}


\section{Interpréteur Python, IPython, et notebooks}
\label{\detokenize{ch01:interpreteur-python-ipython-et-notebooks}}\label{\detokenize{ch01::doc}}

\subsection{Interpréteur Python}
\label{\detokenize{ch01:interpreteur-python}}
\sphinxAtStartPar
Apres l’installation de python sur machine, On peut utiliser l’interpréteur Python de manière interactive. On tape \sphinxcode{\sphinxupquote{python}} dans le terminal et on y est. Ensuite, on peut directement lui fournir du code Python à exécuter et nous allons recevoir les résultats immédiatement.

\begin{figure}[htbp]
\centering
\capstart

\noindent\sphinxincludegraphics[width=400\sphinxpxdimen]{{interpwindows}.PNG}
\caption{Lancer l’interpréteur Python dans le terminal Windows}\label{\detokenize{ch01:markdown-fig}}\end{figure}

\sphinxAtStartPar
Cette option est très utile si on souhaite tester rapidement des bouts de code ou pour l’apprentissage.
En revanche, si on souhaite écrire un programme relativement long, on devrait écrire notre programme dans un éditeur de texte (simple comme Notepad, vim, … ou un éditeur avance comme vscode, pycharm, atom, ….) et on enregistre le fichier sous un intitulé qui se termine par l’extension \sphinxcode{\sphinxupquote{.py}}.

\begin{figure}[htbp]
\centering
\capstart

\noindent\sphinxincludegraphics[width=400\sphinxpxdimen]{{scriptnote}.PNG}
\caption{Les étapes pour écrire et exécuter un programme avec Notepad}\label{\detokenize{ch01:markdown-fig1}}\end{figure}


\subsection{IPython}
\label{\detokenize{ch01:ipython}}
\sphinxAtStartPar
IPython est un terminal interactif plus développé qui propose des fonctionnalités telles que l’introspection, une syntaxe additionnelle, la complétion et un historique riche. IPython doit être installe à l’aide de la commande \sphinxcode{\sphinxupquote{pip install ipython}} (voir \sphinxhref{https://ipython.org/install.html}{ici} pour plus de détails).


\subsection{Jupyter notebook}
\label{\detokenize{ch01:jupyter-notebook}}
\sphinxAtStartPar
Les \sphinxcode{\sphinxupquote{notebooks Jupyter}} sont des cahiers électroniques dont on peut, à la fois, produire du texte, insérer des images, des formules mathématiques et du code informatique exécutable. Ils sont manipulables interactivement dans un navigateur web. Jupyter a été développé dans un premier temps pour les langages de programmation Julia, Python et R (d’où le nom Jupyter). Actuellement les notebooks supportent près de 40 langages différents.

\sphinxAtStartPar
La cellule est l’élément de base d’un notebook Jupyter. Elle peut contenir du texte formaté au format Markdown ou du code informatique qui pourra être exécuté. Le présent cours est écrit avec \sphinxcode{\sphinxupquote{jupyterbook}} et chaque chapitre est un \sphinxcode{\sphinxupquote{notebook Jupyter}}. Le présent texte est écrit dans une cellule Markdown. La cellule suivante une cellule code (qui est par défaut l’interpréteur IPython).

\begin{sphinxVerbatim}[commandchars=\\\{\}]
\PYG{c+c1}{\PYGZsh{} La présente cellule est une cellule code, en dessous de la cellule s\PYGZsq{}affiche le résultat du code}
\PYG{n+nb}{print}\PYG{p}{(}\PYG{l+s+s2}{\PYGZdq{}}\PYG{l+s+s2}{Hello world!}\PYG{l+s+s2}{\PYGZdq{}}\PYG{p}{)}
\PYG{l+m+mi}{5}\PYG{o}{+}\PYG{l+m+mi}{2}
\end{sphinxVerbatim}

\begin{sphinxVerbatim}[commandchars=\\\{\}]
Hello world!
\end{sphinxVerbatim}

\begin{sphinxVerbatim}[commandchars=\\\{\}]
7
\end{sphinxVerbatim}

\sphinxAtStartPar
Voir \sphinxhref{https://docs.jupyter.org/en/latest/}{ici} pour plus de détails sur le projet \sphinxcode{\sphinxupquote{Jupyter}}.


\subsection{Google Colab}
\label{\detokenize{ch01:google-colab}}
\sphinxAtStartPar
Etant un projet incontournable dans le domaine du \sphinxcode{\sphinxupquote{machine learning}} et \sphinxcode{\sphinxupquote{l\textquotesingle{}intelligence artificielle}}, Google propose une version cloud de \sphinxcode{\sphinxupquote{Jupyter}}: \sphinxstylestrong{Google Colab}  ou \sphinxstylestrong{Colaboratory}: il s’agit un service cloud, offert par Google qui est essentiellement destiné à la formation et à la recherche dans les domaines susmentionnés. Google Colab permet d’entraîner des modèles de Machine Learning directement dans le cloud. Ceci dit, nous n’avons besoin d’installer quoi que ce soit (python, anaconda, éditeur…) sur la machine, la seule chose requise est un navigateur. (Voir la section liens utiles pour les personnes intéressées).


\chapter{Les éléments de base}
\label{\detokenize{content1:les-elements-de-base}}\label{\detokenize{content1::doc}}
\sphinxAtStartPar
Le présent chapitre introduit les éléments de base da la programmation sous Python. Il introduit les grandes caractéristiques du langage Python en mettant l’accent sur :
\begin{itemize}
\item {} 
\sphinxAtStartPar
Les opérations arithmétique en Python

\item {} 
\sphinxAtStartPar
Affectation et variables

\item {} 
\sphinxAtStartPar
Types de données de base dans Python

\item {} 
\sphinxAtStartPar
Opérateurs logiques et relationnels

\item {} 
\sphinxAtStartPar
Affichage Lecture d’informations

\end{itemize}


\section{Introduction}
\label{\detokenize{ch1:introduction}}\label{\detokenize{ch1::doc}}

\subsection{Python}
\label{\detokenize{ch1:python}}
\sphinxAtStartPar
Python est un langage de programmation de haut niveau (high\sphinxhyphen{}level programming language), avec des applications dans de nombreux domaines, notamment la programmation Web, le calcul scientifique et l’intelligence artificielle !

\sphinxAtStartPar
Python est très populaire et utilisé par des organisations telles que Google, la NASA, la CIA et Disney.

\sphinxAtStartPar
Le langage Python est un de langage Interprété et dynamique. Il support la programmation orienté objet et il est ineffaçable avec plusieurs autres langages de programmation. L’un des plus grands avantages de Python est sa lisibilité, la simplicité de sa syntaxe et une large communauté qui est derrière son développement.


\subsection{Hello world !}
\label{\detokenize{ch1:hello-world}}
\sphinxAtStartPar
Commençons par créer un programme court qui affiche la fameuse phrase « Hello world! ».
En Python, on utilise la fonction \sphinxcode{\sphinxupquote{print}} pour afficher du texte comme output.

\begin{sphinxVerbatim}[commandchars=\\\{\}]
\PYG{n+nb}{print}\PYG{p}{(}\PYG{l+s+s2}{\PYGZdq{}}\PYG{l+s+s2}{Hello world!}\PYG{l+s+s2}{\PYGZdq{}}\PYG{p}{)}
\end{sphinxVerbatim}

\begin{sphinxVerbatim}[commandchars=\\\{\}]
Hello world!
\end{sphinxVerbatim}

\sphinxAtStartPar
Application : Ecrire un programme qui permet d’afficher « Bonjour tout le monde! ».

\begin{sphinxVerbatim}[commandchars=\\\{\}]
\PYG{c+c1}{\PYGZsh{} votre code ici}
\end{sphinxVerbatim}


\subsection{Les opérations arithmétiques en Python}
\label{\detokenize{ch1:les-operations-arithmetiques-en-python}}
\sphinxAtStartPar
Dans sa forme la plus élémentaire, Python peut être utilisé tout simplement comme une \sphinxcode{\sphinxupquote{calculatrice}}. Il peut réaliser les opérations arithmétiques suivantes :
\begin{itemize}
\item {} 
\sphinxAtStartPar
Addition : \sphinxcode{\sphinxupquote{+}}

\item {} 
\sphinxAtStartPar
Soustraction : \sphinxcode{\sphinxupquote{\sphinxhyphen{}}}

\item {} 
\sphinxAtStartPar
Multiplication : \sphinxcode{\sphinxupquote{*}}

\item {} 
\sphinxAtStartPar
Division entière : \sphinxcode{\sphinxupquote{//}}

\item {} 
\sphinxAtStartPar
Division : \sphinxcode{\sphinxupquote{/}}

\item {} 
\sphinxAtStartPar
Exponentiation : \sphinxcode{\sphinxupquote{**}}

\item {} 
\sphinxAtStartPar
Modulo : \sphinxcode{\sphinxupquote{\%}}

\end{itemize}

\sphinxAtStartPar
On rappelle que la division entière renvoi le quotient de la division euclidienne et le modulo renvoie le reste de la division euclidienne. Par exemple 7 // 4 est 1; 7 \% 4 est 3, 5//2 est 2; 5\%2 est 1.

\sphinxAtStartPar
Commençons maintenant à programmer. Nous voulons calculer 1+5, 5\sphinxhyphen{}4, 4*3, (10+8)/3, 2**4, 28//6 et 28\%6;
Comme dans la plupart des langages de programmation, on utilise les symboles suivants pour les opérations mathématiques:

\begin{sphinxVerbatim}[commandchars=\\\{\}]
\PYG{c+c1}{\PYGZsh{} addition}
\PYG{n+nb}{print}\PYG{p}{(}\PYG{l+m+mi}{1}\PYG{o}{+}\PYG{l+m+mi}{5}\PYG{p}{)}

\PYG{c+c1}{\PYGZsh{} soustraction}
\PYG{n+nb}{print}\PYG{p}{(}\PYG{l+m+mi}{5}\PYG{o}{\PYGZhy{}}\PYG{l+m+mi}{4}\PYG{p}{)}

\PYG{c+c1}{\PYGZsh{} multiplication}
\PYG{n+nb}{print}\PYG{p}{(}\PYG{l+m+mi}{4}\PYG{o}{*}\PYG{l+m+mi}{3}\PYG{p}{)}

\PYG{c+c1}{\PYGZsh{} division}
\PYG{n+nb}{print}\PYG{p}{(}\PYG{p}{(}\PYG{l+m+mi}{10}\PYG{o}{+}\PYG{l+m+mi}{8}\PYG{p}{)}\PYG{o}{/}\PYG{l+m+mi}{3}\PYG{p}{)}

\PYG{c+c1}{\PYGZsh{} exponentiation}
\PYG{n+nb}{print}\PYG{p}{(}\PYG{l+m+mi}{2}\PYG{o}{*}\PYG{o}{*}\PYG{l+m+mi}{4}\PYG{p}{)}

\PYG{c+c1}{\PYGZsh{} division entière}
\PYG{n+nb}{print}\PYG{p}{(}\PYG{l+m+mi}{28}\PYG{o}{/}\PYG{o}{/}\PYG{l+m+mi}{6}\PYG{p}{)}

\PYG{c+c1}{\PYGZsh{} modulo}
\PYG{n+nb}{print}\PYG{p}{(}\PYG{l+m+mi}{28}\PYG{o}{\PYGZpc{}}\PYG{k}{6})
\end{sphinxVerbatim}

\begin{sphinxVerbatim}[commandchars=\\\{\}]
6
1
12
6.0
16
4
4
\end{sphinxVerbatim}

\begin{sphinxadmonition}{note}{Note:}
\sphinxAtStartPar
Toute chose commençant par \sphinxcode{\sphinxupquote{\#}} est un \sphinxstylestrong{commentaire}, il sera ignoré par Python. Les commentaires sont une bonne manière de savoir ce qu’on fait ou pour qu’une autre personne sache ce qu’on fait.
\end{sphinxadmonition}


\subsection{Affectation et variables}
\label{\detokenize{ch1:affectation-et-variables}}
\sphinxAtStartPar
Une variable est concept fondamental en programmation. Elle permet de stocker une valeur (par exemple la valeur 2) ou un objet (par exemple une fonction, liste, tableau, vecteur, …). Cela permet de l’utiliser ultérieurement pour accéder facilement à la valeur ou à l’objet qui est stocké dans cette variable.

\sphinxAtStartPar
On affecte une valeur a une variable en utilisant le symbole \sphinxcode{\sphinxupquote{=}}. Si vous voulez stocker la valeur 4 dans une variable nommée \sphinxcode{\sphinxupquote{ma\_valeur}} on peut faire comme suit :

\begin{sphinxVerbatim}[commandchars=\\\{\}]
\PYG{c+c1}{\PYGZsh{} affecter la valeur 4 a la variable ma\PYGZus{}valeur}
\PYG{n}{ma\PYGZus{}valeur} \PYG{o}{=} \PYG{l+m+mi}{4}
\PYG{c+c1}{\PYGZsh{} afficher le contenu de la variable ma\PYGZus{}valeur}
\PYG{n+nb}{print}\PYG{p}{(}\PYG{n}{ma\PYGZus{}valeur}\PYG{p}{)}
\end{sphinxVerbatim}

\begin{sphinxVerbatim}[commandchars=\\\{\}]
4
\end{sphinxVerbatim}

\sphinxAtStartPar
On peut faire toutes les opérations susmentionnées avec les variables à la place de valeurs. Voici un exemple : Dans une classe, le nombre des hommes est 10 le nombre des femmes est 18.

\begin{sphinxVerbatim}[commandchars=\\\{\}]
\PYG{c+c1}{\PYGZsh{} Affecter le nombre des étudiants de sexe masculin à la variable hommes}
\PYG{n}{hommes} \PYG{o}{=} \PYG{l+m+mi}{10}
\PYG{c+c1}{\PYGZsh{} Affecter le nombre des étudiants de sexe féminin à la variable femmes}
\PYG{n}{femmes} \PYG{o}{=} \PYG{l+m+mi}{18}

\PYG{c+c1}{\PYGZsh{} Affecter le nombre total des étudiants à la variable total\PYGZus{}etudiant}
\PYG{n}{total\PYGZus{}etudiant} \PYG{o}{=} \PYG{n}{hommes} \PYG{o}{+} \PYG{n}{femmes}

\PYG{c+c1}{\PYGZsh{} Afficher le nombre total d\PYGZsq{}étudiants}
\PYG{n+nb}{print}\PYG{p}{(}\PYG{n}{total\PYGZus{}etudiant}\PYG{p}{)}
\end{sphinxVerbatim}

\begin{sphinxVerbatim}[commandchars=\\\{\}]
28
\end{sphinxVerbatim}

\sphinxAtStartPar
Python supporte les opérateurs d’affectation suivants : \sphinxcode{\sphinxupquote{=}}, \sphinxcode{\sphinxupquote{+ =}}, \sphinxcode{\sphinxupquote{\textendash{} =}}, \sphinxcode{\sphinxupquote{*=}}, \sphinxcode{\sphinxupquote{/=}}, \sphinxcode{\sphinxupquote{\%=}}, \sphinxcode{\sphinxupquote{**=}} et \sphinxcode{\sphinxupquote{//=}}.

\sphinxAtStartPar
Hormis le premier opérateur \sphinxcode{\sphinxupquote{=}}, les autres opérateurs ont la même logique. Nous allons illustrer cette logique avec \sphinxcode{\sphinxupquote{+=}} et vous alles comprendre l’utilisation des autres.

\sphinxAtStartPar
Supposons que nous avons une variable (soit \sphinxcode{\sphinxupquote{var}}) qui contient une valeur (soit, par exemple 2) et pour une raison (vous allez confronter plusieurs raisons dans la suite de ce cours) nous voulons que cette variable reçoive l’ancienne valeur (2) plus une nouvelle valeur (soit, par exemple 1). On peut faire ça avec l’opérateur \sphinxcode{\sphinxupquote{=}} comme suit :

\begin{sphinxVerbatim}[commandchars=\\\{\}]
\PYG{c+c1}{\PYGZsh{} la variable var contient la valeur 2}
\PYG{n}{var} \PYG{o}{=} \PYG{l+m+mi}{2}
\PYG{c+c1}{\PYGZsh{} la variable var reçoit l\PYGZsq{}ancienne valeur plus 1}
\PYG{n}{var} \PYG{o}{=} \PYG{n}{var} \PYG{o}{+}\PYG{l+m+mi}{1}
\PYG{c+c1}{\PYGZsh{} afficher le contenu de la variable var}
\PYG{n+nb}{print}\PYG{p}{(}\PYG{n}{var}\PYG{p}{)}
\end{sphinxVerbatim}

\begin{sphinxVerbatim}[commandchars=\\\{\}]
3
\end{sphinxVerbatim}

\sphinxAtStartPar
On peut faire la même instruction avec l’opérateur \sphinxcode{\sphinxupquote{+=}} comme suit :

\begin{sphinxVerbatim}[commandchars=\\\{\}]
\PYG{c+c1}{\PYGZsh{} la variable var contient la valeur 2}
\PYG{n}{var} \PYG{o}{=} \PYG{l+m+mi}{2}
\PYG{c+c1}{\PYGZsh{} la variable var reçoit l\PYGZsq{}ancienne valeur plus 1}
\PYG{n}{var} \PYG{o}{+}\PYG{o}{=} \PYG{l+m+mi}{1}
\PYG{c+c1}{\PYGZsh{} afficher le contenu de la variable var}
\PYG{n+nb}{print}\PYG{p}{(}\PYG{n}{var}\PYG{p}{)}
\end{sphinxVerbatim}

\begin{sphinxVerbatim}[commandchars=\\\{\}]
3
\end{sphinxVerbatim}

\begin{sphinxadmonition}{warning}{Avertissement:}
\sphinxAtStartPar
Sous Python, les noms de variables doivent en outre obéir à quelques règles simples :
\begin{itemize}
\item {} 
\sphinxAtStartPar
Un nom de variable est une séquence de lettres (a \(\rightarrow\) z, A \(\rightarrow\) Z) et de chiffres (0 \(\rightarrow\) 9), qui doit
\sphinxstylestrong{toujours commencer par une lettre}.

\item {} 
\sphinxAtStartPar
Seules les lettres ordinaires sont autorisées. Les lettres accentuées, les cédilles, les espaces, les caractères spéciaux tels que \$, \#, @, etc. sont interdits, à l’exception du caractère \_ (souligné).

\item {} 
\sphinxAtStartPar
Python est « case sensitive » (les caractères majuscules et minuscules sont distingués). Par exemple : Var1, var1, VAR1 sont donc des variables différentes.

\item {} 
\sphinxAtStartPar
il est interdit d’utiliser comme nom de variables les mots réservés a python. Ils sont :
\sphinxstylestrong{and, as, assert, break, class, continue, def,
del, elif, else, except, False, finally, for,
from, global, if, import, in, is, lambda,
None, nonlocal, not, or, pass, raise, return,
True, try, while, with, yield,}

\end{itemize}
\end{sphinxadmonition}


\subsection{Types de données de base dans Python}
\label{\detokenize{ch1:types-de-donnees-de-base-dans-python}}
\sphinxAtStartPar
Il existe de nombreux types de données dans Python. Voici quelques\sphinxhyphen{}uns les plus basiques :
\begin{itemize}
\item {} 
\sphinxAtStartPar
\sphinxcode{\sphinxupquote{float}}: Les valeurs décimales (ou en virgule flottante) telles que 10,15 (attention en utilise \sphinxcode{\sphinxupquote{.}} comme décimal au lieu de \sphinxcode{\sphinxupquote{,}}).

\item {} 
\sphinxAtStartPar
\sphinxcode{\sphinxupquote{int}} (les nombres entiers) »: les valeurs comme 8; 4; 10 sont des entiers (\sphinxcode{\sphinxupquote{int}}). Les entiers font aussi partie des valeurs numériques.

\item {} 
\sphinxAtStartPar
\sphinxcode{\sphinxupquote{bool}} (les valeurs booléennes (\sphinxcode{\sphinxupquote{True}} ou \sphinxcode{\sphinxupquote{False}}) sont dites valeurs logiques.

\item {} 
\sphinxAtStartPar
\sphinxcode{\sphinxupquote{str}} (les caractères, ou chaînes de caractères). Les guillemets \sphinxcode{\sphinxupquote{"texte"}} (ou encore les apostrophes \sphinxcode{\sphinxupquote{\textquotesingle{}texte\textquotesingle{}}}) indiquent que \sphinxcode{\sphinxupquote{texte}} est de type \sphinxcode{\sphinxupquote{str}}.

\item {} 
\sphinxAtStartPar
\sphinxcode{\sphinxupquote{complex}}(les nombre complexes):  sont les nombres contenant une partie réelle et une partie imaginaire. En mathématiques, on note \(i\) le nombre complexe dans le carrée est égale à 1. Cependant, en Python, on utilise la lettre \sphinxcode{\sphinxupquote{j}} (ou \sphinxcode{\sphinxupquote{J}}) pour indiquer ce nombre. Par exemple, si on voulait écrire le nombre \sphinxcode{\sphinxupquote{3+1.5i}} (le nombre complexe dont la partie réelle est 3 et la partie imaginaire est 1.5) en python on écrit \sphinxcode{\sphinxupquote{3 + 1.5j}}.

\item {} 
\sphinxAtStartPar
\sphinxcode{\sphinxupquote{None}}: type Le mot\sphinxhyphen{}clé \sphinxcode{\sphinxupquote{None}} est utilisé pour définir une valeur nulle (pas 0), ou aucune valeur du tout. \sphinxcode{\sphinxupquote{None}} n’est pas la même chose que \sphinxcode{\sphinxupquote{0}}, \sphinxcode{\sphinxupquote{False}} ou une chaîne vide \sphinxcode{\sphinxupquote{\textquotesingle{}\textquotesingle{}}}. \sphinxcode{\sphinxupquote{None}} est un type de données en soi (\sphinxcode{\sphinxupquote{NoneType}}) et la seule valeur qui peut être de type \sphinxcode{\sphinxupquote{None}} est le mot\sphinxhyphen{}clé \sphinxcode{\sphinxupquote{None}}.

\end{itemize}

\sphinxAtStartPar
si nous avons un variable (ou même une valeur) et nous voulons savoir son type de données, on utilise la fonction \sphinxcode{\sphinxupquote{type}}. Il faut être prudent lorsqu’on voulait faire des opérations sur des variables si leur type n’est le même!

\begin{sphinxVerbatim}[commandchars=\\\{\}]
\PYG{c+c1}{\PYGZsh{} cette instruction va afficher int}
\PYG{n+nb}{print}\PYG{p}{(}\PYG{n+nb}{type}\PYG{p}{(}\PYG{l+m+mi}{1}\PYG{p}{)}\PYG{p}{)}
\PYG{c+c1}{\PYGZsh{} cette instruction va afficher float}
\PYG{n+nb}{print}\PYG{p}{(}\PYG{n+nb}{type}\PYG{p}{(}\PYG{l+m+mf}{1.5}\PYG{p}{)}\PYG{p}{)}

\PYG{c+c1}{\PYGZsh{} cette instruction va afficher  bool}
\PYG{n+nb}{print}\PYG{p}{(}\PYG{n+nb}{type}\PYG{p}{(}\PYG{k+kc}{True}\PYG{p}{)}\PYG{p}{)}
\PYG{c+c1}{\PYGZsh{} cette instruction va afficher str}
\PYG{n+nb}{print}\PYG{p}{(}\PYG{n+nb}{type}\PYG{p}{(}\PYG{l+s+s2}{\PYGZdq{}}\PYG{l+s+s2}{bonjour}\PYG{l+s+s2}{\PYGZdq{}}\PYG{p}{)}\PYG{p}{)}

\PYG{c+c1}{\PYGZsh{} cette instruction va afficher complex}
\PYG{n+nb}{print}\PYG{p}{(}\PYG{n+nb}{type}\PYG{p}{(}\PYG{l+m+mi}{5}\PYG{o}{+}\PYG{l+m+mf}{17.89}\PYG{n}{j}\PYG{p}{)}\PYG{p}{)}

\PYG{c+c1}{\PYGZsh{} cette instruction va afficher NoneType}
\PYG{n+nb}{print}\PYG{p}{(}\PYG{n+nb}{type}\PYG{p}{(}\PYG{k+kc}{None}\PYG{p}{)}\PYG{p}{)}
\end{sphinxVerbatim}

\begin{sphinxVerbatim}[commandchars=\\\{\}]
\PYGZlt{}class \PYGZsq{}int\PYGZsq{}\PYGZgt{}
\PYGZlt{}class \PYGZsq{}float\PYGZsq{}\PYGZgt{}
\PYGZlt{}class \PYGZsq{}bool\PYGZsq{}\PYGZgt{}
\PYGZlt{}class \PYGZsq{}str\PYGZsq{}\PYGZgt{}
\PYGZlt{}class \PYGZsq{}complex\PYGZsq{}\PYGZgt{}
\PYGZlt{}class \PYGZsq{}NoneType\PYGZsq{}\PYGZgt{}
\end{sphinxVerbatim}

\sphinxAtStartPar
Sous Python, on peut aussi définir une liste comme une collection d’éléments séparés par des virgules, l’ensemble étant enfermé dans des crochets \sphinxcode{\sphinxupquote{{[} {]}}}. Les éléments de la liste peuvent être de n’importe quel type. Voici un exemple d’un liste en python :\sphinxcode{\sphinxupquote{jour\_et\_nombre = {[}\textquotesingle{}lundi\textquotesingle{}, \textquotesingle{}mardi\textquotesingle{}, \textquotesingle{}mercredi\textquotesingle{}, \textquotesingle{}jeudi\textquotesingle{}, \textquotesingle{}vendredi\textquotesingle{}, \textquotesingle{}samedi\textquotesingle{}, \textquotesingle{}dimanche\textquotesingle{}, 0, 1, 5{]}}}.

\begin{sphinxVerbatim}[commandchars=\\\{\}]
\PYG{c+c1}{\PYGZsh{}}
\PYG{n}{jour\PYGZus{}et\PYGZus{}nombre} \PYG{o}{=} \PYG{p}{[}\PYG{l+s+s1}{\PYGZsq{}}\PYG{l+s+s1}{lundi}\PYG{l+s+s1}{\PYGZsq{}}\PYG{p}{,} \PYG{l+s+s1}{\PYGZsq{}}\PYG{l+s+s1}{mardi}\PYG{l+s+s1}{\PYGZsq{}}\PYG{p}{,} \PYG{l+s+s1}{\PYGZsq{}}\PYG{l+s+s1}{mercredi}\PYG{l+s+s1}{\PYGZsq{}}\PYG{p}{,} \PYG{l+s+s1}{\PYGZsq{}}\PYG{l+s+s1}{jeudi}\PYG{l+s+s1}{\PYGZsq{}}\PYG{p}{,} \PYG{l+s+s1}{\PYGZsq{}}\PYG{l+s+s1}{vendredi}\PYG{l+s+s1}{\PYGZsq{}}\PYG{p}{,} \PYG{l+s+s1}{\PYGZsq{}}\PYG{l+s+s1}{samedi}\PYG{l+s+s1}{\PYGZsq{}}\PYG{p}{,} \PYG{l+s+s1}{\PYGZsq{}}\PYG{l+s+s1}{dimanche}\PYG{l+s+s1}{\PYGZsq{}}\PYG{p}{,} \PYG{l+m+mi}{0}\PYG{p}{,} \PYG{l+m+mi}{1}\PYG{p}{,} \PYG{l+m+mi}{5}\PYG{p}{]}

\PYG{c+c1}{\PYGZsh{}}
\PYG{n+nb}{print}\PYG{p}{(}\PYG{n}{jour\PYGZus{}et\PYGZus{}nombre}\PYG{p}{)}

\PYG{c+c1}{\PYGZsh{}}
\PYG{n+nb}{print}\PYG{p}{(}\PYG{n+nb}{type}\PYG{p}{(}\PYG{n}{jour\PYGZus{}et\PYGZus{}nombre}\PYG{p}{)}\PYG{p}{)}
\end{sphinxVerbatim}

\begin{sphinxVerbatim}[commandchars=\\\{\}]
[\PYGZsq{}lundi\PYGZsq{}, \PYGZsq{}mardi\PYGZsq{}, \PYGZsq{}mercredi\PYGZsq{}, \PYGZsq{}jeudi\PYGZsq{}, \PYGZsq{}vendredi\PYGZsq{}, \PYGZsq{}samedi\PYGZsq{}, \PYGZsq{}dimanche\PYGZsq{}, 0, 1, 5]
\PYGZlt{}class \PYGZsq{}list\PYGZsq{}\PYGZgt{}
\end{sphinxVerbatim}

\sphinxAtStartPar
Lorsqu’on dit que les éléments d’une liste peuvent contenir n’importe quel type, on ne rigole pas !! Voici une illustration :

\begin{sphinxVerbatim}[commandchars=\\\{\}]
\PYG{c+c1}{\PYGZsh{} on peut avoir une liste contenant des listes comme des éléments}
\PYG{c+c1}{\PYGZsh{} une première liste}
\PYG{n}{list1} \PYG{o}{=} \PYG{p}{[}\PYG{l+m+mi}{1}\PYG{p}{,} \PYG{l+m+mi}{3}\PYG{p}{,} \PYG{l+s+s1}{\PYGZsq{}}\PYG{l+s+s1}{bonjour}\PYG{l+s+s1}{\PYGZsq{}}\PYG{p}{]}

\PYG{c+c1}{\PYGZsh{} une deuxième liste}
\PYG{n}{list2} \PYG{o}{=} \PYG{p}{[}\PYG{k+kc}{None}\PYG{p}{,} \PYG{k+kc}{True}\PYG{p}{,} \PYG{l+m+mi}{3}\PYG{o}{+}\PYG{l+m+mi}{2}\PYG{n}{j}\PYG{p}{,} \PYG{l+s+s1}{\PYGZsq{}}\PYG{l+s+s1}{\PYGZsq{}}\PYG{p}{]}

\PYG{c+c1}{\PYGZsh{} une troisième liste contenante list1 et list2 comme éléments}
\PYG{n}{list3} \PYG{o}{=} \PYG{p}{[}\PYG{n}{list1}\PYG{p}{,} \PYG{n}{list2}\PYG{p}{]}

\PYG{c+c1}{\PYGZsh{} une quatrième liste contenante list1, list2, et list3 comme éléments}
\PYG{n}{list4} \PYG{o}{=} \PYG{p}{[}\PYG{n}{list1}\PYG{p}{,} \PYG{n}{list2}\PYG{p}{,} \PYG{n}{list3}\PYG{p}{]}

\PYG{c+c1}{\PYGZsh{} afficher les listes}
\PYG{n+nb}{print}\PYG{p}{(}\PYG{n}{list1}\PYG{p}{)}
\PYG{n+nb}{print}\PYG{p}{(}\PYG{n}{list2}\PYG{p}{)}
\PYG{n+nb}{print}\PYG{p}{(}\PYG{n}{list3}\PYG{p}{)}
\PYG{n+nb}{print}\PYG{p}{(}\PYG{n}{list4}\PYG{p}{)}
\end{sphinxVerbatim}

\begin{sphinxVerbatim}[commandchars=\\\{\}]
[1, 3, \PYGZsq{}bonjour\PYGZsq{}]
[None, True, (3+2j), \PYGZsq{}\PYGZsq{}]
[[1, 3, \PYGZsq{}bonjour\PYGZsq{}], [None, True, (3+2j), \PYGZsq{}\PYGZsq{}]]
[[1, 3, \PYGZsq{}bonjour\PYGZsq{}], [None, True, (3+2j), \PYGZsq{}\PYGZsq{}], [[1, 3, \PYGZsq{}bonjour\PYGZsq{}], [None, True, (3+2j), \PYGZsq{}\PYGZsq{}]]]
\end{sphinxVerbatim}


\subsection{Opérateurs logiques :}
\label{\detokenize{ch1:operateurs-logiques}}
\sphinxAtStartPar
En python, il existe trois opérateurs logiques : \sphinxcode{\sphinxupquote{or}} qui signifie ou, \sphinxcode{\sphinxupquote{and}} qui signifie et, et \sphinxcode{\sphinxupquote{not}} qui signifie non.

\sphinxAtStartPar
Pour mieux illustrer ces opérations, soient \sphinxcode{\sphinxupquote{var1= True}} et \sphinxcode{\sphinxupquote{var2 = False}}. Quel serait le résultat des instructions suivantes :
\begin{itemize}
\item {} 
\sphinxAtStartPar
\sphinxcode{\sphinxupquote{var1 or var2}}

\item {} 
\sphinxAtStartPar
\sphinxcode{\sphinxupquote{var1 and var2}}

\item {} 
\sphinxAtStartPar
\sphinxcode{\sphinxupquote{not var1}}

\item {} 
\sphinxAtStartPar
\sphinxcode{\sphinxupquote{(var1 and var2) or (not var2)}}

\end{itemize}

\begin{sphinxVerbatim}[commandchars=\\\{\}]
\PYG{c+c1}{\PYGZsh{} initialiser les deux variables}
\PYG{n}{var1} \PYG{o}{=} \PYG{k+kc}{True}\PYG{p}{;} \PYG{n}{var2} \PYG{o}{=} \PYG{k+kc}{False} \PYG{c+c1}{\PYGZsh{} remarquer qu\PYGZsq{}on peut faire plusieurs instructions dans une même linge. Cependant, il est déconseillé}
\PYG{c+c1}{\PYGZsh{}}
\PYG{n+nb}{print}\PYG{p}{(}\PYG{n}{var1} \PYG{o+ow}{or} \PYG{n}{var2}\PYG{p}{)}

\PYG{c+c1}{\PYGZsh{}}
\PYG{n+nb}{print}\PYG{p}{(}\PYG{n}{var1} \PYG{o+ow}{and} \PYG{n}{var2}\PYG{p}{)}

\PYG{c+c1}{\PYGZsh{}}
\PYG{n+nb}{print}\PYG{p}{(}\PYG{o+ow}{not} \PYG{n}{var1}\PYG{p}{)}

\PYG{c+c1}{\PYGZsh{} }
\PYG{n+nb}{print}\PYG{p}{(}\PYG{p}{(}\PYG{n}{var1} \PYG{o+ow}{and} \PYG{n}{var2}\PYG{p}{)} \PYG{o+ow}{or} \PYG{p}{(}\PYG{o+ow}{not} \PYG{n}{var2}\PYG{p}{)}\PYG{p}{)}
\end{sphinxVerbatim}

\begin{sphinxVerbatim}[commandchars=\\\{\}]
True
False
False
True
\end{sphinxVerbatim}


\subsection{Opérateurs relationnels (tests) :}
\label{\detokenize{ch1:operateurs-relationnels-tests}}
\sphinxAtStartPar
En python, il existe six opérateurs relationnels :
\begin{itemize}
\item {} 
\sphinxAtStartPar
\sphinxcode{\sphinxupquote{\textless{}}} qui permet de tester si une valeur et strictement inférieure à une autre ;

\item {} 
\sphinxAtStartPar
\sphinxcode{\sphinxupquote{\textless{}=}} qui permet de tester si une valeur et inférieure ou égale à une autre ;

\item {} 
\sphinxAtStartPar
\sphinxcode{\sphinxupquote{\textgreater{}}} qui permet de tester si une valeur et strictement supérieure à une autre ;

\item {} 
\sphinxAtStartPar
\sphinxcode{\sphinxupquote{\textgreater{}=}} qui permet de tester si une valeur et supérieure ou égale à une autre ;

\item {} 
\sphinxAtStartPar
\sphinxcode{\sphinxupquote{!=}} ou \sphinxcode{\sphinxupquote{\textless{}\textgreater{}}} qui permet de tester si une valeur est égale a une autre;

\item {} 
\sphinxAtStartPar
et \sphinxcode{\sphinxupquote{==}} qui permet de tester l’égalité de deux valeurs.

\end{itemize}

\sphinxAtStartPar
Soient \sphinxcode{\sphinxupquote{var1 = 5}}, \sphinxcode{\sphinxupquote{var2 = 2+3}}, \sphinxcode{\sphinxupquote{var3 = 6}}, \sphinxcode{\sphinxupquote{var4 = "bonjour"}}, \sphinxcode{\sphinxupquote{var5 = True}}. Quel serait le résultat des instructions suivantes :
\begin{itemize}
\item {} 
\sphinxAtStartPar
\sphinxcode{\sphinxupquote{var1 == var2}};

\item {} 
\sphinxAtStartPar
\sphinxcode{\sphinxupquote{var2 != var4}};

\item {} 
\sphinxAtStartPar
\sphinxcode{\sphinxupquote{var3 \textgreater{} var1+ var2}};

\item {} 
\sphinxAtStartPar
\sphinxcode{\sphinxupquote{var3\textless{} var1 + var2}};

\end{itemize}

\begin{sphinxVerbatim}[commandchars=\\\{\}]
\PYG{c+c1}{\PYGZsh{}}
\PYG{n}{var1} \PYG{o}{=} \PYG{l+m+mi}{5}\PYG{p}{;} \PYG{n}{var2} \PYG{o}{=} \PYG{l+m+mi}{2}\PYG{o}{+}\PYG{l+m+mi}{3}\PYG{p}{;} \PYG{n}{var3} \PYG{o}{=} \PYG{l+m+mi}{6}\PYG{p}{;} \PYG{n}{var4} \PYG{o}{=} \PYG{l+s+s2}{\PYGZdq{}}\PYG{l+s+s2}{bonjour}\PYG{l+s+s2}{\PYGZdq{}}\PYG{p}{;} \PYG{n}{var5} \PYG{o}{=} \PYG{k+kc}{True}

\PYG{c+c1}{\PYGZsh{}}
\PYG{n+nb}{print}\PYG{p}{(}\PYG{n}{var1} \PYG{o}{==} \PYG{n}{var2}\PYG{p}{)}

\PYG{c+c1}{\PYGZsh{}}
\PYG{n+nb}{print}\PYG{p}{(}\PYG{n}{var1} \PYG{o}{!=} \PYG{n}{var4}\PYG{p}{)}

\PYG{c+c1}{\PYGZsh{}}
\PYG{n+nb}{print}\PYG{p}{(}\PYG{n}{var3} \PYG{o}{\PYGZgt{}} \PYG{n}{var1} \PYG{o}{+} \PYG{n}{var2}\PYG{p}{)}

\PYG{c+c1}{\PYGZsh{}}
\PYG{n+nb}{print}\PYG{p}{(}\PYG{n}{var3} \PYG{o}{\PYGZlt{}} \PYG{n}{var1} \PYG{o}{+} \PYG{n}{var2}\PYG{p}{)}
\end{sphinxVerbatim}

\begin{sphinxVerbatim}[commandchars=\\\{\}]
True
True
False
True
\end{sphinxVerbatim}


\subsection{Affichage}
\label{\detokenize{ch1:affichage}}
\sphinxAtStartPar
Nous avons rencontré, dans ce qui précède, la fonction \sphinxcode{\sphinxupquote{print}} qui affiche une chaîne de caractères (\sphinxcode{\sphinxupquote{print("Hello world!")}}).
Pour avoir des informations sur la fonction \sphinxcode{\sphinxupquote{print}} (éventuellement sur toute autre fonction en python) on utilise la fonction \sphinxcode{\sphinxupquote{help}}.

\begin{sphinxVerbatim}[commandchars=\\\{\}]
\PYG{n}{help}\PYG{p}{(}\PYG{n+nb}{print}\PYG{p}{)}
\end{sphinxVerbatim}

\begin{sphinxVerbatim}[commandchars=\\\{\}]
Help on built\PYGZhy{}in function print in module builtins:

print(...)
    print(value, ..., sep=\PYGZsq{} \PYGZsq{}, end=\PYGZsq{}\PYGZbs{}n\PYGZsq{}, file=sys.stdout, flush=False)
    
    Prints the values to a stream, or to sys.stdout by default.
    Optional keyword arguments:
    file:  a file\PYGZhy{}like object (stream); defaults to the current sys.stdout.
    sep:   string inserted between values, default a space.
    end:   string appended after the last value, default a newline.
    flush: whether to forcibly flush the stream.
\end{sphinxVerbatim}

\sphinxAtStartPar
La fonction \sphinxcode{\sphinxupquote{print}} peut prendre plusieurs arguments, les valeurs de tous types de données, les variables. Par défaut, le séparateur entre les arguments est l’espace (\sphinxcode{\sphinxupquote{sep=\textquotesingle{} \textquotesingle{}}}) et la fin de l’affichage est un retour a la ligne (\sphinxcode{\sphinxupquote{end=\textquotesingle{}\textbackslash{}n\textquotesingle{}}}). Dans les exemples suivants nous allons voir comment cette fonction affiche plusieurs arguments et nous allons modifier le séparateur (sep) et la fin de l’affichage (end).

\begin{sphinxVerbatim}[commandchars=\\\{\}]
\PYG{c+c1}{\PYGZsh{}}
\PYG{n+nb}{print}\PYG{p}{(}\PYG{l+s+s1}{\PYGZsq{}}\PYG{l+s+s1}{Hello}\PYG{l+s+s1}{\PYGZsq{}}\PYG{p}{)}
\PYG{n+nb}{print}\PYG{p}{(}\PYG{l+s+s1}{\PYGZsq{}}\PYG{l+s+s1}{World}\PYG{l+s+s1}{\PYGZsq{}}\PYG{p}{)}
\end{sphinxVerbatim}

\begin{sphinxVerbatim}[commandchars=\\\{\}]
Hello
World
\end{sphinxVerbatim}

\begin{sphinxVerbatim}[commandchars=\\\{\}]
\PYG{c+c1}{\PYGZsh{}}
\PYG{n+nb}{print}\PYG{p}{(}\PYG{l+s+s1}{\PYGZsq{}}\PYG{l+s+s1}{Hello}\PYG{l+s+s1}{\PYGZsq{}}\PYG{p}{,}\PYG{l+s+s1}{\PYGZsq{}}\PYG{l+s+s1}{World}\PYG{l+s+s1}{\PYGZsq{}}\PYG{p}{)}
\end{sphinxVerbatim}

\begin{sphinxVerbatim}[commandchars=\\\{\}]
Hello World
\end{sphinxVerbatim}

\begin{sphinxVerbatim}[commandchars=\\\{\}]
\PYG{c+c1}{\PYGZsh{}}
\PYG{n}{var1} \PYG{o}{=} \PYG{l+s+s1}{\PYGZsq{}}\PYG{l+s+s1}{Hello}\PYG{l+s+s1}{\PYGZsq{}}\PYG{p}{;} \PYG{n}{var2} \PYG{o}{=} \PYG{l+s+s1}{\PYGZsq{}}\PYG{l+s+s1}{World}\PYG{l+s+s1}{\PYGZsq{}}
\PYG{n+nb}{print}\PYG{p}{(}\PYG{n}{var1}\PYG{p}{,}\PYG{n}{var2}\PYG{p}{)}
\end{sphinxVerbatim}

\begin{sphinxVerbatim}[commandchars=\\\{\}]
Hello World
\end{sphinxVerbatim}

\begin{sphinxVerbatim}[commandchars=\\\{\}]
\PYG{c+c1}{\PYGZsh{}}
\PYG{n+nb}{print}\PYG{p}{(}\PYG{n}{var1}\PYG{p}{,} \PYG{n}{end}\PYG{o}{=}\PYG{l+s+s1}{\PYGZsq{}}\PYG{l+s+s1}{\PYGZsq{}}\PYG{p}{)}
\PYG{n+nb}{print}\PYG{p}{(}\PYG{l+s+s1}{\PYGZsq{}}\PYG{l+s+s1}{World}\PYG{l+s+s1}{\PYGZsq{}}\PYG{p}{)}
\end{sphinxVerbatim}

\begin{sphinxVerbatim}[commandchars=\\\{\}]
HelloWorld
\end{sphinxVerbatim}

\begin{sphinxVerbatim}[commandchars=\\\{\}]
\PYG{c+c1}{\PYGZsh{}}
\PYG{n+nb}{print}\PYG{p}{(}\PYG{n}{var1}\PYG{p}{,} \PYG{l+s+s1}{\PYGZsq{}}\PYG{l+s+s1}{world}\PYG{l+s+s1}{\PYGZsq{}}\PYG{p}{,} \PYG{n}{sep} \PYG{o}{=} \PYG{l+s+s1}{\PYGZsq{}}\PYG{l+s+s1}{\PYGZhy{}\PYGZhy{}}\PYG{l+s+s1}{\PYGZsq{}}\PYG{p}{)}
\end{sphinxVerbatim}

\begin{sphinxVerbatim}[commandchars=\\\{\}]
Hello\PYGZhy{}\PYGZhy{}world
\end{sphinxVerbatim}

\begin{sphinxVerbatim}[commandchars=\\\{\}]
\PYG{c+c1}{\PYGZsh{}}
\PYG{n}{x} \PYG{o}{=} \PYG{l+m+mi}{10}
\PYG{n+nb}{print}\PYG{p}{(}\PYG{l+s+s1}{\PYGZsq{}}\PYG{l+s+s1}{la valeur de la variable x est}\PYG{l+s+s1}{\PYGZsq{}}\PYG{p}{,}\PYG{n}{x}\PYG{p}{,} \PYG{n}{sep}\PYG{o}{=}\PYG{l+s+s1}{\PYGZsq{}}\PYG{l+s+s1}{: }\PYG{l+s+s1}{\PYGZsq{}}\PYG{p}{)}
\end{sphinxVerbatim}

\begin{sphinxVerbatim}[commandchars=\\\{\}]
la valeur de la variable x est: 10
\end{sphinxVerbatim}

\sphinxAtStartPar
Nous allons revenir à la fonction \sphinxcode{\sphinxupquote{print}} lorsque nous allons parler de \sphinxcode{\sphinxupquote{l\textquotesingle{}écriture formatée}}.


\subsection{Lecture d’informations}
\label{\detokenize{ch1:lecture-d-informations}}
\sphinxAtStartPar
Parfois, on est amené a ce que notre programme dépend d’intervention de l’utilisateur (entrée d’un paramètre, clic de souris sur un bouton, etc.). Dans un programme simple, on utilise la fonction \sphinxcode{\sphinxupquote{input}}. Cette fonction invite l’utilisateur à entrer des données au clavier puis taper \sphinxcode{\sphinxupquote{\textless{}entrer\textgreater{}}}. Lorsque le programme est exécuté, une case apparait à l’utilisateur pour entrer ces donner. Puis le programme continue à s’exécuter pour rendre un output. Les données entrées par l’utilisateur peuvent être stockées dans une variable. Il est à noter que cette variable sera de type \sphinxcode{\sphinxupquote{str}}. Donc, si l’on veut des données numériques par exemple, on doit convertir cette variable au type de donne que nous souhaitons. Enfin, on peut y mettre comme argument un texte qui oriente à l’utilisateur lors de l’entrée des données.
Voici un exemple dont on demande à l’utilisateur d’entrer son nom.

\begin{sphinxVerbatim}[commandchars=\\\{\}]
\PYG{n}{x} \PYG{o}{=} \PYG{n+nb}{input}\PYG{p}{(}\PYG{p}{)}
\PYG{n+nb}{print}\PYG{p}{(}\PYG{n}{x}\PYG{p}{)}
\end{sphinxVerbatim}

\begin{sphinxVerbatim}[commandchars=\\\{\}]
 Iaousse
\end{sphinxVerbatim}

\begin{sphinxVerbatim}[commandchars=\\\{\}]
Iaousse
\end{sphinxVerbatim}

\sphinxAtStartPar
Voici le même exemple avec un message explicatif:

\begin{sphinxVerbatim}[commandchars=\\\{\}]
\PYG{n}{x} \PYG{o}{=} \PYG{n+nb}{input}\PYG{p}{(}\PYG{l+s+s2}{\PYGZdq{}}\PYG{l+s+s2}{Veuillez entrer votre nom ici: }\PYG{l+s+s2}{\PYGZdq{}}\PYG{p}{)}
\PYG{n+nb}{print}\PYG{p}{(}\PYG{n}{x}\PYG{p}{)}
\end{sphinxVerbatim}

\begin{sphinxVerbatim}[commandchars=\\\{\}]
Veuillez entrer votre nom ici:  Iaousse
\end{sphinxVerbatim}

\begin{sphinxVerbatim}[commandchars=\\\{\}]
Iaousse
\end{sphinxVerbatim}

\sphinxAtStartPar
Le type de la variable qui stocke le contenu de \sphinxcode{\sphinxupquote{input}} est toujours \sphinxcode{\sphinxupquote{str}}:

\begin{sphinxVerbatim}[commandchars=\\\{\}]
\PYG{n}{x} \PYG{o}{=} \PYG{n+nb}{input}\PYG{p}{(}\PYG{l+s+s2}{\PYGZdq{}}\PYG{l+s+s2}{merci d}\PYG{l+s+s2}{\PYGZsq{}}\PYG{l+s+s2}{entrer un chiffre: }\PYG{l+s+s2}{\PYGZdq{}}\PYG{p}{)}
\PYG{n+nb}{print}\PYG{p}{(}\PYG{n+nb}{type}\PYG{p}{(}\PYG{n}{x}\PYG{p}{)}\PYG{p}{)}
\end{sphinxVerbatim}

\begin{sphinxVerbatim}[commandchars=\\\{\}]
merci d\PYGZsq{}entrer un chiffre:  100
\end{sphinxVerbatim}

\begin{sphinxVerbatim}[commandchars=\\\{\}]
\PYGZlt{}class \PYGZsq{}str\PYGZsq{}\PYGZgt{}
\end{sphinxVerbatim}

\begin{sphinxVerbatim}[commandchars=\\\{\}]
\PYG{n}{x} \PYG{o}{=} \PYG{n+nb}{input}\PYG{p}{(}\PYG{l+s+s2}{\PYGZdq{}}\PYG{l+s+s2}{merci d}\PYG{l+s+s2}{\PYGZsq{}}\PYG{l+s+s2}{entrer True: }\PYG{l+s+s2}{\PYGZdq{}}\PYG{p}{)}
\PYG{n+nb}{print}\PYG{p}{(}\PYG{n+nb}{type}\PYG{p}{(}\PYG{n}{x}\PYG{p}{)}\PYG{p}{)}
\end{sphinxVerbatim}

\begin{sphinxVerbatim}[commandchars=\\\{\}]
merci d\PYGZsq{}entrer True:  True
\end{sphinxVerbatim}

\begin{sphinxVerbatim}[commandchars=\\\{\}]
\PYGZlt{}class \PYGZsq{}str\PYGZsq{}\PYGZgt{}
\end{sphinxVerbatim}


\subsection{convertir les types de données}
\label{\detokenize{ch1:convertir-les-types-de-donnees}}
\sphinxAtStartPar
Il existe plusieurs raisons pour lesquelles nous somme devant l’obligation de convertir les types de données (une raison est celle de \sphinxcode{\sphinxupquote{input}} que nous avons vu). Les fonctions qui servent à la conversion entre les types de données sont les suivantes :
\begin{itemize}
\item {} 
\sphinxAtStartPar
\sphinxcode{\sphinxupquote{int()}}: pour convertir un type de données en entier;

\item {} 
\sphinxAtStartPar
\sphinxcode{\sphinxupquote{float()}}: pour convertir un type de données en virgule flottante.

\item {} 
\sphinxAtStartPar
\sphinxcode{\sphinxupquote{str()}}: pour convertir un types de donnes en chaine de caractères;

\item {} 
\sphinxAtStartPar
et \sphinxcode{\sphinxupquote{bool()}}: pour convertir les types de donnes en valeurs booléennes.

\end{itemize}

\sphinxAtStartPar
Les types de données \sphinxcode{\sphinxupquote{int}} et \sphinxcode{\sphinxupquote{float}} peuvent être convertis en tous autre type de données. La seule remarque est que la conversion d’une valeur de type \sphinxcode{\sphinxupquote{float}} en \sphinxcode{\sphinxupquote{int}} rend les chiffres avant la virgule (1.9 devient 1 après conversion). Pour avoir l’entier le plus proche au nombre que nous voulons convertir, nous devons utiliser la fonction \sphinxcode{\sphinxupquote{round()}}.

\begin{sphinxVerbatim}[commandchars=\\\{\}]
\PYG{c+c1}{\PYGZsh{} int() va renvoyer 1}
\PYG{n+nb}{print}\PYG{p}{(}\PYG{n+nb}{int}\PYG{p}{(}\PYG{l+m+mf}{1.9}\PYG{p}{)}\PYG{p}{)}
\PYG{c+c1}{\PYGZsh{} round() va renvoyer 2}
\PYG{n+nb}{print}\PYG{p}{(}\PYG{n+nb}{round}\PYG{p}{(}\PYG{l+m+mf}{1.9}\PYG{p}{)}\PYG{p}{)}
\end{sphinxVerbatim}

\begin{sphinxVerbatim}[commandchars=\\\{\}]
1
2
\end{sphinxVerbatim}

\sphinxAtStartPar
Prenons un exemple d’un nombre entier (\sphinxcode{\sphinxupquote{1}}), un nombre en virgule flottante (\sphinxcode{\sphinxupquote{4.5}}) et une valeur booléenne (\sphinxcode{\sphinxupquote{True}}) puis voyons comment les convertir à chaque type de donnes que nous avons vu jusqu’à maintenant.

\begin{sphinxVerbatim}[commandchars=\\\{\}]
\PYG{n}{x} \PYG{o}{=} \PYG{l+m+mi}{1}
\PYG{n}{x\PYGZus{}float} \PYG{o}{=} \PYG{n+nb}{float}\PYG{p}{(}\PYG{n}{x}\PYG{p}{)}
\PYG{n}{x\PYGZus{}str} \PYG{o}{=} \PYG{n+nb}{str}\PYG{p}{(}\PYG{n}{x}\PYG{p}{)}
\PYG{n}{x\PYGZus{}bool} \PYG{o}{=} \PYG{n+nb}{bool}\PYG{p}{(}\PYG{n}{x}\PYG{p}{)}

\PYG{n+nb}{print} \PYG{p}{(}\PYG{l+s+s2}{\PYGZdq{}}\PYG{l+s+se}{\PYGZbs{}t}\PYG{l+s+se}{\PYGZbs{}t}\PYG{l+s+se}{\PYGZbs{}t}\PYG{l+s+s2}{Avant conversion}\PYG{l+s+s2}{\PYGZdq{}}\PYG{p}{,} \PYG{n}{end}\PYG{o}{=}\PYG{l+s+s2}{\PYGZdq{}}\PYG{l+s+se}{\PYGZbs{}n}\PYG{l+s+s2}{================}\PYG{l+s+se}{\PYGZbs{}n}\PYG{l+s+s2}{\PYGZdq{}}\PYG{p}{)}
\PYG{n+nb}{print} \PYG{p}{(}\PYG{l+s+s2}{\PYGZdq{}}\PYG{l+s+s2}{La valeur de x :}\PYG{l+s+s2}{\PYGZdq{}}\PYG{p}{,} \PYG{n}{x}\PYG{p}{,} \PYG{l+s+s2}{\PYGZdq{}}\PYG{l+s+se}{\PYGZbs{}t}\PYG{l+s+se}{\PYGZbs{}t}\PYG{l+s+se}{\PYGZbs{}t}\PYG{l+s+se}{\PYGZbs{}t}\PYG{l+s+s2}{Le type :}\PYG{l+s+s2}{\PYGZdq{}}\PYG{p}{,} \PYG{n+nb}{type}\PYG{p}{(}\PYG{n}{x}\PYG{p}{)}\PYG{p}{,} \PYG{n}{end}\PYG{o}{=}\PYG{l+s+s2}{\PYGZdq{}}\PYG{l+s+se}{\PYGZbs{}n}\PYG{l+s+s2}{================}\PYG{l+s+se}{\PYGZbs{}n}\PYG{l+s+s2}{\PYGZdq{}}\PYG{p}{)}
\PYG{n+nb}{print} \PYG{p}{(}\PYG{l+s+s2}{\PYGZdq{}}\PYG{l+s+se}{\PYGZbs{}t}\PYG{l+s+se}{\PYGZbs{}t}\PYG{l+s+se}{\PYGZbs{}t}\PYG{l+s+s2}{ Après conversion}\PYG{l+s+s2}{\PYGZdq{}}\PYG{p}{,} \PYG{n}{end}\PYG{o}{=}\PYG{l+s+s2}{\PYGZdq{}}\PYG{l+s+se}{\PYGZbs{}n}\PYG{l+s+s2}{================}\PYG{l+s+se}{\PYGZbs{}n}\PYG{l+s+s2}{\PYGZdq{}}\PYG{p}{)}
\PYG{n+nb}{print} \PYG{p}{(}\PYG{l+s+s2}{\PYGZdq{}}\PYG{l+s+s2}{virgule flottante. }\PYG{l+s+se}{\PYGZbs{}t}\PYG{l+s+s2}{Valeur :}\PYG{l+s+s2}{\PYGZdq{}}\PYG{p}{,} \PYG{n}{x\PYGZus{}float}\PYG{p}{,} \PYG{l+s+s2}{\PYGZdq{}}\PYG{l+s+se}{\PYGZbs{}t}\PYG{l+s+se}{\PYGZbs{}t}\PYG{l+s+s2}{Le type :}\PYG{l+s+s2}{\PYGZdq{}}\PYG{p}{,} \PYG{n+nb}{type}\PYG{p}{(}\PYG{n}{x\PYGZus{}float}\PYG{p}{)}\PYG{p}{,} \PYG{n}{end}\PYG{o}{=}\PYG{l+s+s2}{\PYGZdq{}}\PYG{l+s+se}{\PYGZbs{}n}\PYG{l+s+s2}{================}\PYG{l+s+se}{\PYGZbs{}n}\PYG{l+s+s2}{\PYGZdq{}}\PYG{p}{)}
\PYG{n+nb}{print} \PYG{p}{(}\PYG{l+s+s2}{\PYGZdq{}}\PYG{l+s+s2}{Chaine de caractères. }\PYG{l+s+se}{\PYGZbs{}t}\PYG{l+s+s2}{Valeur :}\PYG{l+s+s2}{\PYGZdq{}}\PYG{p}{,} \PYG{n}{x\PYGZus{}str}\PYG{p}{,} \PYG{l+s+s2}{\PYGZdq{}}\PYG{l+s+se}{\PYGZbs{}t}\PYG{l+s+se}{\PYGZbs{}t}\PYG{l+s+s2}{Le type :}\PYG{l+s+s2}{\PYGZdq{}}\PYG{p}{,} \PYG{n+nb}{type}\PYG{p}{(}\PYG{n}{x\PYGZus{}str}\PYG{p}{)}\PYG{p}{,} \PYG{n}{end}\PYG{o}{=}\PYG{l+s+s2}{\PYGZdq{}}\PYG{l+s+se}{\PYGZbs{}n}\PYG{l+s+s2}{================}\PYG{l+s+se}{\PYGZbs{}n}\PYG{l+s+s2}{\PYGZdq{}}\PYG{p}{)}
\PYG{n+nb}{print} \PYG{p}{(}\PYG{l+s+s2}{\PYGZdq{}}\PYG{l+s+s2}{Valeur booléenne. }\PYG{l+s+se}{\PYGZbs{}t}\PYG{l+s+s2}{Valeur :}\PYG{l+s+s2}{\PYGZdq{}}\PYG{p}{,} \PYG{n}{x\PYGZus{}bool}\PYG{p}{,} \PYG{l+s+s2}{\PYGZdq{}}\PYG{l+s+se}{\PYGZbs{}t}\PYG{l+s+se}{\PYGZbs{}t}\PYG{l+s+s2}{Le type :}\PYG{l+s+s2}{\PYGZdq{}}\PYG{p}{,} \PYG{n+nb}{type}\PYG{p}{(}\PYG{n}{x\PYGZus{}bool}\PYG{p}{)}\PYG{p}{,} \PYG{n}{end}\PYG{o}{=}\PYG{l+s+s2}{\PYGZdq{}}\PYG{l+s+se}{\PYGZbs{}n}\PYG{l+s+s2}{================}\PYG{l+s+se}{\PYGZbs{}n}\PYG{l+s+s2}{\PYGZdq{}}\PYG{p}{)}
\end{sphinxVerbatim}

\begin{sphinxVerbatim}[commandchars=\\\{\}]
			Avant conversion
================
La valeur de x : 1 				Le type : \PYGZlt{}class \PYGZsq{}int\PYGZsq{}\PYGZgt{}
================
			 Après conversion
================
virgule flottante. 	Valeur : 1.0 		Le type : \PYGZlt{}class \PYGZsq{}float\PYGZsq{}\PYGZgt{}
================
Chaine de caractères. 	Valeur : 1 		Le type : \PYGZlt{}class \PYGZsq{}str\PYGZsq{}\PYGZgt{}
================
Valeur booléenne. 	Valeur : True 		Le type : \PYGZlt{}class \PYGZsq{}bool\PYGZsq{}\PYGZgt{}
================
\end{sphinxVerbatim}

\begin{sphinxVerbatim}[commandchars=\\\{\}]
\PYG{n}{y} \PYG{o}{=} \PYG{l+m+mf}{4.5}

\PYG{n}{y\PYGZus{}int} \PYG{o}{=} \PYG{n+nb}{int}\PYG{p}{(}\PYG{n}{y}\PYG{p}{)}
\PYG{n}{y\PYGZus{}str} \PYG{o}{=} \PYG{n+nb}{str}\PYG{p}{(}\PYG{n}{y}\PYG{p}{)}
\PYG{n}{y\PYGZus{}bool} \PYG{o}{=} \PYG{n+nb}{bool}\PYG{p}{(}\PYG{n}{y}\PYG{p}{)}

\PYG{n+nb}{print} \PYG{p}{(}\PYG{l+s+s2}{\PYGZdq{}}\PYG{l+s+se}{\PYGZbs{}t}\PYG{l+s+se}{\PYGZbs{}t}\PYG{l+s+se}{\PYGZbs{}t}\PYG{l+s+s2}{Avant conversion}\PYG{l+s+s2}{\PYGZdq{}}\PYG{p}{,} \PYG{n}{end}\PYG{o}{=}\PYG{l+s+s2}{\PYGZdq{}}\PYG{l+s+se}{\PYGZbs{}n}\PYG{l+s+s2}{================}\PYG{l+s+se}{\PYGZbs{}n}\PYG{l+s+s2}{\PYGZdq{}}\PYG{p}{)}
\PYG{n+nb}{print} \PYG{p}{(}\PYG{l+s+s2}{\PYGZdq{}}\PYG{l+s+s2}{La valeur de y :}\PYG{l+s+s2}{\PYGZdq{}}\PYG{p}{,} \PYG{n}{y}\PYG{p}{,} \PYG{l+s+s2}{\PYGZdq{}}\PYG{l+s+se}{\PYGZbs{}t}\PYG{l+s+se}{\PYGZbs{}t}\PYG{l+s+se}{\PYGZbs{}t}\PYG{l+s+s2}{Le type :}\PYG{l+s+s2}{\PYGZdq{}}\PYG{p}{,} \PYG{n+nb}{type}\PYG{p}{(}\PYG{n}{y}\PYG{p}{)}\PYG{p}{,} \PYG{n}{end}\PYG{o}{=}\PYG{l+s+s2}{\PYGZdq{}}\PYG{l+s+se}{\PYGZbs{}n}\PYG{l+s+s2}{================}\PYG{l+s+se}{\PYGZbs{}n}\PYG{l+s+s2}{\PYGZdq{}}\PYG{p}{)}
\PYG{n+nb}{print} \PYG{p}{(}\PYG{l+s+s2}{\PYGZdq{}}\PYG{l+s+se}{\PYGZbs{}t}\PYG{l+s+se}{\PYGZbs{}t}\PYG{l+s+se}{\PYGZbs{}t}\PYG{l+s+s2}{Après conversion}\PYG{l+s+s2}{\PYGZdq{}}\PYG{p}{,} \PYG{n}{end}\PYG{o}{=}\PYG{l+s+s2}{\PYGZdq{}}\PYG{l+s+se}{\PYGZbs{}n}\PYG{l+s+s2}{================}\PYG{l+s+se}{\PYGZbs{}n}\PYG{l+s+s2}{\PYGZdq{}}\PYG{p}{)}
\PYG{n+nb}{print} \PYG{p}{(}\PYG{l+s+s2}{\PYGZdq{}}\PYG{l+s+s2}{Entier. }\PYG{l+s+se}{\PYGZbs{}t}\PYG{l+s+se}{\PYGZbs{}t}\PYG{l+s+s2}{Valeur :}\PYG{l+s+s2}{\PYGZdq{}}\PYG{p}{,} \PYG{n}{y\PYGZus{}int}\PYG{p}{,} \PYG{l+s+s2}{\PYGZdq{}}\PYG{l+s+se}{\PYGZbs{}t}\PYG{l+s+s2}{Le type :}\PYG{l+s+s2}{\PYGZdq{}}\PYG{p}{,} \PYG{n+nb}{type}\PYG{p}{(}\PYG{n}{y\PYGZus{}int}\PYG{p}{)}\PYG{p}{,} \PYG{n}{end}\PYG{o}{=}\PYG{l+s+s2}{\PYGZdq{}}\PYG{l+s+se}{\PYGZbs{}n}\PYG{l+s+s2}{================}\PYG{l+s+se}{\PYGZbs{}n}\PYG{l+s+s2}{\PYGZdq{}}\PYG{p}{)}
\PYG{n+nb}{print} \PYG{p}{(}\PYG{l+s+s2}{\PYGZdq{}}\PYG{l+s+s2}{Chaine de caractères. }\PYG{l+s+se}{\PYGZbs{}t}\PYG{l+s+s2}{Valeur :}\PYG{l+s+s2}{\PYGZdq{}}\PYG{p}{,} \PYG{n}{y\PYGZus{}str}\PYG{p}{,} \PYG{l+s+s2}{\PYGZdq{}}\PYG{l+s+se}{\PYGZbs{}t}\PYG{l+s+s2}{Le type :}\PYG{l+s+s2}{\PYGZdq{}}\PYG{p}{,} \PYG{n+nb}{type}\PYG{p}{(}\PYG{n}{y\PYGZus{}str}\PYG{p}{)}\PYG{p}{,} \PYG{n}{end}\PYG{o}{=}\PYG{l+s+s2}{\PYGZdq{}}\PYG{l+s+se}{\PYGZbs{}n}\PYG{l+s+s2}{================}\PYG{l+s+se}{\PYGZbs{}n}\PYG{l+s+s2}{\PYGZdq{}}\PYG{p}{)}
\PYG{n+nb}{print} \PYG{p}{(}\PYG{l+s+s2}{\PYGZdq{}}\PYG{l+s+s2}{Valeur booléenne. }\PYG{l+s+se}{\PYGZbs{}t}\PYG{l+s+s2}{Valeur :}\PYG{l+s+s2}{\PYGZdq{}}\PYG{p}{,} \PYG{n}{y\PYGZus{}bool}\PYG{p}{,} \PYG{l+s+s2}{\PYGZdq{}}\PYG{l+s+se}{\PYGZbs{}t}\PYG{l+s+s2}{Le type :}\PYG{l+s+s2}{\PYGZdq{}}\PYG{p}{,} \PYG{n+nb}{type}\PYG{p}{(}\PYG{n}{y\PYGZus{}bool}\PYG{p}{)}\PYG{p}{,} \PYG{n}{end}\PYG{o}{=}\PYG{l+s+s2}{\PYGZdq{}}\PYG{l+s+se}{\PYGZbs{}n}\PYG{l+s+s2}{================}\PYG{l+s+se}{\PYGZbs{}n}\PYG{l+s+s2}{\PYGZdq{}}\PYG{p}{)}
\end{sphinxVerbatim}

\begin{sphinxVerbatim}[commandchars=\\\{\}]
			Avant conversion
================
La valeur de y : 4.5 			Le type : \PYGZlt{}class \PYGZsq{}float\PYGZsq{}\PYGZgt{}
================
			Après conversion
================
Entier. 		Valeur : 4 	Le type : \PYGZlt{}class \PYGZsq{}int\PYGZsq{}\PYGZgt{}
================
Chaine de caractères. 	Valeur : 4.5 	Le type : \PYGZlt{}class \PYGZsq{}str\PYGZsq{}\PYGZgt{}
================
Valeur booléenne. 	Valeur : True 	Le type : \PYGZlt{}class \PYGZsq{}bool\PYGZsq{}\PYGZgt{}
================
\end{sphinxVerbatim}

\begin{sphinxVerbatim}[commandchars=\\\{\}]
\PYG{n}{z} \PYG{o}{=} \PYG{k+kc}{True}
\PYG{n}{z\PYGZus{}float} \PYG{o}{=} \PYG{n+nb}{float}\PYG{p}{(}\PYG{n}{z}\PYG{p}{)}
\PYG{n}{z\PYGZus{}str} \PYG{o}{=} \PYG{n+nb}{str}\PYG{p}{(}\PYG{n}{z}\PYG{p}{)}
\PYG{n}{z\PYGZus{}int} \PYG{o}{=} \PYG{n+nb}{int}\PYG{p}{(}\PYG{n}{z}\PYG{p}{)}


\PYG{n+nb}{print} \PYG{p}{(}\PYG{l+s+s2}{\PYGZdq{}}\PYG{l+s+se}{\PYGZbs{}t}\PYG{l+s+se}{\PYGZbs{}t}\PYG{l+s+se}{\PYGZbs{}t}\PYG{l+s+s2}{Avant conversion}\PYG{l+s+s2}{\PYGZdq{}}\PYG{p}{,} \PYG{n}{end}\PYG{o}{=}\PYG{l+s+s2}{\PYGZdq{}}\PYG{l+s+se}{\PYGZbs{}n}\PYG{l+s+s2}{================}\PYG{l+s+se}{\PYGZbs{}n}\PYG{l+s+s2}{\PYGZdq{}}\PYG{p}{)}
\PYG{n+nb}{print} \PYG{p}{(}\PYG{l+s+s2}{\PYGZdq{}}\PYG{l+s+s2}{La valeur de z :}\PYG{l+s+s2}{\PYGZdq{}}\PYG{p}{,} \PYG{n}{z}\PYG{p}{,} \PYG{l+s+s2}{\PYGZdq{}}\PYG{l+s+se}{\PYGZbs{}t}\PYG{l+s+se}{\PYGZbs{}t}\PYG{l+s+se}{\PYGZbs{}t}\PYG{l+s+s2}{Le type :}\PYG{l+s+s2}{\PYGZdq{}}\PYG{p}{,} \PYG{n+nb}{type}\PYG{p}{(}\PYG{n}{z}\PYG{p}{)}\PYG{p}{,} \PYG{n}{end}\PYG{o}{=}\PYG{l+s+s2}{\PYGZdq{}}\PYG{l+s+se}{\PYGZbs{}n}\PYG{l+s+s2}{================}\PYG{l+s+se}{\PYGZbs{}n}\PYG{l+s+s2}{\PYGZdq{}}\PYG{p}{)}
\PYG{n+nb}{print} \PYG{p}{(}\PYG{l+s+s2}{\PYGZdq{}}\PYG{l+s+se}{\PYGZbs{}t}\PYG{l+s+se}{\PYGZbs{}t}\PYG{l+s+se}{\PYGZbs{}t}\PYG{l+s+s2}{Après conversion}\PYG{l+s+s2}{\PYGZdq{}}\PYG{p}{,} \PYG{n}{end}\PYG{o}{=}\PYG{l+s+s2}{\PYGZdq{}}\PYG{l+s+se}{\PYGZbs{}n}\PYG{l+s+s2}{================}\PYG{l+s+se}{\PYGZbs{}n}\PYG{l+s+s2}{\PYGZdq{}}\PYG{p}{)}
\PYG{n+nb}{print} \PYG{p}{(}\PYG{l+s+s2}{\PYGZdq{}}\PYG{l+s+s2}{virgule flottante. }\PYG{l+s+se}{\PYGZbs{}t}\PYG{l+s+s2}{Valeur:}\PYG{l+s+s2}{\PYGZdq{}}\PYG{p}{,} \PYG{n}{z\PYGZus{}float}\PYG{p}{,} \PYG{l+s+s2}{\PYGZdq{}}\PYG{l+s+se}{\PYGZbs{}t}\PYG{l+s+s2}{Le type :}\PYG{l+s+s2}{\PYGZdq{}}\PYG{p}{,} \PYG{n+nb}{type}\PYG{p}{(}\PYG{n}{z\PYGZus{}float}\PYG{p}{)}\PYG{p}{,} \PYG{n}{end}\PYG{o}{=}\PYG{l+s+s2}{\PYGZdq{}}\PYG{l+s+se}{\PYGZbs{}n}\PYG{l+s+s2}{================}\PYG{l+s+se}{\PYGZbs{}n}\PYG{l+s+s2}{\PYGZdq{}}\PYG{p}{)}
\PYG{n+nb}{print} \PYG{p}{(}\PYG{l+s+s2}{\PYGZdq{}}\PYG{l+s+s2}{Chaine de caractères. }\PYG{l+s+se}{\PYGZbs{}t}\PYG{l+s+s2}{Valeur:}\PYG{l+s+s2}{\PYGZdq{}}\PYG{p}{,} \PYG{n}{z\PYGZus{}str}\PYG{p}{,} \PYG{l+s+s2}{\PYGZdq{}}\PYG{l+s+se}{\PYGZbs{}t}\PYG{l+s+s2}{Le type :}\PYG{l+s+s2}{\PYGZdq{}}\PYG{p}{,} \PYG{n+nb}{type}\PYG{p}{(}\PYG{n}{z\PYGZus{}str}\PYG{p}{)}\PYG{p}{,} \PYG{n}{end}\PYG{o}{=}\PYG{l+s+s2}{\PYGZdq{}}\PYG{l+s+se}{\PYGZbs{}n}\PYG{l+s+s2}{================}\PYG{l+s+se}{\PYGZbs{}n}\PYG{l+s+s2}{\PYGZdq{}}\PYG{p}{)}
\PYG{n+nb}{print} \PYG{p}{(}\PYG{l+s+s2}{\PYGZdq{}}\PYG{l+s+s2}{Entier. }\PYG{l+s+se}{\PYGZbs{}t}\PYG{l+s+se}{\PYGZbs{}t}\PYG{l+s+s2}{Valeur:}\PYG{l+s+s2}{\PYGZdq{}}\PYG{p}{,} \PYG{n}{z\PYGZus{}int}\PYG{p}{,} \PYG{l+s+s2}{\PYGZdq{}}\PYG{l+s+se}{\PYGZbs{}t}\PYG{l+s+s2}{Le type :}\PYG{l+s+s2}{\PYGZdq{}}\PYG{p}{,} \PYG{n+nb}{type}\PYG{p}{(}\PYG{n}{z\PYGZus{}int}\PYG{p}{)}\PYG{p}{,} \PYG{n}{end}\PYG{o}{=}\PYG{l+s+s2}{\PYGZdq{}}\PYG{l+s+se}{\PYGZbs{}n}\PYG{l+s+s2}{================}\PYG{l+s+se}{\PYGZbs{}n}\PYG{l+s+s2}{\PYGZdq{}}\PYG{p}{)}
\end{sphinxVerbatim}

\begin{sphinxVerbatim}[commandchars=\\\{\}]
			Avant conversion
================
La valeur de z : True 			Le type : \PYGZlt{}class \PYGZsq{}bool\PYGZsq{}\PYGZgt{}
================
			Après conversion
================
virgule flottante. 	Valeur: 1.0 	Le type : \PYGZlt{}class \PYGZsq{}float\PYGZsq{}\PYGZgt{}
================
Chaine de caractères. 	Valeur: True 	Le type : \PYGZlt{}class \PYGZsq{}str\PYGZsq{}\PYGZgt{}
================
Entier. 		Valeur: 1 	Le type : \PYGZlt{}class \PYGZsq{}int\PYGZsq{}\PYGZgt{}
================
\end{sphinxVerbatim}

\sphinxAtStartPar
Le type de données \sphinxcode{\sphinxupquote{str}} est un peu particulier. On doit faire attention lors de la conversion. Si la valeur est une chaine de caractères qui contient des lettres (a, b,…) ou d’autres symboles (+, @, …). La conversion en \sphinxcode{\sphinxupquote{int}} ou \sphinxcode{\sphinxupquote{float}} renvoi une erreur.

\begin{sphinxVerbatim}[commandchars=\\\{\}]
\PYG{n}{x} \PYG{o}{=} \PYG{l+s+s2}{\PYGZdq{}}\PYG{l+s+s2}{bonjour}\PYG{l+s+s2}{\PYGZdq{}}
\PYG{n+nb}{print}\PYG{p}{(}\PYG{n+nb}{int}\PYG{p}{(}\PYG{n}{x}\PYG{p}{)}\PYG{p}{)}
\end{sphinxVerbatim}

\begin{sphinxVerbatim}[commandchars=\\\{\}]
\PYG{g+gt}{\PYGZhy{}\PYGZhy{}\PYGZhy{}\PYGZhy{}\PYGZhy{}\PYGZhy{}\PYGZhy{}\PYGZhy{}\PYGZhy{}\PYGZhy{}\PYGZhy{}\PYGZhy{}\PYGZhy{}\PYGZhy{}\PYGZhy{}\PYGZhy{}\PYGZhy{}\PYGZhy{}\PYGZhy{}\PYGZhy{}\PYGZhy{}\PYGZhy{}\PYGZhy{}\PYGZhy{}\PYGZhy{}\PYGZhy{}\PYGZhy{}\PYGZhy{}\PYGZhy{}\PYGZhy{}\PYGZhy{}\PYGZhy{}\PYGZhy{}\PYGZhy{}\PYGZhy{}\PYGZhy{}\PYGZhy{}\PYGZhy{}\PYGZhy{}\PYGZhy{}\PYGZhy{}\PYGZhy{}\PYGZhy{}\PYGZhy{}\PYGZhy{}\PYGZhy{}\PYGZhy{}\PYGZhy{}\PYGZhy{}\PYGZhy{}\PYGZhy{}\PYGZhy{}\PYGZhy{}\PYGZhy{}\PYGZhy{}\PYGZhy{}\PYGZhy{}\PYGZhy{}\PYGZhy{}\PYGZhy{}\PYGZhy{}\PYGZhy{}\PYGZhy{}\PYGZhy{}\PYGZhy{}\PYGZhy{}\PYGZhy{}\PYGZhy{}\PYGZhy{}\PYGZhy{}\PYGZhy{}\PYGZhy{}\PYGZhy{}\PYGZhy{}\PYGZhy{}}
\PYG{n+ne}{ValueError}\PYG{g+gWhitespace}{                                }Traceback (most recent call last)
\PYG{o}{\PYGZti{}}\PYGZbs{}\PYG{n}{AppData}\PYGZbs{}\PYG{n}{Local}\PYGZbs{}\PYG{n}{Temp}\PYGZbs{}\PYG{n}{ipykernel\PYGZus{}14492}\PYGZbs{}\PYG{l+m+mf}{389749341.}\PYG{n}{py} \PYG{o+ow}{in} \PYG{o}{\PYGZlt{}}\PYG{n}{module}\PYG{o}{\PYGZgt{}}
\PYG{g+gWhitespace}{      }\PYG{l+m+mi}{1} \PYG{n}{x} \PYG{o}{=} \PYG{l+s+s2}{\PYGZdq{}}\PYG{l+s+s2}{bonjour}\PYG{l+s+s2}{\PYGZdq{}}
\PYG{n+ne}{\PYGZhy{}\PYGZhy{}\PYGZhy{}\PYGZhy{}\PYGZgt{} }\PYG{l+m+mi}{2} \PYG{n+nb}{print}\PYG{p}{(}\PYG{n+nb}{int}\PYG{p}{(}\PYG{n}{x}\PYG{p}{)}\PYG{p}{)}

\PYG{n+ne}{ValueError}: invalid literal for int() with base 10: \PYGZsq{}bonjour\PYGZsq{}
\end{sphinxVerbatim}

\begin{sphinxVerbatim}[commandchars=\\\{\}]
\PYG{n}{x} \PYG{o}{=} \PYG{l+s+s2}{\PYGZdq{}}\PYG{l+s+s2}{bonjour}\PYG{l+s+s2}{\PYGZdq{}}
\PYG{n+nb}{print}\PYG{p}{(}\PYG{n+nb}{float}\PYG{p}{(}\PYG{n}{x}\PYG{p}{)}\PYG{p}{)}
\end{sphinxVerbatim}

\begin{sphinxVerbatim}[commandchars=\\\{\}]
\PYG{g+gt}{\PYGZhy{}\PYGZhy{}\PYGZhy{}\PYGZhy{}\PYGZhy{}\PYGZhy{}\PYGZhy{}\PYGZhy{}\PYGZhy{}\PYGZhy{}\PYGZhy{}\PYGZhy{}\PYGZhy{}\PYGZhy{}\PYGZhy{}\PYGZhy{}\PYGZhy{}\PYGZhy{}\PYGZhy{}\PYGZhy{}\PYGZhy{}\PYGZhy{}\PYGZhy{}\PYGZhy{}\PYGZhy{}\PYGZhy{}\PYGZhy{}\PYGZhy{}\PYGZhy{}\PYGZhy{}\PYGZhy{}\PYGZhy{}\PYGZhy{}\PYGZhy{}\PYGZhy{}\PYGZhy{}\PYGZhy{}\PYGZhy{}\PYGZhy{}\PYGZhy{}\PYGZhy{}\PYGZhy{}\PYGZhy{}\PYGZhy{}\PYGZhy{}\PYGZhy{}\PYGZhy{}\PYGZhy{}\PYGZhy{}\PYGZhy{}\PYGZhy{}\PYGZhy{}\PYGZhy{}\PYGZhy{}\PYGZhy{}\PYGZhy{}\PYGZhy{}\PYGZhy{}\PYGZhy{}\PYGZhy{}\PYGZhy{}\PYGZhy{}\PYGZhy{}\PYGZhy{}\PYGZhy{}\PYGZhy{}\PYGZhy{}\PYGZhy{}\PYGZhy{}\PYGZhy{}\PYGZhy{}\PYGZhy{}\PYGZhy{}\PYGZhy{}\PYGZhy{}}
\PYG{n+ne}{ValueError}\PYG{g+gWhitespace}{                                }Traceback (most recent call last)
\PYG{o}{\PYGZti{}}\PYGZbs{}\PYG{n}{AppData}\PYGZbs{}\PYG{n}{Local}\PYGZbs{}\PYG{n}{Temp}\PYGZbs{}\PYG{n}{ipykernel\PYGZus{}14492}\PYGZbs{}\PYG{l+m+mf}{1364929283.}\PYG{n}{py} \PYG{o+ow}{in} \PYG{o}{\PYGZlt{}}\PYG{n}{module}\PYG{o}{\PYGZgt{}}
\PYG{g+gWhitespace}{      }\PYG{l+m+mi}{1} \PYG{n}{x} \PYG{o}{=} \PYG{l+s+s2}{\PYGZdq{}}\PYG{l+s+s2}{bonjour}\PYG{l+s+s2}{\PYGZdq{}}
\PYG{n+ne}{\PYGZhy{}\PYGZhy{}\PYGZhy{}\PYGZhy{}\PYGZgt{} }\PYG{l+m+mi}{2} \PYG{n+nb}{print}\PYG{p}{(}\PYG{n+nb}{float}\PYG{p}{(}\PYG{n}{x}\PYG{p}{)}\PYG{p}{)}

\PYG{n+ne}{ValueError}: could not convert string to float: \PYGZsq{}bonjour\PYGZsq{}
\end{sphinxVerbatim}

\begin{sphinxadmonition}{note}{Note:}
\sphinxAtStartPar
Lorsqu’on convertit tout valeur (à l’exception de\sphinxcode{\sphinxupquote{0}}, \sphinxcode{\sphinxupquote{\textquotesingle{}\textquotesingle{}}}, \sphinxcode{\sphinxupquote{None}}, \sphinxcode{\sphinxupquote{{[}{]}}}, et tous les types composés ou structurés vides) on reçoit \sphinxcode{\sphinxupquote{True}}.
\end{sphinxadmonition}

\begin{sphinxVerbatim}[commandchars=\\\{\}]
\PYG{n+nb}{print}\PYG{p}{(}\PYG{n+nb}{bool}\PYG{p}{(}\PYG{l+s+s2}{\PYGZdq{}}\PYG{l+s+s2}{bonjour}\PYG{l+s+s2}{\PYGZdq{}}\PYG{p}{)}\PYG{p}{)}
\PYG{n+nb}{print}\PYG{p}{(}\PYG{n+nb}{bool}\PYG{p}{(}\PYG{l+s+s2}{\PYGZdq{}}\PYG{l+s+s2}{5}\PYG{l+s+s2}{\PYGZdq{}}\PYG{p}{)}\PYG{p}{)}
\PYG{n+nb}{print}\PYG{p}{(}\PYG{n+nb}{bool}\PYG{p}{(}\PYG{l+m+mf}{1.5}\PYG{p}{)}\PYG{p}{)}

\PYG{n+nb}{print}\PYG{p}{(}\PYG{n+nb}{bool}\PYG{p}{(}\PYG{l+m+mi}{0}\PYG{p}{)}\PYG{p}{)}
\PYG{n+nb}{print}\PYG{p}{(}\PYG{n+nb}{bool}\PYG{p}{(}\PYG{l+s+s1}{\PYGZsq{}}\PYG{l+s+s1}{\PYGZsq{}}\PYG{p}{)}\PYG{p}{)}
\PYG{n+nb}{print}\PYG{p}{(}\PYG{n+nb}{bool}\PYG{p}{(}\PYG{k+kc}{None}\PYG{p}{)}\PYG{p}{)}
\PYG{n+nb}{print}\PYG{p}{(}\PYG{n+nb}{bool}\PYG{p}{(}\PYG{p}{[}\PYG{p}{]}\PYG{p}{)}\PYG{p}{)}
\end{sphinxVerbatim}

\begin{sphinxVerbatim}[commandchars=\\\{\}]
True
True
True
False
False
False
False
False
\end{sphinxVerbatim}
\begin{itemize}
\item {} 
\sphinxAtStartPar
Si la chaine de caractères est un nombre (par exemple \sphinxcode{\sphinxupquote{"1.4"}}) on peut la convertir en \sphinxcode{\sphinxupquote{float}}

\item {} 
\sphinxAtStartPar
Si la chaine de caractères est un nombre entier (par exemple \sphinxcode{\sphinxupquote{"5"}}), on peut la convertir en \sphinxcode{\sphinxupquote{int}}

\end{itemize}

\begin{sphinxVerbatim}[commandchars=\\\{\}]
\PYG{c+c1}{\PYGZsh{}}
\PYG{n+nb}{print}\PYG{p}{(}\PYG{n+nb}{float}\PYG{p}{(}\PYG{l+s+s2}{\PYGZdq{}}\PYG{l+s+s2}{1.4}\PYG{l+s+s2}{\PYGZdq{}}\PYG{p}{)}\PYG{p}{)}
\PYG{n+nb}{print}\PYG{p}{(}\PYG{n+nb}{int}\PYG{p}{(}\PYG{l+s+s2}{\PYGZdq{}}\PYG{l+s+s2}{1}\PYG{l+s+s2}{\PYGZdq{}}\PYG{p}{)}\PYG{p}{)}
\end{sphinxVerbatim}

\begin{sphinxVerbatim}[commandchars=\\\{\}]
1.4
1
\end{sphinxVerbatim}

\sphinxAtStartPar
Cependant, On ne peut pas convertir un nombre en virgule flottante directement en \sphinxcode{\sphinxupquote{int}}

\begin{sphinxVerbatim}[commandchars=\\\{\}]
\PYG{n+nb}{print}\PYG{p}{(}\PYG{n+nb}{int}\PYG{p}{(}\PYG{l+s+s2}{\PYGZdq{}}\PYG{l+s+s2}{1.4}\PYG{l+s+s2}{\PYGZdq{}}\PYG{p}{)}\PYG{p}{)}
\end{sphinxVerbatim}

\begin{sphinxVerbatim}[commandchars=\\\{\}]
\PYG{g+gt}{\PYGZhy{}\PYGZhy{}\PYGZhy{}\PYGZhy{}\PYGZhy{}\PYGZhy{}\PYGZhy{}\PYGZhy{}\PYGZhy{}\PYGZhy{}\PYGZhy{}\PYGZhy{}\PYGZhy{}\PYGZhy{}\PYGZhy{}\PYGZhy{}\PYGZhy{}\PYGZhy{}\PYGZhy{}\PYGZhy{}\PYGZhy{}\PYGZhy{}\PYGZhy{}\PYGZhy{}\PYGZhy{}\PYGZhy{}\PYGZhy{}\PYGZhy{}\PYGZhy{}\PYGZhy{}\PYGZhy{}\PYGZhy{}\PYGZhy{}\PYGZhy{}\PYGZhy{}\PYGZhy{}\PYGZhy{}\PYGZhy{}\PYGZhy{}\PYGZhy{}\PYGZhy{}\PYGZhy{}\PYGZhy{}\PYGZhy{}\PYGZhy{}\PYGZhy{}\PYGZhy{}\PYGZhy{}\PYGZhy{}\PYGZhy{}\PYGZhy{}\PYGZhy{}\PYGZhy{}\PYGZhy{}\PYGZhy{}\PYGZhy{}\PYGZhy{}\PYGZhy{}\PYGZhy{}\PYGZhy{}\PYGZhy{}\PYGZhy{}\PYGZhy{}\PYGZhy{}\PYGZhy{}\PYGZhy{}\PYGZhy{}\PYGZhy{}\PYGZhy{}\PYGZhy{}\PYGZhy{}\PYGZhy{}\PYGZhy{}\PYGZhy{}\PYGZhy{}}
\PYG{n+ne}{ValueError}\PYG{g+gWhitespace}{                                }Traceback (most recent call last)
\PYG{o}{\PYGZti{}}\PYGZbs{}\PYG{n}{AppData}\PYGZbs{}\PYG{n}{Local}\PYGZbs{}\PYG{n}{Temp}\PYGZbs{}\PYG{n}{ipykernel\PYGZus{}14492}\PYGZbs{}\PYG{l+m+mf}{2223022832.}\PYG{n}{py} \PYG{o+ow}{in} \PYG{o}{\PYGZlt{}}\PYG{n}{module}\PYG{o}{\PYGZgt{}}
\PYG{n+ne}{\PYGZhy{}\PYGZhy{}\PYGZhy{}\PYGZhy{}\PYGZgt{} }\PYG{l+m+mi}{1} \PYG{n+nb}{print}\PYG{p}{(}\PYG{n+nb}{int}\PYG{p}{(}\PYG{l+s+s2}{\PYGZdq{}}\PYG{l+s+s2}{1.4}\PYG{l+s+s2}{\PYGZdq{}}\PYG{p}{)}\PYG{p}{)}

\PYG{n+ne}{ValueError}: invalid literal for int() with base 10: \PYGZsq{}1.4\PYGZsq{}
\end{sphinxVerbatim}

\sphinxAtStartPar
L’astuce est de le convertir en \sphinxcode{\sphinxupquote{float}} puis en \sphinxcode{\sphinxupquote{int}}:

\begin{sphinxVerbatim}[commandchars=\\\{\}]
\PYG{c+c1}{\PYGZsh{}}
\PYG{n+nb}{int}\PYG{p}{(}\PYG{n+nb}{float}\PYG{p}{(}\PYG{l+s+s2}{\PYGZdq{}}\PYG{l+s+s2}{1.4}\PYG{l+s+s2}{\PYGZdq{}}\PYG{p}{)}\PYG{p}{)}
\end{sphinxVerbatim}

\begin{sphinxVerbatim}[commandchars=\\\{\}]
1
\end{sphinxVerbatim}


\section{Essayez\sphinxhyphen{}vous\sphinxhyphen{}même !}
\label{\detokenize{exo1:essayez-vous-meme}}\label{\detokenize{exo1::doc}}



\subsection{Exercice 1.}
\label{\detokenize{exo1:exercice-1}}
\sphinxAtStartPar
Ordre des opérations : quelle est selon vous le résultat de ces opérations :
\begin{itemize}
\item {} 
\sphinxAtStartPar
\((4/2)^2\times 2 + 1\), \(4/2^{(2\times 2)} + 1\),

\item {} 
\sphinxAtStartPar
\(4/2^2\times (2 + 1)\),

\item {} 
\sphinxAtStartPar
\(4/2^2\times 2 + 1\)
verifier avec Python.

\end{itemize}

\begin{sphinxVerbatim}[commandchars=\\\{\}]
\PYG{c+c1}{\PYGZsh{}\PYGZsh{} votre code ici}
\end{sphinxVerbatim}




\subsection{Exercice 2.}
\label{\detokenize{exo1:exercice-2}}
\sphinxAtStartPar
Un père a une somme d’argent de 1554 dh, il veut la partager sur ses 9 enfants de manière équitable et s’il reste quelque dirham, il va acheter des chocolats a 1 dh l’unité. Combien chaque enfant va recevoir ? combien d’unité de chocolat peut\sphinxhyphen{}il acheter avec le reste?

\begin{sphinxVerbatim}[commandchars=\\\{\}]
\PYG{c+c1}{\PYGZsh{}\PYGZsh{} votre code ici}
\end{sphinxVerbatim}




\subsection{Exercice 3.}
\label{\detokenize{exo1:exercice-3}}
\sphinxAtStartPar
Quel est le type de données de valeurs suivantes :
\begin{itemize}
\item {} 
\sphinxAtStartPar
\sphinxcode{\sphinxupquote{1}}

\item {} 
\sphinxAtStartPar
\sphinxcode{\sphinxupquote{1.}}

\item {} 
\sphinxAtStartPar
\sphinxcode{\sphinxupquote{False}}

\item {} 
\sphinxAtStartPar
\sphinxcode{\sphinxupquote{"False"}}

\item {} 
\sphinxAtStartPar
\sphinxcode{\sphinxupquote{var1/var2}} avec \sphinxcode{\sphinxupquote{var1 = 1}} et \sphinxcode{\sphinxupquote{var2 = 2}}

\end{itemize}

\begin{sphinxVerbatim}[commandchars=\\\{\}]
\PYG{c+c1}{\PYGZsh{}\PYGZsh{} votre code ici}
\end{sphinxVerbatim}




\subsection{Exercice 4.}
\label{\detokenize{exo1:exercice-4}}
\sphinxAtStartPar
Quel est le type de données de valeurs suivantes :
\begin{itemize}
\item {} 
\sphinxAtStartPar
\sphinxcode{\sphinxupquote{1}}

\item {} 
\sphinxAtStartPar
\sphinxcode{\sphinxupquote{1.}}

\item {} 
\sphinxAtStartPar
\sphinxcode{\sphinxupquote{False}}

\item {} 
\sphinxAtStartPar
\sphinxcode{\sphinxupquote{"False"}}

\item {} 
\sphinxAtStartPar
\sphinxcode{\sphinxupquote{"5.4"}}

\item {} 
\sphinxAtStartPar
\sphinxcode{\sphinxupquote{var1/var2}} avec \sphinxcode{\sphinxupquote{var1 = 1}} et \sphinxcode{\sphinxupquote{var2 = 2}}

\item {} 
\sphinxAtStartPar
\sphinxcode{\sphinxupquote{list()}}

\item {} 
\sphinxAtStartPar
\sphinxcode{\sphinxupquote{None}}

\item {} 
\sphinxAtStartPar
\sphinxcode{\sphinxupquote{""}}

\end{itemize}

\begin{sphinxVerbatim}[commandchars=\\\{\}]
\PYG{c+c1}{\PYGZsh{}\PYGZsh{} votre code ici}
\end{sphinxVerbatim}




\subsection{Exercice 5.}
\label{\detokenize{exo1:exercice-5}}
\sphinxAtStartPar
Convertir, si c’est possible, de valeurs suivantes aux types de données que nous avons vues. Expliciter les cas qui ne sont pas possible :
\begin{itemize}
\item {} 
\sphinxAtStartPar
\sphinxcode{\sphinxupquote{1}}

\item {} 
\sphinxAtStartPar
\sphinxcode{\sphinxupquote{1.}}

\item {} 
\sphinxAtStartPar
\sphinxcode{\sphinxupquote{False}}

\item {} 
\sphinxAtStartPar
\sphinxcode{\sphinxupquote{"False"}}

\item {} 
\sphinxAtStartPar
\sphinxcode{\sphinxupquote{"5.4"}}

\item {} 
\sphinxAtStartPar
\sphinxcode{\sphinxupquote{var1/var2}} avec \sphinxcode{\sphinxupquote{var1 = 1}} et \sphinxcode{\sphinxupquote{var2 = 2}}

\item {} 
\sphinxAtStartPar
\sphinxcode{\sphinxupquote{list()}}

\item {} 
\sphinxAtStartPar
\sphinxcode{\sphinxupquote{None}}

\item {} 
\sphinxAtStartPar
\sphinxcode{\sphinxupquote{""}}

\end{itemize}

\begin{sphinxVerbatim}[commandchars=\\\{\}]
\PYG{c+c1}{\PYGZsh{}\PYGZsh{} votre code ici}
\end{sphinxVerbatim}




\subsection{Exercice 6.}
\label{\detokenize{exo1:exercice-6}}
\sphinxAtStartPar
Soient \sphinxcode{\sphinxupquote{x = True}}, \sphinxcode{\sphinxupquote{y= 7\textless{}6}}, and \sphinxcode{\sphinxupquote{z= not y}}. Déterminer la valeur logique de \sphinxcode{\sphinxupquote{x, y,}} et \sphinxcode{\sphinxupquote{z}} (\sphinxcode{\sphinxupquote{True}} ou \sphinxcode{\sphinxupquote{False}}) puis la valeur de chacun des expression suivantes:
\begin{itemize}
\item {} 
\sphinxAtStartPar
\sphinxcode{\sphinxupquote{x != False}}

\item {} 
\sphinxAtStartPar
\sphinxcode{\sphinxupquote{x and y}}

\item {} 
\sphinxAtStartPar
\sphinxcode{\sphinxupquote{x or y}}

\item {} 
\sphinxAtStartPar
\sphinxcode{\sphinxupquote{not y}}

\item {} 
\sphinxAtStartPar
\sphinxcode{\sphinxupquote{x and (y or z)}}

\item {} 
\sphinxAtStartPar
\sphinxcode{\sphinxupquote{(x and y) or z}}

\item {} 
\sphinxAtStartPar
\sphinxcode{\sphinxupquote{(not x or not y) and (not z)}}

\item {} 
\sphinxAtStartPar
\sphinxcode{\sphinxupquote{not ((x and y) or z)}}

\end{itemize}

\begin{sphinxVerbatim}[commandchars=\\\{\}]
\PYG{c+c1}{\PYGZsh{}\PYGZsh{} votre code ici}
\end{sphinxVerbatim}




\subsection{Exercice 7.}
\label{\detokenize{exo1:exercice-7}}
\sphinxAtStartPar
On veut recevoir ce message avec \sphinxcode{\sphinxupquote{print()}}:

\begin{sphinxVerbatim}[commandchars=\\\{\}]
\PYG{n}{La} \PYG{n}{valeur} \PYG{n}{de} \PYG{n}{x} \PYG{n}{est} \PYG{p}{:} \PYG{o}{@}\PYG{n+nd}{@True}\PYG{o}{@}\PYG{o}{@}
\PYG{o}{*}\PYG{o}{*}\PYG{o}{*}\PYG{o}{*}\PYG{o}{*}\PYG{o}{*}\PYG{o}{*}\PYG{o}{*}\PYG{o}{*}\PYG{o}{==}\PYG{o}{==}\PYG{o}{==}\PYG{o}{==}\PYG{o}{==}\PYG{o}{==}\PYG{o}{==}\PYG{o}{*}\PYG{o}{*}\PYG{o}{*}\PYG{o}{*}\PYG{o}{*}\PYG{o}{*}\PYG{o}{*}\PYG{o}{*}
\end{sphinxVerbatim}

\sphinxAtStartPar
En utilisant les arguments suivants :
\begin{itemize}
\item {} 
\sphinxAtStartPar
\sphinxcode{\sphinxupquote{x = True}}

\item {} 
\sphinxAtStartPar
\sphinxcode{\sphinxupquote{a = "la valeur de x est :"}}

\end{itemize}

\sphinxAtStartPar
Toute modification devra être faite au niveau de \sphinxcode{\sphinxupquote{sep=}}, et \sphinxcode{\sphinxupquote{end=}}.

\begin{sphinxVerbatim}[commandchars=\\\{\}]
\PYG{c+c1}{\PYGZsh{}\PYGZsh{} votre code ici}
\end{sphinxVerbatim}




\subsection{Exercice 8.}
\label{\detokenize{exo1:exercice-8}}
\sphinxAtStartPar
Ecrire un petit programme qui permet de demander à l’utilisateur d’entrer son nom, son poids en kilogramme (sans entrer l’unité), et sa taille en mètre (sans entrer l’unité). Puis il affiche l’indice du poids (Body mass index (BMI)) :
\sphinxstyleemphasis{Formule du BMI} : \(BMI = \dfrac{poids}{taille^2}\).

\begin{sphinxVerbatim}[commandchars=\\\{\}]
\PYG{c+c1}{\PYGZsh{}\PYGZsh{} votre code ici}
\end{sphinxVerbatim}


\chapter{Les elements de base}
\label{\detokenize{content2:les-elements-de-base}}\label{\detokenize{content2::doc}}

\section{Instructions conditionnelles et boucles}
\label{\detokenize{ch2:instructions-conditionnelles-et-boucles}}\label{\detokenize{ch2::doc}}

\subsection{Instructions conditionnelles (tests)}
\label{\detokenize{ch2:instructions-conditionnelles-tests}}
\sphinxAtStartPar
En Python, il existe trois structures de test :
\begin{itemize}
\item {} 
\sphinxAtStartPar
La première est la plus simple contenant juste \sphinxcode{\sphinxupquote{if}} :

\end{itemize}

\begin{sphinxVerbatim}[commandchars=\\\{\}]
\PYG{k}{if} \PYG{n}{condition}\PYG{p}{:}
    \PYG{n}{instructions}
\end{sphinxVerbatim}
\begin{itemize}
\item {} 
\sphinxAtStartPar
La deuxième fait apparaitre \sphinxcode{\sphinxupquote{else}} :

\end{itemize}

\begin{sphinxVerbatim}[commandchars=\\\{\}]
\PYG{k}{if} \PYG{n}{condition}\PYG{p}{:}
    \PYG{n}{instructions\PYGZus{}1}
\PYG{k}{else}\PYG{p}{:}
    \PYG{n}{instructions\PYGZus{}2}
\end{sphinxVerbatim}
\begin{itemize}
\item {} 
\sphinxAtStartPar
La dernière fait apparaitre \sphinxcode{\sphinxupquote{elif}} :

\end{itemize}

\begin{sphinxVerbatim}[commandchars=\\\{\}]
\PYG{k}{if} \PYG{n}{condition\PYGZus{}1}\PYG{p}{:}
    \PYG{n}{Bloc} \PYG{n}{d}\PYG{l+s+s1}{\PYGZsq{}}\PYG{l+s+s1}{instructions 1}
\PYG{k}{elif} \PYG{n}{condition\PYGZus{}2}\PYG{p}{:}
    \PYG{n}{Bloc} \PYG{n}{d}\PYG{l+s+s1}{\PYGZsq{}}\PYG{l+s+s1}{instructions 2}
\PYG{k}{elif} \PYG{n}{condition\PYGZus{}3}\PYG{p}{:}
    \PYG{n}{Bloc} \PYG{n}{d}\PYG{l+s+s1}{\PYGZsq{}}\PYG{l+s+s1}{instructions 3}
\PYG{o}{.}\PYG{o}{.}\PYG{o}{.}
\PYG{k}{elif} \PYG{n}{condition\PYGZus{}n}\PYG{p}{:}
    \PYG{n}{Bloc} \PYG{n}{d}\PYG{l+s+s1}{\PYGZsq{}}\PYG{l+s+s1}{instructions n}
\PYG{k}{else}\PYG{p}{:}
    \PYG{n}{Bloc} \PYG{n}{d}\PYG{l+s+s1}{\PYGZsq{}}\PYG{l+s+s1}{instructions n+1}
\end{sphinxVerbatim}

\sphinxAtStartPar
\sphinxstylestrong{Bloc d’instructions \sphinxhyphen{} Indentation}: En python, on définit un bloc d’instructions par une indentation. Ceci dit, on décale le début des instructions vers la droite à l’aide des espaces en début de chaque ligne du bloc (le nombre standard est 4 espaces mais ce n’est pas obligatoire). Toutes les instructions d’un même bloc doivent être indentées exactement au même niveau.

\sphinxAtStartPar
Revenons à nos structures de test. La logique est la suivante :
\begin{itemize}
\item {} 
\sphinxAtStartPar
pour la structure simple \sphinxcode{\sphinxupquote{if}}: si  la condition est vrai, le bloc d’instructions en dessous est exécuté. Sinon, il ne sera pas exécuté.

\end{itemize}

\begin{sphinxVerbatim}[commandchars=\\\{\}]
\PYG{n}{x} \PYG{o}{=} \PYG{l+m+mi}{10}
\PYG{c+c1}{\PYGZsh{} la condition est x\PYGZlt{}15}
\PYG{k}{if} \PYG{n}{x} \PYG{o}{\PYGZlt{}} \PYG{l+m+mi}{15}\PYG{p}{:}
    \PYG{n+nb}{print}\PYG{p}{(}\PYG{n}{x}\PYG{p}{,} \PYG{l+s+s2}{\PYGZdq{}}\PYG{l+s+s2}{est plus petit que 15}\PYG{l+s+s2}{\PYGZdq{}}\PYG{p}{)}
\end{sphinxVerbatim}

\begin{sphinxVerbatim}[commandchars=\\\{\}]
10 est plus petit que 15
\end{sphinxVerbatim}

\begin{sphinxVerbatim}[commandchars=\\\{\}]
\PYG{n}{x} \PYG{o}{=} \PYG{l+m+mi}{10}
\PYG{c+c1}{\PYGZsh{} la condition est x==15}
\PYG{k}{if} \PYG{n}{x} \PYG{o}{==} \PYG{l+m+mi}{15}\PYG{p}{:}
    \PYG{n+nb}{print}\PYG{p}{(}\PYG{n}{x}\PYG{p}{,} \PYG{l+s+s2}{\PYGZdq{}}\PYG{l+s+s2}{est egale a 15}\PYG{l+s+s2}{\PYGZdq{}}\PYG{p}{)}
\end{sphinxVerbatim}
\begin{itemize}
\item {} 
\sphinxAtStartPar
Pour la structure \sphinxcode{\sphinxupquote{if...else}} : si la condition de \sphinxcode{\sphinxupquote{if}} est vrai, le bloc d’instructions en dessous de \sphinxcode{\sphinxupquote{if}} sera exécuté. Sinon, le bloc en dessous de \sphinxcode{\sphinxupquote{else}} sera exécuté.

\end{itemize}

\begin{sphinxVerbatim}[commandchars=\\\{\}]
\PYG{n}{x} \PYG{o}{=} \PYG{l+m+mi}{10}
\PYG{c+c1}{\PYGZsh{} la condition est x\PYGZlt{}10}
\PYG{k}{if} \PYG{n}{x} \PYG{o}{\PYGZlt{}} \PYG{l+m+mi}{10}\PYG{p}{:}
    \PYG{n+nb}{print}\PYG{p}{(}\PYG{n}{x}\PYG{p}{,} \PYG{l+s+s2}{\PYGZdq{}}\PYG{l+s+s2}{est plus grand que 10}\PYG{l+s+s2}{\PYGZdq{}}\PYG{p}{)}
\PYG{k}{else}\PYG{p}{:}
    \PYG{n+nb}{print}\PYG{p}{(}\PYG{n}{x}\PYG{p}{,} \PYG{l+s+s2}{\PYGZdq{}}\PYG{l+s+s2}{est plus petit ou égale à 10}\PYG{l+s+s2}{\PYGZdq{}}\PYG{p}{)}
\end{sphinxVerbatim}

\begin{sphinxVerbatim}[commandchars=\\\{\}]
10 est plus petit ou égale à 10
\end{sphinxVerbatim}
\begin{itemize}
\item {} 
\sphinxAtStartPar
Pour la structure ‘if…elif…..else\sphinxcode{\sphinxupquote{: si  la condition de if est vrai, le bloc d\textquotesingle{}instructions en dessous de }}if\sphinxcode{\sphinxupquote{sera exécuté. Sinon, si le bloc de}}elif\sphinxcode{\sphinxupquote{est vrai le bloc en dessous de}}elif\sphinxcode{\sphinxupquote{sera exécuté…. Sinon, si le bloc de}}else\sphinxcode{\sphinxupquote{est vrai le bloc en dessous de}}else\textasciigrave{} sera exécuté.

\end{itemize}

\begin{sphinxVerbatim}[commandchars=\\\{\}]
\PYG{n}{x} \PYG{o}{=} \PYG{l+m+mi}{10}
\PYG{c+c1}{\PYGZsh{} la condition est x\PYGZgt{}10}
\PYG{k}{if} \PYG{n}{x} \PYG{o}{\PYGZgt{}} \PYG{l+m+mi}{10}\PYG{p}{:}
    \PYG{n+nb}{print}\PYG{p}{(}\PYG{n}{x}\PYG{p}{,} \PYG{l+s+s2}{\PYGZdq{}}\PYG{l+s+s2}{est plus grand que 10}\PYG{l+s+s2}{\PYGZdq{}}\PYG{p}{)}
\PYG{k}{elif} \PYG{n}{x} \PYG{o}{\PYGZlt{}} \PYG{l+m+mi}{10}\PYG{p}{:}
    \PYG{n+nb}{print}\PYG{p}{(}\PYG{n}{x}\PYG{p}{,} \PYG{l+s+s2}{\PYGZdq{}}\PYG{l+s+s2}{est plus petit ou égale à 10}\PYG{l+s+s2}{\PYGZdq{}}\PYG{p}{)}
\PYG{k}{else}\PYG{p}{:}
    \PYG{n+nb}{print}\PYG{p}{(}\PYG{n}{x}\PYG{p}{,} \PYG{l+s+s2}{\PYGZdq{}}\PYG{l+s+s2}{est égale à 10}\PYG{l+s+s2}{\PYGZdq{}}\PYG{p}{)}
\end{sphinxVerbatim}

\begin{sphinxVerbatim}[commandchars=\\\{\}]
10 est égale à 10
\end{sphinxVerbatim}

\sphinxAtStartPar
On peut également avoir une structure de test au sien d’une autre structure de test. Aussi, les conditions peuvent être composé de plusieurs conditions à l’aide des opérateurs logiques (\sphinxcode{\sphinxupquote{or}}, \sphinxcode{\sphinxupquote{and}}, et \sphinxcode{\sphinxupquote{not}}).

\sphinxAtStartPar
Explorer le code ci\sphinxhyphen{}dessous est essayer de savoir son output.

\begin{sphinxVerbatim}[commandchars=\\\{\}]
\PYG{c+c1}{\PYGZsh{}}
\PYG{n}{x} \PYG{o}{=} \PYG{l+m+mi}{6}
\PYG{c+c1}{\PYGZsh{} les conditions}
\PYG{n}{condition1} \PYG{o}{=} \PYG{n}{x}\PYG{o}{\PYGZgt{}}\PYG{l+m+mi}{0}\PYG{p}{;} \PYG{n}{condition2} \PYG{o}{=} \PYG{n}{x}\PYG{o}{\PYGZlt{}}\PYG{l+m+mi}{9}\PYG{p}{;} \PYG{n}{condition3}\PYG{o}{=} \PYG{n}{x}\PYG{o}{==}\PYG{l+m+mi}{2}\PYG{p}{;} \PYG{n}{condition4} \PYG{o}{=} \PYG{n}{x} \PYG{o}{==} \PYG{l+m+mi}{3}\PYG{p}{;} 
\PYG{n}{condition5} \PYG{o}{=} \PYG{p}{(}\PYG{n}{x}\PYG{o}{==}\PYG{l+m+mi}{5}\PYG{p}{)} \PYG{o+ow}{or} \PYG{p}{(}\PYG{n}{x}\PYG{o}{==}\PYG{l+m+mi}{7}\PYG{p}{)}\PYG{p}{;} \PYG{n}{condition6} \PYG{o}{=} \PYG{n}{x}\PYG{o}{\PYGZlt{}}\PYG{l+m+mi}{0}\PYG{p}{;} \PYG{n}{condition7}\PYG{o}{=} \PYG{n}{x}\PYG{o}{\PYGZgt{}}\PYG{l+m+mi}{9}

\PYG{k}{if} \PYG{n}{condition1} \PYG{o+ow}{and} \PYG{n}{condition2}\PYG{p}{:}  \PYG{c+c1}{\PYGZsh{}\PYGZsh{} remarquer qu\PYGZsq{}on a pas besoin de faire condition1==True}
    \PYG{k}{if} \PYG{n}{condition3} \PYG{o+ow}{or} \PYG{n}{condition4} \PYG{o+ow}{or} \PYG{n}{condition5}\PYG{p}{:}
        \PYG{n+nb}{print}\PYG{p}{(}\PYG{n}{x}\PYG{p}{,} \PYG{l+s+s2}{\PYGZdq{}}\PYG{l+s+s2}{est premier}\PYG{l+s+s2}{\PYGZdq{}}\PYG{p}{)}
    \PYG{k}{else}\PYG{p}{:}
        \PYG{n+nb}{print}\PYG{p}{(}\PYG{n}{x}\PYG{p}{,} \PYG{l+s+s2}{\PYGZdq{}}\PYG{l+s+s2}{n}\PYG{l+s+s2}{\PYGZsq{}}\PYG{l+s+s2}{est pas premier}\PYG{l+s+s2}{\PYGZdq{}}\PYG{p}{)}
\PYG{k}{else}\PYG{p}{:}
    \PYG{k}{if} \PYG{n}{condition6}\PYG{p}{:}
        \PYG{n+nb}{print}\PYG{p}{(}\PYG{n}{x}\PYG{p}{,} \PYG{l+s+s2}{\PYGZdq{}}\PYG{l+s+s2}{est négatif}\PYG{l+s+s2}{\PYGZdq{}}\PYG{p}{)}
    \PYG{k}{else}\PYG{p}{:}
        \PYG{n+nb}{print}\PYG{p}{(}\PYG{n}{x}\PYG{p}{,} \PYG{l+s+s2}{\PYGZdq{}}\PYG{l+s+s2}{est supérieur à 9}\PYG{l+s+s2}{\PYGZdq{}}\PYG{p}{)}
\end{sphinxVerbatim}

\begin{sphinxVerbatim}[commandchars=\\\{\}]
6 n\PYGZsq{}est pas premier
\end{sphinxVerbatim}


\subsection{Les Boucles}
\label{\detokenize{ch2:les-boucles}}
\sphinxAtStartPar
Lorsqu’on voulait répéter plusieurs fois l’exécution d’une partie du programme, on utilise les boucles. Il en existe deux types :


\subsubsection{Les boucles bornées}
\label{\detokenize{ch2:les-boucles-bornees}}
\sphinxAtStartPar
Elles sont utilisées lorsqu’on a une idée a priori sur le nombre de répétition. Il s’agit de la boucle \sphinxcode{\sphinxupquote{for}} ; la syntaxe de la boucle for est la suivante :

\begin{sphinxVerbatim}[commandchars=\\\{\}]
\PYG{k}{for} \PYG{n}{var} \PYG{o+ow}{in} \PYG{n}{valeurs\PYGZus{}possibles}\PYG{p}{:}
    \PYG{n}{instructions}
\end{sphinxVerbatim}

\begin{sphinxVerbatim}[commandchars=\\\{\}]
* `var` est le nom d\PYGZsq{}une variable locale à la fonction `for()` 
* `valeurs\PYGZus{}possibles` un objet comprenant des éléments définissant les valeurs que prendra objet pour chaque itération, 
*  `instructions` est le bloc d\PYGZsq{}instructions qui seront exécutées à chaque itération.
\end{sphinxVerbatim}

\sphinxAtStartPar
Voici deux exemples simples de l’utilisation de la boucle \sphinxcode{\sphinxupquote{for}} :
\begin{enumerate}
\sphinxsetlistlabels{\arabic}{enumi}{enumii}{}{.}%
\item {} 
\sphinxAtStartPar
Calculer le carrée de chaque nombre 1, 2, 3 et 4;

\item {} 
\sphinxAtStartPar
Calculer la somme des nombres 1, 2, 3 et 4.

\end{enumerate}

\begin{sphinxVerbatim}[commandchars=\\\{\}]
\PYG{k}{for} \PYG{n}{i} \PYG{o+ow}{in} \PYG{p}{[}\PYG{l+m+mi}{1}\PYG{p}{,} \PYG{l+m+mi}{2}\PYG{p}{,} \PYG{l+m+mi}{3}\PYG{p}{,} \PYG{l+m+mi}{4}\PYG{p}{]}\PYG{p}{:}
    \PYG{n+nb}{print}\PYG{p}{(}\PYG{l+s+s2}{\PYGZdq{}}\PYG{l+s+s2}{le carré de}\PYG{l+s+s2}{\PYGZdq{}}\PYG{p}{,} \PYG{n}{i}\PYG{p}{,} \PYG{l+s+s2}{\PYGZdq{}}\PYG{l+s+s2}{est:}\PYG{l+s+s2}{\PYGZdq{}}\PYG{p}{,} \PYG{n}{i}\PYG{o}{*}\PYG{o}{*}\PYG{l+m+mi}{2}\PYG{p}{)}
\end{sphinxVerbatim}

\begin{sphinxVerbatim}[commandchars=\\\{\}]
le carré de 1 est: 1
le carré de 2 est: 4
le carré de 3 est: 9
le carré de 4 est: 16
\end{sphinxVerbatim}

\begin{sphinxVerbatim}[commandchars=\\\{\}]
\PYG{n}{somme} \PYG{o}{=} \PYG{l+m+mi}{0}
\PYG{k}{for} \PYG{n}{i} \PYG{o+ow}{in} \PYG{p}{[}\PYG{l+m+mi}{1}\PYG{p}{,} \PYG{l+m+mi}{2}\PYG{p}{,} \PYG{l+m+mi}{3}\PYG{p}{,} \PYG{l+m+mi}{4}\PYG{p}{]}\PYG{p}{:}
    \PYG{n}{somme} \PYG{o}{+}\PYG{o}{=} \PYG{n}{i}
\PYG{n+nb}{print}\PYG{p}{(}\PYG{n}{somme}\PYG{p}{)}
\end{sphinxVerbatim}

\begin{sphinxVerbatim}[commandchars=\\\{\}]
10
\end{sphinxVerbatim}

\sphinxAtStartPar
Une fonction très importante qui est souvent utilisée en Python pour la boucle \sphinxcode{\sphinxupquote{for}} est la fonction \sphinxcode{\sphinxupquote{range()}}. Cette fonction prend les arguments suivants :
\begin{itemize}
\item {} 
\sphinxAtStartPar
\sphinxcode{\sphinxupquote{start}} : (optionnel, par défaut, 0) valeur de début pour la séquence (inclue) ;

\item {} 
\sphinxAtStartPar
\sphinxcode{\sphinxupquote{stop}} : valeur de fin de la séquence (\sphinxstylestrong{non inclue}) ;

\item {} 
\sphinxAtStartPar
\sphinxcode{\sphinxupquote{step}} : (optionnel, par défaut 1) le pas.

\end{itemize}

\sphinxAtStartPar
On peut refaire les exemples suivants comme suit :

\begin{sphinxVerbatim}[commandchars=\\\{\}]
\PYG{k}{for} \PYG{n}{i} \PYG{o+ow}{in} \PYG{n+nb}{range}\PYG{p}{(}\PYG{l+m+mi}{1}\PYG{p}{,} \PYG{l+m+mi}{5}\PYG{p}{)}\PYG{p}{:}
    \PYG{n+nb}{print}\PYG{p}{(}\PYG{l+s+s2}{\PYGZdq{}}\PYG{l+s+s2}{le carré de}\PYG{l+s+s2}{\PYGZdq{}}\PYG{p}{,} \PYG{n}{i}\PYG{p}{,} \PYG{l+s+s2}{\PYGZdq{}}\PYG{l+s+s2}{est:}\PYG{l+s+s2}{\PYGZdq{}}\PYG{p}{,} \PYG{n}{i}\PYG{o}{*}\PYG{o}{*}\PYG{l+m+mi}{2}\PYG{p}{)}
\end{sphinxVerbatim}

\begin{sphinxVerbatim}[commandchars=\\\{\}]
le carré de 1 est: 1
le carré de 2 est: 4
le carré de 3 est: 9
le carré de 4 est: 16
\end{sphinxVerbatim}

\begin{sphinxVerbatim}[commandchars=\\\{\}]
\PYG{n}{somme} \PYG{o}{=} \PYG{l+m+mi}{0}
\PYG{k}{for} \PYG{n}{i} \PYG{o+ow}{in} \PYG{n+nb}{range}\PYG{p}{(}\PYG{l+m+mi}{5}\PYG{p}{)}\PYG{p}{:}
    \PYG{n}{somme} \PYG{o}{+}\PYG{o}{=} \PYG{n}{i}
\PYG{n+nb}{print}\PYG{p}{(}\PYG{n}{somme}\PYG{p}{)}
\end{sphinxVerbatim}

\begin{sphinxVerbatim}[commandchars=\\\{\}]
10
\end{sphinxVerbatim}

\sphinxAtStartPar
Rien n’empêche avoir une boucle for dans un bloc d’instructions d’une autre boucle for. Ou encore une instruction conditionnelle dans un bloc de la boucle for (vice versa). Par exemple si on veut calculer le produit de chaque élément de la liste \sphinxcode{\sphinxupquote{{[}1, 5, 9, 10{]}}} par chaque élément de la liste \sphinxcode{\sphinxupquote{{[}2, 3, 5{]}}}:

\begin{sphinxVerbatim}[commandchars=\\\{\}]
\PYG{k}{for} \PYG{n}{i} \PYG{o+ow}{in} \PYG{p}{[}\PYG{l+m+mi}{1}\PYG{p}{,} \PYG{l+m+mi}{5}\PYG{p}{,} \PYG{l+m+mi}{9}\PYG{p}{,} \PYG{l+m+mi}{10}\PYG{p}{]}\PYG{p}{:}
    \PYG{k}{for} \PYG{n}{j} \PYG{o+ow}{in} \PYG{p}{[}\PYG{l+m+mi}{2}\PYG{p}{,} \PYG{l+m+mi}{3}\PYG{p}{,} \PYG{l+m+mi}{5}\PYG{p}{]}\PYG{p}{:}
        \PYG{n+nb}{print}\PYG{p}{(}\PYG{l+s+s2}{\PYGZdq{}}\PYG{l+s+s2}{Le produit de}\PYG{l+s+s2}{\PYGZdq{}}\PYG{p}{,} \PYG{n}{i}\PYG{p}{,} \PYG{l+s+s2}{\PYGZdq{}}\PYG{l+s+s2}{fois}\PYG{l+s+s2}{\PYGZdq{}}\PYG{p}{,} \PYG{n}{j}\PYG{p}{,} \PYG{l+s+s2}{\PYGZdq{}}\PYG{l+s+s2}{est:}\PYG{l+s+s2}{\PYGZdq{}}\PYG{p}{,} \PYG{n}{i}\PYG{o}{*}\PYG{n}{j}\PYG{p}{)}
\end{sphinxVerbatim}

\begin{sphinxVerbatim}[commandchars=\\\{\}]
Le produit de 1 fois 2 est: 2
Le produit de 1 fois 3 est: 3
Le produit de 1 fois 5 est: 5
Le produit de 5 fois 2 est: 10
Le produit de 5 fois 3 est: 15
Le produit de 5 fois 5 est: 25
Le produit de 9 fois 2 est: 18
Le produit de 9 fois 3 est: 27
Le produit de 9 fois 5 est: 45
Le produit de 10 fois 2 est: 20
Le produit de 10 fois 3 est: 30
Le produit de 10 fois 5 est: 50
\end{sphinxVerbatim}

\sphinxAtStartPar
Ou si l’on veut tester la parité de chaque élément entre 3, et 14 \sphinxcode{\sphinxupquote{(range(3,15)}});

\begin{sphinxVerbatim}[commandchars=\\\{\}]
\PYG{k}{for} \PYG{n}{i} \PYG{o+ow}{in} \PYG{n+nb}{range}\PYG{p}{(}\PYG{l+m+mi}{3}\PYG{p}{,} \PYG{l+m+mi}{15}\PYG{p}{)}\PYG{p}{:}
    \PYG{k}{if} \PYG{n}{i}\PYG{o}{\PYGZpc{}}\PYG{k}{2}==0:
        \PYG{n+nb}{print}\PYG{p}{(}\PYG{l+s+s2}{\PYGZdq{}}\PYG{l+s+s2}{Le nombre}\PYG{l+s+s2}{\PYGZdq{}}\PYG{p}{,} \PYG{n}{i}\PYG{p}{,} \PYG{l+s+s2}{\PYGZdq{}}\PYG{l+s+s2}{est pair}\PYG{l+s+s2}{\PYGZdq{}}\PYG{p}{)}
    \PYG{k}{else}\PYG{p}{:}
        \PYG{n+nb}{print}\PYG{p}{(}\PYG{l+s+s2}{\PYGZdq{}}\PYG{l+s+s2}{Le nombre}\PYG{l+s+s2}{\PYGZdq{}}\PYG{p}{,} \PYG{n}{i}\PYG{p}{,} \PYG{l+s+s2}{\PYGZdq{}}\PYG{l+s+s2}{est impair}\PYG{l+s+s2}{\PYGZdq{}}\PYG{p}{)} 
\end{sphinxVerbatim}

\begin{sphinxVerbatim}[commandchars=\\\{\}]
Le nombre 3 est impair
Le nombre 4 est pair
Le nombre 5 est impair
Le nombre 6 est pair
Le nombre 7 est impair
Le nombre 8 est pair
Le nombre 9 est impair
Le nombre 10 est pair
Le nombre 11 est impair
Le nombre 12 est pair
Le nombre 13 est impair
Le nombre 14 est pair
\end{sphinxVerbatim}


\subsubsection{Les boucles non bornées}
\label{\detokenize{ch2:les-boucles-non-bornees}}
\sphinxAtStartPar
Elles sont utilisées si on ne connait pas à l’avance le nombre de répétitions. Il s’agit de la boucle \sphinxcode{\sphinxupquote{while}}.
Le principe d’une boucle \sphinxcode{\sphinxupquote{while()}} est que les instructions à l’intérieur de la boucle seront répétées tant qu’une condition est respectée. L’idée est de faire dépendre cette condition d’un ou plusieurs objets qui seront modifiés au cours des itérations (sans cela, la boucle tournerait à l’infini).

\sphinxAtStartPar
La syntaxe est la suivante :

\begin{sphinxVerbatim}[commandchars=\\\{\}]
\PYG{k}{while} \PYG{n}{condition}\PYG{p}{:}
    \PYG{n}{instructions}
\end{sphinxVerbatim}

\sphinxAtStartPar
Nous allons illustrer le rôle de chaque boucle dans l’exemple suivant. Supposons que nous voulons calculer le factoriel d’un nombre (soit 10), voici le code pour le calculer :

\begin{sphinxVerbatim}[commandchars=\\\{\}]
\PYG{n}{i} \PYG{o}{=} \PYG{l+m+mi}{1}
\PYG{n}{fact} \PYG{o}{=} \PYG{l+m+mi}{1}
\PYG{k}{while} \PYG{n}{i}\PYG{o}{\PYGZlt{}}\PYG{o}{=}\PYG{l+m+mi}{10}\PYG{p}{:}
    \PYG{n}{fact} \PYG{o}{*}\PYG{o}{=} \PYG{n}{i}
    \PYG{n}{i}\PYG{o}{+}\PYG{o}{=}\PYG{l+m+mi}{1}

\PYG{n+nb}{print}\PYG{p}{(}\PYG{n}{fact}\PYG{p}{)}
\end{sphinxVerbatim}

\begin{sphinxVerbatim}[commandchars=\\\{\}]
3628800
\end{sphinxVerbatim}

\sphinxAtStartPar
Reprenons l’exemple précédant dont on veut tester la parité de chaque élément entre 3, et 14 avec la boucle while:

\begin{sphinxVerbatim}[commandchars=\\\{\}]
\PYG{n}{i} \PYG{o}{=} \PYG{l+m+mi}{3}
\PYG{k}{while} \PYG{n}{i}\PYG{o}{\PYGZlt{}}\PYG{l+m+mi}{15}\PYG{p}{:}
    \PYG{k}{if} \PYG{n}{i}\PYG{o}{\PYGZpc{}}\PYG{k}{2}==0:
        \PYG{n+nb}{print}\PYG{p}{(}\PYG{l+s+s2}{\PYGZdq{}}\PYG{l+s+s2}{Le nombre}\PYG{l+s+s2}{\PYGZdq{}}\PYG{p}{,} \PYG{n}{i}\PYG{p}{,} \PYG{l+s+s2}{\PYGZdq{}}\PYG{l+s+s2}{est pair}\PYG{l+s+s2}{\PYGZdq{}}\PYG{p}{)}
    \PYG{k}{else}\PYG{p}{:}
        \PYG{n+nb}{print}\PYG{p}{(}\PYG{l+s+s2}{\PYGZdq{}}\PYG{l+s+s2}{Le nombre}\PYG{l+s+s2}{\PYGZdq{}}\PYG{p}{,} \PYG{n}{i}\PYG{p}{,} \PYG{l+s+s2}{\PYGZdq{}}\PYG{l+s+s2}{est impair}\PYG{l+s+s2}{\PYGZdq{}}\PYG{p}{)}
    \PYG{n}{i}\PYG{o}{+}\PYG{o}{=}\PYG{l+m+mi}{1}
\end{sphinxVerbatim}

\begin{sphinxVerbatim}[commandchars=\\\{\}]
Le nombre 3 est impair
Le nombre 4 est pair
Le nombre 5 est impair
Le nombre 6 est pair
Le nombre 7 est impair
Le nombre 8 est pair
Le nombre 9 est impair
Le nombre 10 est pair
Le nombre 11 est impair
Le nombre 12 est pair
Le nombre 13 est impair
Le nombre 14 est pair
\end{sphinxVerbatim}


\section{Essayez vous\sphinxhyphen{}meme!}
\label{\detokenize{exo2:essayez-vous-meme}}\label{\detokenize{exo2::doc}}



\subsection{Exercice 1.}
\label{\detokenize{exo2:exercice-1}}
\sphinxAtStartPar
Ordre des operations: quelle est selon vous le resulats de ces operations :
\begin{itemize}
\item {} 
\sphinxAtStartPar
\((4/2)^2\times 2 + 1\), \(4/2^{(2\times 2)} + 1\),

\item {} 
\sphinxAtStartPar
\(4/2^2\times (2 + 1)\),

\item {} 
\sphinxAtStartPar
\(4/2^2\times 2 + 1\)
verifier avec Python.

\end{itemize}

\begin{sphinxVerbatim}[commandchars=\\\{\}]
\PYG{c+c1}{\PYGZsh{}\PYGZsh{} votre code ici}
\end{sphinxVerbatim}




\subsection{Exercice 2.}
\label{\detokenize{exo2:exercice-2}}
\sphinxAtStartPar
Un pere a une somme d’argent de 1554 dh, il veut la partager sur ses 9 enfants de maniere equitable et s’il reste quelque dirham, il va acheter des chocolat a 1 dh l’unite. combien chanque enfant va recevoir? combien d’unite de chocolat peut\sphinxhyphen{}il acheter avec le reste?

\begin{sphinxVerbatim}[commandchars=\\\{\}]
\PYG{c+c1}{\PYGZsh{}\PYGZsh{} votre code ici}
\end{sphinxVerbatim}




\subsection{Exercice 3.}
\label{\detokenize{exo2:exercice-3}}
\sphinxAtStartPar
Quel est le type de donnees de valeurs suivantes:
\begin{itemize}
\item {} 
\sphinxAtStartPar
\sphinxcode{\sphinxupquote{1}}

\item {} 
\sphinxAtStartPar
\sphinxcode{\sphinxupquote{1.}}

\item {} 
\sphinxAtStartPar
\sphinxcode{\sphinxupquote{False}}

\item {} 
\sphinxAtStartPar
\sphinxcode{\sphinxupquote{"False"}}

\item {} 
\sphinxAtStartPar
\sphinxcode{\sphinxupquote{var1/var2}} avec \sphinxcode{\sphinxupquote{var1 = 1}} et \sphinxcode{\sphinxupquote{var2 = 2}}

\end{itemize}

\begin{sphinxVerbatim}[commandchars=\\\{\}]
\PYG{c+c1}{\PYGZsh{}\PYGZsh{} votre code ici}
\end{sphinxVerbatim}




\subsection{Exercice 4.}
\label{\detokenize{exo2:exercice-4}}
\sphinxAtStartPar
Quel est le type de donnees de valeurs suivantes:
\begin{itemize}
\item {} 
\sphinxAtStartPar
\sphinxcode{\sphinxupquote{1}}

\item {} 
\sphinxAtStartPar
\sphinxcode{\sphinxupquote{1.}}

\item {} 
\sphinxAtStartPar
\sphinxcode{\sphinxupquote{False}}

\item {} 
\sphinxAtStartPar
\sphinxcode{\sphinxupquote{"False"}}

\item {} 
\sphinxAtStartPar
\sphinxcode{\sphinxupquote{"5.4"}}

\item {} 
\sphinxAtStartPar
\sphinxcode{\sphinxupquote{var1/var2}} avec \sphinxcode{\sphinxupquote{var1 = 1}} et \sphinxcode{\sphinxupquote{var2 = 2}}

\item {} 
\sphinxAtStartPar
\sphinxcode{\sphinxupquote{list()}}

\item {} 
\sphinxAtStartPar
\sphinxcode{\sphinxupquote{None}}

\item {} 
\sphinxAtStartPar
\sphinxcode{\sphinxupquote{""}}

\end{itemize}

\begin{sphinxVerbatim}[commandchars=\\\{\}]
\PYG{c+c1}{\PYGZsh{}\PYGZsh{} votre code ici}
\end{sphinxVerbatim}




\subsection{Exercice 5.}
\label{\detokenize{exo2:exercice-5}}
\sphinxAtStartPar
Convertir, si c’est possible, de valeurs suivantes au types de donnees que nous avons vu. Expliciter les cas qui ne sont pas possible:
\begin{itemize}
\item {} 
\sphinxAtStartPar
\sphinxcode{\sphinxupquote{1}}

\item {} 
\sphinxAtStartPar
\sphinxcode{\sphinxupquote{1.}}

\item {} 
\sphinxAtStartPar
\sphinxcode{\sphinxupquote{False}}

\item {} 
\sphinxAtStartPar
\sphinxcode{\sphinxupquote{"False"}}

\item {} 
\sphinxAtStartPar
\sphinxcode{\sphinxupquote{"5.4"}}

\item {} 
\sphinxAtStartPar
\sphinxcode{\sphinxupquote{var1/var2}} avec \sphinxcode{\sphinxupquote{var1 = 1}} et \sphinxcode{\sphinxupquote{var2 = 2}}

\item {} 
\sphinxAtStartPar
\sphinxcode{\sphinxupquote{list()}}

\item {} 
\sphinxAtStartPar
\sphinxcode{\sphinxupquote{None}}

\item {} 
\sphinxAtStartPar
\sphinxcode{\sphinxupquote{""}}

\end{itemize}

\begin{sphinxVerbatim}[commandchars=\\\{\}]
\PYG{c+c1}{\PYGZsh{}\PYGZsh{} votre code ici}
\end{sphinxVerbatim}




\subsection{Exercice 6.}
\label{\detokenize{exo2:exercice-6}}
\sphinxAtStartPar
Soient \sphinxcode{\sphinxupquote{x = True}}, \sphinxcode{\sphinxupquote{y= 7\textless{}6}}, and \sphinxcode{\sphinxupquote{z= not y}}. Determinier la valeur logique de \sphinxcode{\sphinxupquote{x, y,}} et \sphinxcode{\sphinxupquote{z}} (\sphinxcode{\sphinxupquote{True}} ou \sphinxcode{\sphinxupquote{False}}) puis la valeur de chacun des expression suivantes:
\begin{itemize}
\item {} 
\sphinxAtStartPar
\sphinxcode{\sphinxupquote{x != False}}

\item {} 
\sphinxAtStartPar
\sphinxcode{\sphinxupquote{x and y}}

\item {} 
\sphinxAtStartPar
\sphinxcode{\sphinxupquote{x or y}}

\item {} 
\sphinxAtStartPar
\sphinxcode{\sphinxupquote{not y}}

\item {} 
\sphinxAtStartPar
\sphinxcode{\sphinxupquote{x and (y or z)}}

\item {} 
\sphinxAtStartPar
\sphinxcode{\sphinxupquote{(x and y) or z}}

\item {} 
\sphinxAtStartPar
\sphinxcode{\sphinxupquote{(not x or not y) and (not z)}}

\item {} 
\sphinxAtStartPar
\sphinxcode{\sphinxupquote{not ((x and y) or z)}}

\end{itemize}

\begin{sphinxVerbatim}[commandchars=\\\{\}]
\PYG{c+c1}{\PYGZsh{}\PYGZsh{} votre code ici}
\end{sphinxVerbatim}




\subsection{Exercice 7.}
\label{\detokenize{exo2:exercice-7}}
\sphinxAtStartPar
On veut recevoir ce message avec \sphinxcode{\sphinxupquote{print()}}:

\begin{sphinxVerbatim}[commandchars=\\\{\}]
\PYG{n}{la} \PYG{n}{valeur} \PYG{n}{de} \PYG{n}{x} \PYG{n}{est}\PYG{p}{:} \PYG{o}{@}\PYG{n+nd}{@True}\PYG{o}{@}\PYG{o}{@}
\PYG{o}{*}\PYG{o}{*}\PYG{o}{*}\PYG{o}{*}\PYG{o}{*}\PYG{o}{*}\PYG{o}{*}\PYG{o}{*}\PYG{o}{*}\PYG{o}{==}\PYG{o}{==}\PYG{o}{==}\PYG{o}{==}\PYG{o}{==}\PYG{o}{==}\PYG{o}{==}\PYG{o}{*}\PYG{o}{*}\PYG{o}{*}\PYG{o}{*}\PYG{o}{*}\PYG{o}{*}\PYG{o}{*}\PYG{o}{*}
\end{sphinxVerbatim}

\sphinxAtStartPar
En utilisant les argument suivants:
\begin{itemize}
\item {} 
\sphinxAtStartPar
\sphinxcode{\sphinxupquote{x = True}}

\item {} 
\sphinxAtStartPar
\sphinxcode{\sphinxupquote{a = "la valeur de x est:"}}

\end{itemize}

\sphinxAtStartPar
Toute modification devra etre faite au nivau de \sphinxcode{\sphinxupquote{sep=}}, et \sphinxcode{\sphinxupquote{end=}}.

\begin{sphinxVerbatim}[commandchars=\\\{\}]
\PYG{c+c1}{\PYGZsh{}\PYGZsh{} votre code ici}
\end{sphinxVerbatim}




\subsection{Exercice 8.}
\label{\detokenize{exo2:exercice-8}}
\sphinxAtStartPar
Ecrire un petit programe qui permet de demander a l’utilisatuer d’entrer son nom, son poids en kilograme (sans entrer l’unite), et sa taille en metre (sans entrer l’unite). puis il affiche l’indice du poids (Body mass index (BMI)):
\sphinxstyleemphasis{Formule du BMI}: \(BMI = \dfrac{poids}{taille^2}\)

\begin{sphinxVerbatim}[commandchars=\\\{\}]
\PYG{c+c1}{\PYGZsh{}\PYGZsh{} votre code ici}
\end{sphinxVerbatim}


\chapter{Point sur les chaînes de caractères, les listes, les tuples et les dictionnaires}
\label{\detokenize{content3:point-sur-les-chaines-de-caracteres-les-listes-les-tuples-et-les-dictionnaires}}\label{\detokenize{content3::doc}}

\section{Manupilation des chaînes de caractères (strings)}
\label{\detokenize{ch3:manupilation-des-chaines-de-caracteres-strings}}\label{\detokenize{ch3::doc}}
\begin{sphinxadmonition}{note}{Rappel}

\sphinxAtStartPar
Nous avons vu que:
\begin{itemize}
\item {} 
\sphinxAtStartPar
Une chaîne est une liste ordonnee de caractères;

\item {} 
\sphinxAtStartPar
Un caractère est tout ce que vous pouvez taper sur le clavier en une seule frappe,
comme une lettre (a, b, c, …), un chiffre (0, 1, 3, …), symboles (@, \%, \#,…), espace ou une backslash ();

\item {} 
\sphinxAtStartPar
Une chaîne vide est une chaîne qui contient 0 caractères (“”);

\item {} 
\sphinxAtStartPar
Python reconnaît comme chaînes tout ce qui est délimité par des guillemets
( »  » ou alors “ “).

\item {} 
\sphinxAtStartPar
Pour affecter une chaîne de caractères a une variable:

\end{itemize}

\begin{sphinxVerbatim}[commandchars=\\\{\}]
\PYG{n}{var} \PYG{o}{=} \PYG{l+s+s2}{\PYGZdq{}}\PYG{l+s+s2}{bonjour}\PYG{l+s+s2}{\PYGZdq{}}
\end{sphinxVerbatim}
\end{sphinxadmonition}

\sphinxAtStartPar
Maintenat nous Nous allons voir quelques caracteres speciaux

\sphinxAtStartPar
Sous Python, la backslash \sphinxcode{\sphinxupquote{"\textbackslash{}"}} est un caractère spécial, également appelé caractère \sphinxcode{\sphinxupquote{d\textquotesingle{}échappement}}. Il est utilisé pour représenter certains caractères d’espacement : « \textbackslash{}t » est une tabulation, « \textbackslash{}n » est une nouvelle ligne et « \textbackslash{}r » est un retour chariot. Supposons que nous voulons afficher cette chaine \sphinxcode{\sphinxupquote{"\textbackslash{}nous\textbackslash{}tenons a vous remericer"}}.

\begin{sphinxVerbatim}[commandchars=\\\{\}]
\PYG{n+nb}{print}\PYG{p}{(}\PYG{l+s+s2}{\PYGZdq{}}\PYG{l+s+se}{\PYGZbs{}n}\PYG{l+s+s2}{ous}\PYG{l+s+se}{\PYGZbs{}t}\PYG{l+s+s2}{enons a vous remericer}\PYG{l+s+s2}{\PYGZdq{}}\PYG{p}{)}
\end{sphinxVerbatim}

\begin{sphinxVerbatim}[commandchars=\\\{\}]
ous	enons a vous remericer
\end{sphinxVerbatim}

\sphinxAtStartPar
Python ne comprend que \sphinxcode{\sphinxupquote{\textquotesingle{}\textbackslash{}n\textquotesingle{}}} dons \sphinxcode{\sphinxupquote{"\textbackslash{}nous\textbackslash{}tenons a vous remericer"}} est un retour a linge. La meme chose pour \sphinxcode{\sphinxupquote{\textquotesingle{}\textbackslash{}t\textquotesingle{}}} est une tabulation (nous allons voir en details caractère dits \sphinxcode{\sphinxupquote{caractères speciaux}}). Pour lui informer que nous ne voulons pas ca, on doit ajouter \sphinxcode{\sphinxupquote{\textbackslash{}}} avant \sphinxcode{\sphinxupquote{\textbackslash{}}} dans les deux cas (c’est\sphinxhyphen{}a\sphinxhyphen{}dire \sphinxcode{\sphinxupquote{\textbackslash{}\textbackslash{}}}). Voila comment le faire:

\begin{sphinxVerbatim}[commandchars=\\\{\}]
\PYG{n+nb}{print}\PYG{p}{(}\PYG{l+s+s2}{\PYGZdq{}}\PYG{l+s+se}{\PYGZbs{}\PYGZbs{}}\PYG{l+s+s2}{nous}\PYG{l+s+se}{\PYGZbs{}\PYGZbs{}}\PYG{l+s+s2}{tenons a vous remericer}\PYG{l+s+s2}{\PYGZdq{}}\PYG{p}{)}
\end{sphinxVerbatim}

\begin{sphinxVerbatim}[commandchars=\\\{\}]
\PYGZbs{}nous\PYGZbs{}tenons a vous remericer
\end{sphinxVerbatim}

\sphinxAtStartPar
Aussi, lorsque nous voulons afficher une chaine de caracteres comme \sphinxcode{\sphinxupquote{\textquotesingle{}c\textquotesingle{}est magnifique\textquotesingle{}}}

\begin{sphinxVerbatim}[commandchars=\\\{\}]
\PYG{n+nb}{print}\PYG{p}{(}\PYG{l+s+s1}{\PYGZsq{}}\PYG{l+s+s1}{c}\PYG{l+s+s1}{\PYGZsq{}}\PYG{n}{est} \PYG{n}{magnifique}\PYG{l+s+s1}{\PYGZsq{}}\PYG{l+s+s1}{)}
\end{sphinxVerbatim}

\begin{sphinxVerbatim}[commandchars=\\\{\}]
\PYG{g+gt}{  File}\PYG{n+nn}{ \PYGZdq{}C:\PYGZbs{}Users\PYGZbs{}usr\PYGZbs{}AppData\PYGZbs{}Local\PYGZbs{}Temp\PYGZbs{}ipykernel\PYGZus{}10040\PYGZbs{}892033391.py\PYGZdq{}}\PYG{g+gt}{, line }\PYG{l+m+mi}{1}
    \PYG{n+nb}{print}\PYG{p}{(}\PYG{l+s+s1}{\PYGZsq{}}\PYG{l+s+s1}{c}\PYG{l+s+s1}{\PYGZsq{}}\PYG{n}{est} \PYG{n}{magnifique}\PYG{l+s+s1}{\PYGZsq{}}\PYG{l+s+s1}{)}
               \PYG{o}{\PYGZca{}}
\PYG{n+ne}{SyntaxError}: invalid syntax
\end{sphinxVerbatim}

\sphinxAtStartPar
Python comprend que la deuxiem apostrophe termine la chaine de caractere donc le reste de la chaine (\sphinxcode{\sphinxupquote{est magnifique}}) n’est pas compris. Nous devons mettre \sphinxcode{\sphinxupquote{\textbackslash{}}} avant la deuxieme apostrophe pour elle soit un caractere de cette chaine:

\begin{sphinxVerbatim}[commandchars=\\\{\}]
\PYG{n+nb}{print}\PYG{p}{(}\PYG{l+s+s1}{\PYGZsq{}}\PYG{l+s+s1}{c}\PYG{l+s+se}{\PYGZbs{}\PYGZsq{}}\PYG{l+s+s1}{est magnifique}\PYG{l+s+s1}{\PYGZsq{}}\PYG{p}{)}
\end{sphinxVerbatim}

\begin{sphinxVerbatim}[commandchars=\\\{\}]
c\PYGZsq{}est magnifique
\end{sphinxVerbatim}

\sphinxAtStartPar
La meme chose pour une chaine content \sphinxcode{\sphinxupquote{"}} est declaree par \sphinxcode{\sphinxupquote{" "}} (comme \sphinxcode{\sphinxupquote{""bonjour""}}).

\begin{sphinxVerbatim}[commandchars=\\\{\}]
\PYG{n+nb}{print}\PYG{p}{(}\PYG{l+s+s2}{\PYGZdq{}}\PYG{l+s+s2}{\PYGZdq{}}\PYG{n}{bonjour}\PYG{l+s+s2}{\PYGZdq{}}\PYG{l+s+s2}{\PYGZdq{}}\PYG{p}{)}
\end{sphinxVerbatim}

\begin{sphinxVerbatim}[commandchars=\\\{\}]
\PYG{g+gt}{  File}\PYG{n+nn}{ \PYGZdq{}C:\PYGZbs{}Users\PYGZbs{}usr\PYGZbs{}AppData\PYGZbs{}Local\PYGZbs{}Temp\PYGZbs{}ipykernel\PYGZus{}10040\PYGZbs{}2877536476.py\PYGZdq{}}\PYG{g+gt}{, line }\PYG{l+m+mi}{1}
    \PYG{n+nb}{print}\PYG{p}{(}\PYG{l+s+s2}{\PYGZdq{}}\PYG{l+s+s2}{\PYGZdq{}}\PYG{n}{bonjour}\PYG{l+s+s2}{\PYGZdq{}}\PYG{l+s+s2}{\PYGZdq{}}\PYG{p}{)}
                  \PYG{o}{\PYGZca{}}
\PYG{n+ne}{SyntaxError}: invalid syntax
\end{sphinxVerbatim}

\begin{sphinxVerbatim}[commandchars=\\\{\}]
\PYG{n+nb}{print}\PYG{p}{(}\PYG{l+s+s2}{\PYGZdq{}}\PYG{l+s+se}{\PYGZbs{}\PYGZdq{}}\PYG{l+s+s2}{bonjour}\PYG{l+s+se}{\PYGZbs{}\PYGZdq{}}\PYG{l+s+s2}{\PYGZdq{}}\PYG{p}{)}
\end{sphinxVerbatim}

\begin{sphinxVerbatim}[commandchars=\\\{\}]
\PYGZdq{}bonjour\PYGZdq{}
\end{sphinxVerbatim}

\begin{sphinxadmonition}{note}{Note:}
\sphinxAtStartPar
On peut utiliser \sphinxcode{\sphinxupquote{"c\textquotesingle{}est magnifique"}} ou \sphinxcode{\sphinxupquote{\textquotesingle{}"bonjour"\textquotesingle{}}} sans utiliser \sphinxcode{\sphinxupquote{\textbackslash{}}} (entrer la chaine avec apostrophe s’elle contien des gueimmet ou l’invers).
\end{sphinxadmonition}

\begin{sphinxVerbatim}[commandchars=\\\{\}]
\PYG{n+nb}{print}\PYG{p}{(}\PYG{l+s+s2}{\PYGZdq{}}\PYG{l+s+s2}{c}\PYG{l+s+s2}{\PYGZsq{}}\PYG{l+s+s2}{est magnifique}\PYG{l+s+s2}{\PYGZdq{}}\PYG{p}{)}
\PYG{n+nb}{print}\PYG{p}{(}\PYG{l+s+s1}{\PYGZsq{}}\PYG{l+s+s1}{\PYGZdq{}}\PYG{l+s+s1}{bonjour}\PYG{l+s+s1}{\PYGZdq{}}\PYG{l+s+s1}{\PYGZsq{}}\PYG{p}{)}
\end{sphinxVerbatim}

\begin{sphinxVerbatim}[commandchars=\\\{\}]
c\PYGZsq{}est magnifique
\PYGZdq{}bonjour\PYGZdq{}
\end{sphinxVerbatim}

\sphinxAtStartPar
Voici une qulques des caracteres speciaux et leurs significations:
\begin{itemize}
\item {} 
\sphinxAtStartPar
\sphinxcode{\sphinxupquote{\textbackslash{}n}} \sphinxhyphen{} Nouvelle ligne

\item {} 
\sphinxAtStartPar
\sphinxcode{\sphinxupquote{\textbackslash{}t}} \sphinxhyphen{} tabulation

\item {} 
\sphinxAtStartPar
\sphinxcode{\sphinxupquote{\textbackslash{}r}} \sphinxhyphen{} Retour chariot

\item {} 
\sphinxAtStartPar
\sphinxcode{\sphinxupquote{\textbackslash{}b}} \sphinxhyphen{} Retour en arrière

\item {} 
\sphinxAtStartPar
\sphinxcode{\sphinxupquote{\textbackslash{}f}} \sphinxhyphen{} Saut de page

\item {} 
\sphinxAtStartPar
\sphinxcode{\sphinxupquote{\textbackslash{}\textquotesingle{}}} \sphinxhyphen{} Apostrophe

\item {} 
\sphinxAtStartPar
\sphinxcode{\sphinxupquote{\textbackslash{}"}} \sphinxhyphen{} Guiellmets

\item {} 
\sphinxAtStartPar
\sphinxcode{\sphinxupquote{\textbackslash{}\textbackslash{}}} \sphinxhyphen{}Backslash

\end{itemize}

\begin{sphinxVerbatim}[commandchars=\\\{\}]
\PYG{n+nb}{print}\PYG{p}{(}\PYG{l+s+s1}{\PYGZsq{}}\PYG{l+s+s1}{Bon}\PYG{l+s+se}{\PYGZbs{}n}\PYG{l+s+s1}{jour}\PYG{l+s+s1}{\PYGZsq{}}\PYG{p}{,} \PYG{n}{end}\PYG{o}{=}\PYG{l+s+s1}{\PYGZsq{}}\PYG{l+s+se}{\PYGZbs{}n}\PYG{l+s+s1}{\PYGZhy{}\PYGZhy{}\PYGZhy{}\PYGZhy{}}\PYG{l+s+se}{\PYGZbs{}n}\PYG{l+s+s1}{\PYGZsq{}}\PYG{p}{)}
\PYG{n+nb}{print}\PYG{p}{(}\PYG{l+s+s1}{\PYGZsq{}}\PYG{l+s+s1}{Bon}\PYG{l+s+se}{\PYGZbs{}t}\PYG{l+s+s1}{jour}\PYG{l+s+s1}{\PYGZsq{}}\PYG{p}{,} \PYG{n}{end}\PYG{o}{=}\PYG{l+s+s1}{\PYGZsq{}}\PYG{l+s+se}{\PYGZbs{}n}\PYG{l+s+s1}{\PYGZhy{}\PYGZhy{}\PYGZhy{}\PYGZhy{}}\PYG{l+s+se}{\PYGZbs{}n}\PYG{l+s+s1}{\PYGZsq{}}\PYG{p}{)}
\PYG{n+nb}{print}\PYG{p}{(}\PYG{l+s+s1}{\PYGZsq{}}\PYG{l+s+s1}{Bon}\PYG{l+s+se}{\PYGZbs{}r}\PYG{l+s+s1}{jour}\PYG{l+s+s1}{\PYGZsq{}}\PYG{p}{,} \PYG{n}{end}\PYG{o}{=}\PYG{l+s+s1}{\PYGZsq{}}\PYG{l+s+se}{\PYGZbs{}n}\PYG{l+s+s1}{\PYGZhy{}\PYGZhy{}\PYGZhy{}\PYGZhy{}}\PYG{l+s+se}{\PYGZbs{}n}\PYG{l+s+s1}{\PYGZsq{}}\PYG{p}{)}
\PYG{n+nb}{print}\PYG{p}{(}\PYG{l+s+s1}{\PYGZsq{}}\PYG{l+s+s1}{Bon}\PYG{l+s+se}{\PYGZbs{}b}\PYG{l+s+s1}{jour}\PYG{l+s+s1}{\PYGZsq{}}\PYG{p}{,} \PYG{n}{end}\PYG{o}{=}\PYG{l+s+s1}{\PYGZsq{}}\PYG{l+s+se}{\PYGZbs{}n}\PYG{l+s+s1}{\PYGZhy{}\PYGZhy{}\PYGZhy{}\PYGZhy{}}\PYG{l+s+se}{\PYGZbs{}n}\PYG{l+s+s1}{\PYGZsq{}}\PYG{p}{)}
\PYG{n+nb}{print}\PYG{p}{(}\PYG{l+s+s1}{\PYGZsq{}}\PYG{l+s+s1}{Bon}\PYG{l+s+se}{\PYGZbs{}f}\PYG{l+s+s1}{jour}\PYG{l+s+s1}{\PYGZsq{}}\PYG{p}{,} \PYG{n}{end}\PYG{o}{=}\PYG{l+s+s1}{\PYGZsq{}}\PYG{l+s+se}{\PYGZbs{}n}\PYG{l+s+s1}{\PYGZhy{}\PYGZhy{}\PYGZhy{}\PYGZhy{}}\PYG{l+s+se}{\PYGZbs{}n}\PYG{l+s+s1}{\PYGZsq{}}\PYG{p}{)}
\end{sphinxVerbatim}

\begin{sphinxVerbatim}[commandchars=\\\{\}]
Bon
jour
\PYGZhy{}\PYGZhy{}\PYGZhy{}\PYGZhy{}
Bon	jour
\PYGZhy{}\PYGZhy{}\PYGZhy{}\PYGZhy{}
jour
\PYGZhy{}\PYGZhy{}\PYGZhy{}\PYGZhy{}
Bojour
\PYGZhy{}\PYGZhy{}\PYGZhy{}\PYGZhy{}
Bon\PYGZdq{}jour
\PYGZhy{}\PYGZhy{}\PYGZhy{}\PYGZhy{}
\end{sphinxVerbatim}


\subsection{operations sur les chaînes de caractères}
\label{\detokenize{ch3:operations-sur-les-chaines-de-caracteres}}
\sphinxAtStartPar
On peut realiser quelques orperations sur les chaînes de caractères. Par exemple, pour concatiner deux chaînes de caractères on utilise \sphinxcode{\sphinxupquote{+}}(\sphinxcode{\sphinxupquote{chaine1+chaine2}}), pour repeter une chaîne de caractère n fois on multiplie n par la chaine (\sphinxcode{\sphinxupquote{n*chaine}}).

\begin{sphinxVerbatim}[commandchars=\\\{\}]
\PYG{n}{var1} \PYG{o}{=} \PYG{l+s+s1}{\PYGZsq{}}\PYG{l+s+s1}{hello}\PYG{l+s+s1}{\PYGZsq{}}
\PYG{n}{var2} \PYG{o}{=} \PYG{l+s+s2}{\PYGZdq{}}\PYG{l+s+s2}{world}\PYG{l+s+s2}{\PYGZdq{}}
\PYG{n+nb}{print}\PYG{p}{(}\PYG{n}{var1}\PYG{o}{+}\PYG{n}{var2}\PYG{p}{)}
\PYG{n+nb}{print}\PYG{p}{(}\PYG{n}{var1}\PYG{o}{+}\PYG{l+s+s1}{\PYGZsq{}}\PYG{l+s+s1}{ }\PYG{l+s+s1}{\PYGZsq{}}\PYG{o}{+}\PYG{n}{var2}\PYG{p}{)}
\PYG{n+nb}{print}\PYG{p}{(}\PYG{l+m+mi}{5}\PYG{o}{*}\PYG{n}{var1}\PYG{p}{)}
\PYG{n+nb}{print}\PYG{p}{(}\PYG{l+m+mi}{4}\PYG{o}{*}\PYG{p}{(}\PYG{n}{var1} \PYG{o}{+} \PYG{l+s+s2}{\PYGZdq{}}\PYG{l+s+s2}{ }\PYG{l+s+s2}{\PYGZdq{}}\PYG{o}{+} \PYG{l+m+mi}{2}\PYG{o}{*}\PYG{n}{var2}\PYG{p}{)}\PYG{p}{)}
\end{sphinxVerbatim}

\begin{sphinxVerbatim}[commandchars=\\\{\}]
helloworld
hello world
hellohellohellohellohello
hello worldworldhello worldworldhello worldworldhello worldworld
\end{sphinxVerbatim}

\sphinxAtStartPar
Si l’on désire déterminer le nombre de caractères présents dans une chaîne, on utilise la fonction
intégrée \sphinxcode{\sphinxupquote{len()}} :

\begin{sphinxVerbatim}[commandchars=\\\{\}]
\PYG{n}{var} \PYG{o}{=} \PYG{l+s+s2}{\PYGZdq{}}\PYG{l+s+s2}{bonjour tout le monde}\PYG{l+s+s2}{\PYGZdq{}}
\PYG{n+nb}{print}\PYG{p}{(}\PYG{n+nb}{len}\PYG{p}{(}\PYG{n}{var}\PYG{p}{)}\PYG{p}{)}
\end{sphinxVerbatim}

\begin{sphinxVerbatim}[commandchars=\\\{\}]
21
\end{sphinxVerbatim}

\sphinxAtStartPar
La varaible \sphinxcode{\sphinxupquote{var}} contient une chaine de  21 caracteres (3 espace comptés).

\sphinxAtStartPar
Les chaînes sont des séquences de caractères. Chacun de ceux\sphinxhyphen{}ci occupe une place précise dans cette séquence. Sous Python, les éléments d’une séquence sont toujours indicés (ou numérotés) de la même manière, c’est\sphinxhyphen{}à\sphinxhyphen{}dire à partir de zéro (python commence toujours avec 0 comme premier indice). Pour extraire un caractère specifique (ou plusieurs caractères) on utilise \sphinxcode{\sphinxupquote{{[}{]}}}. Voici quelques exemple en utilisant \sphinxcode{\sphinxupquote{var = "bonjour tout le monde"}}.

\begin{sphinxVerbatim}[commandchars=\\\{\}]
\PYG{n}{var} \PYG{o}{=} \PYG{l+s+s2}{\PYGZdq{}}\PYG{l+s+s2}{bonjour tout le monde}\PYG{l+s+s2}{\PYGZdq{}}

\PYG{n+nb}{print}\PYG{p}{(}\PYG{n}{var}\PYG{p}{[}\PYG{l+m+mi}{0}\PYG{p}{]}\PYG{p}{)}

\PYG{n+nb}{print}\PYG{p}{(}\PYG{n}{var}\PYG{p}{[}\PYG{l+m+mi}{1}\PYG{p}{]}\PYG{p}{)}

\PYG{n+nb}{print}\PYG{p}{(}\PYG{n}{var}\PYG{p}{[}\PYG{l+m+mi}{2}\PYG{p}{]}\PYG{p}{)}

\PYG{n+nb}{print}\PYG{p}{(}\PYG{n}{var}\PYG{p}{[}\PYG{l+m+mi}{3}\PYG{p}{]}\PYG{p}{)}

\PYG{n+nb}{print}\PYG{p}{(}\PYG{n}{var}\PYG{p}{[}\PYG{l+m+mi}{20}\PYG{p}{]}\PYG{p}{)}
\end{sphinxVerbatim}

\begin{sphinxVerbatim}[commandchars=\\\{\}]
b
o
n
j
e
\end{sphinxVerbatim}

\sphinxAtStartPar
Dans certaines situations, Il est utile de pouvoir désigner l’emplacement d’un caractère par rapport à la fin de la chaîne.
Pour cela, nous utilisons des indices négatifs: \sphinxhyphen{}1 désignera le dernier caractère, \sphinxhyphen{}2 l’avant dernier, et ainsi de suite.

\begin{sphinxVerbatim}[commandchars=\\\{\}]
\PYG{n}{var} \PYG{o}{=} \PYG{l+s+s2}{\PYGZdq{}}\PYG{l+s+s2}{bonjour tout le monde}\PYG{l+s+s2}{\PYGZdq{}}

\PYG{n+nb}{print}\PYG{p}{(}\PYG{n}{var}\PYG{p}{[}\PYG{o}{\PYGZhy{}}\PYG{l+m+mi}{1}\PYG{p}{]}\PYG{p}{)}

\PYG{n+nb}{print}\PYG{p}{(}\PYG{n}{var}\PYG{p}{[}\PYG{o}{\PYGZhy{}}\PYG{l+m+mi}{2}\PYG{p}{]}\PYG{p}{)}

\PYG{n+nb}{print}\PYG{p}{(}\PYG{n}{var}\PYG{p}{[}\PYG{o}{\PYGZhy{}}\PYG{l+m+mi}{3}\PYG{p}{]}\PYG{p}{)}

\PYG{n+nb}{print}\PYG{p}{(}\PYG{n}{var}\PYG{p}{[}\PYG{o}{\PYGZhy{}}\PYG{l+m+mi}{21}\PYG{p}{]}\PYG{p}{)}
\end{sphinxVerbatim}

\begin{sphinxVerbatim}[commandchars=\\\{\}]
e
d
n
b
\end{sphinxVerbatim}

\sphinxAtStartPar
Il arrive souvent que l’on souhaite extraire une sous\sphinxhyphen{}sequence de caracteres d’une chaîne. Sous Python, on utilise une technique dite \sphinxcode{\sphinxupquote{slicing}} (). La technique consiste à indiquer entre crochets \sphinxcode{\sphinxupquote{{[}{]}}} les indices correspondant au
début et à la fin de la sous\sphinxhyphen{}sequence que l’on souhaite extraire.

\begin{sphinxadmonition}{warning}{Avertissement:}
\sphinxAtStartPar
Le caracter ayant preimier indice est inclu. Cependant,\sphinxstylestrong{le caracter ayant le dernier indice n’est pas iclu!!}. Si l’on veut, par exemple les trois premiers caracters, on doit indiquer 0 et 4 entre \sphinxcode{\sphinxupquote{{[}{]}}} (\sphinxcode{\sphinxupquote{{[}0:4{]}}}).
\begin{itemize}
\item {} 
\sphinxAtStartPar
Si l’on veut les \sphinxcode{\sphinxupquote{n}} premier caracters: \sphinxcode{\sphinxupquote{{[}:n+1{]}}}

\item {} 
\sphinxAtStartPar
Si l’on veut les caracters a partir de l’indice \sphinxcode{\sphinxupquote{n}} jusqu’a la fin de la chaine: \sphinxcode{\sphinxupquote{{[}n:{]}}}

\end{itemize}
\end{sphinxadmonition}

\sphinxAtStartPar
Illustrons ca avec un exemple:

\begin{sphinxVerbatim}[commandchars=\\\{\}]
\PYG{n}{var} \PYG{o}{=} \PYG{l+s+s2}{\PYGZdq{}}\PYG{l+s+s2}{bonjour tout le monde}\PYG{l+s+s2}{\PYGZdq{}}

\PYG{n+nb}{print}\PYG{p}{(}\PYG{n}{var}\PYG{p}{[}\PYG{l+m+mi}{0}\PYG{p}{:}\PYG{l+m+mi}{0}\PYG{p}{]}\PYG{p}{)}
\PYG{n+nb}{print}\PYG{p}{(}\PYG{n}{var}\PYG{p}{[}\PYG{l+m+mi}{2}\PYG{p}{:}\PYG{l+m+mi}{5}\PYG{p}{]}\PYG{p}{)}
\PYG{n+nb}{print}\PYG{p}{(}\PYG{n}{var}\PYG{p}{[}\PYG{l+m+mi}{0}\PYG{p}{:}\PYG{l+m+mi}{3}\PYG{p}{]}\PYG{p}{)}

\PYG{n+nb}{print}\PYG{p}{(}\PYG{n}{var}\PYG{p}{[}\PYG{p}{:}\PYG{l+m+mi}{3}\PYG{p}{]}\PYG{p}{)}

\PYG{n+nb}{print}\PYG{p}{(}\PYG{n}{var}\PYG{p}{[}\PYG{l+m+mi}{5}\PYG{p}{:}\PYG{l+m+mi}{21}\PYG{p}{]}\PYG{p}{)}

\PYG{n+nb}{print}\PYG{p}{(}\PYG{n}{var}\PYG{p}{[}\PYG{l+m+mi}{5}\PYG{p}{:}\PYG{p}{]}\PYG{p}{)}

\PYG{n+nb}{print}\PYG{p}{(}\PYG{n}{var}\PYG{p}{[}\PYG{p}{:}\PYG{p}{]}\PYG{p}{)}
\end{sphinxVerbatim}

\begin{sphinxVerbatim}[commandchars=\\\{\}]
njo
bon
bon
ur tout le monde
ur tout le monde
bonjour tout le monde
\end{sphinxVerbatim}

\sphinxAtStartPar
Si l’on veut tester l’appartenece d’un caracter a une chaine on utilise l’operateur \sphinxcode{\sphinxupquote{in}}. L’expression est la suivante : \sphinxcode{\sphinxupquote{caractere in chaine}}. Cela nous renvoi \sphinxcode{\sphinxupquote{True}} si \sphinxcode{\sphinxupquote{caractere}}est dans \sphinxcode{\sphinxupquote{chaine}}, sinon \sphinxcode{\sphinxupquote{False}}. On peut aussi ecrire \sphinxcode{\sphinxupquote{caractere not in chaine}} pour tester si le caractere n’est pas dans \sphinxcode{\sphinxupquote{chaine}} (\sphinxcode{\sphinxupquote{True}}) ou s’il est dans \sphinxcode{\sphinxupquote{chaine}} (\sphinxcode{\sphinxupquote{False}}). Voici quelques exemples:

\begin{sphinxVerbatim}[commandchars=\\\{\}]
\PYG{n}{var} \PYG{o}{=} \PYG{l+s+s2}{\PYGZdq{}}\PYG{l+s+s2}{bonjour tout le monde}\PYG{l+s+s2}{\PYGZdq{}}

\PYG{n+nb}{print}\PYG{p}{(}\PYG{l+s+s1}{\PYGZsq{}}\PYG{l+s+s1}{b}\PYG{l+s+s1}{\PYGZsq{}} \PYG{o+ow}{in} \PYG{n}{var}\PYG{p}{)}

\PYG{n+nb}{print}\PYG{p}{(}\PYG{l+s+s1}{\PYGZsq{}}\PYG{l+s+s1}{z}\PYG{l+s+s1}{\PYGZsq{}} \PYG{o+ow}{in} \PYG{n}{var}\PYG{p}{)}

\PYG{n+nb}{print}\PYG{p}{(}\PYG{l+s+s1}{\PYGZsq{}}\PYG{l+s+s1}{o}\PYG{l+s+s1}{\PYGZsq{}} \PYG{o+ow}{not} \PYG{o+ow}{in} \PYG{n}{var}\PYG{p}{)}

\PYG{n+nb}{print}\PYG{p}{(}\PYG{l+s+s1}{\PYGZsq{}}\PYG{l+s+s1}{x}\PYG{l+s+s1}{\PYGZsq{}} \PYG{o+ow}{not} \PYG{o+ow}{in} \PYG{n}{var}\PYG{p}{)}
\end{sphinxVerbatim}

\begin{sphinxVerbatim}[commandchars=\\\{\}]
True
False
False
True
\end{sphinxVerbatim}


\subsection{quelques methodes utiles pour les chaines des caracteres}
\label{\detokenize{ch3:quelques-methodes-utiles-pour-les-chaines-des-caracteres}}
\sphinxAtStartPar
Dans cette sections nous allons voir qulques \sphinxcode{\sphinxupquote{methodes}}  pour les chaines des caracteres (lorsque nous allons etudier la programmation orientee objet, nous allons comprendre la signification de ce terme. Pour l’instant nous allons accepter que \sphinxcode{\sphinxupquote{method}} signifie une fonction dont l’expression generale est \sphinxcode{\sphinxupquote{chqine.method(separateur)}}).
\begin{itemize}
\item {} 
\sphinxAtStartPar
split: cette methode est utlisee pour nous rendre une liste dont les elements sont les caracteres de la chaine.  On peut choisir le caractère séparateur en le fournissant comme argument, sinon c’est un espace par défaut.

\end{itemize}

\begin{sphinxVerbatim}[commandchars=\\\{\}]
\PYG{n}{var} \PYG{o}{=} \PYG{l+s+s2}{\PYGZdq{}}\PYG{l+s+s2}{bonjour tout le monde}\PYG{l+s+s2}{\PYGZdq{}}

\PYG{n}{splitted\PYGZus{}var} \PYG{o}{=} \PYG{n}{var}\PYG{o}{.}\PYG{n}{split}\PYG{p}{(}\PYG{p}{)}
\PYG{n+nb}{print}\PYG{p}{(}\PYG{n}{var}\PYG{p}{)}
\PYG{n+nb}{print}\PYG{p}{(}\PYG{n}{splitted\PYGZus{}var}\PYG{p}{)}
\PYG{n+nb}{print}\PYG{p}{(}\PYG{n+nb}{type}\PYG{p}{(}\PYG{n}{splitted\PYGZus{}var}\PYG{p}{)}\PYG{p}{)}

\PYG{n}{var2} \PYG{o}{=} \PYG{l+s+s2}{\PYGZdq{}}\PYG{l+s+s2}{un, deux, trois, cinq\PYGZhy{}six, sept}\PYG{l+s+s2}{\PYGZdq{}}
\PYG{n+nb}{print}\PYG{p}{(}\PYG{n}{var2}\PYG{o}{.}\PYG{n}{split}\PYG{p}{(}\PYG{p}{)}\PYG{p}{)}
\PYG{n+nb}{print}\PYG{p}{(}\PYG{n}{var2}\PYG{o}{.}\PYG{n}{split}\PYG{p}{(}\PYG{l+s+s1}{\PYGZsq{}}\PYG{l+s+s1}{,}\PYG{l+s+s1}{\PYGZsq{}}\PYG{p}{)}\PYG{p}{)}
\PYG{n+nb}{print}\PYG{p}{(}\PYG{n}{var2}\PYG{o}{.}\PYG{n}{split}\PYG{p}{(}\PYG{l+s+s1}{\PYGZsq{}}\PYG{l+s+s1}{\PYGZhy{}}\PYG{l+s+s1}{\PYGZsq{}}\PYG{p}{)}\PYG{p}{)}
\end{sphinxVerbatim}

\begin{sphinxVerbatim}[commandchars=\\\{\}]
bonjour tout le monde
[\PYGZsq{}bonjour\PYGZsq{}, \PYGZsq{}tout\PYGZsq{}, \PYGZsq{}le\PYGZsq{}, \PYGZsq{}monde\PYGZsq{}]
\PYGZlt{}class \PYGZsq{}list\PYGZsq{}\PYGZgt{}
[\PYGZsq{}un,\PYGZsq{}, \PYGZsq{}deux,\PYGZsq{}, \PYGZsq{}trois,\PYGZsq{}, \PYGZsq{}cinq\PYGZhy{}six,\PYGZsq{}, \PYGZsq{}sept\PYGZsq{}]
[\PYGZsq{}un\PYGZsq{}, \PYGZsq{} deux\PYGZsq{}, \PYGZsq{} trois\PYGZsq{}, \PYGZsq{} cinq\PYGZhy{}six\PYGZsq{}, \PYGZsq{} sept\PYGZsq{}]
[\PYGZsq{}un, deux, trois, cinq\PYGZsq{}, \PYGZsq{}six, sept\PYGZsq{}]
\end{sphinxVerbatim}
\begin{itemize}
\item {} 
\sphinxAtStartPar
count: lorsque elle est appliquee a un chaine de caracteres avec un agrgument qui est aussi une chaine de caracteres (eventuellement un caractere), elle renvoi le nombre d’occurences la derniere chaine dans la premiere chaines. \sphinxcode{\sphinxupquote{ chaine1.count(chaine2)}}.

\end{itemize}

\begin{sphinxVerbatim}[commandchars=\\\{\}]
\PYG{n}{var} \PYG{o}{=} \PYG{l+s+s2}{\PYGZdq{}}\PYG{l+s+s2}{bonjour tout le monde}\PYG{l+s+s2}{\PYGZdq{}}
\PYG{n+nb}{print}\PYG{p}{(}\PYG{n}{var}\PYG{o}{.}\PYG{n}{count}\PYG{p}{(}\PYG{l+s+s2}{\PYGZdq{}}\PYG{l+s+s2}{b}\PYG{l+s+s2}{\PYGZdq{}}\PYG{p}{)}\PYG{p}{)}

\PYG{n+nb}{print}\PYG{p}{(}\PYG{n}{var}\PYG{o}{.}\PYG{n}{count}\PYG{p}{(}\PYG{l+s+s2}{\PYGZdq{}}\PYG{l+s+s2}{on}\PYG{l+s+s2}{\PYGZdq{}}\PYG{p}{)}\PYG{p}{)}

\PYG{n+nb}{print}\PYG{p}{(}\PYG{n}{var}\PYG{o}{.}\PYG{n}{count}\PYG{p}{(}\PYG{l+s+s2}{\PYGZdq{}}\PYG{l+s+s2}{ }\PYG{l+s+s2}{\PYGZdq{}}\PYG{p}{)}\PYG{p}{)}
\end{sphinxVerbatim}

\begin{sphinxVerbatim}[commandchars=\\\{\}]
1
2
3
\end{sphinxVerbatim}
\begin{itemize}
\item {} 
\sphinxAtStartPar
find, index : Lorsque l’une de ces methodes est appliquee a une chaine de caractere avec un agrgument qui est aussi une chaine de caracteres (eventuellement un caractere), elle renvoi l’indicee  la chaine cherchee dans la premiere chaines (l’indice de la primiere occurence). \sphinxcode{\sphinxupquote{find}} renvoit \sphinxhyphen{}1 si la chaine cherchee n’existe pas alors que index revoit une erreur dans la meme situation. \sphinxcode{\sphinxupquote{ chaine1.find(chaine2)}} et \sphinxcode{\sphinxupquote{ chaine1.index(chaine2)}}.

\end{itemize}

\begin{sphinxVerbatim}[commandchars=\\\{\}]
\PYG{n}{var} \PYG{o}{=} \PYG{l+s+s2}{\PYGZdq{}}\PYG{l+s+s2}{bonjour tout le monde}\PYG{l+s+s2}{\PYGZdq{}}
\PYG{n+nb}{print}\PYG{p}{(}\PYG{n}{var}\PYG{o}{.}\PYG{n}{find}\PYG{p}{(}\PYG{l+s+s1}{\PYGZsq{}}\PYG{l+s+s1}{b}\PYG{l+s+s1}{\PYGZsq{}}\PYG{p}{)}\PYG{p}{)}
\PYG{n+nb}{print}\PYG{p}{(}\PYG{n}{var}\PYG{o}{.}\PYG{n}{find}\PYG{p}{(}\PYG{l+s+s1}{\PYGZsq{}}\PYG{l+s+s1}{e}\PYG{l+s+s1}{\PYGZsq{}}\PYG{p}{)}\PYG{p}{)}
\PYG{n+nb}{print}\PYG{p}{(}\PYG{n}{var}\PYG{o}{.}\PYG{n}{find}\PYG{p}{(}\PYG{l+s+s1}{\PYGZsq{}}\PYG{l+s+s1}{tout}\PYG{l+s+s1}{\PYGZsq{}}\PYG{p}{)}\PYG{p}{)}
\PYG{n+nb}{print}\PYG{p}{(}\PYG{n}{var}\PYG{o}{.}\PYG{n}{find}\PYG{p}{(}\PYG{l+s+s1}{\PYGZsq{}}\PYG{l+s+s1}{x}\PYG{l+s+s1}{\PYGZsq{}}\PYG{p}{)}\PYG{p}{)}
\end{sphinxVerbatim}

\begin{sphinxVerbatim}[commandchars=\\\{\}]
0
14
8
\PYGZhy{}1
\end{sphinxVerbatim}

\begin{sphinxVerbatim}[commandchars=\\\{\}]
\PYG{n}{var} \PYG{o}{=} \PYG{l+s+s2}{\PYGZdq{}}\PYG{l+s+s2}{bonjour tout le monde}\PYG{l+s+s2}{\PYGZdq{}}
\PYG{n+nb}{print}\PYG{p}{(}\PYG{n}{var}\PYG{o}{.}\PYG{n}{index}\PYG{p}{(}\PYG{l+s+s1}{\PYGZsq{}}\PYG{l+s+s1}{b}\PYG{l+s+s1}{\PYGZsq{}}\PYG{p}{)}\PYG{p}{)}
\PYG{n+nb}{print}\PYG{p}{(}\PYG{n}{var}\PYG{o}{.}\PYG{n}{index}\PYG{p}{(}\PYG{l+s+s1}{\PYGZsq{}}\PYG{l+s+s1}{e}\PYG{l+s+s1}{\PYGZsq{}}\PYG{p}{)}\PYG{p}{)}
\PYG{n+nb}{print}\PYG{p}{(}\PYG{n}{var}\PYG{o}{.}\PYG{n}{index}\PYG{p}{(}\PYG{l+s+s1}{\PYGZsq{}}\PYG{l+s+s1}{tout}\PYG{l+s+s1}{\PYGZsq{}}\PYG{p}{)}\PYG{p}{)}
\PYG{n+nb}{print}\PYG{p}{(}\PYG{n}{var}\PYG{o}{.}\PYG{n}{index}\PYG{p}{(}\PYG{l+s+s1}{\PYGZsq{}}\PYG{l+s+s1}{x}\PYG{l+s+s1}{\PYGZsq{}}\PYG{p}{)}\PYG{p}{)}
\end{sphinxVerbatim}

\begin{sphinxVerbatim}[commandchars=\\\{\}]
0
14
8
\end{sphinxVerbatim}

\begin{sphinxVerbatim}[commandchars=\\\{\}]
\PYG{g+gt}{\PYGZhy{}\PYGZhy{}\PYGZhy{}\PYGZhy{}\PYGZhy{}\PYGZhy{}\PYGZhy{}\PYGZhy{}\PYGZhy{}\PYGZhy{}\PYGZhy{}\PYGZhy{}\PYGZhy{}\PYGZhy{}\PYGZhy{}\PYGZhy{}\PYGZhy{}\PYGZhy{}\PYGZhy{}\PYGZhy{}\PYGZhy{}\PYGZhy{}\PYGZhy{}\PYGZhy{}\PYGZhy{}\PYGZhy{}\PYGZhy{}\PYGZhy{}\PYGZhy{}\PYGZhy{}\PYGZhy{}\PYGZhy{}\PYGZhy{}\PYGZhy{}\PYGZhy{}\PYGZhy{}\PYGZhy{}\PYGZhy{}\PYGZhy{}\PYGZhy{}\PYGZhy{}\PYGZhy{}\PYGZhy{}\PYGZhy{}\PYGZhy{}\PYGZhy{}\PYGZhy{}\PYGZhy{}\PYGZhy{}\PYGZhy{}\PYGZhy{}\PYGZhy{}\PYGZhy{}\PYGZhy{}\PYGZhy{}\PYGZhy{}\PYGZhy{}\PYGZhy{}\PYGZhy{}\PYGZhy{}\PYGZhy{}\PYGZhy{}\PYGZhy{}\PYGZhy{}\PYGZhy{}\PYGZhy{}\PYGZhy{}\PYGZhy{}\PYGZhy{}\PYGZhy{}\PYGZhy{}\PYGZhy{}\PYGZhy{}\PYGZhy{}\PYGZhy{}}
\PYG{n+ne}{ValueError}\PYG{g+gWhitespace}{                                }Traceback (most recent call last)
\PYG{o}{\PYGZti{}}\PYGZbs{}\PYG{n}{AppData}\PYGZbs{}\PYG{n}{Local}\PYGZbs{}\PYG{n}{Temp}\PYGZbs{}\PYG{n}{ipykernel\PYGZus{}10040}\PYGZbs{}\PYG{l+m+mf}{458526427.}\PYG{n}{py} \PYG{o+ow}{in} \PYG{o}{\PYGZlt{}}\PYG{n}{module}\PYG{o}{\PYGZgt{}}
\PYG{g+gWhitespace}{      }\PYG{l+m+mi}{3} \PYG{n+nb}{print}\PYG{p}{(}\PYG{n}{var}\PYG{o}{.}\PYG{n}{index}\PYG{p}{(}\PYG{l+s+s1}{\PYGZsq{}}\PYG{l+s+s1}{e}\PYG{l+s+s1}{\PYGZsq{}}\PYG{p}{)}\PYG{p}{)}
\PYG{g+gWhitespace}{      }\PYG{l+m+mi}{4} \PYG{n+nb}{print}\PYG{p}{(}\PYG{n}{var}\PYG{o}{.}\PYG{n}{index}\PYG{p}{(}\PYG{l+s+s1}{\PYGZsq{}}\PYG{l+s+s1}{tout}\PYG{l+s+s1}{\PYGZsq{}}\PYG{p}{)}\PYG{p}{)}
\PYG{n+ne}{\PYGZhy{}\PYGZhy{}\PYGZhy{}\PYGZhy{}\PYGZgt{} }\PYG{l+m+mi}{5} \PYG{n+nb}{print}\PYG{p}{(}\PYG{n}{var}\PYG{o}{.}\PYG{n}{index}\PYG{p}{(}\PYG{l+s+s1}{\PYGZsq{}}\PYG{l+s+s1}{x}\PYG{l+s+s1}{\PYGZsq{}}\PYG{p}{)}\PYG{p}{)}

\PYG{n+ne}{ValueError}: substring not found
\end{sphinxVerbatim}
\begin{itemize}
\item {} 
\sphinxAtStartPar
lower, upper, title, capitalize, swapcase: Lorsque l’une de ces methodes est appliquee a une chaine de caractere, elle la convertit on minuscule (\sphinxcode{\sphinxupquote{lower}}), majuscule (\sphinxcode{\sphinxupquote{upper}}), majuscule a l’initial de chaque mot (\sphinxcode{\sphinxupquote{title}}), mjuscule a l’initial du premier mot (\sphinxcode{\sphinxupquote{capitalize}}), et les majuscules en minuscules, et vice\sphinxhyphen{}versa (\sphinxcode{\sphinxupquote{swapecase}}).

\end{itemize}

\begin{sphinxVerbatim}[commandchars=\\\{\}]
\PYG{n}{var1} \PYG{o}{=} \PYG{l+s+s2}{\PYGZdq{}}\PYG{l+s+s2}{bonjour tout le monde}\PYG{l+s+s2}{\PYGZdq{}}
\PYG{n}{var2} \PYG{o}{=} \PYG{l+s+s2}{\PYGZdq{}}\PYG{l+s+s2}{Bonjour tout le monde}\PYG{l+s+s2}{\PYGZdq{}}
\PYG{n}{var3} \PYG{o}{=} \PYG{l+s+s2}{\PYGZdq{}}\PYG{l+s+s2}{BONJOUR tout le monde}\PYG{l+s+s2}{\PYGZdq{}}

\PYG{n+nb}{print}\PYG{p}{(}\PYG{n}{var1}\PYG{o}{.}\PYG{n}{upper}\PYG{p}{(}\PYG{p}{)}\PYG{p}{)}
\PYG{n+nb}{print}\PYG{p}{(}\PYG{n}{var1}\PYG{o}{.}\PYG{n}{title}\PYG{p}{(}\PYG{p}{)}\PYG{p}{)}
\PYG{n+nb}{print}\PYG{p}{(}\PYG{n}{var1}\PYG{o}{.}\PYG{n}{capitalize}\PYG{p}{(}\PYG{p}{)}\PYG{p}{)}
\PYG{n+nb}{print}\PYG{p}{(}\PYG{n}{var3}\PYG{o}{.}\PYG{n}{lower}\PYG{p}{(}\PYG{p}{)}\PYG{p}{)}
\PYG{n+nb}{print}\PYG{p}{(}\PYG{n}{var1}\PYG{o}{.}\PYG{n}{swapcase}\PYG{p}{(}\PYG{p}{)}\PYG{p}{)}
\end{sphinxVerbatim}

\begin{sphinxVerbatim}[commandchars=\\\{\}]
BONJOUR TOUT LE MONDE
Bonjour Tout Le Monde
Bonjour tout le monde
bonjour tout le monde
BONJOUR TOUT LE MONDE
\end{sphinxVerbatim}
\begin{itemize}
\item {} 
\sphinxAtStartPar
strip: enlève les espaces (s’ils existent) au début et à la fin de la chaîne.

\end{itemize}

\begin{sphinxVerbatim}[commandchars=\\\{\}]
\PYG{n}{var1} \PYG{o}{=} \PYG{l+s+s2}{\PYGZdq{}}\PYG{l+s+s2}{bonjour tout le monde}\PYG{l+s+s2}{\PYGZdq{}}
\PYG{n}{var2} \PYG{o}{=} \PYG{l+s+s2}{\PYGZdq{}}\PYG{l+s+s2}{   Bonjour tout le monde   }\PYG{l+s+s2}{\PYGZdq{}}

\PYG{n+nb}{print}\PYG{p}{(}\PYG{n}{var1}\PYG{o}{.}\PYG{n}{strip}\PYG{p}{(}\PYG{p}{)}\PYG{p}{)}
\PYG{n+nb}{print}\PYG{p}{(}\PYG{n}{var2}\PYG{o}{.}\PYG{n}{strip}\PYG{p}{(}\PYG{p}{)}\PYG{p}{)}
\end{sphinxVerbatim}

\begin{sphinxVerbatim}[commandchars=\\\{\}]
bonjour tout le monde
Bonjour tout le monde
\end{sphinxVerbatim}
\begin{itemize}
\item {} 
\sphinxAtStartPar
replace: remplace une cahine de caracteres (eventuellement un seul caractere) par une autre cahine de caractere.

\end{itemize}

\begin{sphinxVerbatim}[commandchars=\\\{\}]
\PYG{n}{var} \PYG{o}{=} \PYG{l+s+s2}{\PYGZdq{}}\PYG{l+s+s2}{bonjour tout le monde}\PYG{l+s+s2}{\PYGZdq{}}
\PYG{n+nb}{print}\PYG{p}{(}\PYG{n}{var}\PYG{o}{.}\PYG{n}{replace}\PYG{p}{(}\PYG{l+s+s1}{\PYGZsq{}}\PYG{l+s+s1}{b}\PYG{l+s+s1}{\PYGZsq{}}\PYG{p}{,} \PYG{l+s+s1}{\PYGZsq{}}\PYG{l+s+s1}{B}\PYG{l+s+s1}{\PYGZsq{}}\PYG{p}{)}\PYG{p}{)}

\PYG{n+nb}{print}\PYG{p}{(}\PYG{n}{var}\PYG{o}{.}\PYG{n}{replace}\PYG{p}{(}\PYG{l+s+s1}{\PYGZsq{}}\PYG{l+s+s1}{bonjour}\PYG{l+s+s1}{\PYGZsq{}}\PYG{p}{,} \PYG{l+s+s1}{\PYGZsq{}}\PYG{l+s+s1}{bonsoir}\PYG{l+s+s1}{\PYGZsq{}}\PYG{p}{)}\PYG{p}{)}
\end{sphinxVerbatim}

\begin{sphinxVerbatim}[commandchars=\\\{\}]
Bonjour tout le monde
bonsoir tout le monde
\end{sphinxVerbatim}

\sphinxAtStartPar
Formatage des chaînes de caractères

\sphinxAtStartPar
Les chaines de caracters en python sont riche en terme de fonctionalite (vous allez, au fur et mesure, decouvrir des nouvelles methods et technique avancees). Dans cette section, nou allons presenter une derniere une technique de traitement très puissante, que l’on appelle formatage des chaînes.

\sphinxAtStartPar
Nous pouvons, d’ailleurs, faire ce que cette technique fait avec la concatenation des chaines (avec \sphinxcode{\sphinxupquote{+}}). Cependant le formatage est plus lisible est ne necessite pas du code additionel. Illustrons ca avec un exemple.

\sphinxAtStartPar
3 \%, format, f”

\begin{sphinxVerbatim}[commandchars=\\\{\}]
\PYG{n}{message} \PYG{o}{=} \PYG{l+s+s1}{\PYGZsq{}}\PYG{l+s+s1}{ est le meilleur language de programmation}\PYG{l+s+s1}{\PYGZsq{}}
\PYG{n}{nom\PYGZus{}du\PYGZus{}langage} \PYG{o}{=} \PYG{l+s+s1}{\PYGZsq{}}\PYG{l+s+s1}{python}\PYG{l+s+s1}{\PYGZsq{}}
\PYG{n+nb}{print}\PYG{p}{(}\PYG{n}{nom\PYGZus{}du\PYGZus{}langage} \PYG{o}{+} \PYG{n}{message}\PYG{p}{)}
\end{sphinxVerbatim}

\begin{sphinxVerbatim}[commandchars=\\\{\}]
python est le meilleur language de programmation
\end{sphinxVerbatim}

\begin{sphinxVerbatim}[commandchars=\\\{\}]
\PYG{n}{message} \PYG{o}{=} \PYG{l+s+s1}{\PYGZsq{}}\PYG{l+s+si}{\PYGZob{}\PYGZcb{}}\PYG{l+s+s1}{ est le meilleur language de programmation}\PYG{l+s+s1}{\PYGZsq{}}
\PYG{n}{nom\PYGZus{}du\PYGZus{}langage} \PYG{o}{=} \PYG{l+s+s1}{\PYGZsq{}}\PYG{l+s+s1}{python}\PYG{l+s+s1}{\PYGZsq{}}
\PYG{n+nb}{print}\PYG{p}{(}\PYG{n}{message}\PYG{o}{.}\PYG{n}{format}\PYG{p}{(}\PYG{n}{nom\PYGZus{}du\PYGZus{}langage}\PYG{p}{)}\PYG{p}{)}
\end{sphinxVerbatim}

\begin{sphinxVerbatim}[commandchars=\\\{\}]
python est le meilleur language de programmation
\end{sphinxVerbatim}

\begin{sphinxVerbatim}[commandchars=\\\{\}]
\PYG{n}{message} \PYG{o}{=} \PYG{l+s+s1}{\PYGZsq{}}\PYG{l+s+si}{\PYGZpc{}s}\PYG{l+s+s1}{ est le meilleur language de programmation}\PYG{l+s+s1}{\PYGZsq{}}
\PYG{n}{nom\PYGZus{}du\PYGZus{}langage} \PYG{o}{=} \PYG{l+s+s1}{\PYGZsq{}}\PYG{l+s+s1}{python}\PYG{l+s+s1}{\PYGZsq{}}
\PYG{n+nb}{print}\PYG{p}{(}\PYG{n}{message} \PYG{o}{\PYGZpc{}} \PYG{n}{nom\PYGZus{}du\PYGZus{}langage}\PYG{p}{)}
\end{sphinxVerbatim}

\begin{sphinxVerbatim}[commandchars=\\\{\}]
python est le meilleur language de programmation
\end{sphinxVerbatim}

\begin{sphinxVerbatim}[commandchars=\\\{\}]
\PYG{n}{nom\PYGZus{}du\PYGZus{}langage} \PYG{o}{=} \PYG{l+s+s1}{\PYGZsq{}}\PYG{l+s+s1}{python}\PYG{l+s+s1}{\PYGZsq{}}
\PYG{n}{message} \PYG{o}{=} \PYG{l+s+sa}{f}\PYG{l+s+s1}{\PYGZsq{}}\PYG{l+s+si}{\PYGZob{}}\PYG{n}{nom\PYGZus{}du\PYGZus{}langage}\PYG{l+s+si}{\PYGZcb{}}\PYG{l+s+s1}{ est le meilleur language de programmation}\PYG{l+s+s1}{\PYGZsq{}}

\PYG{n+nb}{print}\PYG{p}{(}\PYG{n}{message}\PYG{p}{)}
\end{sphinxVerbatim}

\begin{sphinxVerbatim}[commandchars=\\\{\}]
python est le meilleur language de programmation
\end{sphinxVerbatim}


\section{Manupilation des listes et tuples}
\label{\detokenize{ch4:manupilation-des-listes-et-tuples}}\label{\detokenize{ch4::doc}}
\begin{sphinxadmonition}{note}{Rappel sur les listes}

\sphinxAtStartPar
Nous avons vu que:
\begin{itemize}
\item {} 
\sphinxAtStartPar
En Python, une liste est créée en plaçant des éléments entre crochets \sphinxcode{\sphinxupquote{{[}{]}}} , séparés par des virgules(\sphinxcode{\sphinxupquote{,}});

\item {} 
\sphinxAtStartPar
Les elements de la liste peuvent d’etre des objets differents,

\item {} 
\sphinxAtStartPar
Une liste vide est une list qui contient 0 element (\sphinxcode{\sphinxupquote{{[}{]}}} ou \sphinxcode{\sphinxupquote{list()}});

\item {} 
\sphinxAtStartPar
Pour creer une liste et l’affecter a une variable:

\end{itemize}

\begin{sphinxVerbatim}[commandchars=\\\{\}]
\PYG{n}{une\PYGZus{}liste} \PYG{o}{=} \PYG{p}{[}\PYG{l+m+mi}{1}\PYG{p}{,} \PYG{l+s+s1}{\PYGZsq{}}\PYG{l+s+s1}{bonjour}\PYG{l+s+s1}{\PYGZsq{}}\PYG{p}{,} \PYG{l+m+mf}{4.5}\PYG{p}{,} \PYG{k+kc}{True}\PYG{p}{]}
\end{sphinxVerbatim}
\end{sphinxadmonition}

\begin{sphinxadmonition}{note}{Tuples}

\sphinxAtStartPar
Les tuples (ou encore uplets) sont des types construits. Tout comme les listes, les tuples sont une collection ordonees d’elements.
\begin{itemize}
\item {} 
\sphinxAtStartPar
En Python, un tuple est créée en plaçant des éléments entre parentheses \sphinxcode{\sphinxupquote{()}} , séparés par des virgules(\sphinxcode{\sphinxupquote{,}});

\item {} 
\sphinxAtStartPar
Un tuple vide est un tuple qui contient 0 element (\sphinxcode{\sphinxupquote{()}} ou \sphinxcode{\sphinxupquote{tuple()}});

\item {} 
\sphinxAtStartPar
Pour creer un tuple et l’affecter a une variable:

\end{itemize}

\begin{sphinxVerbatim}[commandchars=\\\{\}]
\PYG{n}{un\PYGZus{}tuple} \PYG{o}{=} \PYG{p}{(}\PYG{l+m+mi}{1}\PYG{p}{,} \PYG{l+s+s1}{\PYGZsq{}}\PYG{l+s+s1}{bonjour}\PYG{l+s+s1}{\PYGZsq{}}\PYG{p}{,} \PYG{l+m+mf}{4.5}\PYG{p}{,} \PYG{k+kc}{True}\PYG{p}{)}
\end{sphinxVerbatim}
\end{sphinxadmonition}


\subsection{Operations sur les listes les tuples}
\label{\detokenize{ch4:operations-sur-les-listes-les-tuples}}
\sphinxAtStartPar
Tout comme les chaines de caracteres, on peut realiser quelques orperations sur les listes (tuples). Par exemple, pour concatiner deux listes (deux tuples), on utilise \sphinxcode{\sphinxupquote{+}}(\sphinxcode{\sphinxupquote{liste1+liste2}} ou \sphinxcode{\sphinxupquote{tuple1+tuple2}}), pour repeter une liste (un tuple) n fois, on multiplie n par la liste/tuple (\sphinxcode{\sphinxupquote{n*liste}}ou \sphinxcode{\sphinxupquote{n*tuple}}).

\begin{sphinxVerbatim}[commandchars=\\\{\}]
\PYG{n}{une\PYGZus{}liste} \PYG{o}{=} \PYG{p}{[}\PYG{l+m+mi}{1}\PYG{p}{,} \PYG{l+s+s1}{\PYGZsq{}}\PYG{l+s+s1}{bonjour}\PYG{l+s+s1}{\PYGZsq{}}\PYG{p}{,} \PYG{l+m+mf}{4.5}\PYG{p}{,} \PYG{k+kc}{True}\PYG{p}{]}
\PYG{n}{autre\PYGZus{}liste} \PYG{o}{=} \PYG{p}{[}\PYG{l+m+mi}{2}\PYG{p}{,} \PYG{l+m+mf}{4.5}\PYG{p}{,} \PYG{l+s+s1}{\PYGZsq{}}\PYG{l+s+s1}{tout le monde}\PYG{l+s+s1}{\PYGZsq{}}\PYG{p}{,} \PYG{k+kc}{False}\PYG{p}{]}
\PYG{n+nb}{print}\PYG{p}{(}\PYG{n}{une\PYGZus{}liste}\PYG{o}{+} \PYG{n}{autre\PYGZus{}liste}\PYG{p}{)}
\PYG{n+nb}{print}\PYG{p}{(}\PYG{n}{une\PYGZus{}liste}\PYG{o}{*}\PYG{l+m+mi}{4}\PYG{p}{)}
\end{sphinxVerbatim}

\begin{sphinxVerbatim}[commandchars=\\\{\}]
[1, \PYGZsq{}bonjour\PYGZsq{}, 4.5, True, 2, 4.5, \PYGZsq{}tout le monde\PYGZsq{}, False]
[1, \PYGZsq{}bonjour\PYGZsq{}, 4.5, True, 1, \PYGZsq{}bonjour\PYGZsq{}, 4.5, True, 1, \PYGZsq{}bonjour\PYGZsq{}, 4.5, True, 1, \PYGZsq{}bonjour\PYGZsq{}, 4.5, True]
\end{sphinxVerbatim}

\begin{sphinxVerbatim}[commandchars=\\\{\}]
\PYG{n}{un\PYGZus{}tuple} \PYG{o}{=} \PYG{p}{(}\PYG{l+m+mi}{1}\PYG{p}{,} \PYG{l+s+s1}{\PYGZsq{}}\PYG{l+s+s1}{bonjour}\PYG{l+s+s1}{\PYGZsq{}}\PYG{p}{,} \PYG{l+m+mf}{4.5}\PYG{p}{,} \PYG{k+kc}{True}\PYG{p}{)}
\PYG{n}{autre\PYGZus{}tuple} \PYG{o}{=} \PYG{p}{(}\PYG{l+m+mi}{2}\PYG{p}{,} \PYG{l+m+mf}{4.5}\PYG{p}{,} \PYG{l+s+s1}{\PYGZsq{}}\PYG{l+s+s1}{tout le monde}\PYG{l+s+s1}{\PYGZsq{}}\PYG{p}{,} \PYG{k+kc}{False}\PYG{p}{)}
\PYG{n+nb}{print}\PYG{p}{(}\PYG{n}{un\PYGZus{}tuple}\PYG{o}{+} \PYG{n}{autre\PYGZus{}tuple}\PYG{p}{)}
\PYG{n+nb}{print}\PYG{p}{(}\PYG{n}{un\PYGZus{}tuple}\PYG{o}{*}\PYG{l+m+mi}{4}\PYG{p}{)}            
\end{sphinxVerbatim}

\begin{sphinxVerbatim}[commandchars=\\\{\}]
(1, \PYGZsq{}bonjour\PYGZsq{}, 4.5, True, 2, 4.5, \PYGZsq{}tout le monde\PYGZsq{}, False)
(1, \PYGZsq{}bonjour\PYGZsq{}, 4.5, True, 1, \PYGZsq{}bonjour\PYGZsq{}, 4.5, True, 1, \PYGZsq{}bonjour\PYGZsq{}, 4.5, True, 1, \PYGZsq{}bonjour\PYGZsq{}, 4.5, True)
\end{sphinxVerbatim}

\sphinxAtStartPar
Les listes/tuples sont aussi une séquence d’objets quelconques. Chacun de ceux\sphinxhyphen{}ci occupe une place précise dans cette séquence. Pour extraire un element specifique d’une liste/tuple, on utilise \sphinxcode{\sphinxupquote{{[}{]}}}. Voici quelques exemple en utilisant \sphinxcode{\sphinxupquote{une\_liste = {[}1, \textquotesingle{}bonjour\textquotesingle{}, 4.5, True{]}}}. Nous utilisons aussi les indices négatifs: \sphinxhyphen{}1 désignera le dernier caractère, \sphinxhyphen{}2 l’avant dernier, et ainsi de suite. Nous utilisons le \sphinxcode{\sphinxupquote{slicing}} aussi de la meme maniere que les chaines.

\begin{sphinxadmonition}{warning}{Avertissement:}
\sphinxAtStartPar
Tout comme le cas des \sphinxcode{\sphinxupquote{strings}}, l’element ayant premier indice est inclu. Cependant,\sphinxstylestrong{l’element ayant le dernier indice n’est pas iclu!!}. Si l’on veut, par exemple les trois premiers elements, on doit indiquer 0 et 4 entre \sphinxcode{\sphinxupquote{{[}{]}}} (\sphinxcode{\sphinxupquote{{[}0:4{]}}}).
\begin{itemize}
\item {} 
\sphinxAtStartPar
Si l’on veut les \sphinxcode{\sphinxupquote{n}} premier elements: \sphinxcode{\sphinxupquote{{[}:n+1{]}}}

\item {} 
\sphinxAtStartPar
Si l’on veut les elements a partir de l’indice \sphinxcode{\sphinxupquote{n}} jusqu’a la fin de la liste: \sphinxcode{\sphinxupquote{{[}n:{]}}}

\end{itemize}
\end{sphinxadmonition}

\begin{sphinxVerbatim}[commandchars=\\\{\}]
\PYG{n}{une\PYGZus{}liste} \PYG{o}{=} \PYG{p}{[}\PYG{l+m+mi}{1}\PYG{p}{,} \PYG{l+s+s1}{\PYGZsq{}}\PYG{l+s+s1}{bonjour}\PYG{l+s+s1}{\PYGZsq{}}\PYG{p}{,} \PYG{l+m+mf}{4.5}\PYG{p}{,} \PYG{k+kc}{True}\PYG{p}{]}
\PYG{n+nb}{print}\PYG{p}{(}\PYG{n}{une\PYGZus{}liste}\PYG{p}{[}\PYG{l+m+mi}{0}\PYG{p}{]}\PYG{p}{)}
\PYG{n+nb}{print}\PYG{p}{(}\PYG{n}{une\PYGZus{}liste}\PYG{p}{[}\PYG{l+m+mi}{1}\PYG{p}{]}\PYG{p}{)}

\PYG{n+nb}{print}\PYG{p}{(}\PYG{n}{une\PYGZus{}liste}\PYG{p}{[}\PYG{o}{\PYGZhy{}}\PYG{l+m+mi}{1}\PYG{p}{]}\PYG{p}{)}
\PYG{n+nb}{print}\PYG{p}{(}\PYG{n}{une\PYGZus{}liste}\PYG{p}{[}\PYG{o}{\PYGZhy{}}\PYG{l+m+mi}{2}\PYG{p}{]}\PYG{p}{)}

\PYG{n+nb}{print}\PYG{p}{(}\PYG{n}{une\PYGZus{}liste}\PYG{p}{[}\PYG{l+m+mi}{0}\PYG{p}{:}\PYG{l+m+mi}{0}\PYG{p}{]}\PYG{p}{)}
\PYG{n+nb}{print}\PYG{p}{(}\PYG{n}{une\PYGZus{}liste}\PYG{p}{[}\PYG{l+m+mi}{1}\PYG{p}{:}\PYG{l+m+mi}{3}\PYG{p}{]}\PYG{p}{)}

\PYG{n+nb}{print}\PYG{p}{(}\PYG{n}{une\PYGZus{}liste}\PYG{p}{[}\PYG{p}{:}\PYG{l+m+mi}{3}\PYG{p}{]}\PYG{p}{)}
\PYG{n+nb}{print}\PYG{p}{(}\PYG{n}{une\PYGZus{}liste}\PYG{p}{[}\PYG{l+m+mi}{2}\PYG{p}{:}\PYG{p}{]}\PYG{p}{)}

\PYG{n+nb}{print}\PYG{p}{(}\PYG{n}{une\PYGZus{}liste}\PYG{p}{[}\PYG{p}{:}\PYG{p}{]}\PYG{p}{)}
\end{sphinxVerbatim}

\begin{sphinxVerbatim}[commandchars=\\\{\}]
1
bonjour
True
4.5
[]
[\PYGZsq{}bonjour\PYGZsq{}, 4.5]
[1, \PYGZsq{}bonjour\PYGZsq{}, 4.5]
[4.5, True]
[1, \PYGZsq{}bonjour\PYGZsq{}, 4.5, True]
\end{sphinxVerbatim}

\begin{sphinxVerbatim}[commandchars=\\\{\}]
\PYG{n}{un\PYGZus{}tuple} \PYG{o}{=} \PYG{p}{(}\PYG{l+m+mi}{1}\PYG{p}{,} \PYG{l+s+s1}{\PYGZsq{}}\PYG{l+s+s1}{bonjour}\PYG{l+s+s1}{\PYGZsq{}}\PYG{p}{,} \PYG{l+m+mf}{4.5}\PYG{p}{,} \PYG{k+kc}{True}\PYG{p}{)}
\PYG{n+nb}{print}\PYG{p}{(}\PYG{n}{un\PYGZus{}tuple}\PYG{p}{[}\PYG{l+m+mi}{0}\PYG{p}{]}\PYG{p}{)}
\PYG{n+nb}{print}\PYG{p}{(}\PYG{n}{un\PYGZus{}tuple}\PYG{p}{[}\PYG{l+m+mi}{1}\PYG{p}{]}\PYG{p}{)}

\PYG{n+nb}{print}\PYG{p}{(}\PYG{n}{un\PYGZus{}tuple}\PYG{p}{[}\PYG{o}{\PYGZhy{}}\PYG{l+m+mi}{1}\PYG{p}{]}\PYG{p}{)}
\PYG{n+nb}{print}\PYG{p}{(}\PYG{n}{un\PYGZus{}tuple}\PYG{p}{[}\PYG{o}{\PYGZhy{}}\PYG{l+m+mi}{2}\PYG{p}{]}\PYG{p}{)}

\PYG{n+nb}{print}\PYG{p}{(}\PYG{n}{un\PYGZus{}tuple}\PYG{p}{[}\PYG{l+m+mi}{0}\PYG{p}{:}\PYG{l+m+mi}{0}\PYG{p}{]}\PYG{p}{)}
\PYG{n+nb}{print}\PYG{p}{(}\PYG{n}{un\PYGZus{}tuple}\PYG{p}{[}\PYG{l+m+mi}{1}\PYG{p}{:}\PYG{l+m+mi}{3}\PYG{p}{]}\PYG{p}{)}

\PYG{n+nb}{print}\PYG{p}{(}\PYG{n}{un\PYGZus{}tuple}\PYG{p}{[}\PYG{p}{:}\PYG{l+m+mi}{3}\PYG{p}{]}\PYG{p}{)}
\PYG{n+nb}{print}\PYG{p}{(}\PYG{n}{un\PYGZus{}tuple}\PYG{p}{[}\PYG{l+m+mi}{2}\PYG{p}{:}\PYG{p}{]}\PYG{p}{)}

\PYG{n+nb}{print}\PYG{p}{(}\PYG{n}{un\PYGZus{}tuple}\PYG{p}{[}\PYG{p}{:}\PYG{p}{]}\PYG{p}{)}
\end{sphinxVerbatim}

\begin{sphinxVerbatim}[commandchars=\\\{\}]
1
bonjour
True
4.5
()
(\PYGZsq{}bonjour\PYGZsq{}, 4.5)
(1, \PYGZsq{}bonjour\PYGZsq{}, 4.5)
(4.5, True)
(1, \PYGZsq{}bonjour\PYGZsq{}, 4.5, True)
\end{sphinxVerbatim}

\sphinxAtStartPar
La fonction intégrée \sphinxcode{\sphinxupquote{len()}} est aussi appliquable pour les listes et les tuples. Si l’on désire déterminer le nombre d’elements:

\begin{sphinxVerbatim}[commandchars=\\\{\}]
\PYG{n}{une\PYGZus{}liste} \PYG{o}{=} \PYG{p}{[}\PYG{l+m+mi}{1}\PYG{p}{,} \PYG{l+s+s1}{\PYGZsq{}}\PYG{l+s+s1}{bonjour}\PYG{l+s+s1}{\PYGZsq{}}\PYG{p}{,} \PYG{l+m+mf}{4.5}\PYG{p}{,} \PYG{k+kc}{True}\PYG{p}{]}
\PYG{n}{un\PYGZus{}tuple} \PYG{o}{=} \PYG{p}{(}\PYG{l+m+mi}{1}\PYG{p}{,} \PYG{l+s+s1}{\PYGZsq{}}\PYG{l+s+s1}{bonjour}\PYG{l+s+s1}{\PYGZsq{}}\PYG{p}{,} \PYG{l+m+mf}{4.5}\PYG{p}{,} \PYG{k+kc}{True}\PYG{p}{)}
\PYG{n+nb}{print}\PYG{p}{(}\PYG{n+nb}{len}\PYG{p}{(}\PYG{n}{une\PYGZus{}liste}\PYG{p}{)}\PYG{p}{)}
\PYG{n+nb}{print}\PYG{p}{(}\PYG{n+nb}{len}\PYG{p}{(}\PYG{n}{un\PYGZus{}tuple}\PYG{p}{)}\PYG{p}{)}
\end{sphinxVerbatim}

\begin{sphinxVerbatim}[commandchars=\\\{\}]
4
4
\end{sphinxVerbatim}

\sphinxAtStartPar
Les deux varaibles \sphinxcode{\sphinxupquote{une\_liste}} et \sphinxcode{\sphinxupquote{un\_tuple}} contienent 4 elements.

\sphinxAtStartPar
Si l’on veut tester l’appartenece d’un element a une liste/tuple on utilise l’operateur \sphinxcode{\sphinxupquote{in}}. L’expression est la suivante : \sphinxcode{\sphinxupquote{element in liste}}. Cela nous renvoi \sphinxcode{\sphinxupquote{True}} si \sphinxcode{\sphinxupquote{element}}est dans \sphinxcode{\sphinxupquote{liste}}, sinon \sphinxcode{\sphinxupquote{False}}. On peut aussi ecrire \sphinxcode{\sphinxupquote{element not in liste}} pour tester si l’element n’est pas dans \sphinxcode{\sphinxupquote{liste}} (\sphinxcode{\sphinxupquote{True}}) ou s’il est dans \sphinxcode{\sphinxupquote{liste}} (\sphinxcode{\sphinxupquote{False}}). Voici quelques exemples:

\begin{sphinxVerbatim}[commandchars=\\\{\}]
\PYG{n}{une\PYGZus{}liste} \PYG{o}{=} \PYG{p}{[}\PYG{l+m+mi}{1}\PYG{p}{,} \PYG{l+s+s1}{\PYGZsq{}}\PYG{l+s+s1}{bonjour}\PYG{l+s+s1}{\PYGZsq{}}\PYG{p}{,} \PYG{l+m+mf}{4.5}\PYG{p}{,} \PYG{k+kc}{True}\PYG{p}{]}

\PYG{n+nb}{print}\PYG{p}{(}\PYG{l+m+mi}{1} \PYG{o+ow}{in} \PYG{n}{une\PYGZus{}liste}\PYG{p}{)}

\PYG{n+nb}{print}\PYG{p}{(}\PYG{l+m+mi}{2} \PYG{o+ow}{in} \PYG{n}{une\PYGZus{}liste}\PYG{p}{)}

\PYG{n+nb}{print}\PYG{p}{(}\PYG{l+s+s1}{\PYGZsq{}}\PYG{l+s+s1}{o}\PYG{l+s+s1}{\PYGZsq{}} \PYG{o+ow}{not} \PYG{o+ow}{in} \PYG{n}{une\PYGZus{}liste}\PYG{p}{)}

\PYG{n+nb}{print}\PYG{p}{(}\PYG{l+s+s1}{\PYGZsq{}}\PYG{l+s+s1}{x}\PYG{l+s+s1}{\PYGZsq{}} \PYG{o+ow}{not} \PYG{o+ow}{in} \PYG{n}{une\PYGZus{}liste}\PYG{p}{)}
\end{sphinxVerbatim}

\begin{sphinxVerbatim}[commandchars=\\\{\}]
True
False
True
True
\end{sphinxVerbatim}

\begin{sphinxVerbatim}[commandchars=\\\{\}]
\PYG{n}{un\PYGZus{}tuple} \PYG{o}{=} \PYG{p}{(}\PYG{l+m+mi}{1}\PYG{p}{,} \PYG{l+s+s1}{\PYGZsq{}}\PYG{l+s+s1}{bonjour}\PYG{l+s+s1}{\PYGZsq{}}\PYG{p}{,} \PYG{l+m+mf}{4.5}\PYG{p}{,} \PYG{k+kc}{True}\PYG{p}{)}

\PYG{n+nb}{print}\PYG{p}{(}\PYG{l+m+mi}{1} \PYG{o+ow}{in} \PYG{n}{un\PYGZus{}tuple}\PYG{p}{)}

\PYG{n+nb}{print}\PYG{p}{(}\PYG{l+m+mi}{2} \PYG{o+ow}{in} \PYG{n}{un\PYGZus{}tuple}\PYG{p}{)}

\PYG{n+nb}{print}\PYG{p}{(}\PYG{l+s+s1}{\PYGZsq{}}\PYG{l+s+s1}{o}\PYG{l+s+s1}{\PYGZsq{}} \PYG{o+ow}{not} \PYG{o+ow}{in} \PYG{n}{un\PYGZus{}tuple}\PYG{p}{)}

\PYG{n+nb}{print}\PYG{p}{(}\PYG{l+s+s1}{\PYGZsq{}}\PYG{l+s+s1}{x}\PYG{l+s+s1}{\PYGZsq{}} \PYG{o+ow}{not} \PYG{o+ow}{in} \PYG{n}{un\PYGZus{}tuple}\PYG{p}{)}
\end{sphinxVerbatim}

\begin{sphinxVerbatim}[commandchars=\\\{\}]
True
False
True
True
\end{sphinxVerbatim}


\subsection{Quelle est la difference entre listes et tuples?}
\label{\detokenize{ch4:quelle-est-la-difference-entre-listes-et-tuples}}
\sphinxAtStartPar
Les listes et les tuples sont pareils dans la plupart des contextes. Cepandant, la difference primordiale entre les deux et que les listes sont des \sphinxstylestrong{objets mutables} (modifiables) alors que les tuples sont des \sphinxstylestrong{objets immuables} (ne sont pas modifiable). La question qui se pose est donc: qu’est\sphinxhyphen{}ce qu’un objet mutable et un objet immuable?
Parmi les objet immuable en python, on trouve:
\begin{itemize}
\item {} 
\sphinxAtStartPar
Les nombres entiers (int)

\item {} 
\sphinxAtStartPar
Les nombres décimaux (float)

\item {} 
\sphinxAtStartPar
Les chaînes de caractères (str)

\item {} 
\sphinxAtStartPar
Les booléens (bool)

\item {} 
\sphinxAtStartPar
Les tuples (tuple)
La plus part des autres objets que vous allez confronter en python sont mutables.

\end{itemize}

\sphinxAtStartPar
Nous allons illustre ca dans les exemples suivant:
\begin{enumerate}
\sphinxsetlistlabels{\arabic}{enumi}{enumii}{}{.}%
\item {} 
\sphinxAtStartPar
nous allons creer les variables suivantes:

\end{enumerate}
\begin{itemize}
\item {} 
\sphinxAtStartPar
\sphinxcode{\sphinxupquote{var\_chaine = "bonjour tout le monde"}},

\item {} 
\sphinxAtStartPar
\sphinxcode{\sphinxupquote{var\_liste = {[}1, 2, True, \textquotesingle{}bonjour\textquotesingle{}{]}}},

\item {} 
\sphinxAtStartPar
\sphinxcode{\sphinxupquote{var\_tuple = (1, 2, True, \textquotesingle{}bonjour\textquotesingle{})}}.

\end{itemize}
\begin{enumerate}
\sphinxsetlistlabels{\arabic}{enumi}{enumii}{}{.}%
\item {} 
\sphinxAtStartPar
nous qllons essayer de changer (par exemple) le premier element de chaque variable (par un autre element).

\end{enumerate}

\begin{sphinxVerbatim}[commandchars=\\\{\}]
\PYG{n}{var\PYGZus{}chaine} \PYG{o}{=} \PYG{l+s+s2}{\PYGZdq{}}\PYG{l+s+s2}{bonjour tout le monde}\PYG{l+s+s2}{\PYGZdq{}}
\PYG{n}{var\PYGZus{}liste} \PYG{o}{=} \PYG{p}{[}\PYG{l+m+mi}{1}\PYG{p}{,} \PYG{l+m+mi}{2}\PYG{p}{,} \PYG{k+kc}{True}\PYG{p}{,} \PYG{l+s+s1}{\PYGZsq{}}\PYG{l+s+s1}{bonjour}\PYG{l+s+s1}{\PYGZsq{}}\PYG{p}{]}
\PYG{n}{var\PYGZus{}tuple} \PYG{o}{=} \PYG{p}{(}\PYG{l+m+mi}{1}\PYG{p}{,} \PYG{l+m+mi}{2}\PYG{p}{,} \PYG{k+kc}{True}\PYG{p}{,} \PYG{l+s+s1}{\PYGZsq{}}\PYG{l+s+s1}{bonjour}\PYG{l+s+s1}{\PYGZsq{}}\PYG{p}{)}
\end{sphinxVerbatim}

\begin{sphinxVerbatim}[commandchars=\\\{\}]
\PYG{n}{var\PYGZus{}chaine}\PYG{p}{[}\PYG{l+m+mi}{0}\PYG{p}{]} \PYG{o}{=} \PYG{l+s+s1}{\PYGZsq{}}\PYG{l+s+s1}{B}\PYG{l+s+s1}{\PYGZsq{}}
\PYG{n+nb}{print}\PYG{p}{(}\PYG{n}{var\PYGZus{}chaine}\PYG{p}{)}
\end{sphinxVerbatim}

\begin{sphinxVerbatim}[commandchars=\\\{\}]
\PYG{g+gt}{\PYGZhy{}\PYGZhy{}\PYGZhy{}\PYGZhy{}\PYGZhy{}\PYGZhy{}\PYGZhy{}\PYGZhy{}\PYGZhy{}\PYGZhy{}\PYGZhy{}\PYGZhy{}\PYGZhy{}\PYGZhy{}\PYGZhy{}\PYGZhy{}\PYGZhy{}\PYGZhy{}\PYGZhy{}\PYGZhy{}\PYGZhy{}\PYGZhy{}\PYGZhy{}\PYGZhy{}\PYGZhy{}\PYGZhy{}\PYGZhy{}\PYGZhy{}\PYGZhy{}\PYGZhy{}\PYGZhy{}\PYGZhy{}\PYGZhy{}\PYGZhy{}\PYGZhy{}\PYGZhy{}\PYGZhy{}\PYGZhy{}\PYGZhy{}\PYGZhy{}\PYGZhy{}\PYGZhy{}\PYGZhy{}\PYGZhy{}\PYGZhy{}\PYGZhy{}\PYGZhy{}\PYGZhy{}\PYGZhy{}\PYGZhy{}\PYGZhy{}\PYGZhy{}\PYGZhy{}\PYGZhy{}\PYGZhy{}\PYGZhy{}\PYGZhy{}\PYGZhy{}\PYGZhy{}\PYGZhy{}\PYGZhy{}\PYGZhy{}\PYGZhy{}\PYGZhy{}\PYGZhy{}\PYGZhy{}\PYGZhy{}\PYGZhy{}\PYGZhy{}\PYGZhy{}\PYGZhy{}\PYGZhy{}\PYGZhy{}\PYGZhy{}\PYGZhy{}}
\PYG{n+ne}{TypeError}\PYG{g+gWhitespace}{                                 }Traceback (most recent call last)
\PYG{o}{\PYGZti{}}\PYGZbs{}\PYG{n}{AppData}\PYGZbs{}\PYG{n}{Local}\PYGZbs{}\PYG{n}{Temp}\PYGZbs{}\PYG{n}{ipykernel\PYGZus{}232}\PYGZbs{}\PYG{l+m+mf}{4274672008.}\PYG{n}{py} \PYG{o+ow}{in} \PYG{o}{\PYGZlt{}}\PYG{n}{module}\PYG{o}{\PYGZgt{}}
\PYG{n+ne}{\PYGZhy{}\PYGZhy{}\PYGZhy{}\PYGZhy{}\PYGZgt{} }\PYG{l+m+mi}{1} \PYG{n}{var\PYGZus{}chaine}\PYG{p}{[}\PYG{l+m+mi}{0}\PYG{p}{]} \PYG{o}{=} \PYG{l+s+s1}{\PYGZsq{}}\PYG{l+s+s1}{B}\PYG{l+s+s1}{\PYGZsq{}}
\PYG{g+gWhitespace}{      }\PYG{l+m+mi}{2} \PYG{n+nb}{print}\PYG{p}{(}\PYG{n}{var\PYGZus{}chaine}\PYG{p}{)}

\PYG{n+ne}{TypeError}: \PYGZsq{}str\PYGZsq{} object does not support item assignment
\end{sphinxVerbatim}

\begin{sphinxVerbatim}[commandchars=\\\{\}]
\PYG{n}{var\PYGZus{}liste}\PYG{p}{[}\PYG{l+m+mi}{0}\PYG{p}{]} \PYG{o}{=} \PYG{l+m+mi}{3333}
\PYG{n+nb}{print}\PYG{p}{(}\PYG{n}{var\PYGZus{}liste}\PYG{p}{)}
\end{sphinxVerbatim}

\begin{sphinxVerbatim}[commandchars=\\\{\}]
[3333, 2, True, \PYGZsq{}bonjour\PYGZsq{}]
\end{sphinxVerbatim}

\begin{sphinxVerbatim}[commandchars=\\\{\}]
\PYG{n}{var\PYGZus{}tuple}\PYG{p}{[}\PYG{l+m+mi}{0}\PYG{p}{]} \PYG{o}{=} \PYG{l+m+mi}{3333}
\PYG{n+nb}{print}\PYG{p}{(}\PYG{n}{var\PYGZus{}tuple}\PYG{p}{)}
\end{sphinxVerbatim}

\begin{sphinxVerbatim}[commandchars=\\\{\}]
\PYG{g+gt}{\PYGZhy{}\PYGZhy{}\PYGZhy{}\PYGZhy{}\PYGZhy{}\PYGZhy{}\PYGZhy{}\PYGZhy{}\PYGZhy{}\PYGZhy{}\PYGZhy{}\PYGZhy{}\PYGZhy{}\PYGZhy{}\PYGZhy{}\PYGZhy{}\PYGZhy{}\PYGZhy{}\PYGZhy{}\PYGZhy{}\PYGZhy{}\PYGZhy{}\PYGZhy{}\PYGZhy{}\PYGZhy{}\PYGZhy{}\PYGZhy{}\PYGZhy{}\PYGZhy{}\PYGZhy{}\PYGZhy{}\PYGZhy{}\PYGZhy{}\PYGZhy{}\PYGZhy{}\PYGZhy{}\PYGZhy{}\PYGZhy{}\PYGZhy{}\PYGZhy{}\PYGZhy{}\PYGZhy{}\PYGZhy{}\PYGZhy{}\PYGZhy{}\PYGZhy{}\PYGZhy{}\PYGZhy{}\PYGZhy{}\PYGZhy{}\PYGZhy{}\PYGZhy{}\PYGZhy{}\PYGZhy{}\PYGZhy{}\PYGZhy{}\PYGZhy{}\PYGZhy{}\PYGZhy{}\PYGZhy{}\PYGZhy{}\PYGZhy{}\PYGZhy{}\PYGZhy{}\PYGZhy{}\PYGZhy{}\PYGZhy{}\PYGZhy{}\PYGZhy{}\PYGZhy{}\PYGZhy{}\PYGZhy{}\PYGZhy{}\PYGZhy{}\PYGZhy{}}
\PYG{n+ne}{TypeError}\PYG{g+gWhitespace}{                                 }Traceback (most recent call last)
\PYG{o}{\PYGZti{}}\PYGZbs{}\PYG{n}{AppData}\PYGZbs{}\PYG{n}{Local}\PYGZbs{}\PYG{n}{Temp}\PYGZbs{}\PYG{n}{ipykernel\PYGZus{}232}\PYGZbs{}\PYG{l+m+mf}{613115800.}\PYG{n}{py} \PYG{o+ow}{in} \PYG{o}{\PYGZlt{}}\PYG{n}{module}\PYG{o}{\PYGZgt{}}
\PYG{n+ne}{\PYGZhy{}\PYGZhy{}\PYGZhy{}\PYGZhy{}\PYGZgt{} }\PYG{l+m+mi}{1} \PYG{n}{var\PYGZus{}tuple}\PYG{p}{[}\PYG{l+m+mi}{0}\PYG{p}{]} \PYG{o}{=} \PYG{l+m+mi}{3333}
\PYG{g+gWhitespace}{      }\PYG{l+m+mi}{2} \PYG{n+nb}{print}\PYG{p}{(}\PYG{n}{var\PYGZus{}tuple}\PYG{p}{)}

\PYG{n+ne}{TypeError}: \PYGZsq{}tuple\PYGZsq{} object does not support item assignment
\end{sphinxVerbatim}

\sphinxAtStartPar
Les listes ont une taille variable, les tuples et les chaines de caracteres ont une taille fixe. Enfin, les listes ont plus de fonctionnalités que les tuples. Cependant, c’est le contexte qui nous force a utiliser les listes ou les tuples. Nous allons rencontrer plusieurs contextes ou on est amener a choisir l’un des deux types.


\subsection{quelques methodes utiles pour les listes et les tuples}
\label{\detokenize{ch4:quelques-methodes-utiles-pour-les-listes-et-les-tuples}}
\sphinxAtStartPar
Dans cette sections nous allons voir qulques \sphinxcode{\sphinxupquote{methodes}}  pour les liste et/ou les tuples (les methodes communes et les methodes propres aux listes seulement). Les deux methodes suivantes sont communes aux listes et au tuples:
\begin{itemize}
\item {} 
\sphinxAtStartPar
count: cette methode est utlisee pour compter le nombre d’elements de la liste/tuple.

\item {} 
\sphinxAtStartPar
index: cette method est utilisee pour chercher une valeur spécifiée dans la liste/tuple et renvoie la position de l’endroit où il a été trouvé.

\end{itemize}

\begin{sphinxVerbatim}[commandchars=\\\{\}]
\PYG{n}{var\PYGZus{}liste} \PYG{o}{=} \PYG{p}{[}\PYG{l+m+mi}{1}\PYG{p}{,} \PYG{l+m+mi}{2}\PYG{p}{,} \PYG{l+m+mi}{2}\PYG{p}{,} \PYG{k+kc}{True}\PYG{p}{,} \PYG{l+s+s1}{\PYGZsq{}}\PYG{l+s+s1}{bonjour}\PYG{l+s+s1}{\PYGZsq{}}\PYG{p}{,} \PYG{l+m+mi}{2}\PYG{p}{]}
\PYG{n}{var\PYGZus{}tuple} \PYG{o}{=} \PYG{p}{(}\PYG{l+m+mi}{1}\PYG{p}{,} \PYG{l+m+mi}{2}\PYG{p}{,} \PYG{l+m+mi}{2}\PYG{p}{,} \PYG{k+kc}{True}\PYG{p}{,} \PYG{l+s+s1}{\PYGZsq{}}\PYG{l+s+s1}{bonjour}\PYG{l+s+s1}{\PYGZsq{}}\PYG{p}{,} \PYG{l+m+mi}{2}\PYG{p}{)}

\PYG{n+nb}{print}\PYG{p}{(}\PYG{n}{var\PYGZus{}liste}\PYG{o}{.}\PYG{n}{count}\PYG{p}{(}\PYG{l+m+mi}{2}\PYG{p}{)}\PYG{p}{)} \PYG{c+c1}{\PYGZsh{}\PYGZsh{} print(var\PYGZus{}tuple.count(2))}

\PYG{n+nb}{print}\PYG{p}{(}\PYG{n}{var\PYGZus{}tuple}\PYG{o}{.}\PYG{n}{index}\PYG{p}{(}\PYG{l+m+mi}{2}\PYG{p}{)}\PYG{p}{)}  \PYG{c+c1}{\PYGZsh{}\PYGZsh{} print(var\PYGZus{}tuple.index(2))}
\end{sphinxVerbatim}

\begin{sphinxVerbatim}[commandchars=\\\{\}]
3
\end{sphinxVerbatim}

\begin{sphinxVerbatim}[commandchars=\\\{\}]
\PYG{n}{var\PYGZus{}tuple}\PYG{o}{.}\PYG{n}{index}\PYG{p}{(}\PYG{l+m+mi}{44}\PYG{p}{)} \PYG{c+c1}{\PYGZsh{} var\PYGZus{}tuple.index(44)}
\end{sphinxVerbatim}

\begin{sphinxVerbatim}[commandchars=\\\{\}]
\PYG{g+gt}{\PYGZhy{}\PYGZhy{}\PYGZhy{}\PYGZhy{}\PYGZhy{}\PYGZhy{}\PYGZhy{}\PYGZhy{}\PYGZhy{}\PYGZhy{}\PYGZhy{}\PYGZhy{}\PYGZhy{}\PYGZhy{}\PYGZhy{}\PYGZhy{}\PYGZhy{}\PYGZhy{}\PYGZhy{}\PYGZhy{}\PYGZhy{}\PYGZhy{}\PYGZhy{}\PYGZhy{}\PYGZhy{}\PYGZhy{}\PYGZhy{}\PYGZhy{}\PYGZhy{}\PYGZhy{}\PYGZhy{}\PYGZhy{}\PYGZhy{}\PYGZhy{}\PYGZhy{}\PYGZhy{}\PYGZhy{}\PYGZhy{}\PYGZhy{}\PYGZhy{}\PYGZhy{}\PYGZhy{}\PYGZhy{}\PYGZhy{}\PYGZhy{}\PYGZhy{}\PYGZhy{}\PYGZhy{}\PYGZhy{}\PYGZhy{}\PYGZhy{}\PYGZhy{}\PYGZhy{}\PYGZhy{}\PYGZhy{}\PYGZhy{}\PYGZhy{}\PYGZhy{}\PYGZhy{}\PYGZhy{}\PYGZhy{}\PYGZhy{}\PYGZhy{}\PYGZhy{}\PYGZhy{}\PYGZhy{}\PYGZhy{}\PYGZhy{}\PYGZhy{}\PYGZhy{}\PYGZhy{}\PYGZhy{}\PYGZhy{}\PYGZhy{}\PYGZhy{}}
\PYG{n+ne}{ValueError}\PYG{g+gWhitespace}{                                }Traceback (most recent call last)
\PYG{o}{\PYGZti{}}\PYGZbs{}\PYG{n}{AppData}\PYGZbs{}\PYG{n}{Local}\PYGZbs{}\PYG{n}{Temp}\PYGZbs{}\PYG{n}{ipykernel\PYGZus{}232}\PYGZbs{}\PYG{l+m+mf}{1993419995.}\PYG{n}{py} \PYG{o+ow}{in} \PYG{o}{\PYGZlt{}}\PYG{n}{module}\PYG{o}{\PYGZgt{}}
\PYG{n+ne}{\PYGZhy{}\PYGZhy{}\PYGZhy{}\PYGZhy{}\PYGZgt{} }\PYG{l+m+mi}{1} \PYG{n}{var\PYGZus{}tuple}\PYG{o}{.}\PYG{n}{index}\PYG{p}{(}\PYG{l+m+mi}{44}\PYG{p}{)} \PYG{c+c1}{\PYGZsh{} var\PYGZus{}tuple.index(44)}

\PYG{n+ne}{ValueError}: tuple.index(x): x not in tuple
\end{sphinxVerbatim}

\sphinxAtStartPar
Les methodes decrites dans la table suivante sont appliquee au lites seulement:


\begin{savenotes}\sphinxattablestart
\centering
\begin{tabulary}{\linewidth}[t]{|T|T|}
\hline
\sphinxstyletheadfamily 
\sphinxAtStartPar
Méthode
&\sphinxstyletheadfamily 
\sphinxAtStartPar
Description
\\
\hline
\sphinxAtStartPar
append()
&
\sphinxAtStartPar
Ajoute un élément en fin de liste
\\
\hline
\sphinxAtStartPar
clear()
&
\sphinxAtStartPar
Supprime tous les éléments de la liste
\\
\hline
\sphinxAtStartPar
copy()
&
\sphinxAtStartPar
Renvoie une copie de la liste
\\
\hline
\sphinxAtStartPar
count()
&
\sphinxAtStartPar
Renvoie le nombre d’éléments avec la valeur spécifiée
\\
\hline
\sphinxAtStartPar
extend()
&
\sphinxAtStartPar
Ajouter les éléments d’une liste, à la fin de la liste actuelle
\\
\hline
\sphinxAtStartPar
index()
&
\sphinxAtStartPar
Renvoie l’indice du premier élément avec la valeur spécifiée
\\
\hline
\sphinxAtStartPar
insert()
&
\sphinxAtStartPar
Ajoute un élément à la position spécifiée
\\
\hline
\sphinxAtStartPar
pop()
&
\sphinxAtStartPar
Supprime l’élément à la position spécifiée
\\
\hline
\sphinxAtStartPar
remove()
&
\sphinxAtStartPar
Supprime le premier élément avec la valeur spécifiée
\\
\hline
\sphinxAtStartPar
reverse()
&
\sphinxAtStartPar
Inverse l’ordre de la liste
\\
\hline
\sphinxAtStartPar
sort()
&
\sphinxAtStartPar
Trie la liste
\\
\hline
\end{tabulary}
\par
\sphinxattableend\end{savenotes}


\subsection{Slicing}
\label{\detokenize{ch4:slicing}}
\sphinxAtStartPar
quelques technique en slicing


\section{Manupilation des dictionaires (dictionnaries) et ensembles (sets)}
\label{\detokenize{ch5:manupilation-des-dictionaires-dictionnaries-et-ensembles-sets}}\label{\detokenize{ch5::doc}}

\subsection{Dictionnaire (dictionnary)}
\label{\detokenize{ch5:dictionnaire-dictionnary}}
\sphinxAtStartPar
Contrairment au objets construits que nous avons vu (\sphinxcode{\sphinxupquote{strings}}, \sphinxcode{\sphinxupquote{lists}}, \sphinxcode{\sphinxupquote{tuples}}), les dictionnaires (\sphinxcode{\sphinxupquote{dictionnary}}) sont utilisés pour stocker des valeurs de données dans des paires clé:valeur. Un dictionnaire est une collection ordonnée , modifiable et qui n’autorise pas les doublons (Depuis la version 3.7 de Python, les dictionnaires demeurent ordonnés. Dans les versions antérieures (Python 3.6 et moins), les dictionnaires ne sont pas ordonnés).
\begin{itemize}
\item {} 
\sphinxAtStartPar
Les dictionnaires sont écrits avec des accolades et ont des clés et des valeurs.  Pour creer un dictionnaire et l’affecter a une variable :

\end{itemize}

\begin{sphinxVerbatim}[commandchars=\\\{\}]
\PYG{n}{var\PYGZus{}dict} \PYG{o}{=} \PYG{p}{\PYGZob{}}\PYG{n}{cle\PYGZus{}1}\PYG{p}{:} \PYG{n}{valeur\PYGZus{}1}\PYG{p}{,} \PYG{n}{cle\PYGZus{}2}\PYG{p}{:}\PYG{n}{valeur\PYGZus{}2}\PYG{p}{,}\PYG{o}{.}\PYG{o}{.}\PYG{o}{.}\PYG{o}{.}\PYG{p}{,} \PYG{n}{cle\PYGZus{}n}\PYG{p}{:}\PYG{n}{valeur\PYGZus{}n}\PYG{p}{\PYGZcb{}}
\end{sphinxVerbatim}
\begin{itemize}
\item {} 
\sphinxAtStartPar
Un dictionnaire vide est un dictionnaire qui contient 0 element (\{\} ou dict());

\item {} 
\sphinxAtStartPar
Les dictionnaires sont \sphinxstylestrong{mutable}, on peut modifier leur contenu et leur taille.

\end{itemize}

\begin{sphinxVerbatim}[commandchars=\\\{\}]
\PYG{c+c1}{\PYGZsh{} dictionnaire vide}
\PYG{n}{var} \PYG{o}{=} \PYG{p}{\PYGZob{}}\PYG{p}{\PYGZcb{}} \PYG{c+c1}{\PYGZsh{} ou var = dict()}
\PYG{n+nb}{print}\PYG{p}{(}\PYG{n+nb}{type}\PYG{p}{(}\PYG{n}{var}\PYG{p}{)}\PYG{p}{)}
\PYG{n}{les\PYGZus{}jours} \PYG{o}{=} \PYG{p}{\PYGZob{}}\PYG{l+m+mi}{1}\PYG{p}{:}\PYG{l+s+s2}{\PYGZdq{}}\PYG{l+s+s2}{lundi}\PYG{l+s+s2}{\PYGZdq{}}\PYG{p}{,} \PYG{l+m+mi}{2}\PYG{p}{:}\PYG{l+s+s2}{\PYGZdq{}}\PYG{l+s+s2}{mardi}\PYG{l+s+s2}{\PYGZdq{}}\PYG{p}{,} \PYG{l+m+mi}{3}\PYG{p}{:}\PYG{l+s+s2}{\PYGZdq{}}\PYG{l+s+s2}{mercredi}\PYG{l+s+s2}{\PYGZdq{}}\PYG{p}{,} \PYG{l+m+mi}{4}\PYG{p}{:}\PYG{l+s+s2}{\PYGZdq{}}\PYG{l+s+s2}{jeudi}\PYG{l+s+s2}{\PYGZdq{}}\PYG{p}{,} \PYG{l+m+mi}{5}\PYG{p}{:}\PYG{l+s+s2}{\PYGZdq{}}\PYG{l+s+s2}{vendredi}\PYG{l+s+s2}{\PYGZdq{}}\PYG{p}{,} \PYG{l+m+mi}{6}\PYG{p}{:}\PYG{l+s+s2}{\PYGZdq{}}\PYG{l+s+s2}{samedi}\PYG{l+s+s2}{\PYGZdq{}}\PYG{p}{,} \PYG{l+m+mi}{7}\PYG{p}{:}\PYG{l+s+s2}{\PYGZdq{}}\PYG{l+s+s2}{dimanche}\PYG{l+s+s2}{\PYGZdq{}}\PYG{p}{\PYGZcb{}}
\PYG{n+nb}{print}\PYG{p}{(}\PYG{n}{les\PYGZus{}jours}\PYG{p}{)}
\end{sphinxVerbatim}

\begin{sphinxVerbatim}[commandchars=\\\{\}]
\PYGZlt{}class \PYGZsq{}dict\PYGZsq{}\PYGZgt{}
\PYGZob{}1: \PYGZsq{}lundi\PYGZsq{}, 2: \PYGZsq{}mardi\PYGZsq{}, 3: \PYGZsq{}mercredi\PYGZsq{}, 4: \PYGZsq{}jeudi\PYGZsq{}, 5: \PYGZsq{}vendredi\PYGZsq{}, 6: \PYGZsq{}samedi\PYGZsq{}, 7: \PYGZsq{}dimanche\PYGZsq{}\PYGZcb{}
\end{sphinxVerbatim}

\begin{sphinxVerbatim}[commandchars=\\\{\}]
\PYG{n}{les\PYGZus{}jours2} \PYG{o}{=} \PYG{p}{\PYGZob{}}\PYG{p}{\PYGZcb{}}

\PYG{n}{les\PYGZus{}jours2}\PYG{p}{[}\PYG{l+m+mi}{1}\PYG{p}{]} \PYG{o}{=} \PYG{l+s+s1}{\PYGZsq{}}\PYG{l+s+s1}{lundi}\PYG{l+s+s1}{\PYGZsq{}}
\PYG{n}{les\PYGZus{}jours2}\PYG{p}{[}\PYG{l+m+mi}{2}\PYG{p}{]} \PYG{o}{=} \PYG{l+s+s1}{\PYGZsq{}}\PYG{l+s+s1}{mardi}\PYG{l+s+s1}{\PYGZsq{}}
\PYG{n}{les\PYGZus{}jours2}\PYG{p}{[}\PYG{l+m+mi}{3}\PYG{p}{]} \PYG{o}{=} \PYG{l+s+s1}{\PYGZsq{}}\PYG{l+s+s1}{mercredi}\PYG{l+s+s1}{\PYGZsq{}}
\PYG{n}{les\PYGZus{}jours2}\PYG{p}{[}\PYG{l+m+mi}{4}\PYG{p}{]} \PYG{o}{=} \PYG{l+s+s1}{\PYGZsq{}}\PYG{l+s+s1}{jeudi}\PYG{l+s+s1}{\PYGZsq{}}
\PYG{n}{les\PYGZus{}jours2}\PYG{p}{[}\PYG{l+m+mi}{5}\PYG{p}{]} \PYG{o}{=} \PYG{l+s+s1}{\PYGZsq{}}\PYG{l+s+s1}{vendredi}\PYG{l+s+s1}{\PYGZsq{}}
\PYG{n}{les\PYGZus{}jours2}\PYG{p}{[}\PYG{l+m+mi}{6}\PYG{p}{]} \PYG{o}{=} \PYG{l+s+s1}{\PYGZsq{}}\PYG{l+s+s1}{samedi}\PYG{l+s+s1}{\PYGZsq{}}
\PYG{n}{les\PYGZus{}jours2}\PYG{p}{[}\PYG{l+m+mi}{7}\PYG{p}{]} \PYG{o}{=} \PYG{l+s+s1}{\PYGZsq{}}\PYG{l+s+s1}{dimanche}\PYG{l+s+s1}{\PYGZsq{}}

\PYG{n+nb}{print}\PYG{p}{(}\PYG{n}{les\PYGZus{}jours2}\PYG{p}{)}
\end{sphinxVerbatim}

\begin{sphinxVerbatim}[commandchars=\\\{\}]
\PYGZob{}1: \PYGZsq{}lundi\PYGZsq{}, 2: \PYGZsq{}mardi\PYGZsq{}, 3: \PYGZsq{}mercredi\PYGZsq{}, 4: \PYGZsq{}jeudi\PYGZsq{}, 5: \PYGZsq{}vendredi\PYGZsq{}, 6: \PYGZsq{}samedi\PYGZsq{}, 7: \PYGZsq{}dimanche\PYGZsq{}\PYGZcb{}
\end{sphinxVerbatim}

\sphinxAtStartPar
les cles et les valeurs peuvent etre de n’importe quel types de donnees. Par exemple on peut creer le dictionnaire suivant:

\begin{sphinxVerbatim}[commandchars=\\\{\}]
\PYG{n}{les\PYGZus{}jours3} \PYG{o}{=} \PYG{p}{\PYGZob{}}\PYG{p}{\PYGZcb{}}

\PYG{n}{les\PYGZus{}jours3}\PYG{p}{[}\PYG{l+s+s1}{\PYGZsq{}}\PYG{l+s+s1}{lundi}\PYG{l+s+s1}{\PYGZsq{}}\PYG{p}{]} \PYG{o}{=} \PYG{l+m+mi}{1}
\PYG{n}{les\PYGZus{}jours3}\PYG{p}{[}\PYG{l+s+s1}{\PYGZsq{}}\PYG{l+s+s1}{mardi}\PYG{l+s+s1}{\PYGZsq{}}\PYG{p}{]} \PYG{o}{=} \PYG{l+m+mi}{2}
\PYG{n}{les\PYGZus{}jours3}\PYG{p}{[}\PYG{l+s+s1}{\PYGZsq{}}\PYG{l+s+s1}{mercredi}\PYG{l+s+s1}{\PYGZsq{}}\PYG{p}{]} \PYG{o}{=} \PYG{l+m+mi}{3}
\PYG{n}{les\PYGZus{}jours3}\PYG{p}{[}\PYG{l+s+s1}{\PYGZsq{}}\PYG{l+s+s1}{jeudi}\PYG{l+s+s1}{\PYGZsq{}}\PYG{p}{]} \PYG{o}{=} \PYG{l+m+mi}{4}
\PYG{n}{les\PYGZus{}jours3}\PYG{p}{[}\PYG{l+s+s1}{\PYGZsq{}}\PYG{l+s+s1}{vendredi}\PYG{l+s+s1}{\PYGZsq{}}\PYG{p}{]} \PYG{o}{=} \PYG{l+m+mi}{5}
\PYG{n}{les\PYGZus{}jours3}\PYG{p}{[}\PYG{l+s+s1}{\PYGZsq{}}\PYG{l+s+s1}{samedi}\PYG{l+s+s1}{\PYGZsq{}}\PYG{p}{]} \PYG{o}{=} \PYG{l+m+mi}{6}
\PYG{n}{les\PYGZus{}jours3}\PYG{p}{[}\PYG{l+s+s1}{\PYGZsq{}}\PYG{l+s+s1}{dimanche}\PYG{l+s+s1}{\PYGZsq{}}\PYG{p}{]} \PYG{o}{=} \PYG{l+m+mi}{7}

\PYG{n+nb}{print}\PYG{p}{(}\PYG{n}{les\PYGZus{}jours3}\PYG{p}{)}
\end{sphinxVerbatim}

\begin{sphinxVerbatim}[commandchars=\\\{\}]
\PYGZob{}\PYGZsq{}lundi\PYGZsq{}: 1, \PYGZsq{}mardi\PYGZsq{}: 2, \PYGZsq{}mercredi\PYGZsq{}: 3, \PYGZsq{}jeudi\PYGZsq{}: 4, \PYGZsq{}vendredi\PYGZsq{}: 5, \PYGZsq{}samedi\PYGZsq{}: 6, \PYGZsq{}dimanche\PYGZsq{}: 7\PYGZcb{}
\end{sphinxVerbatim}

\sphinxAtStartPar
Contrairment aux chaines de caracteres et aux listes, pour acceder a une valeur dans un dictionnaire nous deveons utiliser les cles. Par exemple, si nous voulons acceder a la valeur associee a la cle \sphinxcode{\sphinxupquote{jeudi}} dans \sphinxcode{\sphinxupquote{les\_jours3}}, la maniere de le faire est la suivante:

\begin{sphinxVerbatim}[commandchars=\\\{\}]
\PYG{n+nb}{print}\PYG{p}{(}\PYG{n}{les\PYGZus{}jours3}\PYG{p}{[}\PYG{l+s+s1}{\PYGZsq{}}\PYG{l+s+s1}{jeudi}\PYG{l+s+s1}{\PYGZsq{}}\PYG{p}{]}\PYG{p}{)}
\end{sphinxVerbatim}

\begin{sphinxVerbatim}[commandchars=\\\{\}]
4
\end{sphinxVerbatim}

\sphinxAtStartPar
Les dictionnaires ne peuvent pas avoir des cles dupliqee (les valeurs peuvent etre dupliquees). En effet, les valeurs en double écraseront les valeurs existantes. Illustrons ca avec l’exemple suivant:

\begin{sphinxVerbatim}[commandchars=\\\{\}]
\PYG{n}{les\PYGZus{}jours4} \PYG{o}{=} \PYG{p}{\PYGZob{}}\PYG{p}{\PYGZcb{}}

\PYG{n}{les\PYGZus{}jours4}\PYG{p}{[}\PYG{l+s+s1}{\PYGZsq{}}\PYG{l+s+s1}{lundi}\PYG{l+s+s1}{\PYGZsq{}}\PYG{p}{]} \PYG{o}{=} \PYG{l+m+mi}{0}
\PYG{n}{les\PYGZus{}jours4}\PYG{p}{[}\PYG{l+s+s1}{\PYGZsq{}}\PYG{l+s+s1}{mardi}\PYG{l+s+s1}{\PYGZsq{}}\PYG{p}{]} \PYG{o}{=} \PYG{l+m+mi}{1}
\PYG{n}{les\PYGZus{}jours4}\PYG{p}{[}\PYG{l+s+s1}{\PYGZsq{}}\PYG{l+s+s1}{mardi}\PYG{l+s+s1}{\PYGZsq{}}\PYG{p}{]} \PYG{o}{=} \PYG{l+m+mi}{2}
\PYG{n}{les\PYGZus{}jours4}\PYG{p}{[}\PYG{l+s+s1}{\PYGZsq{}}\PYG{l+s+s1}{mercredi}\PYG{l+s+s1}{\PYGZsq{}}\PYG{p}{]} \PYG{o}{=} \PYG{l+m+mi}{3}
\PYG{n}{les\PYGZus{}jours4}\PYG{p}{[}\PYG{l+s+s1}{\PYGZsq{}}\PYG{l+s+s1}{jeudi}\PYG{l+s+s1}{\PYGZsq{}}\PYG{p}{]} \PYG{o}{=} \PYG{l+m+mi}{4}
\PYG{n}{les\PYGZus{}jours4}\PYG{p}{[}\PYG{l+s+s1}{\PYGZsq{}}\PYG{l+s+s1}{vendredi}\PYG{l+s+s1}{\PYGZsq{}}\PYG{p}{]} \PYG{o}{=} \PYG{l+m+mi}{5}
\PYG{n}{les\PYGZus{}jours4}\PYG{p}{[}\PYG{l+s+s1}{\PYGZsq{}}\PYG{l+s+s1}{samedi}\PYG{l+s+s1}{\PYGZsq{}}\PYG{p}{]} \PYG{o}{=} \PYG{l+m+mi}{6}
\PYG{n}{les\PYGZus{}jours4}\PYG{p}{[}\PYG{l+s+s1}{\PYGZsq{}}\PYG{l+s+s1}{dimanche}\PYG{l+s+s1}{\PYGZsq{}}\PYG{p}{]} \PYG{o}{=} \PYG{l+m+mi}{7}

\PYG{n+nb}{print}\PYG{p}{(}\PYG{n}{les\PYGZus{}jours4}\PYG{p}{)}
\end{sphinxVerbatim}

\begin{sphinxVerbatim}[commandchars=\\\{\}]
\PYGZob{}\PYGZsq{}lundi\PYGZsq{}: 0, \PYGZsq{}mardi\PYGZsq{}: 2, \PYGZsq{}mercredi\PYGZsq{}: 3, \PYGZsq{}jeudi\PYGZsq{}: 4, \PYGZsq{}vendredi\PYGZsq{}: 5, \PYGZsq{}samedi\PYGZsq{}: 6, \PYGZsq{}dimanche\PYGZsq{}: 7\PYGZcb{}
\end{sphinxVerbatim}

\begin{sphinxVerbatim}[commandchars=\\\{\}]
\PYG{n}{les\PYGZus{}jours5} \PYG{o}{=} \PYG{p}{\PYGZob{}}\PYG{p}{\PYGZcb{}}

\PYG{n}{les\PYGZus{}jours5}\PYG{p}{[}\PYG{l+s+s1}{\PYGZsq{}}\PYG{l+s+s1}{lundi}\PYG{l+s+s1}{\PYGZsq{}}\PYG{p}{]} \PYG{o}{=} \PYG{l+m+mi}{0}
\PYG{n}{les\PYGZus{}jours5}\PYG{p}{[}\PYG{l+s+s1}{\PYGZsq{}}\PYG{l+s+s1}{mardi}\PYG{l+s+s1}{\PYGZsq{}}\PYG{p}{]} \PYG{o}{=} \PYG{l+m+mi}{0}
\PYG{n}{les\PYGZus{}jours5}\PYG{p}{[}\PYG{l+s+s1}{\PYGZsq{}}\PYG{l+s+s1}{mercredi}\PYG{l+s+s1}{\PYGZsq{}}\PYG{p}{]} \PYG{o}{=} \PYG{l+m+mi}{0}
\PYG{n}{les\PYGZus{}jours5}\PYG{p}{[}\PYG{l+s+s1}{\PYGZsq{}}\PYG{l+s+s1}{jeudi}\PYG{l+s+s1}{\PYGZsq{}}\PYG{p}{]} \PYG{o}{=} \PYG{l+m+mi}{5}
\PYG{n}{les\PYGZus{}jours5}\PYG{p}{[}\PYG{l+s+s1}{\PYGZsq{}}\PYG{l+s+s1}{vendredi}\PYG{l+s+s1}{\PYGZsq{}}\PYG{p}{]} \PYG{o}{=} \PYG{l+m+mi}{0}
\PYG{n}{les\PYGZus{}jours5}\PYG{p}{[}\PYG{l+s+s1}{\PYGZsq{}}\PYG{l+s+s1}{samedi}\PYG{l+s+s1}{\PYGZsq{}}\PYG{p}{]} \PYG{o}{=} \PYG{l+m+mi}{0}
\PYG{n}{les\PYGZus{}jours5}\PYG{p}{[}\PYG{l+s+s1}{\PYGZsq{}}\PYG{l+s+s1}{dimanche}\PYG{l+s+s1}{\PYGZsq{}}\PYG{p}{]} \PYG{o}{=} \PYG{l+m+mi}{0}

\PYG{n+nb}{print}\PYG{p}{(}\PYG{n}{les\PYGZus{}jours5}\PYG{p}{)}
\end{sphinxVerbatim}

\begin{sphinxVerbatim}[commandchars=\\\{\}]
\PYGZob{}\PYGZsq{}lundi\PYGZsq{}: 0, \PYGZsq{}mardi\PYGZsq{}: 0, \PYGZsq{}mercredi\PYGZsq{}: 0, \PYGZsq{}jeudi\PYGZsq{}: 5, \PYGZsq{}vendredi\PYGZsq{}: 0, \PYGZsq{}samedi\PYGZsq{}: 0, \PYGZsq{}dimanche\PYGZsq{}: 0\PYGZcb{}
\end{sphinxVerbatim}


\subsection{Operations sur les dictionnaires}
\label{\detokenize{ch5:operations-sur-les-dictionnaires}}
\sphinxAtStartPar
Python possède un ensemble de méthodes intégrées que nous pouvons utiliser pour manipuler les dictionnaires:


\begin{savenotes}\sphinxattablestart
\centering
\begin{tabulary}{\linewidth}[t]{|T|T|}
\hline
\sphinxstyletheadfamily 
\sphinxAtStartPar
Methode
&\sphinxstyletheadfamily 
\sphinxAtStartPar
Description
\\
\hline
\sphinxAtStartPar
clear()
&
\sphinxAtStartPar
Supprime tous les éléments du dictionnaire
\\
\hline
\sphinxAtStartPar
copy()
&
\sphinxAtStartPar
Renvoie une copie du dictionnaire
\\
\hline
\sphinxAtStartPar
fromkeys()
&
\sphinxAtStartPar
Renvoie un dictionnaire avec les clés et la valeur spécifiées
\\
\hline
\sphinxAtStartPar
get()
&
\sphinxAtStartPar
Renvoie la valeur de la clé spécifiée
\\
\hline
\sphinxAtStartPar
items()
&
\sphinxAtStartPar
Renvoie une liste contenant un tuple pour chaque paire clé\sphinxhyphen{}valeur
\\
\hline
\sphinxAtStartPar
keys()
&
\sphinxAtStartPar
Retourne une liste contenant les clés du dictionnaire
\\
\hline
\sphinxAtStartPar
pop()
&
\sphinxAtStartPar
Supprime l’élément avec la clé spécifiée
\\
\hline
\sphinxAtStartPar
popitem()
&
\sphinxAtStartPar
Supprime la dernière paire clé\sphinxhyphen{}valeur insérée
\\
\hline
\sphinxAtStartPar
setdefault()
&
\sphinxAtStartPar
Renvoie la valeur de la clé spécifiée. Si la clé n’existe pas : insérez la clé, avec la valeur spécifiée
\\
\hline
\sphinxAtStartPar
update()
&
\sphinxAtStartPar
Met à jour le dictionnaire avec les paires clé\sphinxhyphen{}valeur spécifiées
\\
\hline
\sphinxAtStartPar
values()
&
\sphinxAtStartPar
Renvoie une liste de toutes les valeurs du dictionnaire
\\
\hline
\end{tabulary}
\par
\sphinxattableend\end{savenotes}

\begin{sphinxVerbatim}[commandchars=\\\{\}]
\PYG{c+c1}{\PYGZsh{} clear}
\PYG{n}{les\PYGZus{}jours}\PYG{o}{.}\PYG{n}{clear}\PYG{p}{(}\PYG{p}{)}

\PYG{n+nb}{print}\PYG{p}{(}\PYG{n}{les\PYGZus{}jours}\PYG{p}{)}
\end{sphinxVerbatim}

\begin{sphinxVerbatim}[commandchars=\\\{\}]
\PYGZob{}\PYGZcb{}
\end{sphinxVerbatim}

\begin{sphinxVerbatim}[commandchars=\\\{\}]
\PYG{c+c1}{\PYGZsh{} copy}
\PYG{n}{les\PYGZus{}jours2\PYGZus{}copie} \PYG{o}{=} \PYG{n}{les\PYGZus{}jours2}\PYG{o}{.}\PYG{n}{copy}\PYG{p}{(}\PYG{p}{)}

\PYG{n+nb}{print}\PYG{p}{(}\PYG{n}{les\PYGZus{}jours2\PYGZus{}copie}\PYG{p}{)}
\end{sphinxVerbatim}

\begin{sphinxVerbatim}[commandchars=\\\{\}]
\PYGZob{}1: \PYGZsq{}lundi\PYGZsq{}, 2: \PYGZsq{}mardi\PYGZsq{}, 3: \PYGZsq{}mercredi\PYGZsq{}, 4: \PYGZsq{}jeudi\PYGZsq{}, 5: \PYGZsq{}vendredi\PYGZsq{}, 6: \PYGZsq{}samedi\PYGZsq{}, 7: \PYGZsq{}dimanche\PYGZsq{}\PYGZcb{}
\end{sphinxVerbatim}

\begin{sphinxVerbatim}[commandchars=\\\{\}]
\PYG{c+c1}{\PYGZsh{} fromkeys}
\PYG{c+c1}{\PYGZsh{} creer un dictionnaire avec des cles mais pas de valeurs (None)}

\PYG{n}{cles} \PYG{o}{=} \PYG{n+nb}{range}\PYG{p}{(}\PYG{l+m+mi}{6}\PYG{p}{)}
\PYG{n}{dict\PYGZus{}vide} \PYG{o}{=} \PYG{n+nb}{dict}\PYG{o}{.}\PYG{n}{fromkeys}\PYG{p}{(}\PYG{n}{cles}\PYG{p}{)}
\PYG{n+nb}{print}\PYG{p}{(}\PYG{n}{dict\PYGZus{}vide}\PYG{p}{)}
\end{sphinxVerbatim}

\begin{sphinxVerbatim}[commandchars=\\\{\}]
\PYGZob{}0: None, 1: None, 2: None, 3: None, 4: None, 5: None\PYGZcb{}
\end{sphinxVerbatim}

\begin{sphinxVerbatim}[commandchars=\\\{\}]
\PYG{c+c1}{\PYGZsh{} fromkeys}
\PYG{c+c1}{\PYGZsh{} creer un dictionnaire avec des cles a la meme valeur}
\PYG{n}{var} \PYG{o}{=} \PYG{p}{[}\PYG{l+s+s1}{\PYGZsq{}}\PYG{l+s+s1}{a}\PYG{l+s+s1}{\PYGZsq{}}\PYG{p}{,} \PYG{l+s+s1}{\PYGZsq{}}\PYG{l+s+s1}{e}\PYG{l+s+s1}{\PYGZsq{}}\PYG{p}{,} \PYG{l+s+s1}{\PYGZsq{}}\PYG{l+s+s1}{i}\PYG{l+s+s1}{\PYGZsq{}}\PYG{p}{,} \PYG{l+s+s1}{\PYGZsq{}}\PYG{l+s+s1}{o}\PYG{l+s+s1}{\PYGZsq{}}\PYG{p}{,} \PYG{l+s+s1}{\PYGZsq{}}\PYG{l+s+s1}{u}\PYG{l+s+s1}{\PYGZsq{}}\PYG{p}{,} \PYG{l+s+s1}{\PYGZsq{}}\PYG{l+s+s1}{y}\PYG{l+s+s1}{\PYGZsq{}}\PYG{p}{]}
\PYG{n}{valeur} \PYG{o}{=} \PYG{l+s+s1}{\PYGZsq{}}\PYG{l+s+s1}{voyelle}\PYG{l+s+s1}{\PYGZsq{}}
\PYG{n}{dict\PYGZus{}voyelles} \PYG{o}{=} \PYG{n+nb}{dict}\PYG{o}{.}\PYG{n}{fromkeys}\PYG{p}{(}\PYG{n}{var}\PYG{p}{,} \PYG{n}{valeur}\PYG{p}{)}

\PYG{n+nb}{print}\PYG{p}{(}\PYG{n}{dict\PYGZus{}voyelles}\PYG{p}{)}
\end{sphinxVerbatim}

\begin{sphinxVerbatim}[commandchars=\\\{\}]
\PYGZob{}\PYGZsq{}a\PYGZsq{}: \PYGZsq{}voyelle\PYGZsq{}, \PYGZsq{}e\PYGZsq{}: \PYGZsq{}voyelle\PYGZsq{}, \PYGZsq{}i\PYGZsq{}: \PYGZsq{}voyelle\PYGZsq{}, \PYGZsq{}o\PYGZsq{}: \PYGZsq{}voyelle\PYGZsq{}, \PYGZsq{}u\PYGZsq{}: \PYGZsq{}voyelle\PYGZsq{}, \PYGZsq{}y\PYGZsq{}: \PYGZsq{}voyelle\PYGZsq{}\PYGZcb{}
\end{sphinxVerbatim}

\begin{sphinxVerbatim}[commandchars=\\\{\}]
\PYG{c+c1}{\PYGZsh{} get}
\PYG{n+nb}{print}\PYG{p}{(}\PYG{n}{les\PYGZus{}jours2}\PYG{o}{.}\PYG{n}{get}\PYG{p}{(}\PYG{l+m+mi}{3}\PYG{p}{)}\PYG{p}{)}
\end{sphinxVerbatim}

\begin{sphinxVerbatim}[commandchars=\\\{\}]
mercredi
\end{sphinxVerbatim}

\begin{sphinxVerbatim}[commandchars=\\\{\}]
\PYG{c+c1}{\PYGZsh{} items}
\PYG{n+nb}{print}\PYG{p}{(}\PYG{n}{les\PYGZus{}jours2}\PYG{o}{.}\PYG{n}{items}\PYG{p}{(}\PYG{p}{)}\PYG{p}{)}
\end{sphinxVerbatim}

\begin{sphinxVerbatim}[commandchars=\\\{\}]
dict\PYGZus{}items([(1, \PYGZsq{}lundi\PYGZsq{}), (2, \PYGZsq{}mardi\PYGZsq{}), (3, \PYGZsq{}mercredi\PYGZsq{}), (4, \PYGZsq{}jeudi\PYGZsq{}), (5, \PYGZsq{}vendredi\PYGZsq{}), (6, \PYGZsq{}samedi\PYGZsq{}), (7, \PYGZsq{}dimanche\PYGZsq{})])
\end{sphinxVerbatim}

\begin{sphinxVerbatim}[commandchars=\\\{\}]
\PYG{c+c1}{\PYGZsh{} keys}
\PYG{n+nb}{print}\PYG{p}{(}\PYG{n}{les\PYGZus{}jours2}\PYG{o}{.}\PYG{n}{keys}\PYG{p}{(}\PYG{p}{)}\PYG{p}{)}
\end{sphinxVerbatim}

\begin{sphinxVerbatim}[commandchars=\\\{\}]
dict\PYGZus{}keys([1, 2, 3, 4, 5, 6, 7])
\end{sphinxVerbatim}

\begin{sphinxVerbatim}[commandchars=\\\{\}]
\PYG{c+c1}{\PYGZsh{} values}
\PYG{n+nb}{print}\PYG{p}{(}\PYG{n}{les\PYGZus{}jours2}\PYG{o}{.}\PYG{n}{values}\PYG{p}{(}\PYG{p}{)}\PYG{p}{)}
\end{sphinxVerbatim}

\begin{sphinxVerbatim}[commandchars=\\\{\}]
dict\PYGZus{}values([\PYGZsq{}lundi\PYGZsq{}, \PYGZsq{}mardi\PYGZsq{}, \PYGZsq{}mercredi\PYGZsq{}, \PYGZsq{}jeudi\PYGZsq{}, \PYGZsq{}vendredi\PYGZsq{}, \PYGZsq{}samedi\PYGZsq{}, \PYGZsq{}dimanche\PYGZsq{}])
\end{sphinxVerbatim}

\begin{sphinxVerbatim}[commandchars=\\\{\}]
\PYG{c+c1}{\PYGZsh{} pop}
\PYG{n}{les\PYGZus{}jours2}\PYG{o}{.}\PYG{n}{pop}\PYG{p}{(}\PYG{l+m+mi}{4}\PYG{p}{)}

\PYG{n+nb}{print}\PYG{p}{(}\PYG{n}{les\PYGZus{}jours2}\PYG{p}{)}
\end{sphinxVerbatim}

\begin{sphinxVerbatim}[commandchars=\\\{\}]
\PYGZob{}1: \PYGZsq{}lundi\PYGZsq{}, 2: \PYGZsq{}mardi\PYGZsq{}, 3: \PYGZsq{}mercredi\PYGZsq{}, 5: \PYGZsq{}vendredi\PYGZsq{}, 6: \PYGZsq{}samedi\PYGZsq{}, 7: \PYGZsq{}dimanche\PYGZsq{}\PYGZcb{}
\end{sphinxVerbatim}

\begin{sphinxVerbatim}[commandchars=\\\{\}]
\PYG{c+c1}{\PYGZsh{} popitem }
\PYG{c+c1}{\PYGZsh{} inserer une cles\PYGZhy{}valeur arbitraire}
\PYG{n}{les\PYGZus{}jours2}\PYG{p}{[}\PYG{l+s+s1}{\PYGZsq{}}\PYG{l+s+s1}{la dirniere cles ajoutee}\PYG{l+s+s1}{\PYGZsq{}}\PYG{p}{]} \PYG{o}{=} \PYG{l+s+s1}{\PYGZsq{}}\PYG{l+s+s1}{La valeur associee a la derniere cles}\PYG{l+s+s1}{\PYGZsq{}}

\PYG{n+nb}{print}\PYG{p}{(}\PYG{n}{les\PYGZus{}jours2}\PYG{p}{)}
\end{sphinxVerbatim}

\begin{sphinxVerbatim}[commandchars=\\\{\}]
\PYGZob{}1: \PYGZsq{}lundi\PYGZsq{}, 2: \PYGZsq{}mardi\PYGZsq{}, 3: \PYGZsq{}mercredi\PYGZsq{}, 5: \PYGZsq{}vendredi\PYGZsq{}, 6: \PYGZsq{}samedi\PYGZsq{}, 7: \PYGZsq{}dimanche\PYGZsq{}, \PYGZsq{}la dirniere cles ajoutee\PYGZsq{}: \PYGZsq{}La valeur associee a la derniere cles\PYGZsq{}\PYGZcb{}
\end{sphinxVerbatim}

\begin{sphinxVerbatim}[commandchars=\\\{\}]
\PYG{c+c1}{\PYGZsh{} popitem}
\PYG{n}{les\PYGZus{}jours2}\PYG{o}{.}\PYG{n}{popitem}\PYG{p}{(}\PYG{p}{)}

\PYG{n+nb}{print}\PYG{p}{(}\PYG{n}{les\PYGZus{}jours2}\PYG{p}{)}
\end{sphinxVerbatim}

\begin{sphinxVerbatim}[commandchars=\\\{\}]
\PYGZob{}1: \PYGZsq{}lundi\PYGZsq{}, 2: \PYGZsq{}mardi\PYGZsq{}, 3: \PYGZsq{}mercredi\PYGZsq{}, 5: \PYGZsq{}vendredi\PYGZsq{}, 6: \PYGZsq{}samedi\PYGZsq{}, 7: \PYGZsq{}dimanche\PYGZsq{}\PYGZcb{}
\end{sphinxVerbatim}

\begin{sphinxVerbatim}[commandchars=\\\{\}]
\PYG{c+c1}{\PYGZsh{} setdefault()}
\PYG{c+c1}{\PYGZsh{} si la cles existe}
\PYG{n+nb}{print}\PYG{p}{(}\PYG{n}{les\PYGZus{}jours2}\PYG{o}{.}\PYG{n}{setdefault}\PYG{p}{(}\PYG{l+m+mi}{1}\PYG{p}{,} \PYG{l+s+s2}{\PYGZdq{}}\PYG{l+s+s2}{LUNDI}\PYG{l+s+s2}{\PYGZdq{}}\PYG{p}{)}\PYG{p}{)}
\PYG{c+c1}{\PYGZsh{} pusique la cles existe, la valeur ne va pas changer. Elle va etre affichee. Le dictionnaire ne va pas changer aussi.}
\PYG{n+nb}{print}\PYG{p}{(}\PYG{n}{les\PYGZus{}jours2}\PYG{p}{)}
\end{sphinxVerbatim}

\begin{sphinxVerbatim}[commandchars=\\\{\}]
lundi
\PYGZob{}1: \PYGZsq{}lundi\PYGZsq{}, 2: \PYGZsq{}mardi\PYGZsq{}, 3: \PYGZsq{}mercredi\PYGZsq{}, 5: \PYGZsq{}vendredi\PYGZsq{}, 6: \PYGZsq{}samedi\PYGZsq{}, 7: \PYGZsq{}dimanche\PYGZsq{}\PYGZcb{}
\end{sphinxVerbatim}

\begin{sphinxVerbatim}[commandchars=\\\{\}]
\PYG{c+c1}{\PYGZsh{} setdefault()}
\PYG{c+c1}{\PYGZsh{} si la cles existe}
\PYG{c+c1}{\PYGZsh{} puisque la cles 4 n\PYGZsq{}existe pas. La valeur \PYGZsq{}mercredi\PYGZsq{} va etre affichee. Le dictionnaire va etre modifie aussi}
\PYG{n+nb}{print}\PYG{p}{(}\PYG{n}{les\PYGZus{}jours2}\PYG{o}{.}\PYG{n}{setdefault}\PYG{p}{(}\PYG{l+m+mi}{4}\PYG{p}{,} \PYG{l+s+s2}{\PYGZdq{}}\PYG{l+s+s2}{mercredi}\PYG{l+s+s2}{\PYGZdq{}}\PYG{p}{)}\PYG{p}{)}
\PYG{n+nb}{print}\PYG{p}{(}\PYG{n}{les\PYGZus{}jours2}\PYG{p}{)}
\end{sphinxVerbatim}

\begin{sphinxVerbatim}[commandchars=\\\{\}]
mercredi
\PYGZob{}1: \PYGZsq{}lundi\PYGZsq{}, 2: \PYGZsq{}mardi\PYGZsq{}, 3: \PYGZsq{}mercredi\PYGZsq{}, 5: \PYGZsq{}vendredi\PYGZsq{}, 6: \PYGZsq{}samedi\PYGZsq{}, 7: \PYGZsq{}dimanche\PYGZsq{}, 4: \PYGZsq{}mercredi\PYGZsq{}\PYGZcb{}
\end{sphinxVerbatim}

\begin{sphinxVerbatim}[commandchars=\\\{\}]
\PYG{c+c1}{\PYGZsh{} update}
\PYG{n}{premier}
\end{sphinxVerbatim}

\sphinxAtStartPar
La fonction intégrée \sphinxcode{\sphinxupquote{len()}} est aussi appliquable pour les dictionnaires. Si l’on désire déterminer le nombre d’elements:

\begin{sphinxVerbatim}[commandchars=\\\{\}]
\PYG{n+nb}{len}\PYG{p}{(}\PYG{n}{les\PYGZus{}jours}\PYG{p}{)}
\end{sphinxVerbatim}

\begin{sphinxVerbatim}[commandchars=\\\{\}]
7
\end{sphinxVerbatim}

\sphinxAtStartPar
La varaible \sphinxcode{\sphinxupquote{les\_jours}} contient 7 elements.

\sphinxAtStartPar
Si l’on veut tester l’appartenece d’un element a une liste/tuple on utilise l’operateur \sphinxcode{\sphinxupquote{in}}. L’expression est la suivante : \sphinxcode{\sphinxupquote{element in liste}}. Cela nous renvoi \sphinxcode{\sphinxupquote{True}} si \sphinxcode{\sphinxupquote{element}}est dans \sphinxcode{\sphinxupquote{liste}}, sinon \sphinxcode{\sphinxupquote{False}}. On peut aussi ecrire \sphinxcode{\sphinxupquote{element not in liste}} pour tester si l’element n’est pas dans \sphinxcode{\sphinxupquote{liste}} (\sphinxcode{\sphinxupquote{True}}) ou s’il est dans \sphinxcode{\sphinxupquote{liste}} (\sphinxcode{\sphinxupquote{False}}). Voici quelques exemples:

\begin{sphinxVerbatim}[commandchars=\\\{\}]
\PYG{n}{les\PYGZus{}jours}
\end{sphinxVerbatim}

\begin{sphinxVerbatim}[commandchars=\\\{\}]
\PYGZob{}1: \PYGZsq{}lundi\PYGZsq{},
 2: \PYGZsq{}mardi\PYGZsq{},
 3: \PYGZsq{}mercredi\PYGZsq{},
 4: \PYGZsq{}jeudi\PYGZsq{},
 5: \PYGZsq{}vendredi\PYGZsq{},
 6: \PYGZsq{}samedi\PYGZsq{},
 7: \PYGZsq{}dimanche\PYGZsq{}\PYGZcb{}
\end{sphinxVerbatim}

\begin{sphinxVerbatim}[commandchars=\\\{\}]
\PYG{n}{les\PYGZus{}jours}\PYG{o}{.}\PYG{n}{values}\PYG{p}{(}\PYG{p}{)}
\end{sphinxVerbatim}

\begin{sphinxVerbatim}[commandchars=\\\{\}]
dict\PYGZus{}values([\PYGZsq{}lundi\PYGZsq{}, \PYGZsq{}mardi\PYGZsq{}, \PYGZsq{}mercredi\PYGZsq{}, \PYGZsq{}jeudi\PYGZsq{}, \PYGZsq{}vendredi\PYGZsq{}, \PYGZsq{}samedi\PYGZsq{}, \PYGZsq{}dimanche\PYGZsq{}])
\end{sphinxVerbatim}

\begin{sphinxVerbatim}[commandchars=\\\{\}]
\PYG{n}{un\PYGZus{}tuple} \PYG{o}{=} \PYG{p}{(}\PYG{l+m+mi}{1}\PYG{p}{,} \PYG{l+s+s1}{\PYGZsq{}}\PYG{l+s+s1}{bonjour}\PYG{l+s+s1}{\PYGZsq{}}\PYG{p}{,} \PYG{l+m+mf}{4.5}\PYG{p}{,} \PYG{k+kc}{True}\PYG{p}{)}

\PYG{n+nb}{print}\PYG{p}{(}\PYG{l+m+mi}{1} \PYG{o+ow}{in} \PYG{n}{un\PYGZus{}tuple}\PYG{p}{)}

\PYG{n+nb}{print}\PYG{p}{(}\PYG{l+m+mi}{2} \PYG{o+ow}{in} \PYG{n}{un\PYGZus{}tuple}\PYG{p}{)}

\PYG{n+nb}{print}\PYG{p}{(}\PYG{l+s+s1}{\PYGZsq{}}\PYG{l+s+s1}{o}\PYG{l+s+s1}{\PYGZsq{}} \PYG{o+ow}{not} \PYG{o+ow}{in} \PYG{n}{un\PYGZus{}tuple}\PYG{p}{)}

\PYG{n+nb}{print}\PYG{p}{(}\PYG{l+s+s1}{\PYGZsq{}}\PYG{l+s+s1}{x}\PYG{l+s+s1}{\PYGZsq{}} \PYG{o+ow}{not} \PYG{o+ow}{in} \PYG{n}{un\PYGZus{}tuple}\PYG{p}{)}
\end{sphinxVerbatim}

\begin{sphinxVerbatim}[commandchars=\\\{\}]
True
False
True
True
\end{sphinxVerbatim}


\subsection{Quelle est la difference entre listes et tuples?}
\label{\detokenize{ch5:quelle-est-la-difference-entre-listes-et-tuples}}
\sphinxAtStartPar
Les listes et les tuples sont pareils dans la plupart des contextes. Cepandant, la difference primordiale entre les deux et que les listes sont des \sphinxstylestrong{objets mutables} (modifiables) alors que les tuples sont des \sphinxstylestrong{objets immuables} (ne sont pas modifiable). La question qui se pose est donc: qu’est\sphinxhyphen{}ce qu’un objet mutable et un objet immuable?
Parmi les objet immuable en python, on trouve:
\begin{itemize}
\item {} 
\sphinxAtStartPar
Les nombres entiers (int)

\item {} 
\sphinxAtStartPar
Les nombres décimaux (float)

\item {} 
\sphinxAtStartPar
Les chaînes de caractères (str)

\item {} 
\sphinxAtStartPar
Les booléens (bool)

\item {} 
\sphinxAtStartPar
Les tuples (tuple)
La plus part des autres objets que vous allez confronter en python sont mutables.

\end{itemize}

\sphinxAtStartPar
Nous allons illustre ca dans les exemples suivant:
\begin{enumerate}
\sphinxsetlistlabels{\arabic}{enumi}{enumii}{}{.}%
\item {} 
\sphinxAtStartPar
nous allons creer les variables suivantes:

\end{enumerate}
\begin{itemize}
\item {} 
\sphinxAtStartPar
\sphinxcode{\sphinxupquote{var\_chaine = "bonjour tout le monde"}},

\item {} 
\sphinxAtStartPar
\sphinxcode{\sphinxupquote{var\_liste = {[}1, 2, True, \textquotesingle{}bonjour\textquotesingle{}{]}}},

\item {} 
\sphinxAtStartPar
\sphinxcode{\sphinxupquote{var\_tuple = (1, 2, True, \textquotesingle{}bonjour\textquotesingle{})}}.

\end{itemize}
\begin{enumerate}
\sphinxsetlistlabels{\arabic}{enumi}{enumii}{}{.}%
\item {} 
\sphinxAtStartPar
nous qllons essayer de changer (par exemple) le premier element de chaque variable (par un autre element).

\end{enumerate}

\begin{sphinxVerbatim}[commandchars=\\\{\}]
\PYG{n}{var\PYGZus{}chaine} \PYG{o}{=} \PYG{l+s+s2}{\PYGZdq{}}\PYG{l+s+s2}{bonjour tout le monde}\PYG{l+s+s2}{\PYGZdq{}}
\PYG{n}{var\PYGZus{}liste} \PYG{o}{=} \PYG{p}{[}\PYG{l+m+mi}{1}\PYG{p}{,} \PYG{l+m+mi}{2}\PYG{p}{,} \PYG{k+kc}{True}\PYG{p}{,} \PYG{l+s+s1}{\PYGZsq{}}\PYG{l+s+s1}{bonjour}\PYG{l+s+s1}{\PYGZsq{}}\PYG{p}{]}
\PYG{n}{var\PYGZus{}tuple} \PYG{o}{=} \PYG{p}{(}\PYG{l+m+mi}{1}\PYG{p}{,} \PYG{l+m+mi}{2}\PYG{p}{,} \PYG{k+kc}{True}\PYG{p}{,} \PYG{l+s+s1}{\PYGZsq{}}\PYG{l+s+s1}{bonjour}\PYG{l+s+s1}{\PYGZsq{}}\PYG{p}{)}
\end{sphinxVerbatim}

\begin{sphinxVerbatim}[commandchars=\\\{\}]
\PYG{n}{var\PYGZus{}chaine}\PYG{p}{[}\PYG{l+m+mi}{0}\PYG{p}{]} \PYG{o}{=} \PYG{l+s+s1}{\PYGZsq{}}\PYG{l+s+s1}{B}\PYG{l+s+s1}{\PYGZsq{}}
\PYG{n+nb}{print}\PYG{p}{(}\PYG{n}{var\PYGZus{}chaine}\PYG{p}{)}
\end{sphinxVerbatim}

\begin{sphinxVerbatim}[commandchars=\\\{\}]
\PYG{g+gt}{\PYGZhy{}\PYGZhy{}\PYGZhy{}\PYGZhy{}\PYGZhy{}\PYGZhy{}\PYGZhy{}\PYGZhy{}\PYGZhy{}\PYGZhy{}\PYGZhy{}\PYGZhy{}\PYGZhy{}\PYGZhy{}\PYGZhy{}\PYGZhy{}\PYGZhy{}\PYGZhy{}\PYGZhy{}\PYGZhy{}\PYGZhy{}\PYGZhy{}\PYGZhy{}\PYGZhy{}\PYGZhy{}\PYGZhy{}\PYGZhy{}\PYGZhy{}\PYGZhy{}\PYGZhy{}\PYGZhy{}\PYGZhy{}\PYGZhy{}\PYGZhy{}\PYGZhy{}\PYGZhy{}\PYGZhy{}\PYGZhy{}\PYGZhy{}\PYGZhy{}\PYGZhy{}\PYGZhy{}\PYGZhy{}\PYGZhy{}\PYGZhy{}\PYGZhy{}\PYGZhy{}\PYGZhy{}\PYGZhy{}\PYGZhy{}\PYGZhy{}\PYGZhy{}\PYGZhy{}\PYGZhy{}\PYGZhy{}\PYGZhy{}\PYGZhy{}\PYGZhy{}\PYGZhy{}\PYGZhy{}\PYGZhy{}\PYGZhy{}\PYGZhy{}\PYGZhy{}\PYGZhy{}\PYGZhy{}\PYGZhy{}\PYGZhy{}\PYGZhy{}\PYGZhy{}\PYGZhy{}\PYGZhy{}\PYGZhy{}\PYGZhy{}\PYGZhy{}}
\PYG{n+ne}{TypeError}\PYG{g+gWhitespace}{                                 }Traceback (most recent call last)
\PYG{o}{\PYGZti{}}\PYGZbs{}\PYG{n}{AppData}\PYGZbs{}\PYG{n}{Local}\PYGZbs{}\PYG{n}{Temp}\PYGZbs{}\PYG{n}{ipykernel\PYGZus{}232}\PYGZbs{}\PYG{l+m+mf}{4274672008.}\PYG{n}{py} \PYG{o+ow}{in} \PYG{o}{\PYGZlt{}}\PYG{n}{module}\PYG{o}{\PYGZgt{}}
\PYG{n+ne}{\PYGZhy{}\PYGZhy{}\PYGZhy{}\PYGZhy{}\PYGZgt{} }\PYG{l+m+mi}{1} \PYG{n}{var\PYGZus{}chaine}\PYG{p}{[}\PYG{l+m+mi}{0}\PYG{p}{]} \PYG{o}{=} \PYG{l+s+s1}{\PYGZsq{}}\PYG{l+s+s1}{B}\PYG{l+s+s1}{\PYGZsq{}}
\PYG{g+gWhitespace}{      }\PYG{l+m+mi}{2} \PYG{n+nb}{print}\PYG{p}{(}\PYG{n}{var\PYGZus{}chaine}\PYG{p}{)}

\PYG{n+ne}{TypeError}: \PYGZsq{}str\PYGZsq{} object does not support item assignment
\end{sphinxVerbatim}

\begin{sphinxVerbatim}[commandchars=\\\{\}]
\PYG{n}{var\PYGZus{}liste}\PYG{p}{[}\PYG{l+m+mi}{0}\PYG{p}{]} \PYG{o}{=} \PYG{l+m+mi}{3333}
\PYG{n+nb}{print}\PYG{p}{(}\PYG{n}{var\PYGZus{}liste}\PYG{p}{)}
\end{sphinxVerbatim}

\begin{sphinxVerbatim}[commandchars=\\\{\}]
[3333, 2, True, \PYGZsq{}bonjour\PYGZsq{}]
\end{sphinxVerbatim}

\begin{sphinxVerbatim}[commandchars=\\\{\}]
\PYG{n}{var\PYGZus{}tuple}\PYG{p}{[}\PYG{l+m+mi}{0}\PYG{p}{]} \PYG{o}{=} \PYG{l+m+mi}{3333}
\PYG{n+nb}{print}\PYG{p}{(}\PYG{n}{var\PYGZus{}tuple}\PYG{p}{)}
\end{sphinxVerbatim}

\begin{sphinxVerbatim}[commandchars=\\\{\}]
\PYG{g+gt}{\PYGZhy{}\PYGZhy{}\PYGZhy{}\PYGZhy{}\PYGZhy{}\PYGZhy{}\PYGZhy{}\PYGZhy{}\PYGZhy{}\PYGZhy{}\PYGZhy{}\PYGZhy{}\PYGZhy{}\PYGZhy{}\PYGZhy{}\PYGZhy{}\PYGZhy{}\PYGZhy{}\PYGZhy{}\PYGZhy{}\PYGZhy{}\PYGZhy{}\PYGZhy{}\PYGZhy{}\PYGZhy{}\PYGZhy{}\PYGZhy{}\PYGZhy{}\PYGZhy{}\PYGZhy{}\PYGZhy{}\PYGZhy{}\PYGZhy{}\PYGZhy{}\PYGZhy{}\PYGZhy{}\PYGZhy{}\PYGZhy{}\PYGZhy{}\PYGZhy{}\PYGZhy{}\PYGZhy{}\PYGZhy{}\PYGZhy{}\PYGZhy{}\PYGZhy{}\PYGZhy{}\PYGZhy{}\PYGZhy{}\PYGZhy{}\PYGZhy{}\PYGZhy{}\PYGZhy{}\PYGZhy{}\PYGZhy{}\PYGZhy{}\PYGZhy{}\PYGZhy{}\PYGZhy{}\PYGZhy{}\PYGZhy{}\PYGZhy{}\PYGZhy{}\PYGZhy{}\PYGZhy{}\PYGZhy{}\PYGZhy{}\PYGZhy{}\PYGZhy{}\PYGZhy{}\PYGZhy{}\PYGZhy{}\PYGZhy{}\PYGZhy{}\PYGZhy{}}
\PYG{n+ne}{TypeError}\PYG{g+gWhitespace}{                                 }Traceback (most recent call last)
\PYG{o}{\PYGZti{}}\PYGZbs{}\PYG{n}{AppData}\PYGZbs{}\PYG{n}{Local}\PYGZbs{}\PYG{n}{Temp}\PYGZbs{}\PYG{n}{ipykernel\PYGZus{}232}\PYGZbs{}\PYG{l+m+mf}{613115800.}\PYG{n}{py} \PYG{o+ow}{in} \PYG{o}{\PYGZlt{}}\PYG{n}{module}\PYG{o}{\PYGZgt{}}
\PYG{n+ne}{\PYGZhy{}\PYGZhy{}\PYGZhy{}\PYGZhy{}\PYGZgt{} }\PYG{l+m+mi}{1} \PYG{n}{var\PYGZus{}tuple}\PYG{p}{[}\PYG{l+m+mi}{0}\PYG{p}{]} \PYG{o}{=} \PYG{l+m+mi}{3333}
\PYG{g+gWhitespace}{      }\PYG{l+m+mi}{2} \PYG{n+nb}{print}\PYG{p}{(}\PYG{n}{var\PYGZus{}tuple}\PYG{p}{)}

\PYG{n+ne}{TypeError}: \PYGZsq{}tuple\PYGZsq{} object does not support item assignment
\end{sphinxVerbatim}

\sphinxAtStartPar
Les listes ont une taille variable, les tuples et les chaines de caracteres ont une taille fixe. Enfin, les listes ont plus de fonctionnalités que les tuples. Cependant, c’est le contexte qui nous force a utiliser les listes ou les tuples. Nous allons rencontrer plusieurs contextes ou on est amener a choisir l’un des deux types.


\subsection{quelques methodes utiles pour les listes et les tuples}
\label{\detokenize{ch5:quelques-methodes-utiles-pour-les-listes-et-les-tuples}}
\sphinxAtStartPar
Dans cette sections nous allons voir qulques \sphinxcode{\sphinxupquote{methodes}}  pour les liste et/ou les tuples (les methodes communes et les methodes propres aux listes seulement). Les deux methodes suivantes sont communes aux listes et au tuples:
\begin{itemize}
\item {} 
\sphinxAtStartPar
count: cette methode est utlisee pour compter le nombre d’elements de la liste/tuple.

\item {} 
\sphinxAtStartPar
index: cette method est utilisee pour chercher une valeur spécifiée dans la liste/tuple et renvoie la position de l’endroit où il a été trouvé.

\end{itemize}

\begin{sphinxVerbatim}[commandchars=\\\{\}]
\PYG{n}{var\PYGZus{}liste} \PYG{o}{=} \PYG{p}{[}\PYG{l+m+mi}{1}\PYG{p}{,} \PYG{l+m+mi}{2}\PYG{p}{,} \PYG{l+m+mi}{2}\PYG{p}{,} \PYG{k+kc}{True}\PYG{p}{,} \PYG{l+s+s1}{\PYGZsq{}}\PYG{l+s+s1}{bonjour}\PYG{l+s+s1}{\PYGZsq{}}\PYG{p}{,} \PYG{l+m+mi}{2}\PYG{p}{]}
\PYG{n}{var\PYGZus{}tuple} \PYG{o}{=} \PYG{p}{(}\PYG{l+m+mi}{1}\PYG{p}{,} \PYG{l+m+mi}{2}\PYG{p}{,} \PYG{l+m+mi}{2}\PYG{p}{,} \PYG{k+kc}{True}\PYG{p}{,} \PYG{l+s+s1}{\PYGZsq{}}\PYG{l+s+s1}{bonjour}\PYG{l+s+s1}{\PYGZsq{}}\PYG{p}{,} \PYG{l+m+mi}{2}\PYG{p}{)}

\PYG{n+nb}{print}\PYG{p}{(}\PYG{n}{var\PYGZus{}liste}\PYG{o}{.}\PYG{n}{count}\PYG{p}{(}\PYG{l+m+mi}{2}\PYG{p}{)}\PYG{p}{)} \PYG{c+c1}{\PYGZsh{}\PYGZsh{} print(var\PYGZus{}tuple.count(2))}

\PYG{n+nb}{print}\PYG{p}{(}\PYG{n}{var\PYGZus{}tuple}\PYG{o}{.}\PYG{n}{index}\PYG{p}{(}\PYG{l+m+mi}{2}\PYG{p}{)}\PYG{p}{)}  \PYG{c+c1}{\PYGZsh{}\PYGZsh{} print(var\PYGZus{}tuple.index(2))}
\end{sphinxVerbatim}

\begin{sphinxVerbatim}[commandchars=\\\{\}]
3
\end{sphinxVerbatim}

\begin{sphinxVerbatim}[commandchars=\\\{\}]
\PYG{n}{var\PYGZus{}tuple}\PYG{o}{.}\PYG{n}{index}\PYG{p}{(}\PYG{l+m+mi}{44}\PYG{p}{)} \PYG{c+c1}{\PYGZsh{} var\PYGZus{}tuple.index(44)}
\end{sphinxVerbatim}

\begin{sphinxVerbatim}[commandchars=\\\{\}]
\PYG{g+gt}{\PYGZhy{}\PYGZhy{}\PYGZhy{}\PYGZhy{}\PYGZhy{}\PYGZhy{}\PYGZhy{}\PYGZhy{}\PYGZhy{}\PYGZhy{}\PYGZhy{}\PYGZhy{}\PYGZhy{}\PYGZhy{}\PYGZhy{}\PYGZhy{}\PYGZhy{}\PYGZhy{}\PYGZhy{}\PYGZhy{}\PYGZhy{}\PYGZhy{}\PYGZhy{}\PYGZhy{}\PYGZhy{}\PYGZhy{}\PYGZhy{}\PYGZhy{}\PYGZhy{}\PYGZhy{}\PYGZhy{}\PYGZhy{}\PYGZhy{}\PYGZhy{}\PYGZhy{}\PYGZhy{}\PYGZhy{}\PYGZhy{}\PYGZhy{}\PYGZhy{}\PYGZhy{}\PYGZhy{}\PYGZhy{}\PYGZhy{}\PYGZhy{}\PYGZhy{}\PYGZhy{}\PYGZhy{}\PYGZhy{}\PYGZhy{}\PYGZhy{}\PYGZhy{}\PYGZhy{}\PYGZhy{}\PYGZhy{}\PYGZhy{}\PYGZhy{}\PYGZhy{}\PYGZhy{}\PYGZhy{}\PYGZhy{}\PYGZhy{}\PYGZhy{}\PYGZhy{}\PYGZhy{}\PYGZhy{}\PYGZhy{}\PYGZhy{}\PYGZhy{}\PYGZhy{}\PYGZhy{}\PYGZhy{}\PYGZhy{}\PYGZhy{}\PYGZhy{}}
\PYG{n+ne}{ValueError}\PYG{g+gWhitespace}{                                }Traceback (most recent call last)
\PYG{o}{\PYGZti{}}\PYGZbs{}\PYG{n}{AppData}\PYGZbs{}\PYG{n}{Local}\PYGZbs{}\PYG{n}{Temp}\PYGZbs{}\PYG{n}{ipykernel\PYGZus{}232}\PYGZbs{}\PYG{l+m+mf}{1993419995.}\PYG{n}{py} \PYG{o+ow}{in} \PYG{o}{\PYGZlt{}}\PYG{n}{module}\PYG{o}{\PYGZgt{}}
\PYG{n+ne}{\PYGZhy{}\PYGZhy{}\PYGZhy{}\PYGZhy{}\PYGZgt{} }\PYG{l+m+mi}{1} \PYG{n}{var\PYGZus{}tuple}\PYG{o}{.}\PYG{n}{index}\PYG{p}{(}\PYG{l+m+mi}{44}\PYG{p}{)} \PYG{c+c1}{\PYGZsh{} var\PYGZus{}tuple.index(44)}

\PYG{n+ne}{ValueError}: tuple.index(x): x not in tuple
\end{sphinxVerbatim}

\sphinxAtStartPar
Les methodes decrites dans la table suivante sont appliquee au lites seulement:


\begin{savenotes}\sphinxattablestart
\centering
\begin{tabulary}{\linewidth}[t]{|T|T|}
\hline
\sphinxstyletheadfamily 
\sphinxAtStartPar
Méthode
&\sphinxstyletheadfamily 
\sphinxAtStartPar
Description
\\
\hline
\sphinxAtStartPar
append()
&
\sphinxAtStartPar
Ajoute un élément en fin de liste
\\
\hline
\sphinxAtStartPar
clear()
&
\sphinxAtStartPar
Supprime tous les éléments de la liste
\\
\hline
\sphinxAtStartPar
copy()
&
\sphinxAtStartPar
Renvoie une copie de la liste
\\
\hline
\sphinxAtStartPar
count()
&
\sphinxAtStartPar
Renvoie le nombre d’éléments avec la valeur spécifiée
\\
\hline
\sphinxAtStartPar
extend()
&
\sphinxAtStartPar
Ajouter les éléments d’une liste, à la fin de la liste actuelle
\\
\hline
\sphinxAtStartPar
index()
&
\sphinxAtStartPar
Renvoie l’indice du premier élément avec la valeur spécifiée
\\
\hline
\sphinxAtStartPar
insert()
&
\sphinxAtStartPar
Ajoute un élément à la position spécifiée
\\
\hline
\sphinxAtStartPar
pop()
&
\sphinxAtStartPar
Supprime l’élément à la position spécifiée
\\
\hline
\sphinxAtStartPar
remove()
&
\sphinxAtStartPar
Supprime le premier élément avec la valeur spécifiée
\\
\hline
\sphinxAtStartPar
reverse()
&
\sphinxAtStartPar
Inverse l’ordre de la liste
\\
\hline
\sphinxAtStartPar
sort()
&
\sphinxAtStartPar
Trie la liste
\\
\hline
\end{tabulary}
\par
\sphinxattableend\end{savenotes}


\subsection{sets}
\label{\detokenize{ch5:sets}}
\sphinxAtStartPar
On a vu que les chaines de caracteres, les liste et tuples sont des sequences ordonnees d’elements


\section{Essayez vous\sphinxhyphen{}meme!}
\label{\detokenize{exo3:essayez-vous-meme}}\label{\detokenize{exo3::doc}}



\subsection{Exercice 1.}
\label{\detokenize{exo3:exercice-1}}
\sphinxAtStartPar
Ordre des operations: quelle est selon vous le resulats de ces operations :
\begin{itemize}
\item {} 
\sphinxAtStartPar
\((4/2)^2\times 2 + 1\), \(4/2^{(2\times 2)} + 1\),

\item {} 
\sphinxAtStartPar
\(4/2^2\times (2 + 1)\),

\item {} 
\sphinxAtStartPar
\(4/2^2\times 2 + 1\)
verifier avec Python.

\end{itemize}

\begin{sphinxVerbatim}[commandchars=\\\{\}]
\PYG{c+c1}{\PYGZsh{}\PYGZsh{} votre code ici}
\end{sphinxVerbatim}




\subsection{Exercice 2.}
\label{\detokenize{exo3:exercice-2}}
\sphinxAtStartPar
Un pere a une somme d’argent de 1554 dh, il veut la partager sur ses 9 enfants de maniere equitable et s’il reste quelque dirham, il va acheter des chocolat a 1 dh l’unite. combien chanque enfant va recevoir? combien d’unite de chocolat peut\sphinxhyphen{}il acheter avec le reste?

\begin{sphinxVerbatim}[commandchars=\\\{\}]
\PYG{c+c1}{\PYGZsh{}\PYGZsh{} votre code ici}
\end{sphinxVerbatim}




\subsection{Exercice 3.}
\label{\detokenize{exo3:exercice-3}}
\sphinxAtStartPar
Quel est le type de donnees de valeurs suivantes:
\begin{itemize}
\item {} 
\sphinxAtStartPar
\sphinxcode{\sphinxupquote{1}}

\item {} 
\sphinxAtStartPar
\sphinxcode{\sphinxupquote{1.}}

\item {} 
\sphinxAtStartPar
\sphinxcode{\sphinxupquote{False}}

\item {} 
\sphinxAtStartPar
\sphinxcode{\sphinxupquote{"False"}}

\item {} 
\sphinxAtStartPar
\sphinxcode{\sphinxupquote{var1/var2}} avec \sphinxcode{\sphinxupquote{var1 = 1}} et \sphinxcode{\sphinxupquote{var2 = 2}}

\end{itemize}

\begin{sphinxVerbatim}[commandchars=\\\{\}]
\PYG{c+c1}{\PYGZsh{}\PYGZsh{} votre code ici}
\end{sphinxVerbatim}




\subsection{Exercice 4.}
\label{\detokenize{exo3:exercice-4}}
\sphinxAtStartPar
Quel est le type de donnees de valeurs suivantes:
\begin{itemize}
\item {} 
\sphinxAtStartPar
\sphinxcode{\sphinxupquote{1}}

\item {} 
\sphinxAtStartPar
\sphinxcode{\sphinxupquote{1.}}

\item {} 
\sphinxAtStartPar
\sphinxcode{\sphinxupquote{False}}

\item {} 
\sphinxAtStartPar
\sphinxcode{\sphinxupquote{"False"}}

\item {} 
\sphinxAtStartPar
\sphinxcode{\sphinxupquote{"5.4"}}

\item {} 
\sphinxAtStartPar
\sphinxcode{\sphinxupquote{var1/var2}} avec \sphinxcode{\sphinxupquote{var1 = 1}} et \sphinxcode{\sphinxupquote{var2 = 2}}

\item {} 
\sphinxAtStartPar
\sphinxcode{\sphinxupquote{list()}}

\item {} 
\sphinxAtStartPar
\sphinxcode{\sphinxupquote{None}}

\item {} 
\sphinxAtStartPar
\sphinxcode{\sphinxupquote{""}}

\end{itemize}

\begin{sphinxVerbatim}[commandchars=\\\{\}]
\PYG{c+c1}{\PYGZsh{}\PYGZsh{} votre code ici}
\end{sphinxVerbatim}




\subsection{Exercice 5.}
\label{\detokenize{exo3:exercice-5}}
\sphinxAtStartPar
Convertir, si c’est possible, de valeurs suivantes au types de donnees que nous avons vu. Expliciter les cas qui ne sont pas possible:
\begin{itemize}
\item {} 
\sphinxAtStartPar
\sphinxcode{\sphinxupquote{1}}

\item {} 
\sphinxAtStartPar
\sphinxcode{\sphinxupquote{1.}}

\item {} 
\sphinxAtStartPar
\sphinxcode{\sphinxupquote{False}}

\item {} 
\sphinxAtStartPar
\sphinxcode{\sphinxupquote{"False"}}

\item {} 
\sphinxAtStartPar
\sphinxcode{\sphinxupquote{"5.4"}}

\item {} 
\sphinxAtStartPar
\sphinxcode{\sphinxupquote{var1/var2}} avec \sphinxcode{\sphinxupquote{var1 = 1}} et \sphinxcode{\sphinxupquote{var2 = 2}}

\item {} 
\sphinxAtStartPar
\sphinxcode{\sphinxupquote{list()}}

\item {} 
\sphinxAtStartPar
\sphinxcode{\sphinxupquote{None}}

\item {} 
\sphinxAtStartPar
\sphinxcode{\sphinxupquote{""}}

\end{itemize}

\begin{sphinxVerbatim}[commandchars=\\\{\}]
\PYG{c+c1}{\PYGZsh{}\PYGZsh{} votre code ici}
\end{sphinxVerbatim}




\subsection{Exercice 6.}
\label{\detokenize{exo3:exercice-6}}
\sphinxAtStartPar
Soient \sphinxcode{\sphinxupquote{x = True}}, \sphinxcode{\sphinxupquote{y= 7\textless{}6}}, and \sphinxcode{\sphinxupquote{z= not y}}. Determinier la valeur logique de \sphinxcode{\sphinxupquote{x, y,}} et \sphinxcode{\sphinxupquote{z}} (\sphinxcode{\sphinxupquote{True}} ou \sphinxcode{\sphinxupquote{False}}) puis la valeur de chacun des expression suivantes:
\begin{itemize}
\item {} 
\sphinxAtStartPar
\sphinxcode{\sphinxupquote{x != False}}

\item {} 
\sphinxAtStartPar
\sphinxcode{\sphinxupquote{x and y}}

\item {} 
\sphinxAtStartPar
\sphinxcode{\sphinxupquote{x or y}}

\item {} 
\sphinxAtStartPar
\sphinxcode{\sphinxupquote{not y}}

\item {} 
\sphinxAtStartPar
\sphinxcode{\sphinxupquote{x and (y or z)}}

\item {} 
\sphinxAtStartPar
\sphinxcode{\sphinxupquote{(x and y) or z}}

\item {} 
\sphinxAtStartPar
\sphinxcode{\sphinxupquote{(not x or not y) and (not z)}}

\item {} 
\sphinxAtStartPar
\sphinxcode{\sphinxupquote{not ((x and y) or z)}}

\end{itemize}

\begin{sphinxVerbatim}[commandchars=\\\{\}]
\PYG{c+c1}{\PYGZsh{}\PYGZsh{} votre code ici}
\end{sphinxVerbatim}




\subsection{Exercice 7.}
\label{\detokenize{exo3:exercice-7}}
\sphinxAtStartPar
On veut recevoir ce message avec \sphinxcode{\sphinxupquote{print()}}:

\begin{sphinxVerbatim}[commandchars=\\\{\}]
\PYG{n}{la} \PYG{n}{valeur} \PYG{n}{de} \PYG{n}{x} \PYG{n}{est}\PYG{p}{:} \PYG{o}{@}\PYG{n+nd}{@True}\PYG{o}{@}\PYG{o}{@}
\PYG{o}{*}\PYG{o}{*}\PYG{o}{*}\PYG{o}{*}\PYG{o}{*}\PYG{o}{*}\PYG{o}{*}\PYG{o}{*}\PYG{o}{*}\PYG{o}{==}\PYG{o}{==}\PYG{o}{==}\PYG{o}{==}\PYG{o}{==}\PYG{o}{==}\PYG{o}{==}\PYG{o}{*}\PYG{o}{*}\PYG{o}{*}\PYG{o}{*}\PYG{o}{*}\PYG{o}{*}\PYG{o}{*}\PYG{o}{*}
\end{sphinxVerbatim}

\sphinxAtStartPar
En utilisant les argument suivants:
\begin{itemize}
\item {} 
\sphinxAtStartPar
\sphinxcode{\sphinxupquote{x = True}}

\item {} 
\sphinxAtStartPar
\sphinxcode{\sphinxupquote{a = "la valeur de x est:"}}

\end{itemize}

\sphinxAtStartPar
Toute modification devra etre faite au nivau de \sphinxcode{\sphinxupquote{sep=}}, et \sphinxcode{\sphinxupquote{end=}}.

\begin{sphinxVerbatim}[commandchars=\\\{\}]
\PYG{c+c1}{\PYGZsh{}\PYGZsh{} votre code ici}
\end{sphinxVerbatim}




\subsection{Exercice 8.}
\label{\detokenize{exo3:exercice-8}}
\sphinxAtStartPar
Ecrire un petit programe qui permet de demander a l’utilisatuer d’entrer son nom, son poids en kilograme (sans entrer l’unite), et sa taille en metre (sans entrer l’unite). puis il affiche l’indice du poids (Body mass index (BMI)):
\sphinxstyleemphasis{Formule du BMI}: \(BMI = \dfrac{poids}{taille^2}\)

\begin{sphinxVerbatim}[commandchars=\\\{\}]
\PYG{c+c1}{\PYGZsh{}\PYGZsh{} votre code ici}
\end{sphinxVerbatim}


\chapter{Les fonctions et les modules:}
\label{\detokenize{content4:les-fonctions-et-les-modules}}\label{\detokenize{content4::doc}}

\section{Les fonctions}
\label{\detokenize{ch6:les-fonctions}}\label{\detokenize{ch6::doc}}

\subsection{Dictionnaire (dictionnary)}
\label{\detokenize{ch6:dictionnaire-dictionnary}}
\sphinxAtStartPar
Contrairment au objets construits que nous avons vu (\sphinxcode{\sphinxupquote{strings}}, \sphinxcode{\sphinxupquote{lists}}, \sphinxcode{\sphinxupquote{tuples}}), les dictionnaires (\sphinxcode{\sphinxupquote{dictionnary}}) sont utilisés pour stocker des valeurs de données dans des paires clé:valeur. Un dictionnaire est une collection ordonnée , modifiable et qui n’autorise pas les doublons (Depuis la version 3.7 de Python, les dictionnaires demeurent ordonnés. Dans les versions antérieures (Python 3.6 et moins), les dictionnaires ne sont pas ordonnés).
\begin{itemize}
\item {} 
\sphinxAtStartPar
Les dictionnaires sont écrits avec des accolades et ont des clés et des valeurs.  Pour creer un dictionnaire et l’affecter a une variable :

\end{itemize}

\begin{sphinxVerbatim}[commandchars=\\\{\}]
\PYG{n}{var\PYGZus{}dict} \PYG{o}{=} \PYG{p}{\PYGZob{}}\PYG{n}{cle\PYGZus{}1}\PYG{p}{:} \PYG{n}{valeur\PYGZus{}1}\PYG{p}{,} \PYG{n}{cle\PYGZus{}2}\PYG{p}{:}\PYG{n}{valeur\PYGZus{}2}\PYG{p}{,}\PYG{o}{.}\PYG{o}{.}\PYG{o}{.}\PYG{o}{.}\PYG{p}{,} \PYG{n}{cle\PYGZus{}n}\PYG{p}{:}\PYG{n}{valeur\PYGZus{}n}\PYG{p}{\PYGZcb{}}
\end{sphinxVerbatim}
\begin{itemize}
\item {} 
\sphinxAtStartPar
Un dictionnaire vide est un dictionnaire qui contient 0 element (\{\} ou dict());

\item {} 
\sphinxAtStartPar
Les dictionnaires sont \sphinxstylestrong{mutable}, on peut modifier leur contenu et leur taille.

\end{itemize}

\begin{sphinxVerbatim}[commandchars=\\\{\}]
\PYG{c+c1}{\PYGZsh{} dictionnaire vide}
\PYG{n}{var} \PYG{o}{=} \PYG{p}{\PYGZob{}}\PYG{p}{\PYGZcb{}} \PYG{c+c1}{\PYGZsh{} ou var = dict()}
\PYG{n+nb}{print}\PYG{p}{(}\PYG{n+nb}{type}\PYG{p}{(}\PYG{n}{var}\PYG{p}{)}\PYG{p}{)}
\PYG{n}{les\PYGZus{}jours} \PYG{o}{=} \PYG{p}{\PYGZob{}}\PYG{l+m+mi}{1}\PYG{p}{:}\PYG{l+s+s2}{\PYGZdq{}}\PYG{l+s+s2}{lundi}\PYG{l+s+s2}{\PYGZdq{}}\PYG{p}{,} \PYG{l+m+mi}{2}\PYG{p}{:}\PYG{l+s+s2}{\PYGZdq{}}\PYG{l+s+s2}{mardi}\PYG{l+s+s2}{\PYGZdq{}}\PYG{p}{,} \PYG{l+m+mi}{3}\PYG{p}{:}\PYG{l+s+s2}{\PYGZdq{}}\PYG{l+s+s2}{mercredi}\PYG{l+s+s2}{\PYGZdq{}}\PYG{p}{,} \PYG{l+m+mi}{4}\PYG{p}{:}\PYG{l+s+s2}{\PYGZdq{}}\PYG{l+s+s2}{jeudi}\PYG{l+s+s2}{\PYGZdq{}}\PYG{p}{,} \PYG{l+m+mi}{5}\PYG{p}{:}\PYG{l+s+s2}{\PYGZdq{}}\PYG{l+s+s2}{vendredi}\PYG{l+s+s2}{\PYGZdq{}}\PYG{p}{,} \PYG{l+m+mi}{6}\PYG{p}{:}\PYG{l+s+s2}{\PYGZdq{}}\PYG{l+s+s2}{samedi}\PYG{l+s+s2}{\PYGZdq{}}\PYG{p}{,} \PYG{l+m+mi}{7}\PYG{p}{:}\PYG{l+s+s2}{\PYGZdq{}}\PYG{l+s+s2}{dimanche}\PYG{l+s+s2}{\PYGZdq{}}\PYG{p}{\PYGZcb{}}
\PYG{n+nb}{print}\PYG{p}{(}\PYG{n}{les\PYGZus{}jours}\PYG{p}{)}
\end{sphinxVerbatim}

\begin{sphinxVerbatim}[commandchars=\\\{\}]
\PYGZlt{}class \PYGZsq{}dict\PYGZsq{}\PYGZgt{}
\PYGZob{}1: \PYGZsq{}lundi\PYGZsq{}, 2: \PYGZsq{}mardi\PYGZsq{}, 3: \PYGZsq{}mercredi\PYGZsq{}, 4: \PYGZsq{}jeudi\PYGZsq{}, 5: \PYGZsq{}vendredi\PYGZsq{}, 6: \PYGZsq{}samedi\PYGZsq{}, 7: \PYGZsq{}dimanche\PYGZsq{}\PYGZcb{}
\end{sphinxVerbatim}

\begin{sphinxVerbatim}[commandchars=\\\{\}]
\PYG{n}{les\PYGZus{}jours2} \PYG{o}{=} \PYG{p}{\PYGZob{}}\PYG{p}{\PYGZcb{}}

\PYG{n}{les\PYGZus{}jours2}\PYG{p}{[}\PYG{l+m+mi}{1}\PYG{p}{]} \PYG{o}{=} \PYG{l+s+s1}{\PYGZsq{}}\PYG{l+s+s1}{lundi}\PYG{l+s+s1}{\PYGZsq{}}
\PYG{n}{les\PYGZus{}jours2}\PYG{p}{[}\PYG{l+m+mi}{2}\PYG{p}{]} \PYG{o}{=} \PYG{l+s+s1}{\PYGZsq{}}\PYG{l+s+s1}{mardi}\PYG{l+s+s1}{\PYGZsq{}}
\PYG{n}{les\PYGZus{}jours2}\PYG{p}{[}\PYG{l+m+mi}{3}\PYG{p}{]} \PYG{o}{=} \PYG{l+s+s1}{\PYGZsq{}}\PYG{l+s+s1}{mercredi}\PYG{l+s+s1}{\PYGZsq{}}
\PYG{n}{les\PYGZus{}jours2}\PYG{p}{[}\PYG{l+m+mi}{4}\PYG{p}{]} \PYG{o}{=} \PYG{l+s+s1}{\PYGZsq{}}\PYG{l+s+s1}{jeudi}\PYG{l+s+s1}{\PYGZsq{}}
\PYG{n}{les\PYGZus{}jours2}\PYG{p}{[}\PYG{l+m+mi}{5}\PYG{p}{]} \PYG{o}{=} \PYG{l+s+s1}{\PYGZsq{}}\PYG{l+s+s1}{vendredi}\PYG{l+s+s1}{\PYGZsq{}}
\PYG{n}{les\PYGZus{}jours2}\PYG{p}{[}\PYG{l+m+mi}{6}\PYG{p}{]} \PYG{o}{=} \PYG{l+s+s1}{\PYGZsq{}}\PYG{l+s+s1}{samedi}\PYG{l+s+s1}{\PYGZsq{}}
\PYG{n}{les\PYGZus{}jours2}\PYG{p}{[}\PYG{l+m+mi}{7}\PYG{p}{]} \PYG{o}{=} \PYG{l+s+s1}{\PYGZsq{}}\PYG{l+s+s1}{dimanche}\PYG{l+s+s1}{\PYGZsq{}}

\PYG{n+nb}{print}\PYG{p}{(}\PYG{n}{les\PYGZus{}jours2}\PYG{p}{)}
\end{sphinxVerbatim}

\begin{sphinxVerbatim}[commandchars=\\\{\}]
\PYGZob{}1: \PYGZsq{}lundi\PYGZsq{}, 2: \PYGZsq{}mardi\PYGZsq{}, 3: \PYGZsq{}mercredi\PYGZsq{}, 4: \PYGZsq{}jeudi\PYGZsq{}, 5: \PYGZsq{}vendredi\PYGZsq{}, 6: \PYGZsq{}samedi\PYGZsq{}, 7: \PYGZsq{}dimanche\PYGZsq{}\PYGZcb{}
\end{sphinxVerbatim}

\sphinxAtStartPar
les cles et les valeurs peuvent etre de n’importe quel types de donnees. Par exemple on peut creer le dictionnaire suivant:

\begin{sphinxVerbatim}[commandchars=\\\{\}]
\PYG{n}{les\PYGZus{}jours3} \PYG{o}{=} \PYG{p}{\PYGZob{}}\PYG{p}{\PYGZcb{}}

\PYG{n}{les\PYGZus{}jours3}\PYG{p}{[}\PYG{l+s+s1}{\PYGZsq{}}\PYG{l+s+s1}{lundi}\PYG{l+s+s1}{\PYGZsq{}}\PYG{p}{]} \PYG{o}{=} \PYG{l+m+mi}{1}
\PYG{n}{les\PYGZus{}jours3}\PYG{p}{[}\PYG{l+s+s1}{\PYGZsq{}}\PYG{l+s+s1}{mardi}\PYG{l+s+s1}{\PYGZsq{}}\PYG{p}{]} \PYG{o}{=} \PYG{l+m+mi}{2}
\PYG{n}{les\PYGZus{}jours3}\PYG{p}{[}\PYG{l+s+s1}{\PYGZsq{}}\PYG{l+s+s1}{mercredi}\PYG{l+s+s1}{\PYGZsq{}}\PYG{p}{]} \PYG{o}{=} \PYG{l+m+mi}{3}
\PYG{n}{les\PYGZus{}jours3}\PYG{p}{[}\PYG{l+s+s1}{\PYGZsq{}}\PYG{l+s+s1}{jeudi}\PYG{l+s+s1}{\PYGZsq{}}\PYG{p}{]} \PYG{o}{=} \PYG{l+m+mi}{4}
\PYG{n}{les\PYGZus{}jours3}\PYG{p}{[}\PYG{l+s+s1}{\PYGZsq{}}\PYG{l+s+s1}{vendredi}\PYG{l+s+s1}{\PYGZsq{}}\PYG{p}{]} \PYG{o}{=} \PYG{l+m+mi}{5}
\PYG{n}{les\PYGZus{}jours3}\PYG{p}{[}\PYG{l+s+s1}{\PYGZsq{}}\PYG{l+s+s1}{samedi}\PYG{l+s+s1}{\PYGZsq{}}\PYG{p}{]} \PYG{o}{=} \PYG{l+m+mi}{6}
\PYG{n}{les\PYGZus{}jours3}\PYG{p}{[}\PYG{l+s+s1}{\PYGZsq{}}\PYG{l+s+s1}{dimanche}\PYG{l+s+s1}{\PYGZsq{}}\PYG{p}{]} \PYG{o}{=} \PYG{l+m+mi}{7}

\PYG{n+nb}{print}\PYG{p}{(}\PYG{n}{les\PYGZus{}jours3}\PYG{p}{)}
\end{sphinxVerbatim}

\begin{sphinxVerbatim}[commandchars=\\\{\}]
\PYGZob{}\PYGZsq{}lundi\PYGZsq{}: 1, \PYGZsq{}mardi\PYGZsq{}: 2, \PYGZsq{}mercredi\PYGZsq{}: 3, \PYGZsq{}jeudi\PYGZsq{}: 4, \PYGZsq{}vendredi\PYGZsq{}: 5, \PYGZsq{}samedi\PYGZsq{}: 6, \PYGZsq{}dimanche\PYGZsq{}: 7\PYGZcb{}
\end{sphinxVerbatim}

\sphinxAtStartPar
Contrairment aux chaines de caracteres et aux listes, pour acceder a une valeur dans un dictionnaire nous deveons utiliser les cles. Par exemple, si nous voulons acceder a la valeur associee a la cle \sphinxcode{\sphinxupquote{jeudi}} dans \sphinxcode{\sphinxupquote{les\_jours3}}, la maniere de le faire est la suivante:

\begin{sphinxVerbatim}[commandchars=\\\{\}]
\PYG{n+nb}{print}\PYG{p}{(}\PYG{n}{les\PYGZus{}jours3}\PYG{p}{[}\PYG{l+s+s1}{\PYGZsq{}}\PYG{l+s+s1}{jeudi}\PYG{l+s+s1}{\PYGZsq{}}\PYG{p}{]}\PYG{p}{)}
\end{sphinxVerbatim}

\begin{sphinxVerbatim}[commandchars=\\\{\}]
4
\end{sphinxVerbatim}

\sphinxAtStartPar
Les dictionnaires ne peuvent pas avoir des cles dupliqee (les valeurs peuvent etre dupliquees). En effet, les valeurs en double écraseront les valeurs existantes. Illustrons ca avec l’exemple suivant:

\begin{sphinxVerbatim}[commandchars=\\\{\}]
\PYG{n}{les\PYGZus{}jours4} \PYG{o}{=} \PYG{p}{\PYGZob{}}\PYG{p}{\PYGZcb{}}

\PYG{n}{les\PYGZus{}jours4}\PYG{p}{[}\PYG{l+s+s1}{\PYGZsq{}}\PYG{l+s+s1}{lundi}\PYG{l+s+s1}{\PYGZsq{}}\PYG{p}{]} \PYG{o}{=} \PYG{l+m+mi}{0}
\PYG{n}{les\PYGZus{}jours4}\PYG{p}{[}\PYG{l+s+s1}{\PYGZsq{}}\PYG{l+s+s1}{mardi}\PYG{l+s+s1}{\PYGZsq{}}\PYG{p}{]} \PYG{o}{=} \PYG{l+m+mi}{1}
\PYG{n}{les\PYGZus{}jours4}\PYG{p}{[}\PYG{l+s+s1}{\PYGZsq{}}\PYG{l+s+s1}{mardi}\PYG{l+s+s1}{\PYGZsq{}}\PYG{p}{]} \PYG{o}{=} \PYG{l+m+mi}{2}
\PYG{n}{les\PYGZus{}jours4}\PYG{p}{[}\PYG{l+s+s1}{\PYGZsq{}}\PYG{l+s+s1}{mercredi}\PYG{l+s+s1}{\PYGZsq{}}\PYG{p}{]} \PYG{o}{=} \PYG{l+m+mi}{3}
\PYG{n}{les\PYGZus{}jours4}\PYG{p}{[}\PYG{l+s+s1}{\PYGZsq{}}\PYG{l+s+s1}{jeudi}\PYG{l+s+s1}{\PYGZsq{}}\PYG{p}{]} \PYG{o}{=} \PYG{l+m+mi}{4}
\PYG{n}{les\PYGZus{}jours4}\PYG{p}{[}\PYG{l+s+s1}{\PYGZsq{}}\PYG{l+s+s1}{vendredi}\PYG{l+s+s1}{\PYGZsq{}}\PYG{p}{]} \PYG{o}{=} \PYG{l+m+mi}{5}
\PYG{n}{les\PYGZus{}jours4}\PYG{p}{[}\PYG{l+s+s1}{\PYGZsq{}}\PYG{l+s+s1}{samedi}\PYG{l+s+s1}{\PYGZsq{}}\PYG{p}{]} \PYG{o}{=} \PYG{l+m+mi}{6}
\PYG{n}{les\PYGZus{}jours4}\PYG{p}{[}\PYG{l+s+s1}{\PYGZsq{}}\PYG{l+s+s1}{dimanche}\PYG{l+s+s1}{\PYGZsq{}}\PYG{p}{]} \PYG{o}{=} \PYG{l+m+mi}{7}

\PYG{n+nb}{print}\PYG{p}{(}\PYG{n}{les\PYGZus{}jours4}\PYG{p}{)}
\end{sphinxVerbatim}

\begin{sphinxVerbatim}[commandchars=\\\{\}]
\PYGZob{}\PYGZsq{}lundi\PYGZsq{}: 0, \PYGZsq{}mardi\PYGZsq{}: 2, \PYGZsq{}mercredi\PYGZsq{}: 3, \PYGZsq{}jeudi\PYGZsq{}: 4, \PYGZsq{}vendredi\PYGZsq{}: 5, \PYGZsq{}samedi\PYGZsq{}: 6, \PYGZsq{}dimanche\PYGZsq{}: 7\PYGZcb{}
\end{sphinxVerbatim}

\begin{sphinxVerbatim}[commandchars=\\\{\}]
\PYG{n}{les\PYGZus{}jours5} \PYG{o}{=} \PYG{p}{\PYGZob{}}\PYG{p}{\PYGZcb{}}

\PYG{n}{les\PYGZus{}jours5}\PYG{p}{[}\PYG{l+s+s1}{\PYGZsq{}}\PYG{l+s+s1}{lundi}\PYG{l+s+s1}{\PYGZsq{}}\PYG{p}{]} \PYG{o}{=} \PYG{l+m+mi}{0}
\PYG{n}{les\PYGZus{}jours5}\PYG{p}{[}\PYG{l+s+s1}{\PYGZsq{}}\PYG{l+s+s1}{mardi}\PYG{l+s+s1}{\PYGZsq{}}\PYG{p}{]} \PYG{o}{=} \PYG{l+m+mi}{0}
\PYG{n}{les\PYGZus{}jours5}\PYG{p}{[}\PYG{l+s+s1}{\PYGZsq{}}\PYG{l+s+s1}{mercredi}\PYG{l+s+s1}{\PYGZsq{}}\PYG{p}{]} \PYG{o}{=} \PYG{l+m+mi}{0}
\PYG{n}{les\PYGZus{}jours5}\PYG{p}{[}\PYG{l+s+s1}{\PYGZsq{}}\PYG{l+s+s1}{jeudi}\PYG{l+s+s1}{\PYGZsq{}}\PYG{p}{]} \PYG{o}{=} \PYG{l+m+mi}{5}
\PYG{n}{les\PYGZus{}jours5}\PYG{p}{[}\PYG{l+s+s1}{\PYGZsq{}}\PYG{l+s+s1}{vendredi}\PYG{l+s+s1}{\PYGZsq{}}\PYG{p}{]} \PYG{o}{=} \PYG{l+m+mi}{0}
\PYG{n}{les\PYGZus{}jours5}\PYG{p}{[}\PYG{l+s+s1}{\PYGZsq{}}\PYG{l+s+s1}{samedi}\PYG{l+s+s1}{\PYGZsq{}}\PYG{p}{]} \PYG{o}{=} \PYG{l+m+mi}{0}
\PYG{n}{les\PYGZus{}jours5}\PYG{p}{[}\PYG{l+s+s1}{\PYGZsq{}}\PYG{l+s+s1}{dimanche}\PYG{l+s+s1}{\PYGZsq{}}\PYG{p}{]} \PYG{o}{=} \PYG{l+m+mi}{0}

\PYG{n+nb}{print}\PYG{p}{(}\PYG{n}{les\PYGZus{}jours5}\PYG{p}{)}
\end{sphinxVerbatim}

\begin{sphinxVerbatim}[commandchars=\\\{\}]
\PYGZob{}\PYGZsq{}lundi\PYGZsq{}: 0, \PYGZsq{}mardi\PYGZsq{}: 0, \PYGZsq{}mercredi\PYGZsq{}: 0, \PYGZsq{}jeudi\PYGZsq{}: 5, \PYGZsq{}vendredi\PYGZsq{}: 0, \PYGZsq{}samedi\PYGZsq{}: 0, \PYGZsq{}dimanche\PYGZsq{}: 0\PYGZcb{}
\end{sphinxVerbatim}


\subsection{Operations sur les dictionnaires}
\label{\detokenize{ch6:operations-sur-les-dictionnaires}}
\sphinxAtStartPar
Python possède un ensemble de méthodes intégrées que nous pouvons utiliser pour manipuler les dictionnaires:


\begin{savenotes}\sphinxattablestart
\centering
\begin{tabulary}{\linewidth}[t]{|T|T|}
\hline
\sphinxstyletheadfamily 
\sphinxAtStartPar
Methode
&\sphinxstyletheadfamily 
\sphinxAtStartPar
Description
\\
\hline
\sphinxAtStartPar
clear()
&
\sphinxAtStartPar
Supprime tous les éléments du dictionnaire
\\
\hline
\sphinxAtStartPar
copy()
&
\sphinxAtStartPar
Renvoie une copie du dictionnaire
\\
\hline
\sphinxAtStartPar
fromkeys()
&
\sphinxAtStartPar
Renvoie un dictionnaire avec les clés et la valeur spécifiées
\\
\hline
\sphinxAtStartPar
get()
&
\sphinxAtStartPar
Renvoie la valeur de la clé spécifiée
\\
\hline
\sphinxAtStartPar
items()
&
\sphinxAtStartPar
Renvoie une liste contenant un tuple pour chaque paire clé\sphinxhyphen{}valeur
\\
\hline
\sphinxAtStartPar
keys()
&
\sphinxAtStartPar
Retourne une liste contenant les clés du dictionnaire
\\
\hline
\sphinxAtStartPar
pop()
&
\sphinxAtStartPar
Supprime l’élément avec la clé spécifiée
\\
\hline
\sphinxAtStartPar
popitem()
&
\sphinxAtStartPar
Supprime la dernière paire clé\sphinxhyphen{}valeur insérée
\\
\hline
\sphinxAtStartPar
setdefault()
&
\sphinxAtStartPar
Renvoie la valeur de la clé spécifiée. Si la clé n’existe pas : insérez la clé, avec la valeur spécifiée
\\
\hline
\sphinxAtStartPar
update()
&
\sphinxAtStartPar
Met à jour le dictionnaire avec les paires clé\sphinxhyphen{}valeur spécifiées
\\
\hline
\sphinxAtStartPar
values()
&
\sphinxAtStartPar
Renvoie une liste de toutes les valeurs du dictionnaire
\\
\hline
\end{tabulary}
\par
\sphinxattableend\end{savenotes}

\begin{sphinxVerbatim}[commandchars=\\\{\}]
\PYG{c+c1}{\PYGZsh{} clear}
\PYG{n}{les\PYGZus{}jours}\PYG{o}{.}\PYG{n}{clear}\PYG{p}{(}\PYG{p}{)}

\PYG{n+nb}{print}\PYG{p}{(}\PYG{n}{les\PYGZus{}jours}\PYG{p}{)}
\end{sphinxVerbatim}

\begin{sphinxVerbatim}[commandchars=\\\{\}]
\PYGZob{}\PYGZcb{}
\end{sphinxVerbatim}

\begin{sphinxVerbatim}[commandchars=\\\{\}]
\PYG{c+c1}{\PYGZsh{} copy}
\PYG{n}{les\PYGZus{}jours2\PYGZus{}copie} \PYG{o}{=} \PYG{n}{les\PYGZus{}jours2}\PYG{o}{.}\PYG{n}{copy}\PYG{p}{(}\PYG{p}{)}

\PYG{n+nb}{print}\PYG{p}{(}\PYG{n}{les\PYGZus{}jours2\PYGZus{}copie}\PYG{p}{)}
\end{sphinxVerbatim}

\begin{sphinxVerbatim}[commandchars=\\\{\}]
\PYGZob{}1: \PYGZsq{}lundi\PYGZsq{}, 2: \PYGZsq{}mardi\PYGZsq{}, 3: \PYGZsq{}mercredi\PYGZsq{}, 4: \PYGZsq{}jeudi\PYGZsq{}, 5: \PYGZsq{}vendredi\PYGZsq{}, 6: \PYGZsq{}samedi\PYGZsq{}, 7: \PYGZsq{}dimanche\PYGZsq{}\PYGZcb{}
\end{sphinxVerbatim}

\begin{sphinxVerbatim}[commandchars=\\\{\}]
\PYG{c+c1}{\PYGZsh{} fromkeys}
\PYG{c+c1}{\PYGZsh{} creer un dictionnaire avec des cles mais pas de valeurs (None)}

\PYG{n}{cles} \PYG{o}{=} \PYG{n+nb}{range}\PYG{p}{(}\PYG{l+m+mi}{6}\PYG{p}{)}
\PYG{n}{dict\PYGZus{}vide} \PYG{o}{=} \PYG{n+nb}{dict}\PYG{o}{.}\PYG{n}{fromkeys}\PYG{p}{(}\PYG{n}{cles}\PYG{p}{)}
\PYG{n+nb}{print}\PYG{p}{(}\PYG{n}{dict\PYGZus{}vide}\PYG{p}{)}
\end{sphinxVerbatim}

\begin{sphinxVerbatim}[commandchars=\\\{\}]
\PYGZob{}0: None, 1: None, 2: None, 3: None, 4: None, 5: None\PYGZcb{}
\end{sphinxVerbatim}

\begin{sphinxVerbatim}[commandchars=\\\{\}]
\PYG{c+c1}{\PYGZsh{} fromkeys}
\PYG{c+c1}{\PYGZsh{} creer un dictionnaire avec des cles a la meme valeur}
\PYG{n}{var} \PYG{o}{=} \PYG{p}{[}\PYG{l+s+s1}{\PYGZsq{}}\PYG{l+s+s1}{a}\PYG{l+s+s1}{\PYGZsq{}}\PYG{p}{,} \PYG{l+s+s1}{\PYGZsq{}}\PYG{l+s+s1}{e}\PYG{l+s+s1}{\PYGZsq{}}\PYG{p}{,} \PYG{l+s+s1}{\PYGZsq{}}\PYG{l+s+s1}{i}\PYG{l+s+s1}{\PYGZsq{}}\PYG{p}{,} \PYG{l+s+s1}{\PYGZsq{}}\PYG{l+s+s1}{o}\PYG{l+s+s1}{\PYGZsq{}}\PYG{p}{,} \PYG{l+s+s1}{\PYGZsq{}}\PYG{l+s+s1}{u}\PYG{l+s+s1}{\PYGZsq{}}\PYG{p}{,} \PYG{l+s+s1}{\PYGZsq{}}\PYG{l+s+s1}{y}\PYG{l+s+s1}{\PYGZsq{}}\PYG{p}{]}
\PYG{n}{valeur} \PYG{o}{=} \PYG{l+s+s1}{\PYGZsq{}}\PYG{l+s+s1}{voyelle}\PYG{l+s+s1}{\PYGZsq{}}
\PYG{n}{dict\PYGZus{}voyelles} \PYG{o}{=} \PYG{n+nb}{dict}\PYG{o}{.}\PYG{n}{fromkeys}\PYG{p}{(}\PYG{n}{var}\PYG{p}{,} \PYG{n}{valeur}\PYG{p}{)}

\PYG{n+nb}{print}\PYG{p}{(}\PYG{n}{dict\PYGZus{}voyelles}\PYG{p}{)}
\end{sphinxVerbatim}

\begin{sphinxVerbatim}[commandchars=\\\{\}]
\PYGZob{}\PYGZsq{}a\PYGZsq{}: \PYGZsq{}voyelle\PYGZsq{}, \PYGZsq{}e\PYGZsq{}: \PYGZsq{}voyelle\PYGZsq{}, \PYGZsq{}i\PYGZsq{}: \PYGZsq{}voyelle\PYGZsq{}, \PYGZsq{}o\PYGZsq{}: \PYGZsq{}voyelle\PYGZsq{}, \PYGZsq{}u\PYGZsq{}: \PYGZsq{}voyelle\PYGZsq{}, \PYGZsq{}y\PYGZsq{}: \PYGZsq{}voyelle\PYGZsq{}\PYGZcb{}
\end{sphinxVerbatim}

\begin{sphinxVerbatim}[commandchars=\\\{\}]
\PYG{c+c1}{\PYGZsh{} get}
\PYG{n+nb}{print}\PYG{p}{(}\PYG{n}{les\PYGZus{}jours2}\PYG{o}{.}\PYG{n}{get}\PYG{p}{(}\PYG{l+m+mi}{3}\PYG{p}{)}\PYG{p}{)}
\end{sphinxVerbatim}

\begin{sphinxVerbatim}[commandchars=\\\{\}]
mercredi
\end{sphinxVerbatim}

\begin{sphinxVerbatim}[commandchars=\\\{\}]
\PYG{c+c1}{\PYGZsh{} items}
\PYG{n+nb}{print}\PYG{p}{(}\PYG{n}{les\PYGZus{}jours2}\PYG{o}{.}\PYG{n}{items}\PYG{p}{(}\PYG{p}{)}\PYG{p}{)}
\end{sphinxVerbatim}

\begin{sphinxVerbatim}[commandchars=\\\{\}]
dict\PYGZus{}items([(1, \PYGZsq{}lundi\PYGZsq{}), (2, \PYGZsq{}mardi\PYGZsq{}), (3, \PYGZsq{}mercredi\PYGZsq{}), (4, \PYGZsq{}jeudi\PYGZsq{}), (5, \PYGZsq{}vendredi\PYGZsq{}), (6, \PYGZsq{}samedi\PYGZsq{}), (7, \PYGZsq{}dimanche\PYGZsq{})])
\end{sphinxVerbatim}

\begin{sphinxVerbatim}[commandchars=\\\{\}]
\PYG{c+c1}{\PYGZsh{} keys}
\PYG{n+nb}{print}\PYG{p}{(}\PYG{n}{les\PYGZus{}jours2}\PYG{o}{.}\PYG{n}{keys}\PYG{p}{(}\PYG{p}{)}\PYG{p}{)}
\end{sphinxVerbatim}

\begin{sphinxVerbatim}[commandchars=\\\{\}]
dict\PYGZus{}keys([1, 2, 3, 4, 5, 6, 7])
\end{sphinxVerbatim}

\begin{sphinxVerbatim}[commandchars=\\\{\}]
\PYG{c+c1}{\PYGZsh{} values}
\PYG{n+nb}{print}\PYG{p}{(}\PYG{n}{les\PYGZus{}jours2}\PYG{o}{.}\PYG{n}{values}\PYG{p}{(}\PYG{p}{)}\PYG{p}{)}
\end{sphinxVerbatim}

\begin{sphinxVerbatim}[commandchars=\\\{\}]
dict\PYGZus{}values([\PYGZsq{}lundi\PYGZsq{}, \PYGZsq{}mardi\PYGZsq{}, \PYGZsq{}mercredi\PYGZsq{}, \PYGZsq{}jeudi\PYGZsq{}, \PYGZsq{}vendredi\PYGZsq{}, \PYGZsq{}samedi\PYGZsq{}, \PYGZsq{}dimanche\PYGZsq{}])
\end{sphinxVerbatim}

\begin{sphinxVerbatim}[commandchars=\\\{\}]
\PYG{c+c1}{\PYGZsh{} pop}
\PYG{n}{les\PYGZus{}jours2}\PYG{o}{.}\PYG{n}{pop}\PYG{p}{(}\PYG{l+m+mi}{4}\PYG{p}{)}

\PYG{n+nb}{print}\PYG{p}{(}\PYG{n}{les\PYGZus{}jours2}\PYG{p}{)}
\end{sphinxVerbatim}

\begin{sphinxVerbatim}[commandchars=\\\{\}]
\PYGZob{}1: \PYGZsq{}lundi\PYGZsq{}, 2: \PYGZsq{}mardi\PYGZsq{}, 3: \PYGZsq{}mercredi\PYGZsq{}, 5: \PYGZsq{}vendredi\PYGZsq{}, 6: \PYGZsq{}samedi\PYGZsq{}, 7: \PYGZsq{}dimanche\PYGZsq{}\PYGZcb{}
\end{sphinxVerbatim}

\begin{sphinxVerbatim}[commandchars=\\\{\}]
\PYG{c+c1}{\PYGZsh{} popitem }
\PYG{c+c1}{\PYGZsh{} inserer une cles\PYGZhy{}valeur arbitraire}
\PYG{n}{les\PYGZus{}jours2}\PYG{p}{[}\PYG{l+s+s1}{\PYGZsq{}}\PYG{l+s+s1}{la dirniere cles ajoutee}\PYG{l+s+s1}{\PYGZsq{}}\PYG{p}{]} \PYG{o}{=} \PYG{l+s+s1}{\PYGZsq{}}\PYG{l+s+s1}{La valeur associee a la derniere cles}\PYG{l+s+s1}{\PYGZsq{}}

\PYG{n+nb}{print}\PYG{p}{(}\PYG{n}{les\PYGZus{}jours2}\PYG{p}{)}
\end{sphinxVerbatim}

\begin{sphinxVerbatim}[commandchars=\\\{\}]
\PYGZob{}1: \PYGZsq{}lundi\PYGZsq{}, 2: \PYGZsq{}mardi\PYGZsq{}, 3: \PYGZsq{}mercredi\PYGZsq{}, 5: \PYGZsq{}vendredi\PYGZsq{}, 6: \PYGZsq{}samedi\PYGZsq{}, 7: \PYGZsq{}dimanche\PYGZsq{}, \PYGZsq{}la dirniere cles ajoutee\PYGZsq{}: \PYGZsq{}La valeur associee a la derniere cles\PYGZsq{}\PYGZcb{}
\end{sphinxVerbatim}

\begin{sphinxVerbatim}[commandchars=\\\{\}]
\PYG{c+c1}{\PYGZsh{} popitem}
\PYG{n}{les\PYGZus{}jours2}\PYG{o}{.}\PYG{n}{popitem}\PYG{p}{(}\PYG{p}{)}

\PYG{n+nb}{print}\PYG{p}{(}\PYG{n}{les\PYGZus{}jours2}\PYG{p}{)}
\end{sphinxVerbatim}

\begin{sphinxVerbatim}[commandchars=\\\{\}]
\PYGZob{}1: \PYGZsq{}lundi\PYGZsq{}, 2: \PYGZsq{}mardi\PYGZsq{}, 3: \PYGZsq{}mercredi\PYGZsq{}, 5: \PYGZsq{}vendredi\PYGZsq{}, 6: \PYGZsq{}samedi\PYGZsq{}, 7: \PYGZsq{}dimanche\PYGZsq{}\PYGZcb{}
\end{sphinxVerbatim}

\begin{sphinxVerbatim}[commandchars=\\\{\}]
\PYG{c+c1}{\PYGZsh{} setdefault()}
\PYG{c+c1}{\PYGZsh{} si la cles existe}
\PYG{n+nb}{print}\PYG{p}{(}\PYG{n}{les\PYGZus{}jours2}\PYG{o}{.}\PYG{n}{setdefault}\PYG{p}{(}\PYG{l+m+mi}{1}\PYG{p}{,} \PYG{l+s+s2}{\PYGZdq{}}\PYG{l+s+s2}{LUNDI}\PYG{l+s+s2}{\PYGZdq{}}\PYG{p}{)}\PYG{p}{)}
\PYG{c+c1}{\PYGZsh{} pusique la cles existe, la valeur ne va pas changer. Elle va etre affichee. Le dictionnaire ne va pas changer aussi.}
\PYG{n+nb}{print}\PYG{p}{(}\PYG{n}{les\PYGZus{}jours2}\PYG{p}{)}
\end{sphinxVerbatim}

\begin{sphinxVerbatim}[commandchars=\\\{\}]
lundi
\PYGZob{}1: \PYGZsq{}lundi\PYGZsq{}, 2: \PYGZsq{}mardi\PYGZsq{}, 3: \PYGZsq{}mercredi\PYGZsq{}, 5: \PYGZsq{}vendredi\PYGZsq{}, 6: \PYGZsq{}samedi\PYGZsq{}, 7: \PYGZsq{}dimanche\PYGZsq{}\PYGZcb{}
\end{sphinxVerbatim}

\begin{sphinxVerbatim}[commandchars=\\\{\}]
\PYG{c+c1}{\PYGZsh{} setdefault()}
\PYG{c+c1}{\PYGZsh{} si la cles existe}
\PYG{c+c1}{\PYGZsh{} puisque la cles 4 n\PYGZsq{}existe pas. La valeur \PYGZsq{}mercredi\PYGZsq{} va etre affichee. Le dictionnaire va etre modifie aussi}
\PYG{n+nb}{print}\PYG{p}{(}\PYG{n}{les\PYGZus{}jours2}\PYG{o}{.}\PYG{n}{setdefault}\PYG{p}{(}\PYG{l+m+mi}{4}\PYG{p}{,} \PYG{l+s+s2}{\PYGZdq{}}\PYG{l+s+s2}{mercredi}\PYG{l+s+s2}{\PYGZdq{}}\PYG{p}{)}\PYG{p}{)}
\PYG{n+nb}{print}\PYG{p}{(}\PYG{n}{les\PYGZus{}jours2}\PYG{p}{)}
\end{sphinxVerbatim}

\begin{sphinxVerbatim}[commandchars=\\\{\}]
mercredi
\PYGZob{}1: \PYGZsq{}lundi\PYGZsq{}, 2: \PYGZsq{}mardi\PYGZsq{}, 3: \PYGZsq{}mercredi\PYGZsq{}, 5: \PYGZsq{}vendredi\PYGZsq{}, 6: \PYGZsq{}samedi\PYGZsq{}, 7: \PYGZsq{}dimanche\PYGZsq{}, 4: \PYGZsq{}mercredi\PYGZsq{}\PYGZcb{}
\end{sphinxVerbatim}

\begin{sphinxVerbatim}[commandchars=\\\{\}]
\PYG{c+c1}{\PYGZsh{} update}
\PYG{n}{premier}
\end{sphinxVerbatim}

\sphinxAtStartPar
La fonction intégrée \sphinxcode{\sphinxupquote{len()}} est aussi appliquable pour les dictionnaires. Si l’on désire déterminer le nombre d’elements:

\begin{sphinxVerbatim}[commandchars=\\\{\}]
\PYG{n+nb}{len}\PYG{p}{(}\PYG{n}{les\PYGZus{}jours}\PYG{p}{)}
\end{sphinxVerbatim}

\begin{sphinxVerbatim}[commandchars=\\\{\}]
7
\end{sphinxVerbatim}

\sphinxAtStartPar
La varaible \sphinxcode{\sphinxupquote{les\_jours}} contient 7 elements.

\sphinxAtStartPar
Si l’on veut tester l’appartenece d’un element a une liste/tuple on utilise l’operateur \sphinxcode{\sphinxupquote{in}}. L’expression est la suivante : \sphinxcode{\sphinxupquote{element in liste}}. Cela nous renvoi \sphinxcode{\sphinxupquote{True}} si \sphinxcode{\sphinxupquote{element}}est dans \sphinxcode{\sphinxupquote{liste}}, sinon \sphinxcode{\sphinxupquote{False}}. On peut aussi ecrire \sphinxcode{\sphinxupquote{element not in liste}} pour tester si l’element n’est pas dans \sphinxcode{\sphinxupquote{liste}} (\sphinxcode{\sphinxupquote{True}}) ou s’il est dans \sphinxcode{\sphinxupquote{liste}} (\sphinxcode{\sphinxupquote{False}}). Voici quelques exemples:

\begin{sphinxVerbatim}[commandchars=\\\{\}]
\PYG{n}{les\PYGZus{}jours}
\end{sphinxVerbatim}

\begin{sphinxVerbatim}[commandchars=\\\{\}]
\PYGZob{}1: \PYGZsq{}lundi\PYGZsq{},
 2: \PYGZsq{}mardi\PYGZsq{},
 3: \PYGZsq{}mercredi\PYGZsq{},
 4: \PYGZsq{}jeudi\PYGZsq{},
 5: \PYGZsq{}vendredi\PYGZsq{},
 6: \PYGZsq{}samedi\PYGZsq{},
 7: \PYGZsq{}dimanche\PYGZsq{}\PYGZcb{}
\end{sphinxVerbatim}

\begin{sphinxVerbatim}[commandchars=\\\{\}]
\PYG{n}{les\PYGZus{}jours}\PYG{o}{.}\PYG{n}{values}\PYG{p}{(}\PYG{p}{)}
\end{sphinxVerbatim}

\begin{sphinxVerbatim}[commandchars=\\\{\}]
dict\PYGZus{}values([\PYGZsq{}lundi\PYGZsq{}, \PYGZsq{}mardi\PYGZsq{}, \PYGZsq{}mercredi\PYGZsq{}, \PYGZsq{}jeudi\PYGZsq{}, \PYGZsq{}vendredi\PYGZsq{}, \PYGZsq{}samedi\PYGZsq{}, \PYGZsq{}dimanche\PYGZsq{}])
\end{sphinxVerbatim}

\begin{sphinxVerbatim}[commandchars=\\\{\}]
\PYG{n}{un\PYGZus{}tuple} \PYG{o}{=} \PYG{p}{(}\PYG{l+m+mi}{1}\PYG{p}{,} \PYG{l+s+s1}{\PYGZsq{}}\PYG{l+s+s1}{bonjour}\PYG{l+s+s1}{\PYGZsq{}}\PYG{p}{,} \PYG{l+m+mf}{4.5}\PYG{p}{,} \PYG{k+kc}{True}\PYG{p}{)}

\PYG{n+nb}{print}\PYG{p}{(}\PYG{l+m+mi}{1} \PYG{o+ow}{in} \PYG{n}{un\PYGZus{}tuple}\PYG{p}{)}

\PYG{n+nb}{print}\PYG{p}{(}\PYG{l+m+mi}{2} \PYG{o+ow}{in} \PYG{n}{un\PYGZus{}tuple}\PYG{p}{)}

\PYG{n+nb}{print}\PYG{p}{(}\PYG{l+s+s1}{\PYGZsq{}}\PYG{l+s+s1}{o}\PYG{l+s+s1}{\PYGZsq{}} \PYG{o+ow}{not} \PYG{o+ow}{in} \PYG{n}{un\PYGZus{}tuple}\PYG{p}{)}

\PYG{n+nb}{print}\PYG{p}{(}\PYG{l+s+s1}{\PYGZsq{}}\PYG{l+s+s1}{x}\PYG{l+s+s1}{\PYGZsq{}} \PYG{o+ow}{not} \PYG{o+ow}{in} \PYG{n}{un\PYGZus{}tuple}\PYG{p}{)}
\end{sphinxVerbatim}

\begin{sphinxVerbatim}[commandchars=\\\{\}]
True
False
True
True
\end{sphinxVerbatim}


\subsection{Quelle est la difference entre listes et tuples?}
\label{\detokenize{ch6:quelle-est-la-difference-entre-listes-et-tuples}}
\sphinxAtStartPar
Les listes et les tuples sont pareils dans la plupart des contextes. Cepandant, la difference primordiale entre les deux et que les listes sont des \sphinxstylestrong{objets mutables} (modifiables) alors que les tuples sont des \sphinxstylestrong{objets immuables} (ne sont pas modifiable). La question qui se pose est donc: qu’est\sphinxhyphen{}ce qu’un objet mutable et un objet immuable?
Parmi les objet immuable en python, on trouve:
\begin{itemize}
\item {} 
\sphinxAtStartPar
Les nombres entiers (int)

\item {} 
\sphinxAtStartPar
Les nombres décimaux (float)

\item {} 
\sphinxAtStartPar
Les chaînes de caractères (str)

\item {} 
\sphinxAtStartPar
Les booléens (bool)

\item {} 
\sphinxAtStartPar
Les tuples (tuple)
La plus part des autres objets que vous allez confronter en python sont mutables.

\end{itemize}

\sphinxAtStartPar
Nous allons illustre ca dans les exemples suivant:
\begin{enumerate}
\sphinxsetlistlabels{\arabic}{enumi}{enumii}{}{.}%
\item {} 
\sphinxAtStartPar
nous allons creer les variables suivantes:

\end{enumerate}
\begin{itemize}
\item {} 
\sphinxAtStartPar
\sphinxcode{\sphinxupquote{var\_chaine = "bonjour tout le monde"}},

\item {} 
\sphinxAtStartPar
\sphinxcode{\sphinxupquote{var\_liste = {[}1, 2, True, \textquotesingle{}bonjour\textquotesingle{}{]}}},

\item {} 
\sphinxAtStartPar
\sphinxcode{\sphinxupquote{var\_tuple = (1, 2, True, \textquotesingle{}bonjour\textquotesingle{})}}.

\end{itemize}
\begin{enumerate}
\sphinxsetlistlabels{\arabic}{enumi}{enumii}{}{.}%
\item {} 
\sphinxAtStartPar
nous qllons essayer de changer (par exemple) le premier element de chaque variable (par un autre element).

\end{enumerate}

\begin{sphinxVerbatim}[commandchars=\\\{\}]
\PYG{n}{var\PYGZus{}chaine} \PYG{o}{=} \PYG{l+s+s2}{\PYGZdq{}}\PYG{l+s+s2}{bonjour tout le monde}\PYG{l+s+s2}{\PYGZdq{}}
\PYG{n}{var\PYGZus{}liste} \PYG{o}{=} \PYG{p}{[}\PYG{l+m+mi}{1}\PYG{p}{,} \PYG{l+m+mi}{2}\PYG{p}{,} \PYG{k+kc}{True}\PYG{p}{,} \PYG{l+s+s1}{\PYGZsq{}}\PYG{l+s+s1}{bonjour}\PYG{l+s+s1}{\PYGZsq{}}\PYG{p}{]}
\PYG{n}{var\PYGZus{}tuple} \PYG{o}{=} \PYG{p}{(}\PYG{l+m+mi}{1}\PYG{p}{,} \PYG{l+m+mi}{2}\PYG{p}{,} \PYG{k+kc}{True}\PYG{p}{,} \PYG{l+s+s1}{\PYGZsq{}}\PYG{l+s+s1}{bonjour}\PYG{l+s+s1}{\PYGZsq{}}\PYG{p}{)}
\end{sphinxVerbatim}

\begin{sphinxVerbatim}[commandchars=\\\{\}]
\PYG{n}{var\PYGZus{}chaine}\PYG{p}{[}\PYG{l+m+mi}{0}\PYG{p}{]} \PYG{o}{=} \PYG{l+s+s1}{\PYGZsq{}}\PYG{l+s+s1}{B}\PYG{l+s+s1}{\PYGZsq{}}
\PYG{n+nb}{print}\PYG{p}{(}\PYG{n}{var\PYGZus{}chaine}\PYG{p}{)}
\end{sphinxVerbatim}

\begin{sphinxVerbatim}[commandchars=\\\{\}]
\PYG{g+gt}{\PYGZhy{}\PYGZhy{}\PYGZhy{}\PYGZhy{}\PYGZhy{}\PYGZhy{}\PYGZhy{}\PYGZhy{}\PYGZhy{}\PYGZhy{}\PYGZhy{}\PYGZhy{}\PYGZhy{}\PYGZhy{}\PYGZhy{}\PYGZhy{}\PYGZhy{}\PYGZhy{}\PYGZhy{}\PYGZhy{}\PYGZhy{}\PYGZhy{}\PYGZhy{}\PYGZhy{}\PYGZhy{}\PYGZhy{}\PYGZhy{}\PYGZhy{}\PYGZhy{}\PYGZhy{}\PYGZhy{}\PYGZhy{}\PYGZhy{}\PYGZhy{}\PYGZhy{}\PYGZhy{}\PYGZhy{}\PYGZhy{}\PYGZhy{}\PYGZhy{}\PYGZhy{}\PYGZhy{}\PYGZhy{}\PYGZhy{}\PYGZhy{}\PYGZhy{}\PYGZhy{}\PYGZhy{}\PYGZhy{}\PYGZhy{}\PYGZhy{}\PYGZhy{}\PYGZhy{}\PYGZhy{}\PYGZhy{}\PYGZhy{}\PYGZhy{}\PYGZhy{}\PYGZhy{}\PYGZhy{}\PYGZhy{}\PYGZhy{}\PYGZhy{}\PYGZhy{}\PYGZhy{}\PYGZhy{}\PYGZhy{}\PYGZhy{}\PYGZhy{}\PYGZhy{}\PYGZhy{}\PYGZhy{}\PYGZhy{}\PYGZhy{}\PYGZhy{}}
\PYG{n+ne}{TypeError}\PYG{g+gWhitespace}{                                 }Traceback (most recent call last)
\PYG{o}{\PYGZti{}}\PYGZbs{}\PYG{n}{AppData}\PYGZbs{}\PYG{n}{Local}\PYGZbs{}\PYG{n}{Temp}\PYGZbs{}\PYG{n}{ipykernel\PYGZus{}232}\PYGZbs{}\PYG{l+m+mf}{4274672008.}\PYG{n}{py} \PYG{o+ow}{in} \PYG{o}{\PYGZlt{}}\PYG{n}{module}\PYG{o}{\PYGZgt{}}
\PYG{n+ne}{\PYGZhy{}\PYGZhy{}\PYGZhy{}\PYGZhy{}\PYGZgt{} }\PYG{l+m+mi}{1} \PYG{n}{var\PYGZus{}chaine}\PYG{p}{[}\PYG{l+m+mi}{0}\PYG{p}{]} \PYG{o}{=} \PYG{l+s+s1}{\PYGZsq{}}\PYG{l+s+s1}{B}\PYG{l+s+s1}{\PYGZsq{}}
\PYG{g+gWhitespace}{      }\PYG{l+m+mi}{2} \PYG{n+nb}{print}\PYG{p}{(}\PYG{n}{var\PYGZus{}chaine}\PYG{p}{)}

\PYG{n+ne}{TypeError}: \PYGZsq{}str\PYGZsq{} object does not support item assignment
\end{sphinxVerbatim}

\begin{sphinxVerbatim}[commandchars=\\\{\}]
\PYG{n}{var\PYGZus{}liste}\PYG{p}{[}\PYG{l+m+mi}{0}\PYG{p}{]} \PYG{o}{=} \PYG{l+m+mi}{3333}
\PYG{n+nb}{print}\PYG{p}{(}\PYG{n}{var\PYGZus{}liste}\PYG{p}{)}
\end{sphinxVerbatim}

\begin{sphinxVerbatim}[commandchars=\\\{\}]
[3333, 2, True, \PYGZsq{}bonjour\PYGZsq{}]
\end{sphinxVerbatim}

\begin{sphinxVerbatim}[commandchars=\\\{\}]
\PYG{n}{var\PYGZus{}tuple}\PYG{p}{[}\PYG{l+m+mi}{0}\PYG{p}{]} \PYG{o}{=} \PYG{l+m+mi}{3333}
\PYG{n+nb}{print}\PYG{p}{(}\PYG{n}{var\PYGZus{}tuple}\PYG{p}{)}
\end{sphinxVerbatim}

\begin{sphinxVerbatim}[commandchars=\\\{\}]
\PYG{g+gt}{\PYGZhy{}\PYGZhy{}\PYGZhy{}\PYGZhy{}\PYGZhy{}\PYGZhy{}\PYGZhy{}\PYGZhy{}\PYGZhy{}\PYGZhy{}\PYGZhy{}\PYGZhy{}\PYGZhy{}\PYGZhy{}\PYGZhy{}\PYGZhy{}\PYGZhy{}\PYGZhy{}\PYGZhy{}\PYGZhy{}\PYGZhy{}\PYGZhy{}\PYGZhy{}\PYGZhy{}\PYGZhy{}\PYGZhy{}\PYGZhy{}\PYGZhy{}\PYGZhy{}\PYGZhy{}\PYGZhy{}\PYGZhy{}\PYGZhy{}\PYGZhy{}\PYGZhy{}\PYGZhy{}\PYGZhy{}\PYGZhy{}\PYGZhy{}\PYGZhy{}\PYGZhy{}\PYGZhy{}\PYGZhy{}\PYGZhy{}\PYGZhy{}\PYGZhy{}\PYGZhy{}\PYGZhy{}\PYGZhy{}\PYGZhy{}\PYGZhy{}\PYGZhy{}\PYGZhy{}\PYGZhy{}\PYGZhy{}\PYGZhy{}\PYGZhy{}\PYGZhy{}\PYGZhy{}\PYGZhy{}\PYGZhy{}\PYGZhy{}\PYGZhy{}\PYGZhy{}\PYGZhy{}\PYGZhy{}\PYGZhy{}\PYGZhy{}\PYGZhy{}\PYGZhy{}\PYGZhy{}\PYGZhy{}\PYGZhy{}\PYGZhy{}\PYGZhy{}}
\PYG{n+ne}{TypeError}\PYG{g+gWhitespace}{                                 }Traceback (most recent call last)
\PYG{o}{\PYGZti{}}\PYGZbs{}\PYG{n}{AppData}\PYGZbs{}\PYG{n}{Local}\PYGZbs{}\PYG{n}{Temp}\PYGZbs{}\PYG{n}{ipykernel\PYGZus{}232}\PYGZbs{}\PYG{l+m+mf}{613115800.}\PYG{n}{py} \PYG{o+ow}{in} \PYG{o}{\PYGZlt{}}\PYG{n}{module}\PYG{o}{\PYGZgt{}}
\PYG{n+ne}{\PYGZhy{}\PYGZhy{}\PYGZhy{}\PYGZhy{}\PYGZgt{} }\PYG{l+m+mi}{1} \PYG{n}{var\PYGZus{}tuple}\PYG{p}{[}\PYG{l+m+mi}{0}\PYG{p}{]} \PYG{o}{=} \PYG{l+m+mi}{3333}
\PYG{g+gWhitespace}{      }\PYG{l+m+mi}{2} \PYG{n+nb}{print}\PYG{p}{(}\PYG{n}{var\PYGZus{}tuple}\PYG{p}{)}

\PYG{n+ne}{TypeError}: \PYGZsq{}tuple\PYGZsq{} object does not support item assignment
\end{sphinxVerbatim}

\sphinxAtStartPar
Les listes ont une taille variable, les tuples et les chaines de caracteres ont une taille fixe. Enfin, les listes ont plus de fonctionnalités que les tuples. Cependant, c’est le contexte qui nous force a utiliser les listes ou les tuples. Nous allons rencontrer plusieurs contextes ou on est amener a choisir l’un des deux types.


\subsection{quelques methodes utiles pour les listes et les tuples}
\label{\detokenize{ch6:quelques-methodes-utiles-pour-les-listes-et-les-tuples}}
\sphinxAtStartPar
Dans cette sections nous allons voir qulques \sphinxcode{\sphinxupquote{methodes}}  pour les liste et/ou les tuples (les methodes communes et les methodes propres aux listes seulement). Les deux methodes suivantes sont communes aux listes et au tuples:
\begin{itemize}
\item {} 
\sphinxAtStartPar
count: cette methode est utlisee pour compter le nombre d’elements de la liste/tuple.

\item {} 
\sphinxAtStartPar
index: cette method est utilisee pour chercher une valeur spécifiée dans la liste/tuple et renvoie la position de l’endroit où il a été trouvé.

\end{itemize}

\begin{sphinxVerbatim}[commandchars=\\\{\}]
\PYG{n}{var\PYGZus{}liste} \PYG{o}{=} \PYG{p}{[}\PYG{l+m+mi}{1}\PYG{p}{,} \PYG{l+m+mi}{2}\PYG{p}{,} \PYG{l+m+mi}{2}\PYG{p}{,} \PYG{k+kc}{True}\PYG{p}{,} \PYG{l+s+s1}{\PYGZsq{}}\PYG{l+s+s1}{bonjour}\PYG{l+s+s1}{\PYGZsq{}}\PYG{p}{,} \PYG{l+m+mi}{2}\PYG{p}{]}
\PYG{n}{var\PYGZus{}tuple} \PYG{o}{=} \PYG{p}{(}\PYG{l+m+mi}{1}\PYG{p}{,} \PYG{l+m+mi}{2}\PYG{p}{,} \PYG{l+m+mi}{2}\PYG{p}{,} \PYG{k+kc}{True}\PYG{p}{,} \PYG{l+s+s1}{\PYGZsq{}}\PYG{l+s+s1}{bonjour}\PYG{l+s+s1}{\PYGZsq{}}\PYG{p}{,} \PYG{l+m+mi}{2}\PYG{p}{)}

\PYG{n+nb}{print}\PYG{p}{(}\PYG{n}{var\PYGZus{}liste}\PYG{o}{.}\PYG{n}{count}\PYG{p}{(}\PYG{l+m+mi}{2}\PYG{p}{)}\PYG{p}{)} \PYG{c+c1}{\PYGZsh{}\PYGZsh{} print(var\PYGZus{}tuple.count(2))}

\PYG{n+nb}{print}\PYG{p}{(}\PYG{n}{var\PYGZus{}tuple}\PYG{o}{.}\PYG{n}{index}\PYG{p}{(}\PYG{l+m+mi}{2}\PYG{p}{)}\PYG{p}{)}  \PYG{c+c1}{\PYGZsh{}\PYGZsh{} print(var\PYGZus{}tuple.index(2))}
\end{sphinxVerbatim}

\begin{sphinxVerbatim}[commandchars=\\\{\}]
3
\end{sphinxVerbatim}

\begin{sphinxVerbatim}[commandchars=\\\{\}]
\PYG{n}{var\PYGZus{}tuple}\PYG{o}{.}\PYG{n}{index}\PYG{p}{(}\PYG{l+m+mi}{44}\PYG{p}{)} \PYG{c+c1}{\PYGZsh{} var\PYGZus{}tuple.index(44)}
\end{sphinxVerbatim}

\begin{sphinxVerbatim}[commandchars=\\\{\}]
\PYG{g+gt}{\PYGZhy{}\PYGZhy{}\PYGZhy{}\PYGZhy{}\PYGZhy{}\PYGZhy{}\PYGZhy{}\PYGZhy{}\PYGZhy{}\PYGZhy{}\PYGZhy{}\PYGZhy{}\PYGZhy{}\PYGZhy{}\PYGZhy{}\PYGZhy{}\PYGZhy{}\PYGZhy{}\PYGZhy{}\PYGZhy{}\PYGZhy{}\PYGZhy{}\PYGZhy{}\PYGZhy{}\PYGZhy{}\PYGZhy{}\PYGZhy{}\PYGZhy{}\PYGZhy{}\PYGZhy{}\PYGZhy{}\PYGZhy{}\PYGZhy{}\PYGZhy{}\PYGZhy{}\PYGZhy{}\PYGZhy{}\PYGZhy{}\PYGZhy{}\PYGZhy{}\PYGZhy{}\PYGZhy{}\PYGZhy{}\PYGZhy{}\PYGZhy{}\PYGZhy{}\PYGZhy{}\PYGZhy{}\PYGZhy{}\PYGZhy{}\PYGZhy{}\PYGZhy{}\PYGZhy{}\PYGZhy{}\PYGZhy{}\PYGZhy{}\PYGZhy{}\PYGZhy{}\PYGZhy{}\PYGZhy{}\PYGZhy{}\PYGZhy{}\PYGZhy{}\PYGZhy{}\PYGZhy{}\PYGZhy{}\PYGZhy{}\PYGZhy{}\PYGZhy{}\PYGZhy{}\PYGZhy{}\PYGZhy{}\PYGZhy{}\PYGZhy{}\PYGZhy{}}
\PYG{n+ne}{ValueError}\PYG{g+gWhitespace}{                                }Traceback (most recent call last)
\PYG{o}{\PYGZti{}}\PYGZbs{}\PYG{n}{AppData}\PYGZbs{}\PYG{n}{Local}\PYGZbs{}\PYG{n}{Temp}\PYGZbs{}\PYG{n}{ipykernel\PYGZus{}232}\PYGZbs{}\PYG{l+m+mf}{1993419995.}\PYG{n}{py} \PYG{o+ow}{in} \PYG{o}{\PYGZlt{}}\PYG{n}{module}\PYG{o}{\PYGZgt{}}
\PYG{n+ne}{\PYGZhy{}\PYGZhy{}\PYGZhy{}\PYGZhy{}\PYGZgt{} }\PYG{l+m+mi}{1} \PYG{n}{var\PYGZus{}tuple}\PYG{o}{.}\PYG{n}{index}\PYG{p}{(}\PYG{l+m+mi}{44}\PYG{p}{)} \PYG{c+c1}{\PYGZsh{} var\PYGZus{}tuple.index(44)}

\PYG{n+ne}{ValueError}: tuple.index(x): x not in tuple
\end{sphinxVerbatim}

\sphinxAtStartPar
Les methodes decrites dans la table suivante sont appliquee au lites seulement:


\begin{savenotes}\sphinxattablestart
\centering
\begin{tabulary}{\linewidth}[t]{|T|T|}
\hline
\sphinxstyletheadfamily 
\sphinxAtStartPar
Méthode
&\sphinxstyletheadfamily 
\sphinxAtStartPar
Description
\\
\hline
\sphinxAtStartPar
append()
&
\sphinxAtStartPar
Ajoute un élément en fin de liste
\\
\hline
\sphinxAtStartPar
clear()
&
\sphinxAtStartPar
Supprime tous les éléments de la liste
\\
\hline
\sphinxAtStartPar
copy()
&
\sphinxAtStartPar
Renvoie une copie de la liste
\\
\hline
\sphinxAtStartPar
count()
&
\sphinxAtStartPar
Renvoie le nombre d’éléments avec la valeur spécifiée
\\
\hline
\sphinxAtStartPar
extend()
&
\sphinxAtStartPar
Ajouter les éléments d’une liste, à la fin de la liste actuelle
\\
\hline
\sphinxAtStartPar
index()
&
\sphinxAtStartPar
Renvoie l’indice du premier élément avec la valeur spécifiée
\\
\hline
\sphinxAtStartPar
insert()
&
\sphinxAtStartPar
Ajoute un élément à la position spécifiée
\\
\hline
\sphinxAtStartPar
pop()
&
\sphinxAtStartPar
Supprime l’élément à la position spécifiée
\\
\hline
\sphinxAtStartPar
remove()
&
\sphinxAtStartPar
Supprime le premier élément avec la valeur spécifiée
\\
\hline
\sphinxAtStartPar
reverse()
&
\sphinxAtStartPar
Inverse l’ordre de la liste
\\
\hline
\sphinxAtStartPar
sort()
&
\sphinxAtStartPar
Trie la liste
\\
\hline
\end{tabulary}
\par
\sphinxattableend\end{savenotes}


\subsection{sets}
\label{\detokenize{ch6:sets}}
\sphinxAtStartPar
On a vu que les chaines de caracteres, les liste et tuples sont des sequences ordonnees d’elements


\section{Les modules}
\label{\detokenize{ch7:les-modules}}\label{\detokenize{ch7::doc}}

\subsection{Dictionnaire (dictionnary)}
\label{\detokenize{ch7:dictionnaire-dictionnary}}
\sphinxAtStartPar
Contrairment au objets construits que nous avons vu (\sphinxcode{\sphinxupquote{strings}}, \sphinxcode{\sphinxupquote{lists}}, \sphinxcode{\sphinxupquote{tuples}}), les dictionnaires (\sphinxcode{\sphinxupquote{dictionnary}}) sont utilisés pour stocker des valeurs de données dans des paires clé:valeur. Un dictionnaire est une collection ordonnée , modifiable et qui n’autorise pas les doublons (Depuis la version 3.7 de Python, les dictionnaires demeurent ordonnés. Dans les versions antérieures (Python 3.6 et moins), les dictionnaires ne sont pas ordonnés).
\begin{itemize}
\item {} 
\sphinxAtStartPar
Les dictionnaires sont écrits avec des accolades et ont des clés et des valeurs.  Pour creer un dictionnaire et l’affecter a une variable :

\end{itemize}

\begin{sphinxVerbatim}[commandchars=\\\{\}]
\PYG{n}{var\PYGZus{}dict} \PYG{o}{=} \PYG{p}{\PYGZob{}}\PYG{n}{cle\PYGZus{}1}\PYG{p}{:} \PYG{n}{valeur\PYGZus{}1}\PYG{p}{,} \PYG{n}{cle\PYGZus{}2}\PYG{p}{:}\PYG{n}{valeur\PYGZus{}2}\PYG{p}{,}\PYG{o}{.}\PYG{o}{.}\PYG{o}{.}\PYG{o}{.}\PYG{p}{,} \PYG{n}{cle\PYGZus{}n}\PYG{p}{:}\PYG{n}{valeur\PYGZus{}n}\PYG{p}{\PYGZcb{}}
\end{sphinxVerbatim}
\begin{itemize}
\item {} 
\sphinxAtStartPar
Un dictionnaire vide est un dictionnaire qui contient 0 element (\{\} ou dict());

\item {} 
\sphinxAtStartPar
Les dictionnaires sont \sphinxstylestrong{mutable}, on peut modifier leur contenu et leur taille.

\end{itemize}

\begin{sphinxVerbatim}[commandchars=\\\{\}]
\PYG{c+c1}{\PYGZsh{} dictionnaire vide}
\PYG{n}{var} \PYG{o}{=} \PYG{p}{\PYGZob{}}\PYG{p}{\PYGZcb{}} \PYG{c+c1}{\PYGZsh{} ou var = dict()}
\PYG{n+nb}{print}\PYG{p}{(}\PYG{n+nb}{type}\PYG{p}{(}\PYG{n}{var}\PYG{p}{)}\PYG{p}{)}
\PYG{n}{les\PYGZus{}jours} \PYG{o}{=} \PYG{p}{\PYGZob{}}\PYG{l+m+mi}{1}\PYG{p}{:}\PYG{l+s+s2}{\PYGZdq{}}\PYG{l+s+s2}{lundi}\PYG{l+s+s2}{\PYGZdq{}}\PYG{p}{,} \PYG{l+m+mi}{2}\PYG{p}{:}\PYG{l+s+s2}{\PYGZdq{}}\PYG{l+s+s2}{mardi}\PYG{l+s+s2}{\PYGZdq{}}\PYG{p}{,} \PYG{l+m+mi}{3}\PYG{p}{:}\PYG{l+s+s2}{\PYGZdq{}}\PYG{l+s+s2}{mercredi}\PYG{l+s+s2}{\PYGZdq{}}\PYG{p}{,} \PYG{l+m+mi}{4}\PYG{p}{:}\PYG{l+s+s2}{\PYGZdq{}}\PYG{l+s+s2}{jeudi}\PYG{l+s+s2}{\PYGZdq{}}\PYG{p}{,} \PYG{l+m+mi}{5}\PYG{p}{:}\PYG{l+s+s2}{\PYGZdq{}}\PYG{l+s+s2}{vendredi}\PYG{l+s+s2}{\PYGZdq{}}\PYG{p}{,} \PYG{l+m+mi}{6}\PYG{p}{:}\PYG{l+s+s2}{\PYGZdq{}}\PYG{l+s+s2}{samedi}\PYG{l+s+s2}{\PYGZdq{}}\PYG{p}{,} \PYG{l+m+mi}{7}\PYG{p}{:}\PYG{l+s+s2}{\PYGZdq{}}\PYG{l+s+s2}{dimanche}\PYG{l+s+s2}{\PYGZdq{}}\PYG{p}{\PYGZcb{}}
\PYG{n+nb}{print}\PYG{p}{(}\PYG{n}{les\PYGZus{}jours}\PYG{p}{)}
\end{sphinxVerbatim}

\begin{sphinxVerbatim}[commandchars=\\\{\}]
\PYGZlt{}class \PYGZsq{}dict\PYGZsq{}\PYGZgt{}
\PYGZob{}1: \PYGZsq{}lundi\PYGZsq{}, 2: \PYGZsq{}mardi\PYGZsq{}, 3: \PYGZsq{}mercredi\PYGZsq{}, 4: \PYGZsq{}jeudi\PYGZsq{}, 5: \PYGZsq{}vendredi\PYGZsq{}, 6: \PYGZsq{}samedi\PYGZsq{}, 7: \PYGZsq{}dimanche\PYGZsq{}\PYGZcb{}
\end{sphinxVerbatim}

\begin{sphinxVerbatim}[commandchars=\\\{\}]
\PYG{n}{les\PYGZus{}jours2} \PYG{o}{=} \PYG{p}{\PYGZob{}}\PYG{p}{\PYGZcb{}}

\PYG{n}{les\PYGZus{}jours2}\PYG{p}{[}\PYG{l+m+mi}{1}\PYG{p}{]} \PYG{o}{=} \PYG{l+s+s1}{\PYGZsq{}}\PYG{l+s+s1}{lundi}\PYG{l+s+s1}{\PYGZsq{}}
\PYG{n}{les\PYGZus{}jours2}\PYG{p}{[}\PYG{l+m+mi}{2}\PYG{p}{]} \PYG{o}{=} \PYG{l+s+s1}{\PYGZsq{}}\PYG{l+s+s1}{mardi}\PYG{l+s+s1}{\PYGZsq{}}
\PYG{n}{les\PYGZus{}jours2}\PYG{p}{[}\PYG{l+m+mi}{3}\PYG{p}{]} \PYG{o}{=} \PYG{l+s+s1}{\PYGZsq{}}\PYG{l+s+s1}{mercredi}\PYG{l+s+s1}{\PYGZsq{}}
\PYG{n}{les\PYGZus{}jours2}\PYG{p}{[}\PYG{l+m+mi}{4}\PYG{p}{]} \PYG{o}{=} \PYG{l+s+s1}{\PYGZsq{}}\PYG{l+s+s1}{jeudi}\PYG{l+s+s1}{\PYGZsq{}}
\PYG{n}{les\PYGZus{}jours2}\PYG{p}{[}\PYG{l+m+mi}{5}\PYG{p}{]} \PYG{o}{=} \PYG{l+s+s1}{\PYGZsq{}}\PYG{l+s+s1}{vendredi}\PYG{l+s+s1}{\PYGZsq{}}
\PYG{n}{les\PYGZus{}jours2}\PYG{p}{[}\PYG{l+m+mi}{6}\PYG{p}{]} \PYG{o}{=} \PYG{l+s+s1}{\PYGZsq{}}\PYG{l+s+s1}{samedi}\PYG{l+s+s1}{\PYGZsq{}}
\PYG{n}{les\PYGZus{}jours2}\PYG{p}{[}\PYG{l+m+mi}{7}\PYG{p}{]} \PYG{o}{=} \PYG{l+s+s1}{\PYGZsq{}}\PYG{l+s+s1}{dimanche}\PYG{l+s+s1}{\PYGZsq{}}

\PYG{n+nb}{print}\PYG{p}{(}\PYG{n}{les\PYGZus{}jours2}\PYG{p}{)}
\end{sphinxVerbatim}

\begin{sphinxVerbatim}[commandchars=\\\{\}]
\PYGZob{}1: \PYGZsq{}lundi\PYGZsq{}, 2: \PYGZsq{}mardi\PYGZsq{}, 3: \PYGZsq{}mercredi\PYGZsq{}, 4: \PYGZsq{}jeudi\PYGZsq{}, 5: \PYGZsq{}vendredi\PYGZsq{}, 6: \PYGZsq{}samedi\PYGZsq{}, 7: \PYGZsq{}dimanche\PYGZsq{}\PYGZcb{}
\end{sphinxVerbatim}

\sphinxAtStartPar
les cles et les valeurs peuvent etre de n’importe quel types de donnees. Par exemple on peut creer le dictionnaire suivant:

\begin{sphinxVerbatim}[commandchars=\\\{\}]
\PYG{n}{les\PYGZus{}jours3} \PYG{o}{=} \PYG{p}{\PYGZob{}}\PYG{p}{\PYGZcb{}}

\PYG{n}{les\PYGZus{}jours3}\PYG{p}{[}\PYG{l+s+s1}{\PYGZsq{}}\PYG{l+s+s1}{lundi}\PYG{l+s+s1}{\PYGZsq{}}\PYG{p}{]} \PYG{o}{=} \PYG{l+m+mi}{1}
\PYG{n}{les\PYGZus{}jours3}\PYG{p}{[}\PYG{l+s+s1}{\PYGZsq{}}\PYG{l+s+s1}{mardi}\PYG{l+s+s1}{\PYGZsq{}}\PYG{p}{]} \PYG{o}{=} \PYG{l+m+mi}{2}
\PYG{n}{les\PYGZus{}jours3}\PYG{p}{[}\PYG{l+s+s1}{\PYGZsq{}}\PYG{l+s+s1}{mercredi}\PYG{l+s+s1}{\PYGZsq{}}\PYG{p}{]} \PYG{o}{=} \PYG{l+m+mi}{3}
\PYG{n}{les\PYGZus{}jours3}\PYG{p}{[}\PYG{l+s+s1}{\PYGZsq{}}\PYG{l+s+s1}{jeudi}\PYG{l+s+s1}{\PYGZsq{}}\PYG{p}{]} \PYG{o}{=} \PYG{l+m+mi}{4}
\PYG{n}{les\PYGZus{}jours3}\PYG{p}{[}\PYG{l+s+s1}{\PYGZsq{}}\PYG{l+s+s1}{vendredi}\PYG{l+s+s1}{\PYGZsq{}}\PYG{p}{]} \PYG{o}{=} \PYG{l+m+mi}{5}
\PYG{n}{les\PYGZus{}jours3}\PYG{p}{[}\PYG{l+s+s1}{\PYGZsq{}}\PYG{l+s+s1}{samedi}\PYG{l+s+s1}{\PYGZsq{}}\PYG{p}{]} \PYG{o}{=} \PYG{l+m+mi}{6}
\PYG{n}{les\PYGZus{}jours3}\PYG{p}{[}\PYG{l+s+s1}{\PYGZsq{}}\PYG{l+s+s1}{dimanche}\PYG{l+s+s1}{\PYGZsq{}}\PYG{p}{]} \PYG{o}{=} \PYG{l+m+mi}{7}

\PYG{n+nb}{print}\PYG{p}{(}\PYG{n}{les\PYGZus{}jours3}\PYG{p}{)}
\end{sphinxVerbatim}

\begin{sphinxVerbatim}[commandchars=\\\{\}]
\PYGZob{}\PYGZsq{}lundi\PYGZsq{}: 1, \PYGZsq{}mardi\PYGZsq{}: 2, \PYGZsq{}mercredi\PYGZsq{}: 3, \PYGZsq{}jeudi\PYGZsq{}: 4, \PYGZsq{}vendredi\PYGZsq{}: 5, \PYGZsq{}samedi\PYGZsq{}: 6, \PYGZsq{}dimanche\PYGZsq{}: 7\PYGZcb{}
\end{sphinxVerbatim}

\sphinxAtStartPar
Contrairment aux chaines de caracteres et aux listes, pour acceder a une valeur dans un dictionnaire nous deveons utiliser les cles. Par exemple, si nous voulons acceder a la valeur associee a la cle \sphinxcode{\sphinxupquote{jeudi}} dans \sphinxcode{\sphinxupquote{les\_jours3}}, la maniere de le faire est la suivante:

\begin{sphinxVerbatim}[commandchars=\\\{\}]
\PYG{n+nb}{print}\PYG{p}{(}\PYG{n}{les\PYGZus{}jours3}\PYG{p}{[}\PYG{l+s+s1}{\PYGZsq{}}\PYG{l+s+s1}{jeudi}\PYG{l+s+s1}{\PYGZsq{}}\PYG{p}{]}\PYG{p}{)}
\end{sphinxVerbatim}

\begin{sphinxVerbatim}[commandchars=\\\{\}]
4
\end{sphinxVerbatim}

\sphinxAtStartPar
Les dictionnaires ne peuvent pas avoir des cles dupliqee (les valeurs peuvent etre dupliquees). En effet, les valeurs en double écraseront les valeurs existantes. Illustrons ca avec l’exemple suivant:

\begin{sphinxVerbatim}[commandchars=\\\{\}]
\PYG{n}{les\PYGZus{}jours4} \PYG{o}{=} \PYG{p}{\PYGZob{}}\PYG{p}{\PYGZcb{}}

\PYG{n}{les\PYGZus{}jours4}\PYG{p}{[}\PYG{l+s+s1}{\PYGZsq{}}\PYG{l+s+s1}{lundi}\PYG{l+s+s1}{\PYGZsq{}}\PYG{p}{]} \PYG{o}{=} \PYG{l+m+mi}{0}
\PYG{n}{les\PYGZus{}jours4}\PYG{p}{[}\PYG{l+s+s1}{\PYGZsq{}}\PYG{l+s+s1}{mardi}\PYG{l+s+s1}{\PYGZsq{}}\PYG{p}{]} \PYG{o}{=} \PYG{l+m+mi}{1}
\PYG{n}{les\PYGZus{}jours4}\PYG{p}{[}\PYG{l+s+s1}{\PYGZsq{}}\PYG{l+s+s1}{mardi}\PYG{l+s+s1}{\PYGZsq{}}\PYG{p}{]} \PYG{o}{=} \PYG{l+m+mi}{2}
\PYG{n}{les\PYGZus{}jours4}\PYG{p}{[}\PYG{l+s+s1}{\PYGZsq{}}\PYG{l+s+s1}{mercredi}\PYG{l+s+s1}{\PYGZsq{}}\PYG{p}{]} \PYG{o}{=} \PYG{l+m+mi}{3}
\PYG{n}{les\PYGZus{}jours4}\PYG{p}{[}\PYG{l+s+s1}{\PYGZsq{}}\PYG{l+s+s1}{jeudi}\PYG{l+s+s1}{\PYGZsq{}}\PYG{p}{]} \PYG{o}{=} \PYG{l+m+mi}{4}
\PYG{n}{les\PYGZus{}jours4}\PYG{p}{[}\PYG{l+s+s1}{\PYGZsq{}}\PYG{l+s+s1}{vendredi}\PYG{l+s+s1}{\PYGZsq{}}\PYG{p}{]} \PYG{o}{=} \PYG{l+m+mi}{5}
\PYG{n}{les\PYGZus{}jours4}\PYG{p}{[}\PYG{l+s+s1}{\PYGZsq{}}\PYG{l+s+s1}{samedi}\PYG{l+s+s1}{\PYGZsq{}}\PYG{p}{]} \PYG{o}{=} \PYG{l+m+mi}{6}
\PYG{n}{les\PYGZus{}jours4}\PYG{p}{[}\PYG{l+s+s1}{\PYGZsq{}}\PYG{l+s+s1}{dimanche}\PYG{l+s+s1}{\PYGZsq{}}\PYG{p}{]} \PYG{o}{=} \PYG{l+m+mi}{7}

\PYG{n+nb}{print}\PYG{p}{(}\PYG{n}{les\PYGZus{}jours4}\PYG{p}{)}
\end{sphinxVerbatim}

\begin{sphinxVerbatim}[commandchars=\\\{\}]
\PYGZob{}\PYGZsq{}lundi\PYGZsq{}: 0, \PYGZsq{}mardi\PYGZsq{}: 2, \PYGZsq{}mercredi\PYGZsq{}: 3, \PYGZsq{}jeudi\PYGZsq{}: 4, \PYGZsq{}vendredi\PYGZsq{}: 5, \PYGZsq{}samedi\PYGZsq{}: 6, \PYGZsq{}dimanche\PYGZsq{}: 7\PYGZcb{}
\end{sphinxVerbatim}

\begin{sphinxVerbatim}[commandchars=\\\{\}]
\PYG{n}{les\PYGZus{}jours5} \PYG{o}{=} \PYG{p}{\PYGZob{}}\PYG{p}{\PYGZcb{}}

\PYG{n}{les\PYGZus{}jours5}\PYG{p}{[}\PYG{l+s+s1}{\PYGZsq{}}\PYG{l+s+s1}{lundi}\PYG{l+s+s1}{\PYGZsq{}}\PYG{p}{]} \PYG{o}{=} \PYG{l+m+mi}{0}
\PYG{n}{les\PYGZus{}jours5}\PYG{p}{[}\PYG{l+s+s1}{\PYGZsq{}}\PYG{l+s+s1}{mardi}\PYG{l+s+s1}{\PYGZsq{}}\PYG{p}{]} \PYG{o}{=} \PYG{l+m+mi}{0}
\PYG{n}{les\PYGZus{}jours5}\PYG{p}{[}\PYG{l+s+s1}{\PYGZsq{}}\PYG{l+s+s1}{mercredi}\PYG{l+s+s1}{\PYGZsq{}}\PYG{p}{]} \PYG{o}{=} \PYG{l+m+mi}{0}
\PYG{n}{les\PYGZus{}jours5}\PYG{p}{[}\PYG{l+s+s1}{\PYGZsq{}}\PYG{l+s+s1}{jeudi}\PYG{l+s+s1}{\PYGZsq{}}\PYG{p}{]} \PYG{o}{=} \PYG{l+m+mi}{5}
\PYG{n}{les\PYGZus{}jours5}\PYG{p}{[}\PYG{l+s+s1}{\PYGZsq{}}\PYG{l+s+s1}{vendredi}\PYG{l+s+s1}{\PYGZsq{}}\PYG{p}{]} \PYG{o}{=} \PYG{l+m+mi}{0}
\PYG{n}{les\PYGZus{}jours5}\PYG{p}{[}\PYG{l+s+s1}{\PYGZsq{}}\PYG{l+s+s1}{samedi}\PYG{l+s+s1}{\PYGZsq{}}\PYG{p}{]} \PYG{o}{=} \PYG{l+m+mi}{0}
\PYG{n}{les\PYGZus{}jours5}\PYG{p}{[}\PYG{l+s+s1}{\PYGZsq{}}\PYG{l+s+s1}{dimanche}\PYG{l+s+s1}{\PYGZsq{}}\PYG{p}{]} \PYG{o}{=} \PYG{l+m+mi}{0}

\PYG{n+nb}{print}\PYG{p}{(}\PYG{n}{les\PYGZus{}jours5}\PYG{p}{)}
\end{sphinxVerbatim}

\begin{sphinxVerbatim}[commandchars=\\\{\}]
\PYGZob{}\PYGZsq{}lundi\PYGZsq{}: 0, \PYGZsq{}mardi\PYGZsq{}: 0, \PYGZsq{}mercredi\PYGZsq{}: 0, \PYGZsq{}jeudi\PYGZsq{}: 5, \PYGZsq{}vendredi\PYGZsq{}: 0, \PYGZsq{}samedi\PYGZsq{}: 0, \PYGZsq{}dimanche\PYGZsq{}: 0\PYGZcb{}
\end{sphinxVerbatim}


\subsection{Operations sur les dictionnaires}
\label{\detokenize{ch7:operations-sur-les-dictionnaires}}
\sphinxAtStartPar
Python possède un ensemble de méthodes intégrées que nous pouvons utiliser pour manipuler les dictionnaires:


\begin{savenotes}\sphinxattablestart
\centering
\begin{tabulary}{\linewidth}[t]{|T|T|}
\hline
\sphinxstyletheadfamily 
\sphinxAtStartPar
Methode
&\sphinxstyletheadfamily 
\sphinxAtStartPar
Description
\\
\hline
\sphinxAtStartPar
clear()
&
\sphinxAtStartPar
Supprime tous les éléments du dictionnaire
\\
\hline
\sphinxAtStartPar
copy()
&
\sphinxAtStartPar
Renvoie une copie du dictionnaire
\\
\hline
\sphinxAtStartPar
fromkeys()
&
\sphinxAtStartPar
Renvoie un dictionnaire avec les clés et la valeur spécifiées
\\
\hline
\sphinxAtStartPar
get()
&
\sphinxAtStartPar
Renvoie la valeur de la clé spécifiée
\\
\hline
\sphinxAtStartPar
items()
&
\sphinxAtStartPar
Renvoie une liste contenant un tuple pour chaque paire clé\sphinxhyphen{}valeur
\\
\hline
\sphinxAtStartPar
keys()
&
\sphinxAtStartPar
Retourne une liste contenant les clés du dictionnaire
\\
\hline
\sphinxAtStartPar
pop()
&
\sphinxAtStartPar
Supprime l’élément avec la clé spécifiée
\\
\hline
\sphinxAtStartPar
popitem()
&
\sphinxAtStartPar
Supprime la dernière paire clé\sphinxhyphen{}valeur insérée
\\
\hline
\sphinxAtStartPar
setdefault()
&
\sphinxAtStartPar
Renvoie la valeur de la clé spécifiée. Si la clé n’existe pas : insérez la clé, avec la valeur spécifiée
\\
\hline
\sphinxAtStartPar
update()
&
\sphinxAtStartPar
Met à jour le dictionnaire avec les paires clé\sphinxhyphen{}valeur spécifiées
\\
\hline
\sphinxAtStartPar
values()
&
\sphinxAtStartPar
Renvoie une liste de toutes les valeurs du dictionnaire
\\
\hline
\end{tabulary}
\par
\sphinxattableend\end{savenotes}

\begin{sphinxVerbatim}[commandchars=\\\{\}]
\PYG{c+c1}{\PYGZsh{} clear}
\PYG{n}{les\PYGZus{}jours}\PYG{o}{.}\PYG{n}{clear}\PYG{p}{(}\PYG{p}{)}

\PYG{n+nb}{print}\PYG{p}{(}\PYG{n}{les\PYGZus{}jours}\PYG{p}{)}
\end{sphinxVerbatim}

\begin{sphinxVerbatim}[commandchars=\\\{\}]
\PYGZob{}\PYGZcb{}
\end{sphinxVerbatim}

\begin{sphinxVerbatim}[commandchars=\\\{\}]
\PYG{c+c1}{\PYGZsh{} copy}
\PYG{n}{les\PYGZus{}jours2\PYGZus{}copie} \PYG{o}{=} \PYG{n}{les\PYGZus{}jours2}\PYG{o}{.}\PYG{n}{copy}\PYG{p}{(}\PYG{p}{)}

\PYG{n+nb}{print}\PYG{p}{(}\PYG{n}{les\PYGZus{}jours2\PYGZus{}copie}\PYG{p}{)}
\end{sphinxVerbatim}

\begin{sphinxVerbatim}[commandchars=\\\{\}]
\PYGZob{}1: \PYGZsq{}lundi\PYGZsq{}, 2: \PYGZsq{}mardi\PYGZsq{}, 3: \PYGZsq{}mercredi\PYGZsq{}, 4: \PYGZsq{}jeudi\PYGZsq{}, 5: \PYGZsq{}vendredi\PYGZsq{}, 6: \PYGZsq{}samedi\PYGZsq{}, 7: \PYGZsq{}dimanche\PYGZsq{}\PYGZcb{}
\end{sphinxVerbatim}

\begin{sphinxVerbatim}[commandchars=\\\{\}]
\PYG{c+c1}{\PYGZsh{} fromkeys}
\PYG{c+c1}{\PYGZsh{} creer un dictionnaire avec des cles mais pas de valeurs (None)}

\PYG{n}{cles} \PYG{o}{=} \PYG{n+nb}{range}\PYG{p}{(}\PYG{l+m+mi}{6}\PYG{p}{)}
\PYG{n}{dict\PYGZus{}vide} \PYG{o}{=} \PYG{n+nb}{dict}\PYG{o}{.}\PYG{n}{fromkeys}\PYG{p}{(}\PYG{n}{cles}\PYG{p}{)}
\PYG{n+nb}{print}\PYG{p}{(}\PYG{n}{dict\PYGZus{}vide}\PYG{p}{)}
\end{sphinxVerbatim}

\begin{sphinxVerbatim}[commandchars=\\\{\}]
\PYGZob{}0: None, 1: None, 2: None, 3: None, 4: None, 5: None\PYGZcb{}
\end{sphinxVerbatim}

\begin{sphinxVerbatim}[commandchars=\\\{\}]
\PYG{c+c1}{\PYGZsh{} fromkeys}
\PYG{c+c1}{\PYGZsh{} creer un dictionnaire avec des cles a la meme valeur}
\PYG{n}{var} \PYG{o}{=} \PYG{p}{[}\PYG{l+s+s1}{\PYGZsq{}}\PYG{l+s+s1}{a}\PYG{l+s+s1}{\PYGZsq{}}\PYG{p}{,} \PYG{l+s+s1}{\PYGZsq{}}\PYG{l+s+s1}{e}\PYG{l+s+s1}{\PYGZsq{}}\PYG{p}{,} \PYG{l+s+s1}{\PYGZsq{}}\PYG{l+s+s1}{i}\PYG{l+s+s1}{\PYGZsq{}}\PYG{p}{,} \PYG{l+s+s1}{\PYGZsq{}}\PYG{l+s+s1}{o}\PYG{l+s+s1}{\PYGZsq{}}\PYG{p}{,} \PYG{l+s+s1}{\PYGZsq{}}\PYG{l+s+s1}{u}\PYG{l+s+s1}{\PYGZsq{}}\PYG{p}{,} \PYG{l+s+s1}{\PYGZsq{}}\PYG{l+s+s1}{y}\PYG{l+s+s1}{\PYGZsq{}}\PYG{p}{]}
\PYG{n}{valeur} \PYG{o}{=} \PYG{l+s+s1}{\PYGZsq{}}\PYG{l+s+s1}{voyelle}\PYG{l+s+s1}{\PYGZsq{}}
\PYG{n}{dict\PYGZus{}voyelles} \PYG{o}{=} \PYG{n+nb}{dict}\PYG{o}{.}\PYG{n}{fromkeys}\PYG{p}{(}\PYG{n}{var}\PYG{p}{,} \PYG{n}{valeur}\PYG{p}{)}

\PYG{n+nb}{print}\PYG{p}{(}\PYG{n}{dict\PYGZus{}voyelles}\PYG{p}{)}
\end{sphinxVerbatim}

\begin{sphinxVerbatim}[commandchars=\\\{\}]
\PYGZob{}\PYGZsq{}a\PYGZsq{}: \PYGZsq{}voyelle\PYGZsq{}, \PYGZsq{}e\PYGZsq{}: \PYGZsq{}voyelle\PYGZsq{}, \PYGZsq{}i\PYGZsq{}: \PYGZsq{}voyelle\PYGZsq{}, \PYGZsq{}o\PYGZsq{}: \PYGZsq{}voyelle\PYGZsq{}, \PYGZsq{}u\PYGZsq{}: \PYGZsq{}voyelle\PYGZsq{}, \PYGZsq{}y\PYGZsq{}: \PYGZsq{}voyelle\PYGZsq{}\PYGZcb{}
\end{sphinxVerbatim}

\begin{sphinxVerbatim}[commandchars=\\\{\}]
\PYG{c+c1}{\PYGZsh{} get}
\PYG{n+nb}{print}\PYG{p}{(}\PYG{n}{les\PYGZus{}jours2}\PYG{o}{.}\PYG{n}{get}\PYG{p}{(}\PYG{l+m+mi}{3}\PYG{p}{)}\PYG{p}{)}
\end{sphinxVerbatim}

\begin{sphinxVerbatim}[commandchars=\\\{\}]
mercredi
\end{sphinxVerbatim}

\begin{sphinxVerbatim}[commandchars=\\\{\}]
\PYG{c+c1}{\PYGZsh{} items}
\PYG{n+nb}{print}\PYG{p}{(}\PYG{n}{les\PYGZus{}jours2}\PYG{o}{.}\PYG{n}{items}\PYG{p}{(}\PYG{p}{)}\PYG{p}{)}
\end{sphinxVerbatim}

\begin{sphinxVerbatim}[commandchars=\\\{\}]
dict\PYGZus{}items([(1, \PYGZsq{}lundi\PYGZsq{}), (2, \PYGZsq{}mardi\PYGZsq{}), (3, \PYGZsq{}mercredi\PYGZsq{}), (4, \PYGZsq{}jeudi\PYGZsq{}), (5, \PYGZsq{}vendredi\PYGZsq{}), (6, \PYGZsq{}samedi\PYGZsq{}), (7, \PYGZsq{}dimanche\PYGZsq{})])
\end{sphinxVerbatim}

\begin{sphinxVerbatim}[commandchars=\\\{\}]
\PYG{c+c1}{\PYGZsh{} keys}
\PYG{n+nb}{print}\PYG{p}{(}\PYG{n}{les\PYGZus{}jours2}\PYG{o}{.}\PYG{n}{keys}\PYG{p}{(}\PYG{p}{)}\PYG{p}{)}
\end{sphinxVerbatim}

\begin{sphinxVerbatim}[commandchars=\\\{\}]
dict\PYGZus{}keys([1, 2, 3, 4, 5, 6, 7])
\end{sphinxVerbatim}

\begin{sphinxVerbatim}[commandchars=\\\{\}]
\PYG{c+c1}{\PYGZsh{} values}
\PYG{n+nb}{print}\PYG{p}{(}\PYG{n}{les\PYGZus{}jours2}\PYG{o}{.}\PYG{n}{values}\PYG{p}{(}\PYG{p}{)}\PYG{p}{)}
\end{sphinxVerbatim}

\begin{sphinxVerbatim}[commandchars=\\\{\}]
dict\PYGZus{}values([\PYGZsq{}lundi\PYGZsq{}, \PYGZsq{}mardi\PYGZsq{}, \PYGZsq{}mercredi\PYGZsq{}, \PYGZsq{}jeudi\PYGZsq{}, \PYGZsq{}vendredi\PYGZsq{}, \PYGZsq{}samedi\PYGZsq{}, \PYGZsq{}dimanche\PYGZsq{}])
\end{sphinxVerbatim}

\begin{sphinxVerbatim}[commandchars=\\\{\}]
\PYG{c+c1}{\PYGZsh{} pop}
\PYG{n}{les\PYGZus{}jours2}\PYG{o}{.}\PYG{n}{pop}\PYG{p}{(}\PYG{l+m+mi}{4}\PYG{p}{)}

\PYG{n+nb}{print}\PYG{p}{(}\PYG{n}{les\PYGZus{}jours2}\PYG{p}{)}
\end{sphinxVerbatim}

\begin{sphinxVerbatim}[commandchars=\\\{\}]
\PYGZob{}1: \PYGZsq{}lundi\PYGZsq{}, 2: \PYGZsq{}mardi\PYGZsq{}, 3: \PYGZsq{}mercredi\PYGZsq{}, 5: \PYGZsq{}vendredi\PYGZsq{}, 6: \PYGZsq{}samedi\PYGZsq{}, 7: \PYGZsq{}dimanche\PYGZsq{}\PYGZcb{}
\end{sphinxVerbatim}

\begin{sphinxVerbatim}[commandchars=\\\{\}]
\PYG{c+c1}{\PYGZsh{} popitem }
\PYG{c+c1}{\PYGZsh{} inserer une cles\PYGZhy{}valeur arbitraire}
\PYG{n}{les\PYGZus{}jours2}\PYG{p}{[}\PYG{l+s+s1}{\PYGZsq{}}\PYG{l+s+s1}{la dirniere cles ajoutee}\PYG{l+s+s1}{\PYGZsq{}}\PYG{p}{]} \PYG{o}{=} \PYG{l+s+s1}{\PYGZsq{}}\PYG{l+s+s1}{La valeur associee a la derniere cles}\PYG{l+s+s1}{\PYGZsq{}}

\PYG{n+nb}{print}\PYG{p}{(}\PYG{n}{les\PYGZus{}jours2}\PYG{p}{)}
\end{sphinxVerbatim}

\begin{sphinxVerbatim}[commandchars=\\\{\}]
\PYGZob{}1: \PYGZsq{}lundi\PYGZsq{}, 2: \PYGZsq{}mardi\PYGZsq{}, 3: \PYGZsq{}mercredi\PYGZsq{}, 5: \PYGZsq{}vendredi\PYGZsq{}, 6: \PYGZsq{}samedi\PYGZsq{}, 7: \PYGZsq{}dimanche\PYGZsq{}, \PYGZsq{}la dirniere cles ajoutee\PYGZsq{}: \PYGZsq{}La valeur associee a la derniere cles\PYGZsq{}\PYGZcb{}
\end{sphinxVerbatim}

\begin{sphinxVerbatim}[commandchars=\\\{\}]
\PYG{c+c1}{\PYGZsh{} popitem}
\PYG{n}{les\PYGZus{}jours2}\PYG{o}{.}\PYG{n}{popitem}\PYG{p}{(}\PYG{p}{)}

\PYG{n+nb}{print}\PYG{p}{(}\PYG{n}{les\PYGZus{}jours2}\PYG{p}{)}
\end{sphinxVerbatim}

\begin{sphinxVerbatim}[commandchars=\\\{\}]
\PYGZob{}1: \PYGZsq{}lundi\PYGZsq{}, 2: \PYGZsq{}mardi\PYGZsq{}, 3: \PYGZsq{}mercredi\PYGZsq{}, 5: \PYGZsq{}vendredi\PYGZsq{}, 6: \PYGZsq{}samedi\PYGZsq{}, 7: \PYGZsq{}dimanche\PYGZsq{}\PYGZcb{}
\end{sphinxVerbatim}

\begin{sphinxVerbatim}[commandchars=\\\{\}]
\PYG{c+c1}{\PYGZsh{} setdefault()}
\PYG{c+c1}{\PYGZsh{} si la cles existe}
\PYG{n+nb}{print}\PYG{p}{(}\PYG{n}{les\PYGZus{}jours2}\PYG{o}{.}\PYG{n}{setdefault}\PYG{p}{(}\PYG{l+m+mi}{1}\PYG{p}{,} \PYG{l+s+s2}{\PYGZdq{}}\PYG{l+s+s2}{LUNDI}\PYG{l+s+s2}{\PYGZdq{}}\PYG{p}{)}\PYG{p}{)}
\PYG{c+c1}{\PYGZsh{} pusique la cles existe, la valeur ne va pas changer. Elle va etre affichee. Le dictionnaire ne va pas changer aussi.}
\PYG{n+nb}{print}\PYG{p}{(}\PYG{n}{les\PYGZus{}jours2}\PYG{p}{)}
\end{sphinxVerbatim}

\begin{sphinxVerbatim}[commandchars=\\\{\}]
lundi
\PYGZob{}1: \PYGZsq{}lundi\PYGZsq{}, 2: \PYGZsq{}mardi\PYGZsq{}, 3: \PYGZsq{}mercredi\PYGZsq{}, 5: \PYGZsq{}vendredi\PYGZsq{}, 6: \PYGZsq{}samedi\PYGZsq{}, 7: \PYGZsq{}dimanche\PYGZsq{}\PYGZcb{}
\end{sphinxVerbatim}

\begin{sphinxVerbatim}[commandchars=\\\{\}]
\PYG{c+c1}{\PYGZsh{} setdefault()}
\PYG{c+c1}{\PYGZsh{} si la cles existe}
\PYG{c+c1}{\PYGZsh{} puisque la cles 4 n\PYGZsq{}existe pas. La valeur \PYGZsq{}mercredi\PYGZsq{} va etre affichee. Le dictionnaire va etre modifie aussi}
\PYG{n+nb}{print}\PYG{p}{(}\PYG{n}{les\PYGZus{}jours2}\PYG{o}{.}\PYG{n}{setdefault}\PYG{p}{(}\PYG{l+m+mi}{4}\PYG{p}{,} \PYG{l+s+s2}{\PYGZdq{}}\PYG{l+s+s2}{mercredi}\PYG{l+s+s2}{\PYGZdq{}}\PYG{p}{)}\PYG{p}{)}
\PYG{n+nb}{print}\PYG{p}{(}\PYG{n}{les\PYGZus{}jours2}\PYG{p}{)}
\end{sphinxVerbatim}

\begin{sphinxVerbatim}[commandchars=\\\{\}]
mercredi
\PYGZob{}1: \PYGZsq{}lundi\PYGZsq{}, 2: \PYGZsq{}mardi\PYGZsq{}, 3: \PYGZsq{}mercredi\PYGZsq{}, 5: \PYGZsq{}vendredi\PYGZsq{}, 6: \PYGZsq{}samedi\PYGZsq{}, 7: \PYGZsq{}dimanche\PYGZsq{}, 4: \PYGZsq{}mercredi\PYGZsq{}\PYGZcb{}
\end{sphinxVerbatim}

\begin{sphinxVerbatim}[commandchars=\\\{\}]
\PYG{c+c1}{\PYGZsh{} update}
\PYG{n}{premier}
\end{sphinxVerbatim}

\sphinxAtStartPar
La fonction intégrée \sphinxcode{\sphinxupquote{len()}} est aussi appliquable pour les dictionnaires. Si l’on désire déterminer le nombre d’elements:

\begin{sphinxVerbatim}[commandchars=\\\{\}]
\PYG{n+nb}{len}\PYG{p}{(}\PYG{n}{les\PYGZus{}jours}\PYG{p}{)}
\end{sphinxVerbatim}

\begin{sphinxVerbatim}[commandchars=\\\{\}]
7
\end{sphinxVerbatim}

\sphinxAtStartPar
La varaible \sphinxcode{\sphinxupquote{les\_jours}} contient 7 elements.

\sphinxAtStartPar
Si l’on veut tester l’appartenece d’un element a une liste/tuple on utilise l’operateur \sphinxcode{\sphinxupquote{in}}. L’expression est la suivante : \sphinxcode{\sphinxupquote{element in liste}}. Cela nous renvoi \sphinxcode{\sphinxupquote{True}} si \sphinxcode{\sphinxupquote{element}}est dans \sphinxcode{\sphinxupquote{liste}}, sinon \sphinxcode{\sphinxupquote{False}}. On peut aussi ecrire \sphinxcode{\sphinxupquote{element not in liste}} pour tester si l’element n’est pas dans \sphinxcode{\sphinxupquote{liste}} (\sphinxcode{\sphinxupquote{True}}) ou s’il est dans \sphinxcode{\sphinxupquote{liste}} (\sphinxcode{\sphinxupquote{False}}). Voici quelques exemples:

\begin{sphinxVerbatim}[commandchars=\\\{\}]
\PYG{n}{les\PYGZus{}jours}
\end{sphinxVerbatim}

\begin{sphinxVerbatim}[commandchars=\\\{\}]
\PYGZob{}1: \PYGZsq{}lundi\PYGZsq{},
 2: \PYGZsq{}mardi\PYGZsq{},
 3: \PYGZsq{}mercredi\PYGZsq{},
 4: \PYGZsq{}jeudi\PYGZsq{},
 5: \PYGZsq{}vendredi\PYGZsq{},
 6: \PYGZsq{}samedi\PYGZsq{},
 7: \PYGZsq{}dimanche\PYGZsq{}\PYGZcb{}
\end{sphinxVerbatim}

\begin{sphinxVerbatim}[commandchars=\\\{\}]
\PYG{n}{les\PYGZus{}jours}\PYG{o}{.}\PYG{n}{values}\PYG{p}{(}\PYG{p}{)}
\end{sphinxVerbatim}

\begin{sphinxVerbatim}[commandchars=\\\{\}]
dict\PYGZus{}values([\PYGZsq{}lundi\PYGZsq{}, \PYGZsq{}mardi\PYGZsq{}, \PYGZsq{}mercredi\PYGZsq{}, \PYGZsq{}jeudi\PYGZsq{}, \PYGZsq{}vendredi\PYGZsq{}, \PYGZsq{}samedi\PYGZsq{}, \PYGZsq{}dimanche\PYGZsq{}])
\end{sphinxVerbatim}

\begin{sphinxVerbatim}[commandchars=\\\{\}]
\PYG{n}{un\PYGZus{}tuple} \PYG{o}{=} \PYG{p}{(}\PYG{l+m+mi}{1}\PYG{p}{,} \PYG{l+s+s1}{\PYGZsq{}}\PYG{l+s+s1}{bonjour}\PYG{l+s+s1}{\PYGZsq{}}\PYG{p}{,} \PYG{l+m+mf}{4.5}\PYG{p}{,} \PYG{k+kc}{True}\PYG{p}{)}

\PYG{n+nb}{print}\PYG{p}{(}\PYG{l+m+mi}{1} \PYG{o+ow}{in} \PYG{n}{un\PYGZus{}tuple}\PYG{p}{)}

\PYG{n+nb}{print}\PYG{p}{(}\PYG{l+m+mi}{2} \PYG{o+ow}{in} \PYG{n}{un\PYGZus{}tuple}\PYG{p}{)}

\PYG{n+nb}{print}\PYG{p}{(}\PYG{l+s+s1}{\PYGZsq{}}\PYG{l+s+s1}{o}\PYG{l+s+s1}{\PYGZsq{}} \PYG{o+ow}{not} \PYG{o+ow}{in} \PYG{n}{un\PYGZus{}tuple}\PYG{p}{)}

\PYG{n+nb}{print}\PYG{p}{(}\PYG{l+s+s1}{\PYGZsq{}}\PYG{l+s+s1}{x}\PYG{l+s+s1}{\PYGZsq{}} \PYG{o+ow}{not} \PYG{o+ow}{in} \PYG{n}{un\PYGZus{}tuple}\PYG{p}{)}
\end{sphinxVerbatim}

\begin{sphinxVerbatim}[commandchars=\\\{\}]
True
False
True
True
\end{sphinxVerbatim}


\subsection{Quelle est la difference entre listes et tuples?}
\label{\detokenize{ch7:quelle-est-la-difference-entre-listes-et-tuples}}
\sphinxAtStartPar
Les listes et les tuples sont pareils dans la plupart des contextes. Cepandant, la difference primordiale entre les deux et que les listes sont des \sphinxstylestrong{objets mutables} (modifiables) alors que les tuples sont des \sphinxstylestrong{objets immuables} (ne sont pas modifiable). La question qui se pose est donc: qu’est\sphinxhyphen{}ce qu’un objet mutable et un objet immuable?
Parmi les objet immuable en python, on trouve:
\begin{itemize}
\item {} 
\sphinxAtStartPar
Les nombres entiers (int)

\item {} 
\sphinxAtStartPar
Les nombres décimaux (float)

\item {} 
\sphinxAtStartPar
Les chaînes de caractères (str)

\item {} 
\sphinxAtStartPar
Les booléens (bool)

\item {} 
\sphinxAtStartPar
Les tuples (tuple)
La plus part des autres objets que vous allez confronter en python sont mutables.

\end{itemize}

\sphinxAtStartPar
Nous allons illustre ca dans les exemples suivant:
\begin{enumerate}
\sphinxsetlistlabels{\arabic}{enumi}{enumii}{}{.}%
\item {} 
\sphinxAtStartPar
nous allons creer les variables suivantes:

\end{enumerate}
\begin{itemize}
\item {} 
\sphinxAtStartPar
\sphinxcode{\sphinxupquote{var\_chaine = "bonjour tout le monde"}},

\item {} 
\sphinxAtStartPar
\sphinxcode{\sphinxupquote{var\_liste = {[}1, 2, True, \textquotesingle{}bonjour\textquotesingle{}{]}}},

\item {} 
\sphinxAtStartPar
\sphinxcode{\sphinxupquote{var\_tuple = (1, 2, True, \textquotesingle{}bonjour\textquotesingle{})}}.

\end{itemize}
\begin{enumerate}
\sphinxsetlistlabels{\arabic}{enumi}{enumii}{}{.}%
\item {} 
\sphinxAtStartPar
nous qllons essayer de changer (par exemple) le premier element de chaque variable (par un autre element).

\end{enumerate}

\begin{sphinxVerbatim}[commandchars=\\\{\}]
\PYG{n}{var\PYGZus{}chaine} \PYG{o}{=} \PYG{l+s+s2}{\PYGZdq{}}\PYG{l+s+s2}{bonjour tout le monde}\PYG{l+s+s2}{\PYGZdq{}}
\PYG{n}{var\PYGZus{}liste} \PYG{o}{=} \PYG{p}{[}\PYG{l+m+mi}{1}\PYG{p}{,} \PYG{l+m+mi}{2}\PYG{p}{,} \PYG{k+kc}{True}\PYG{p}{,} \PYG{l+s+s1}{\PYGZsq{}}\PYG{l+s+s1}{bonjour}\PYG{l+s+s1}{\PYGZsq{}}\PYG{p}{]}
\PYG{n}{var\PYGZus{}tuple} \PYG{o}{=} \PYG{p}{(}\PYG{l+m+mi}{1}\PYG{p}{,} \PYG{l+m+mi}{2}\PYG{p}{,} \PYG{k+kc}{True}\PYG{p}{,} \PYG{l+s+s1}{\PYGZsq{}}\PYG{l+s+s1}{bonjour}\PYG{l+s+s1}{\PYGZsq{}}\PYG{p}{)}
\end{sphinxVerbatim}

\begin{sphinxVerbatim}[commandchars=\\\{\}]
\PYG{n}{var\PYGZus{}chaine}\PYG{p}{[}\PYG{l+m+mi}{0}\PYG{p}{]} \PYG{o}{=} \PYG{l+s+s1}{\PYGZsq{}}\PYG{l+s+s1}{B}\PYG{l+s+s1}{\PYGZsq{}}
\PYG{n+nb}{print}\PYG{p}{(}\PYG{n}{var\PYGZus{}chaine}\PYG{p}{)}
\end{sphinxVerbatim}

\begin{sphinxVerbatim}[commandchars=\\\{\}]
\PYG{g+gt}{\PYGZhy{}\PYGZhy{}\PYGZhy{}\PYGZhy{}\PYGZhy{}\PYGZhy{}\PYGZhy{}\PYGZhy{}\PYGZhy{}\PYGZhy{}\PYGZhy{}\PYGZhy{}\PYGZhy{}\PYGZhy{}\PYGZhy{}\PYGZhy{}\PYGZhy{}\PYGZhy{}\PYGZhy{}\PYGZhy{}\PYGZhy{}\PYGZhy{}\PYGZhy{}\PYGZhy{}\PYGZhy{}\PYGZhy{}\PYGZhy{}\PYGZhy{}\PYGZhy{}\PYGZhy{}\PYGZhy{}\PYGZhy{}\PYGZhy{}\PYGZhy{}\PYGZhy{}\PYGZhy{}\PYGZhy{}\PYGZhy{}\PYGZhy{}\PYGZhy{}\PYGZhy{}\PYGZhy{}\PYGZhy{}\PYGZhy{}\PYGZhy{}\PYGZhy{}\PYGZhy{}\PYGZhy{}\PYGZhy{}\PYGZhy{}\PYGZhy{}\PYGZhy{}\PYGZhy{}\PYGZhy{}\PYGZhy{}\PYGZhy{}\PYGZhy{}\PYGZhy{}\PYGZhy{}\PYGZhy{}\PYGZhy{}\PYGZhy{}\PYGZhy{}\PYGZhy{}\PYGZhy{}\PYGZhy{}\PYGZhy{}\PYGZhy{}\PYGZhy{}\PYGZhy{}\PYGZhy{}\PYGZhy{}\PYGZhy{}\PYGZhy{}\PYGZhy{}}
\PYG{n+ne}{TypeError}\PYG{g+gWhitespace}{                                 }Traceback (most recent call last)
\PYG{o}{\PYGZti{}}\PYGZbs{}\PYG{n}{AppData}\PYGZbs{}\PYG{n}{Local}\PYGZbs{}\PYG{n}{Temp}\PYGZbs{}\PYG{n}{ipykernel\PYGZus{}232}\PYGZbs{}\PYG{l+m+mf}{4274672008.}\PYG{n}{py} \PYG{o+ow}{in} \PYG{o}{\PYGZlt{}}\PYG{n}{module}\PYG{o}{\PYGZgt{}}
\PYG{n+ne}{\PYGZhy{}\PYGZhy{}\PYGZhy{}\PYGZhy{}\PYGZgt{} }\PYG{l+m+mi}{1} \PYG{n}{var\PYGZus{}chaine}\PYG{p}{[}\PYG{l+m+mi}{0}\PYG{p}{]} \PYG{o}{=} \PYG{l+s+s1}{\PYGZsq{}}\PYG{l+s+s1}{B}\PYG{l+s+s1}{\PYGZsq{}}
\PYG{g+gWhitespace}{      }\PYG{l+m+mi}{2} \PYG{n+nb}{print}\PYG{p}{(}\PYG{n}{var\PYGZus{}chaine}\PYG{p}{)}

\PYG{n+ne}{TypeError}: \PYGZsq{}str\PYGZsq{} object does not support item assignment
\end{sphinxVerbatim}

\begin{sphinxVerbatim}[commandchars=\\\{\}]
\PYG{n}{var\PYGZus{}liste}\PYG{p}{[}\PYG{l+m+mi}{0}\PYG{p}{]} \PYG{o}{=} \PYG{l+m+mi}{3333}
\PYG{n+nb}{print}\PYG{p}{(}\PYG{n}{var\PYGZus{}liste}\PYG{p}{)}
\end{sphinxVerbatim}

\begin{sphinxVerbatim}[commandchars=\\\{\}]
[3333, 2, True, \PYGZsq{}bonjour\PYGZsq{}]
\end{sphinxVerbatim}

\begin{sphinxVerbatim}[commandchars=\\\{\}]
\PYG{n}{var\PYGZus{}tuple}\PYG{p}{[}\PYG{l+m+mi}{0}\PYG{p}{]} \PYG{o}{=} \PYG{l+m+mi}{3333}
\PYG{n+nb}{print}\PYG{p}{(}\PYG{n}{var\PYGZus{}tuple}\PYG{p}{)}
\end{sphinxVerbatim}

\begin{sphinxVerbatim}[commandchars=\\\{\}]
\PYG{g+gt}{\PYGZhy{}\PYGZhy{}\PYGZhy{}\PYGZhy{}\PYGZhy{}\PYGZhy{}\PYGZhy{}\PYGZhy{}\PYGZhy{}\PYGZhy{}\PYGZhy{}\PYGZhy{}\PYGZhy{}\PYGZhy{}\PYGZhy{}\PYGZhy{}\PYGZhy{}\PYGZhy{}\PYGZhy{}\PYGZhy{}\PYGZhy{}\PYGZhy{}\PYGZhy{}\PYGZhy{}\PYGZhy{}\PYGZhy{}\PYGZhy{}\PYGZhy{}\PYGZhy{}\PYGZhy{}\PYGZhy{}\PYGZhy{}\PYGZhy{}\PYGZhy{}\PYGZhy{}\PYGZhy{}\PYGZhy{}\PYGZhy{}\PYGZhy{}\PYGZhy{}\PYGZhy{}\PYGZhy{}\PYGZhy{}\PYGZhy{}\PYGZhy{}\PYGZhy{}\PYGZhy{}\PYGZhy{}\PYGZhy{}\PYGZhy{}\PYGZhy{}\PYGZhy{}\PYGZhy{}\PYGZhy{}\PYGZhy{}\PYGZhy{}\PYGZhy{}\PYGZhy{}\PYGZhy{}\PYGZhy{}\PYGZhy{}\PYGZhy{}\PYGZhy{}\PYGZhy{}\PYGZhy{}\PYGZhy{}\PYGZhy{}\PYGZhy{}\PYGZhy{}\PYGZhy{}\PYGZhy{}\PYGZhy{}\PYGZhy{}\PYGZhy{}\PYGZhy{}}
\PYG{n+ne}{TypeError}\PYG{g+gWhitespace}{                                 }Traceback (most recent call last)
\PYG{o}{\PYGZti{}}\PYGZbs{}\PYG{n}{AppData}\PYGZbs{}\PYG{n}{Local}\PYGZbs{}\PYG{n}{Temp}\PYGZbs{}\PYG{n}{ipykernel\PYGZus{}232}\PYGZbs{}\PYG{l+m+mf}{613115800.}\PYG{n}{py} \PYG{o+ow}{in} \PYG{o}{\PYGZlt{}}\PYG{n}{module}\PYG{o}{\PYGZgt{}}
\PYG{n+ne}{\PYGZhy{}\PYGZhy{}\PYGZhy{}\PYGZhy{}\PYGZgt{} }\PYG{l+m+mi}{1} \PYG{n}{var\PYGZus{}tuple}\PYG{p}{[}\PYG{l+m+mi}{0}\PYG{p}{]} \PYG{o}{=} \PYG{l+m+mi}{3333}
\PYG{g+gWhitespace}{      }\PYG{l+m+mi}{2} \PYG{n+nb}{print}\PYG{p}{(}\PYG{n}{var\PYGZus{}tuple}\PYG{p}{)}

\PYG{n+ne}{TypeError}: \PYGZsq{}tuple\PYGZsq{} object does not support item assignment
\end{sphinxVerbatim}

\sphinxAtStartPar
Les listes ont une taille variable, les tuples et les chaines de caracteres ont une taille fixe. Enfin, les listes ont plus de fonctionnalités que les tuples. Cependant, c’est le contexte qui nous force a utiliser les listes ou les tuples. Nous allons rencontrer plusieurs contextes ou on est amener a choisir l’un des deux types.


\subsection{quelques methodes utiles pour les listes et les tuples}
\label{\detokenize{ch7:quelques-methodes-utiles-pour-les-listes-et-les-tuples}}
\sphinxAtStartPar
Dans cette sections nous allons voir qulques \sphinxcode{\sphinxupquote{methodes}}  pour les liste et/ou les tuples (les methodes communes et les methodes propres aux listes seulement). Les deux methodes suivantes sont communes aux listes et au tuples:
\begin{itemize}
\item {} 
\sphinxAtStartPar
count: cette methode est utlisee pour compter le nombre d’elements de la liste/tuple.

\item {} 
\sphinxAtStartPar
index: cette method est utilisee pour chercher une valeur spécifiée dans la liste/tuple et renvoie la position de l’endroit où il a été trouvé.

\end{itemize}

\begin{sphinxVerbatim}[commandchars=\\\{\}]
\PYG{n}{var\PYGZus{}liste} \PYG{o}{=} \PYG{p}{[}\PYG{l+m+mi}{1}\PYG{p}{,} \PYG{l+m+mi}{2}\PYG{p}{,} \PYG{l+m+mi}{2}\PYG{p}{,} \PYG{k+kc}{True}\PYG{p}{,} \PYG{l+s+s1}{\PYGZsq{}}\PYG{l+s+s1}{bonjour}\PYG{l+s+s1}{\PYGZsq{}}\PYG{p}{,} \PYG{l+m+mi}{2}\PYG{p}{]}
\PYG{n}{var\PYGZus{}tuple} \PYG{o}{=} \PYG{p}{(}\PYG{l+m+mi}{1}\PYG{p}{,} \PYG{l+m+mi}{2}\PYG{p}{,} \PYG{l+m+mi}{2}\PYG{p}{,} \PYG{k+kc}{True}\PYG{p}{,} \PYG{l+s+s1}{\PYGZsq{}}\PYG{l+s+s1}{bonjour}\PYG{l+s+s1}{\PYGZsq{}}\PYG{p}{,} \PYG{l+m+mi}{2}\PYG{p}{)}

\PYG{n+nb}{print}\PYG{p}{(}\PYG{n}{var\PYGZus{}liste}\PYG{o}{.}\PYG{n}{count}\PYG{p}{(}\PYG{l+m+mi}{2}\PYG{p}{)}\PYG{p}{)} \PYG{c+c1}{\PYGZsh{}\PYGZsh{} print(var\PYGZus{}tuple.count(2))}

\PYG{n+nb}{print}\PYG{p}{(}\PYG{n}{var\PYGZus{}tuple}\PYG{o}{.}\PYG{n}{index}\PYG{p}{(}\PYG{l+m+mi}{2}\PYG{p}{)}\PYG{p}{)}  \PYG{c+c1}{\PYGZsh{}\PYGZsh{} print(var\PYGZus{}tuple.index(2))}
\end{sphinxVerbatim}

\begin{sphinxVerbatim}[commandchars=\\\{\}]
3
\end{sphinxVerbatim}

\begin{sphinxVerbatim}[commandchars=\\\{\}]
\PYG{n}{var\PYGZus{}tuple}\PYG{o}{.}\PYG{n}{index}\PYG{p}{(}\PYG{l+m+mi}{44}\PYG{p}{)} \PYG{c+c1}{\PYGZsh{} var\PYGZus{}tuple.index(44)}
\end{sphinxVerbatim}

\begin{sphinxVerbatim}[commandchars=\\\{\}]
\PYG{g+gt}{\PYGZhy{}\PYGZhy{}\PYGZhy{}\PYGZhy{}\PYGZhy{}\PYGZhy{}\PYGZhy{}\PYGZhy{}\PYGZhy{}\PYGZhy{}\PYGZhy{}\PYGZhy{}\PYGZhy{}\PYGZhy{}\PYGZhy{}\PYGZhy{}\PYGZhy{}\PYGZhy{}\PYGZhy{}\PYGZhy{}\PYGZhy{}\PYGZhy{}\PYGZhy{}\PYGZhy{}\PYGZhy{}\PYGZhy{}\PYGZhy{}\PYGZhy{}\PYGZhy{}\PYGZhy{}\PYGZhy{}\PYGZhy{}\PYGZhy{}\PYGZhy{}\PYGZhy{}\PYGZhy{}\PYGZhy{}\PYGZhy{}\PYGZhy{}\PYGZhy{}\PYGZhy{}\PYGZhy{}\PYGZhy{}\PYGZhy{}\PYGZhy{}\PYGZhy{}\PYGZhy{}\PYGZhy{}\PYGZhy{}\PYGZhy{}\PYGZhy{}\PYGZhy{}\PYGZhy{}\PYGZhy{}\PYGZhy{}\PYGZhy{}\PYGZhy{}\PYGZhy{}\PYGZhy{}\PYGZhy{}\PYGZhy{}\PYGZhy{}\PYGZhy{}\PYGZhy{}\PYGZhy{}\PYGZhy{}\PYGZhy{}\PYGZhy{}\PYGZhy{}\PYGZhy{}\PYGZhy{}\PYGZhy{}\PYGZhy{}\PYGZhy{}\PYGZhy{}}
\PYG{n+ne}{ValueError}\PYG{g+gWhitespace}{                                }Traceback (most recent call last)
\PYG{o}{\PYGZti{}}\PYGZbs{}\PYG{n}{AppData}\PYGZbs{}\PYG{n}{Local}\PYGZbs{}\PYG{n}{Temp}\PYGZbs{}\PYG{n}{ipykernel\PYGZus{}232}\PYGZbs{}\PYG{l+m+mf}{1993419995.}\PYG{n}{py} \PYG{o+ow}{in} \PYG{o}{\PYGZlt{}}\PYG{n}{module}\PYG{o}{\PYGZgt{}}
\PYG{n+ne}{\PYGZhy{}\PYGZhy{}\PYGZhy{}\PYGZhy{}\PYGZgt{} }\PYG{l+m+mi}{1} \PYG{n}{var\PYGZus{}tuple}\PYG{o}{.}\PYG{n}{index}\PYG{p}{(}\PYG{l+m+mi}{44}\PYG{p}{)} \PYG{c+c1}{\PYGZsh{} var\PYGZus{}tuple.index(44)}

\PYG{n+ne}{ValueError}: tuple.index(x): x not in tuple
\end{sphinxVerbatim}

\sphinxAtStartPar
Les methodes decrites dans la table suivante sont appliquee au lites seulement:


\begin{savenotes}\sphinxattablestart
\centering
\begin{tabulary}{\linewidth}[t]{|T|T|}
\hline
\sphinxstyletheadfamily 
\sphinxAtStartPar
Méthode
&\sphinxstyletheadfamily 
\sphinxAtStartPar
Description
\\
\hline
\sphinxAtStartPar
append()
&
\sphinxAtStartPar
Ajoute un élément en fin de liste
\\
\hline
\sphinxAtStartPar
clear()
&
\sphinxAtStartPar
Supprime tous les éléments de la liste
\\
\hline
\sphinxAtStartPar
copy()
&
\sphinxAtStartPar
Renvoie une copie de la liste
\\
\hline
\sphinxAtStartPar
count()
&
\sphinxAtStartPar
Renvoie le nombre d’éléments avec la valeur spécifiée
\\
\hline
\sphinxAtStartPar
extend()
&
\sphinxAtStartPar
Ajouter les éléments d’une liste, à la fin de la liste actuelle
\\
\hline
\sphinxAtStartPar
index()
&
\sphinxAtStartPar
Renvoie l’indice du premier élément avec la valeur spécifiée
\\
\hline
\sphinxAtStartPar
insert()
&
\sphinxAtStartPar
Ajoute un élément à la position spécifiée
\\
\hline
\sphinxAtStartPar
pop()
&
\sphinxAtStartPar
Supprime l’élément à la position spécifiée
\\
\hline
\sphinxAtStartPar
remove()
&
\sphinxAtStartPar
Supprime le premier élément avec la valeur spécifiée
\\
\hline
\sphinxAtStartPar
reverse()
&
\sphinxAtStartPar
Inverse l’ordre de la liste
\\
\hline
\sphinxAtStartPar
sort()
&
\sphinxAtStartPar
Trie la liste
\\
\hline
\end{tabulary}
\par
\sphinxattableend\end{savenotes}


\subsection{sets}
\label{\detokenize{ch7:sets}}
\sphinxAtStartPar
On a vu que les chaines de caracteres, les liste et tuples sont des sequences ordonnees d’elements


\section{Essayez vous\sphinxhyphen{}meme!}
\label{\detokenize{exo4:essayez-vous-meme}}\label{\detokenize{exo4::doc}}



\subsection{Exercice 1.}
\label{\detokenize{exo4:exercice-1}}
\sphinxAtStartPar
Ordre des operations: quelle est selon vous le resulats de ces operations :
\begin{itemize}
\item {} 
\sphinxAtStartPar
\((4/2)^2\times 2 + 1\), \(4/2^{(2\times 2)} + 1\),

\item {} 
\sphinxAtStartPar
\(4/2^2\times (2 + 1)\),

\item {} 
\sphinxAtStartPar
\(4/2^2\times 2 + 1\)
verifier avec Python.

\end{itemize}

\begin{sphinxVerbatim}[commandchars=\\\{\}]
\PYG{c+c1}{\PYGZsh{}\PYGZsh{} votre code ici}
\end{sphinxVerbatim}




\subsection{Exercice 2.}
\label{\detokenize{exo4:exercice-2}}
\sphinxAtStartPar
Un pere a une somme d’argent de 1554 dh, il veut la partager sur ses 9 enfants de maniere equitable et s’il reste quelque dirham, il va acheter des chocolat a 1 dh l’unite. combien chanque enfant va recevoir? combien d’unite de chocolat peut\sphinxhyphen{}il acheter avec le reste?

\begin{sphinxVerbatim}[commandchars=\\\{\}]
\PYG{c+c1}{\PYGZsh{}\PYGZsh{} votre code ici}
\end{sphinxVerbatim}




\subsection{Exercice 3.}
\label{\detokenize{exo4:exercice-3}}
\sphinxAtStartPar
Quel est le type de donnees de valeurs suivantes:
\begin{itemize}
\item {} 
\sphinxAtStartPar
\sphinxcode{\sphinxupquote{1}}

\item {} 
\sphinxAtStartPar
\sphinxcode{\sphinxupquote{1.}}

\item {} 
\sphinxAtStartPar
\sphinxcode{\sphinxupquote{False}}

\item {} 
\sphinxAtStartPar
\sphinxcode{\sphinxupquote{"False"}}

\item {} 
\sphinxAtStartPar
\sphinxcode{\sphinxupquote{var1/var2}} avec \sphinxcode{\sphinxupquote{var1 = 1}} et \sphinxcode{\sphinxupquote{var2 = 2}}

\end{itemize}

\begin{sphinxVerbatim}[commandchars=\\\{\}]
\PYG{c+c1}{\PYGZsh{}\PYGZsh{} votre code ici}
\end{sphinxVerbatim}




\subsection{Exercice 4.}
\label{\detokenize{exo4:exercice-4}}
\sphinxAtStartPar
Quel est le type de donnees de valeurs suivantes:
\begin{itemize}
\item {} 
\sphinxAtStartPar
\sphinxcode{\sphinxupquote{1}}

\item {} 
\sphinxAtStartPar
\sphinxcode{\sphinxupquote{1.}}

\item {} 
\sphinxAtStartPar
\sphinxcode{\sphinxupquote{False}}

\item {} 
\sphinxAtStartPar
\sphinxcode{\sphinxupquote{"False"}}

\item {} 
\sphinxAtStartPar
\sphinxcode{\sphinxupquote{"5.4"}}

\item {} 
\sphinxAtStartPar
\sphinxcode{\sphinxupquote{var1/var2}} avec \sphinxcode{\sphinxupquote{var1 = 1}} et \sphinxcode{\sphinxupquote{var2 = 2}}

\item {} 
\sphinxAtStartPar
\sphinxcode{\sphinxupquote{list()}}

\item {} 
\sphinxAtStartPar
\sphinxcode{\sphinxupquote{None}}

\item {} 
\sphinxAtStartPar
\sphinxcode{\sphinxupquote{""}}

\end{itemize}

\begin{sphinxVerbatim}[commandchars=\\\{\}]
\PYG{c+c1}{\PYGZsh{}\PYGZsh{} votre code ici}
\end{sphinxVerbatim}




\subsection{Exercice 5.}
\label{\detokenize{exo4:exercice-5}}
\sphinxAtStartPar
Convertir, si c’est possible, de valeurs suivantes au types de donnees que nous avons vu. Expliciter les cas qui ne sont pas possible:
\begin{itemize}
\item {} 
\sphinxAtStartPar
\sphinxcode{\sphinxupquote{1}}

\item {} 
\sphinxAtStartPar
\sphinxcode{\sphinxupquote{1.}}

\item {} 
\sphinxAtStartPar
\sphinxcode{\sphinxupquote{False}}

\item {} 
\sphinxAtStartPar
\sphinxcode{\sphinxupquote{"False"}}

\item {} 
\sphinxAtStartPar
\sphinxcode{\sphinxupquote{"5.4"}}

\item {} 
\sphinxAtStartPar
\sphinxcode{\sphinxupquote{var1/var2}} avec \sphinxcode{\sphinxupquote{var1 = 1}} et \sphinxcode{\sphinxupquote{var2 = 2}}

\item {} 
\sphinxAtStartPar
\sphinxcode{\sphinxupquote{list()}}

\item {} 
\sphinxAtStartPar
\sphinxcode{\sphinxupquote{None}}

\item {} 
\sphinxAtStartPar
\sphinxcode{\sphinxupquote{""}}

\end{itemize}

\begin{sphinxVerbatim}[commandchars=\\\{\}]
\PYG{c+c1}{\PYGZsh{}\PYGZsh{} votre code ici}
\end{sphinxVerbatim}




\subsection{Exercice 6.}
\label{\detokenize{exo4:exercice-6}}
\sphinxAtStartPar
Soient \sphinxcode{\sphinxupquote{x = True}}, \sphinxcode{\sphinxupquote{y= 7\textless{}6}}, and \sphinxcode{\sphinxupquote{z= not y}}. Determinier la valeur logique de \sphinxcode{\sphinxupquote{x, y,}} et \sphinxcode{\sphinxupquote{z}} (\sphinxcode{\sphinxupquote{True}} ou \sphinxcode{\sphinxupquote{False}}) puis la valeur de chacun des expression suivantes:
\begin{itemize}
\item {} 
\sphinxAtStartPar
\sphinxcode{\sphinxupquote{x != False}}

\item {} 
\sphinxAtStartPar
\sphinxcode{\sphinxupquote{x and y}}

\item {} 
\sphinxAtStartPar
\sphinxcode{\sphinxupquote{x or y}}

\item {} 
\sphinxAtStartPar
\sphinxcode{\sphinxupquote{not y}}

\item {} 
\sphinxAtStartPar
\sphinxcode{\sphinxupquote{x and (y or z)}}

\item {} 
\sphinxAtStartPar
\sphinxcode{\sphinxupquote{(x and y) or z}}

\item {} 
\sphinxAtStartPar
\sphinxcode{\sphinxupquote{(not x or not y) and (not z)}}

\item {} 
\sphinxAtStartPar
\sphinxcode{\sphinxupquote{not ((x and y) or z)}}

\end{itemize}

\begin{sphinxVerbatim}[commandchars=\\\{\}]
\PYG{c+c1}{\PYGZsh{}\PYGZsh{} votre code ici}
\end{sphinxVerbatim}




\subsection{Exercice 7.}
\label{\detokenize{exo4:exercice-7}}
\sphinxAtStartPar
On veut recevoir ce message avec \sphinxcode{\sphinxupquote{print()}}:

\begin{sphinxVerbatim}[commandchars=\\\{\}]
\PYG{n}{la} \PYG{n}{valeur} \PYG{n}{de} \PYG{n}{x} \PYG{n}{est}\PYG{p}{:} \PYG{o}{@}\PYG{n+nd}{@True}\PYG{o}{@}\PYG{o}{@}
\PYG{o}{*}\PYG{o}{*}\PYG{o}{*}\PYG{o}{*}\PYG{o}{*}\PYG{o}{*}\PYG{o}{*}\PYG{o}{*}\PYG{o}{*}\PYG{o}{==}\PYG{o}{==}\PYG{o}{==}\PYG{o}{==}\PYG{o}{==}\PYG{o}{==}\PYG{o}{==}\PYG{o}{*}\PYG{o}{*}\PYG{o}{*}\PYG{o}{*}\PYG{o}{*}\PYG{o}{*}\PYG{o}{*}\PYG{o}{*}
\end{sphinxVerbatim}

\sphinxAtStartPar
En utilisant les argument suivants:
\begin{itemize}
\item {} 
\sphinxAtStartPar
\sphinxcode{\sphinxupquote{x = True}}

\item {} 
\sphinxAtStartPar
\sphinxcode{\sphinxupquote{a = "la valeur de x est:"}}

\end{itemize}

\sphinxAtStartPar
Toute modification devra etre faite au nivau de \sphinxcode{\sphinxupquote{sep=}}, et \sphinxcode{\sphinxupquote{end=}}.

\begin{sphinxVerbatim}[commandchars=\\\{\}]
\PYG{c+c1}{\PYGZsh{}\PYGZsh{} votre code ici}
\end{sphinxVerbatim}




\subsection{Exercice 8.}
\label{\detokenize{exo4:exercice-8}}
\sphinxAtStartPar
Ecrire un petit programe qui permet de demander a l’utilisatuer d’entrer son nom, son poids en kilograme (sans entrer l’unite), et sa taille en metre (sans entrer l’unite). puis il affiche l’indice du poids (Body mass index (BMI)):
\sphinxstyleemphasis{Formule du BMI}: \(BMI = \dfrac{poids}{taille^2}\)

\begin{sphinxVerbatim}[commandchars=\\\{\}]
\PYG{c+c1}{\PYGZsh{}\PYGZsh{} votre code ici}
\end{sphinxVerbatim}


\chapter{Gestion des exceptions}
\label{\detokenize{content5:gestion-des-exceptions}}\label{\detokenize{content5::doc}}

\section{Les execptions}
\label{\detokenize{ch8:les-execptions}}\label{\detokenize{ch8::doc}}

\subsection{Dictionnaire (dictionnary)}
\label{\detokenize{ch8:dictionnaire-dictionnary}}
\sphinxAtStartPar
Contrairment au objets construits que nous avons vu (\sphinxcode{\sphinxupquote{strings}}, \sphinxcode{\sphinxupquote{lists}}, \sphinxcode{\sphinxupquote{tuples}}), les dictionnaires (\sphinxcode{\sphinxupquote{dictionnary}}) sont utilisés pour stocker des valeurs de données dans des paires clé:valeur. Un dictionnaire est une collection ordonnée , modifiable et qui n’autorise pas les doublons (Depuis la version 3.7 de Python, les dictionnaires demeurent ordonnés. Dans les versions antérieures (Python 3.6 et moins), les dictionnaires ne sont pas ordonnés).
\begin{itemize}
\item {} 
\sphinxAtStartPar
Les dictionnaires sont écrits avec des accolades et ont des clés et des valeurs.  Pour creer un dictionnaire et l’affecter a une variable :

\end{itemize}

\begin{sphinxVerbatim}[commandchars=\\\{\}]
\PYG{n}{var\PYGZus{}dict} \PYG{o}{=} \PYG{p}{\PYGZob{}}\PYG{n}{cle\PYGZus{}1}\PYG{p}{:} \PYG{n}{valeur\PYGZus{}1}\PYG{p}{,} \PYG{n}{cle\PYGZus{}2}\PYG{p}{:}\PYG{n}{valeur\PYGZus{}2}\PYG{p}{,}\PYG{o}{.}\PYG{o}{.}\PYG{o}{.}\PYG{o}{.}\PYG{p}{,} \PYG{n}{cle\PYGZus{}n}\PYG{p}{:}\PYG{n}{valeur\PYGZus{}n}\PYG{p}{\PYGZcb{}}
\end{sphinxVerbatim}
\begin{itemize}
\item {} 
\sphinxAtStartPar
Un dictionnaire vide est un dictionnaire qui contient 0 element (\{\} ou dict());

\item {} 
\sphinxAtStartPar
Les dictionnaires sont \sphinxstylestrong{mutable}, on peut modifier leur contenu et leur taille.

\end{itemize}

\begin{sphinxVerbatim}[commandchars=\\\{\}]
\PYG{c+c1}{\PYGZsh{} dictionnaire vide}
\PYG{n}{var} \PYG{o}{=} \PYG{p}{\PYGZob{}}\PYG{p}{\PYGZcb{}} \PYG{c+c1}{\PYGZsh{} ou var = dict()}
\PYG{n+nb}{print}\PYG{p}{(}\PYG{n+nb}{type}\PYG{p}{(}\PYG{n}{var}\PYG{p}{)}\PYG{p}{)}
\PYG{n}{les\PYGZus{}jours} \PYG{o}{=} \PYG{p}{\PYGZob{}}\PYG{l+m+mi}{1}\PYG{p}{:}\PYG{l+s+s2}{\PYGZdq{}}\PYG{l+s+s2}{lundi}\PYG{l+s+s2}{\PYGZdq{}}\PYG{p}{,} \PYG{l+m+mi}{2}\PYG{p}{:}\PYG{l+s+s2}{\PYGZdq{}}\PYG{l+s+s2}{mardi}\PYG{l+s+s2}{\PYGZdq{}}\PYG{p}{,} \PYG{l+m+mi}{3}\PYG{p}{:}\PYG{l+s+s2}{\PYGZdq{}}\PYG{l+s+s2}{mercredi}\PYG{l+s+s2}{\PYGZdq{}}\PYG{p}{,} \PYG{l+m+mi}{4}\PYG{p}{:}\PYG{l+s+s2}{\PYGZdq{}}\PYG{l+s+s2}{jeudi}\PYG{l+s+s2}{\PYGZdq{}}\PYG{p}{,} \PYG{l+m+mi}{5}\PYG{p}{:}\PYG{l+s+s2}{\PYGZdq{}}\PYG{l+s+s2}{vendredi}\PYG{l+s+s2}{\PYGZdq{}}\PYG{p}{,} \PYG{l+m+mi}{6}\PYG{p}{:}\PYG{l+s+s2}{\PYGZdq{}}\PYG{l+s+s2}{samedi}\PYG{l+s+s2}{\PYGZdq{}}\PYG{p}{,} \PYG{l+m+mi}{7}\PYG{p}{:}\PYG{l+s+s2}{\PYGZdq{}}\PYG{l+s+s2}{dimanche}\PYG{l+s+s2}{\PYGZdq{}}\PYG{p}{\PYGZcb{}}
\PYG{n+nb}{print}\PYG{p}{(}\PYG{n}{les\PYGZus{}jours}\PYG{p}{)}
\end{sphinxVerbatim}

\begin{sphinxVerbatim}[commandchars=\\\{\}]
\PYGZlt{}class \PYGZsq{}dict\PYGZsq{}\PYGZgt{}
\PYGZob{}1: \PYGZsq{}lundi\PYGZsq{}, 2: \PYGZsq{}mardi\PYGZsq{}, 3: \PYGZsq{}mercredi\PYGZsq{}, 4: \PYGZsq{}jeudi\PYGZsq{}, 5: \PYGZsq{}vendredi\PYGZsq{}, 6: \PYGZsq{}samedi\PYGZsq{}, 7: \PYGZsq{}dimanche\PYGZsq{}\PYGZcb{}
\end{sphinxVerbatim}

\begin{sphinxVerbatim}[commandchars=\\\{\}]
\PYG{n}{les\PYGZus{}jours2} \PYG{o}{=} \PYG{p}{\PYGZob{}}\PYG{p}{\PYGZcb{}}

\PYG{n}{les\PYGZus{}jours2}\PYG{p}{[}\PYG{l+m+mi}{1}\PYG{p}{]} \PYG{o}{=} \PYG{l+s+s1}{\PYGZsq{}}\PYG{l+s+s1}{lundi}\PYG{l+s+s1}{\PYGZsq{}}
\PYG{n}{les\PYGZus{}jours2}\PYG{p}{[}\PYG{l+m+mi}{2}\PYG{p}{]} \PYG{o}{=} \PYG{l+s+s1}{\PYGZsq{}}\PYG{l+s+s1}{mardi}\PYG{l+s+s1}{\PYGZsq{}}
\PYG{n}{les\PYGZus{}jours2}\PYG{p}{[}\PYG{l+m+mi}{3}\PYG{p}{]} \PYG{o}{=} \PYG{l+s+s1}{\PYGZsq{}}\PYG{l+s+s1}{mercredi}\PYG{l+s+s1}{\PYGZsq{}}
\PYG{n}{les\PYGZus{}jours2}\PYG{p}{[}\PYG{l+m+mi}{4}\PYG{p}{]} \PYG{o}{=} \PYG{l+s+s1}{\PYGZsq{}}\PYG{l+s+s1}{jeudi}\PYG{l+s+s1}{\PYGZsq{}}
\PYG{n}{les\PYGZus{}jours2}\PYG{p}{[}\PYG{l+m+mi}{5}\PYG{p}{]} \PYG{o}{=} \PYG{l+s+s1}{\PYGZsq{}}\PYG{l+s+s1}{vendredi}\PYG{l+s+s1}{\PYGZsq{}}
\PYG{n}{les\PYGZus{}jours2}\PYG{p}{[}\PYG{l+m+mi}{6}\PYG{p}{]} \PYG{o}{=} \PYG{l+s+s1}{\PYGZsq{}}\PYG{l+s+s1}{samedi}\PYG{l+s+s1}{\PYGZsq{}}
\PYG{n}{les\PYGZus{}jours2}\PYG{p}{[}\PYG{l+m+mi}{7}\PYG{p}{]} \PYG{o}{=} \PYG{l+s+s1}{\PYGZsq{}}\PYG{l+s+s1}{dimanche}\PYG{l+s+s1}{\PYGZsq{}}

\PYG{n+nb}{print}\PYG{p}{(}\PYG{n}{les\PYGZus{}jours2}\PYG{p}{)}
\end{sphinxVerbatim}

\begin{sphinxVerbatim}[commandchars=\\\{\}]
\PYGZob{}1: \PYGZsq{}lundi\PYGZsq{}, 2: \PYGZsq{}mardi\PYGZsq{}, 3: \PYGZsq{}mercredi\PYGZsq{}, 4: \PYGZsq{}jeudi\PYGZsq{}, 5: \PYGZsq{}vendredi\PYGZsq{}, 6: \PYGZsq{}samedi\PYGZsq{}, 7: \PYGZsq{}dimanche\PYGZsq{}\PYGZcb{}
\end{sphinxVerbatim}

\sphinxAtStartPar
les cles et les valeurs peuvent etre de n’importe quel types de donnees. Par exemple on peut creer le dictionnaire suivant:

\begin{sphinxVerbatim}[commandchars=\\\{\}]
\PYG{n}{les\PYGZus{}jours3} \PYG{o}{=} \PYG{p}{\PYGZob{}}\PYG{p}{\PYGZcb{}}

\PYG{n}{les\PYGZus{}jours3}\PYG{p}{[}\PYG{l+s+s1}{\PYGZsq{}}\PYG{l+s+s1}{lundi}\PYG{l+s+s1}{\PYGZsq{}}\PYG{p}{]} \PYG{o}{=} \PYG{l+m+mi}{1}
\PYG{n}{les\PYGZus{}jours3}\PYG{p}{[}\PYG{l+s+s1}{\PYGZsq{}}\PYG{l+s+s1}{mardi}\PYG{l+s+s1}{\PYGZsq{}}\PYG{p}{]} \PYG{o}{=} \PYG{l+m+mi}{2}
\PYG{n}{les\PYGZus{}jours3}\PYG{p}{[}\PYG{l+s+s1}{\PYGZsq{}}\PYG{l+s+s1}{mercredi}\PYG{l+s+s1}{\PYGZsq{}}\PYG{p}{]} \PYG{o}{=} \PYG{l+m+mi}{3}
\PYG{n}{les\PYGZus{}jours3}\PYG{p}{[}\PYG{l+s+s1}{\PYGZsq{}}\PYG{l+s+s1}{jeudi}\PYG{l+s+s1}{\PYGZsq{}}\PYG{p}{]} \PYG{o}{=} \PYG{l+m+mi}{4}
\PYG{n}{les\PYGZus{}jours3}\PYG{p}{[}\PYG{l+s+s1}{\PYGZsq{}}\PYG{l+s+s1}{vendredi}\PYG{l+s+s1}{\PYGZsq{}}\PYG{p}{]} \PYG{o}{=} \PYG{l+m+mi}{5}
\PYG{n}{les\PYGZus{}jours3}\PYG{p}{[}\PYG{l+s+s1}{\PYGZsq{}}\PYG{l+s+s1}{samedi}\PYG{l+s+s1}{\PYGZsq{}}\PYG{p}{]} \PYG{o}{=} \PYG{l+m+mi}{6}
\PYG{n}{les\PYGZus{}jours3}\PYG{p}{[}\PYG{l+s+s1}{\PYGZsq{}}\PYG{l+s+s1}{dimanche}\PYG{l+s+s1}{\PYGZsq{}}\PYG{p}{]} \PYG{o}{=} \PYG{l+m+mi}{7}

\PYG{n+nb}{print}\PYG{p}{(}\PYG{n}{les\PYGZus{}jours3}\PYG{p}{)}
\end{sphinxVerbatim}

\begin{sphinxVerbatim}[commandchars=\\\{\}]
\PYGZob{}\PYGZsq{}lundi\PYGZsq{}: 1, \PYGZsq{}mardi\PYGZsq{}: 2, \PYGZsq{}mercredi\PYGZsq{}: 3, \PYGZsq{}jeudi\PYGZsq{}: 4, \PYGZsq{}vendredi\PYGZsq{}: 5, \PYGZsq{}samedi\PYGZsq{}: 6, \PYGZsq{}dimanche\PYGZsq{}: 7\PYGZcb{}
\end{sphinxVerbatim}

\sphinxAtStartPar
Contrairment aux chaines de caracteres et aux listes, pour acceder a une valeur dans un dictionnaire nous deveons utiliser les cles. Par exemple, si nous voulons acceder a la valeur associee a la cle \sphinxcode{\sphinxupquote{jeudi}} dans \sphinxcode{\sphinxupquote{les\_jours3}}, la maniere de le faire est la suivante:

\begin{sphinxVerbatim}[commandchars=\\\{\}]
\PYG{n+nb}{print}\PYG{p}{(}\PYG{n}{les\PYGZus{}jours3}\PYG{p}{[}\PYG{l+s+s1}{\PYGZsq{}}\PYG{l+s+s1}{jeudi}\PYG{l+s+s1}{\PYGZsq{}}\PYG{p}{]}\PYG{p}{)}
\end{sphinxVerbatim}

\begin{sphinxVerbatim}[commandchars=\\\{\}]
4
\end{sphinxVerbatim}

\sphinxAtStartPar
Les dictionnaires ne peuvent pas avoir des cles dupliqee (les valeurs peuvent etre dupliquees). En effet, les valeurs en double écraseront les valeurs existantes. Illustrons ca avec l’exemple suivant:

\begin{sphinxVerbatim}[commandchars=\\\{\}]
\PYG{n}{les\PYGZus{}jours4} \PYG{o}{=} \PYG{p}{\PYGZob{}}\PYG{p}{\PYGZcb{}}

\PYG{n}{les\PYGZus{}jours4}\PYG{p}{[}\PYG{l+s+s1}{\PYGZsq{}}\PYG{l+s+s1}{lundi}\PYG{l+s+s1}{\PYGZsq{}}\PYG{p}{]} \PYG{o}{=} \PYG{l+m+mi}{0}
\PYG{n}{les\PYGZus{}jours4}\PYG{p}{[}\PYG{l+s+s1}{\PYGZsq{}}\PYG{l+s+s1}{mardi}\PYG{l+s+s1}{\PYGZsq{}}\PYG{p}{]} \PYG{o}{=} \PYG{l+m+mi}{1}
\PYG{n}{les\PYGZus{}jours4}\PYG{p}{[}\PYG{l+s+s1}{\PYGZsq{}}\PYG{l+s+s1}{mardi}\PYG{l+s+s1}{\PYGZsq{}}\PYG{p}{]} \PYG{o}{=} \PYG{l+m+mi}{2}
\PYG{n}{les\PYGZus{}jours4}\PYG{p}{[}\PYG{l+s+s1}{\PYGZsq{}}\PYG{l+s+s1}{mercredi}\PYG{l+s+s1}{\PYGZsq{}}\PYG{p}{]} \PYG{o}{=} \PYG{l+m+mi}{3}
\PYG{n}{les\PYGZus{}jours4}\PYG{p}{[}\PYG{l+s+s1}{\PYGZsq{}}\PYG{l+s+s1}{jeudi}\PYG{l+s+s1}{\PYGZsq{}}\PYG{p}{]} \PYG{o}{=} \PYG{l+m+mi}{4}
\PYG{n}{les\PYGZus{}jours4}\PYG{p}{[}\PYG{l+s+s1}{\PYGZsq{}}\PYG{l+s+s1}{vendredi}\PYG{l+s+s1}{\PYGZsq{}}\PYG{p}{]} \PYG{o}{=} \PYG{l+m+mi}{5}
\PYG{n}{les\PYGZus{}jours4}\PYG{p}{[}\PYG{l+s+s1}{\PYGZsq{}}\PYG{l+s+s1}{samedi}\PYG{l+s+s1}{\PYGZsq{}}\PYG{p}{]} \PYG{o}{=} \PYG{l+m+mi}{6}
\PYG{n}{les\PYGZus{}jours4}\PYG{p}{[}\PYG{l+s+s1}{\PYGZsq{}}\PYG{l+s+s1}{dimanche}\PYG{l+s+s1}{\PYGZsq{}}\PYG{p}{]} \PYG{o}{=} \PYG{l+m+mi}{7}

\PYG{n+nb}{print}\PYG{p}{(}\PYG{n}{les\PYGZus{}jours4}\PYG{p}{)}
\end{sphinxVerbatim}

\begin{sphinxVerbatim}[commandchars=\\\{\}]
\PYGZob{}\PYGZsq{}lundi\PYGZsq{}: 0, \PYGZsq{}mardi\PYGZsq{}: 2, \PYGZsq{}mercredi\PYGZsq{}: 3, \PYGZsq{}jeudi\PYGZsq{}: 4, \PYGZsq{}vendredi\PYGZsq{}: 5, \PYGZsq{}samedi\PYGZsq{}: 6, \PYGZsq{}dimanche\PYGZsq{}: 7\PYGZcb{}
\end{sphinxVerbatim}

\begin{sphinxVerbatim}[commandchars=\\\{\}]
\PYG{n}{les\PYGZus{}jours5} \PYG{o}{=} \PYG{p}{\PYGZob{}}\PYG{p}{\PYGZcb{}}

\PYG{n}{les\PYGZus{}jours5}\PYG{p}{[}\PYG{l+s+s1}{\PYGZsq{}}\PYG{l+s+s1}{lundi}\PYG{l+s+s1}{\PYGZsq{}}\PYG{p}{]} \PYG{o}{=} \PYG{l+m+mi}{0}
\PYG{n}{les\PYGZus{}jours5}\PYG{p}{[}\PYG{l+s+s1}{\PYGZsq{}}\PYG{l+s+s1}{mardi}\PYG{l+s+s1}{\PYGZsq{}}\PYG{p}{]} \PYG{o}{=} \PYG{l+m+mi}{0}
\PYG{n}{les\PYGZus{}jours5}\PYG{p}{[}\PYG{l+s+s1}{\PYGZsq{}}\PYG{l+s+s1}{mercredi}\PYG{l+s+s1}{\PYGZsq{}}\PYG{p}{]} \PYG{o}{=} \PYG{l+m+mi}{0}
\PYG{n}{les\PYGZus{}jours5}\PYG{p}{[}\PYG{l+s+s1}{\PYGZsq{}}\PYG{l+s+s1}{jeudi}\PYG{l+s+s1}{\PYGZsq{}}\PYG{p}{]} \PYG{o}{=} \PYG{l+m+mi}{5}
\PYG{n}{les\PYGZus{}jours5}\PYG{p}{[}\PYG{l+s+s1}{\PYGZsq{}}\PYG{l+s+s1}{vendredi}\PYG{l+s+s1}{\PYGZsq{}}\PYG{p}{]} \PYG{o}{=} \PYG{l+m+mi}{0}
\PYG{n}{les\PYGZus{}jours5}\PYG{p}{[}\PYG{l+s+s1}{\PYGZsq{}}\PYG{l+s+s1}{samedi}\PYG{l+s+s1}{\PYGZsq{}}\PYG{p}{]} \PYG{o}{=} \PYG{l+m+mi}{0}
\PYG{n}{les\PYGZus{}jours5}\PYG{p}{[}\PYG{l+s+s1}{\PYGZsq{}}\PYG{l+s+s1}{dimanche}\PYG{l+s+s1}{\PYGZsq{}}\PYG{p}{]} \PYG{o}{=} \PYG{l+m+mi}{0}

\PYG{n+nb}{print}\PYG{p}{(}\PYG{n}{les\PYGZus{}jours5}\PYG{p}{)}
\end{sphinxVerbatim}

\begin{sphinxVerbatim}[commandchars=\\\{\}]
\PYGZob{}\PYGZsq{}lundi\PYGZsq{}: 0, \PYGZsq{}mardi\PYGZsq{}: 0, \PYGZsq{}mercredi\PYGZsq{}: 0, \PYGZsq{}jeudi\PYGZsq{}: 5, \PYGZsq{}vendredi\PYGZsq{}: 0, \PYGZsq{}samedi\PYGZsq{}: 0, \PYGZsq{}dimanche\PYGZsq{}: 0\PYGZcb{}
\end{sphinxVerbatim}


\subsection{Operations sur les dictionnaires}
\label{\detokenize{ch8:operations-sur-les-dictionnaires}}
\sphinxAtStartPar
Python possède un ensemble de méthodes intégrées que nous pouvons utiliser pour manipuler les dictionnaires:


\begin{savenotes}\sphinxattablestart
\centering
\begin{tabulary}{\linewidth}[t]{|T|T|}
\hline
\sphinxstyletheadfamily 
\sphinxAtStartPar
Methode
&\sphinxstyletheadfamily 
\sphinxAtStartPar
Description
\\
\hline
\sphinxAtStartPar
clear()
&
\sphinxAtStartPar
Supprime tous les éléments du dictionnaire
\\
\hline
\sphinxAtStartPar
copy()
&
\sphinxAtStartPar
Renvoie une copie du dictionnaire
\\
\hline
\sphinxAtStartPar
fromkeys()
&
\sphinxAtStartPar
Renvoie un dictionnaire avec les clés et la valeur spécifiées
\\
\hline
\sphinxAtStartPar
get()
&
\sphinxAtStartPar
Renvoie la valeur de la clé spécifiée
\\
\hline
\sphinxAtStartPar
items()
&
\sphinxAtStartPar
Renvoie une liste contenant un tuple pour chaque paire clé\sphinxhyphen{}valeur
\\
\hline
\sphinxAtStartPar
keys()
&
\sphinxAtStartPar
Retourne une liste contenant les clés du dictionnaire
\\
\hline
\sphinxAtStartPar
pop()
&
\sphinxAtStartPar
Supprime l’élément avec la clé spécifiée
\\
\hline
\sphinxAtStartPar
popitem()
&
\sphinxAtStartPar
Supprime la dernière paire clé\sphinxhyphen{}valeur insérée
\\
\hline
\sphinxAtStartPar
setdefault()
&
\sphinxAtStartPar
Renvoie la valeur de la clé spécifiée. Si la clé n’existe pas : insérez la clé, avec la valeur spécifiée
\\
\hline
\sphinxAtStartPar
update()
&
\sphinxAtStartPar
Met à jour le dictionnaire avec les paires clé\sphinxhyphen{}valeur spécifiées
\\
\hline
\sphinxAtStartPar
values()
&
\sphinxAtStartPar
Renvoie une liste de toutes les valeurs du dictionnaire
\\
\hline
\end{tabulary}
\par
\sphinxattableend\end{savenotes}

\begin{sphinxVerbatim}[commandchars=\\\{\}]
\PYG{c+c1}{\PYGZsh{} clear}
\PYG{n}{les\PYGZus{}jours}\PYG{o}{.}\PYG{n}{clear}\PYG{p}{(}\PYG{p}{)}

\PYG{n+nb}{print}\PYG{p}{(}\PYG{n}{les\PYGZus{}jours}\PYG{p}{)}
\end{sphinxVerbatim}

\begin{sphinxVerbatim}[commandchars=\\\{\}]
\PYGZob{}\PYGZcb{}
\end{sphinxVerbatim}

\begin{sphinxVerbatim}[commandchars=\\\{\}]
\PYG{c+c1}{\PYGZsh{} copy}
\PYG{n}{les\PYGZus{}jours2\PYGZus{}copie} \PYG{o}{=} \PYG{n}{les\PYGZus{}jours2}\PYG{o}{.}\PYG{n}{copy}\PYG{p}{(}\PYG{p}{)}

\PYG{n+nb}{print}\PYG{p}{(}\PYG{n}{les\PYGZus{}jours2\PYGZus{}copie}\PYG{p}{)}
\end{sphinxVerbatim}

\begin{sphinxVerbatim}[commandchars=\\\{\}]
\PYGZob{}1: \PYGZsq{}lundi\PYGZsq{}, 2: \PYGZsq{}mardi\PYGZsq{}, 3: \PYGZsq{}mercredi\PYGZsq{}, 4: \PYGZsq{}jeudi\PYGZsq{}, 5: \PYGZsq{}vendredi\PYGZsq{}, 6: \PYGZsq{}samedi\PYGZsq{}, 7: \PYGZsq{}dimanche\PYGZsq{}\PYGZcb{}
\end{sphinxVerbatim}

\begin{sphinxVerbatim}[commandchars=\\\{\}]
\PYG{c+c1}{\PYGZsh{} fromkeys}
\PYG{c+c1}{\PYGZsh{} creer un dictionnaire avec des cles mais pas de valeurs (None)}

\PYG{n}{cles} \PYG{o}{=} \PYG{n+nb}{range}\PYG{p}{(}\PYG{l+m+mi}{6}\PYG{p}{)}
\PYG{n}{dict\PYGZus{}vide} \PYG{o}{=} \PYG{n+nb}{dict}\PYG{o}{.}\PYG{n}{fromkeys}\PYG{p}{(}\PYG{n}{cles}\PYG{p}{)}
\PYG{n+nb}{print}\PYG{p}{(}\PYG{n}{dict\PYGZus{}vide}\PYG{p}{)}
\end{sphinxVerbatim}

\begin{sphinxVerbatim}[commandchars=\\\{\}]
\PYGZob{}0: None, 1: None, 2: None, 3: None, 4: None, 5: None\PYGZcb{}
\end{sphinxVerbatim}

\begin{sphinxVerbatim}[commandchars=\\\{\}]
\PYG{c+c1}{\PYGZsh{} fromkeys}
\PYG{c+c1}{\PYGZsh{} creer un dictionnaire avec des cles a la meme valeur}
\PYG{n}{var} \PYG{o}{=} \PYG{p}{[}\PYG{l+s+s1}{\PYGZsq{}}\PYG{l+s+s1}{a}\PYG{l+s+s1}{\PYGZsq{}}\PYG{p}{,} \PYG{l+s+s1}{\PYGZsq{}}\PYG{l+s+s1}{e}\PYG{l+s+s1}{\PYGZsq{}}\PYG{p}{,} \PYG{l+s+s1}{\PYGZsq{}}\PYG{l+s+s1}{i}\PYG{l+s+s1}{\PYGZsq{}}\PYG{p}{,} \PYG{l+s+s1}{\PYGZsq{}}\PYG{l+s+s1}{o}\PYG{l+s+s1}{\PYGZsq{}}\PYG{p}{,} \PYG{l+s+s1}{\PYGZsq{}}\PYG{l+s+s1}{u}\PYG{l+s+s1}{\PYGZsq{}}\PYG{p}{,} \PYG{l+s+s1}{\PYGZsq{}}\PYG{l+s+s1}{y}\PYG{l+s+s1}{\PYGZsq{}}\PYG{p}{]}
\PYG{n}{valeur} \PYG{o}{=} \PYG{l+s+s1}{\PYGZsq{}}\PYG{l+s+s1}{voyelle}\PYG{l+s+s1}{\PYGZsq{}}
\PYG{n}{dict\PYGZus{}voyelles} \PYG{o}{=} \PYG{n+nb}{dict}\PYG{o}{.}\PYG{n}{fromkeys}\PYG{p}{(}\PYG{n}{var}\PYG{p}{,} \PYG{n}{valeur}\PYG{p}{)}

\PYG{n+nb}{print}\PYG{p}{(}\PYG{n}{dict\PYGZus{}voyelles}\PYG{p}{)}
\end{sphinxVerbatim}

\begin{sphinxVerbatim}[commandchars=\\\{\}]
\PYGZob{}\PYGZsq{}a\PYGZsq{}: \PYGZsq{}voyelle\PYGZsq{}, \PYGZsq{}e\PYGZsq{}: \PYGZsq{}voyelle\PYGZsq{}, \PYGZsq{}i\PYGZsq{}: \PYGZsq{}voyelle\PYGZsq{}, \PYGZsq{}o\PYGZsq{}: \PYGZsq{}voyelle\PYGZsq{}, \PYGZsq{}u\PYGZsq{}: \PYGZsq{}voyelle\PYGZsq{}, \PYGZsq{}y\PYGZsq{}: \PYGZsq{}voyelle\PYGZsq{}\PYGZcb{}
\end{sphinxVerbatim}

\begin{sphinxVerbatim}[commandchars=\\\{\}]
\PYG{c+c1}{\PYGZsh{} get}
\PYG{n+nb}{print}\PYG{p}{(}\PYG{n}{les\PYGZus{}jours2}\PYG{o}{.}\PYG{n}{get}\PYG{p}{(}\PYG{l+m+mi}{3}\PYG{p}{)}\PYG{p}{)}
\end{sphinxVerbatim}

\begin{sphinxVerbatim}[commandchars=\\\{\}]
mercredi
\end{sphinxVerbatim}

\begin{sphinxVerbatim}[commandchars=\\\{\}]
\PYG{c+c1}{\PYGZsh{} items}
\PYG{n+nb}{print}\PYG{p}{(}\PYG{n}{les\PYGZus{}jours2}\PYG{o}{.}\PYG{n}{items}\PYG{p}{(}\PYG{p}{)}\PYG{p}{)}
\end{sphinxVerbatim}

\begin{sphinxVerbatim}[commandchars=\\\{\}]
dict\PYGZus{}items([(1, \PYGZsq{}lundi\PYGZsq{}), (2, \PYGZsq{}mardi\PYGZsq{}), (3, \PYGZsq{}mercredi\PYGZsq{}), (4, \PYGZsq{}jeudi\PYGZsq{}), (5, \PYGZsq{}vendredi\PYGZsq{}), (6, \PYGZsq{}samedi\PYGZsq{}), (7, \PYGZsq{}dimanche\PYGZsq{})])
\end{sphinxVerbatim}

\begin{sphinxVerbatim}[commandchars=\\\{\}]
\PYG{c+c1}{\PYGZsh{} keys}
\PYG{n+nb}{print}\PYG{p}{(}\PYG{n}{les\PYGZus{}jours2}\PYG{o}{.}\PYG{n}{keys}\PYG{p}{(}\PYG{p}{)}\PYG{p}{)}
\end{sphinxVerbatim}

\begin{sphinxVerbatim}[commandchars=\\\{\}]
dict\PYGZus{}keys([1, 2, 3, 4, 5, 6, 7])
\end{sphinxVerbatim}

\begin{sphinxVerbatim}[commandchars=\\\{\}]
\PYG{c+c1}{\PYGZsh{} values}
\PYG{n+nb}{print}\PYG{p}{(}\PYG{n}{les\PYGZus{}jours2}\PYG{o}{.}\PYG{n}{values}\PYG{p}{(}\PYG{p}{)}\PYG{p}{)}
\end{sphinxVerbatim}

\begin{sphinxVerbatim}[commandchars=\\\{\}]
dict\PYGZus{}values([\PYGZsq{}lundi\PYGZsq{}, \PYGZsq{}mardi\PYGZsq{}, \PYGZsq{}mercredi\PYGZsq{}, \PYGZsq{}jeudi\PYGZsq{}, \PYGZsq{}vendredi\PYGZsq{}, \PYGZsq{}samedi\PYGZsq{}, \PYGZsq{}dimanche\PYGZsq{}])
\end{sphinxVerbatim}

\begin{sphinxVerbatim}[commandchars=\\\{\}]
\PYG{c+c1}{\PYGZsh{} pop}
\PYG{n}{les\PYGZus{}jours2}\PYG{o}{.}\PYG{n}{pop}\PYG{p}{(}\PYG{l+m+mi}{4}\PYG{p}{)}

\PYG{n+nb}{print}\PYG{p}{(}\PYG{n}{les\PYGZus{}jours2}\PYG{p}{)}
\end{sphinxVerbatim}

\begin{sphinxVerbatim}[commandchars=\\\{\}]
\PYGZob{}1: \PYGZsq{}lundi\PYGZsq{}, 2: \PYGZsq{}mardi\PYGZsq{}, 3: \PYGZsq{}mercredi\PYGZsq{}, 5: \PYGZsq{}vendredi\PYGZsq{}, 6: \PYGZsq{}samedi\PYGZsq{}, 7: \PYGZsq{}dimanche\PYGZsq{}\PYGZcb{}
\end{sphinxVerbatim}

\begin{sphinxVerbatim}[commandchars=\\\{\}]
\PYG{c+c1}{\PYGZsh{} popitem }
\PYG{c+c1}{\PYGZsh{} inserer une cles\PYGZhy{}valeur arbitraire}
\PYG{n}{les\PYGZus{}jours2}\PYG{p}{[}\PYG{l+s+s1}{\PYGZsq{}}\PYG{l+s+s1}{la dirniere cles ajoutee}\PYG{l+s+s1}{\PYGZsq{}}\PYG{p}{]} \PYG{o}{=} \PYG{l+s+s1}{\PYGZsq{}}\PYG{l+s+s1}{La valeur associee a la derniere cles}\PYG{l+s+s1}{\PYGZsq{}}

\PYG{n+nb}{print}\PYG{p}{(}\PYG{n}{les\PYGZus{}jours2}\PYG{p}{)}
\end{sphinxVerbatim}

\begin{sphinxVerbatim}[commandchars=\\\{\}]
\PYGZob{}1: \PYGZsq{}lundi\PYGZsq{}, 2: \PYGZsq{}mardi\PYGZsq{}, 3: \PYGZsq{}mercredi\PYGZsq{}, 5: \PYGZsq{}vendredi\PYGZsq{}, 6: \PYGZsq{}samedi\PYGZsq{}, 7: \PYGZsq{}dimanche\PYGZsq{}, \PYGZsq{}la dirniere cles ajoutee\PYGZsq{}: \PYGZsq{}La valeur associee a la derniere cles\PYGZsq{}\PYGZcb{}
\end{sphinxVerbatim}

\begin{sphinxVerbatim}[commandchars=\\\{\}]
\PYG{c+c1}{\PYGZsh{} popitem}
\PYG{n}{les\PYGZus{}jours2}\PYG{o}{.}\PYG{n}{popitem}\PYG{p}{(}\PYG{p}{)}

\PYG{n+nb}{print}\PYG{p}{(}\PYG{n}{les\PYGZus{}jours2}\PYG{p}{)}
\end{sphinxVerbatim}

\begin{sphinxVerbatim}[commandchars=\\\{\}]
\PYGZob{}1: \PYGZsq{}lundi\PYGZsq{}, 2: \PYGZsq{}mardi\PYGZsq{}, 3: \PYGZsq{}mercredi\PYGZsq{}, 5: \PYGZsq{}vendredi\PYGZsq{}, 6: \PYGZsq{}samedi\PYGZsq{}, 7: \PYGZsq{}dimanche\PYGZsq{}\PYGZcb{}
\end{sphinxVerbatim}

\begin{sphinxVerbatim}[commandchars=\\\{\}]
\PYG{c+c1}{\PYGZsh{} setdefault()}
\PYG{c+c1}{\PYGZsh{} si la cles existe}
\PYG{n+nb}{print}\PYG{p}{(}\PYG{n}{les\PYGZus{}jours2}\PYG{o}{.}\PYG{n}{setdefault}\PYG{p}{(}\PYG{l+m+mi}{1}\PYG{p}{,} \PYG{l+s+s2}{\PYGZdq{}}\PYG{l+s+s2}{LUNDI}\PYG{l+s+s2}{\PYGZdq{}}\PYG{p}{)}\PYG{p}{)}
\PYG{c+c1}{\PYGZsh{} pusique la cles existe, la valeur ne va pas changer. Elle va etre affichee. Le dictionnaire ne va pas changer aussi.}
\PYG{n+nb}{print}\PYG{p}{(}\PYG{n}{les\PYGZus{}jours2}\PYG{p}{)}
\end{sphinxVerbatim}

\begin{sphinxVerbatim}[commandchars=\\\{\}]
lundi
\PYGZob{}1: \PYGZsq{}lundi\PYGZsq{}, 2: \PYGZsq{}mardi\PYGZsq{}, 3: \PYGZsq{}mercredi\PYGZsq{}, 5: \PYGZsq{}vendredi\PYGZsq{}, 6: \PYGZsq{}samedi\PYGZsq{}, 7: \PYGZsq{}dimanche\PYGZsq{}\PYGZcb{}
\end{sphinxVerbatim}

\begin{sphinxVerbatim}[commandchars=\\\{\}]
\PYG{c+c1}{\PYGZsh{} setdefault()}
\PYG{c+c1}{\PYGZsh{} si la cles existe}
\PYG{c+c1}{\PYGZsh{} puisque la cles 4 n\PYGZsq{}existe pas. La valeur \PYGZsq{}mercredi\PYGZsq{} va etre affichee. Le dictionnaire va etre modifie aussi}
\PYG{n+nb}{print}\PYG{p}{(}\PYG{n}{les\PYGZus{}jours2}\PYG{o}{.}\PYG{n}{setdefault}\PYG{p}{(}\PYG{l+m+mi}{4}\PYG{p}{,} \PYG{l+s+s2}{\PYGZdq{}}\PYG{l+s+s2}{mercredi}\PYG{l+s+s2}{\PYGZdq{}}\PYG{p}{)}\PYG{p}{)}
\PYG{n+nb}{print}\PYG{p}{(}\PYG{n}{les\PYGZus{}jours2}\PYG{p}{)}
\end{sphinxVerbatim}

\begin{sphinxVerbatim}[commandchars=\\\{\}]
mercredi
\PYGZob{}1: \PYGZsq{}lundi\PYGZsq{}, 2: \PYGZsq{}mardi\PYGZsq{}, 3: \PYGZsq{}mercredi\PYGZsq{}, 5: \PYGZsq{}vendredi\PYGZsq{}, 6: \PYGZsq{}samedi\PYGZsq{}, 7: \PYGZsq{}dimanche\PYGZsq{}, 4: \PYGZsq{}mercredi\PYGZsq{}\PYGZcb{}
\end{sphinxVerbatim}

\begin{sphinxVerbatim}[commandchars=\\\{\}]
\PYG{c+c1}{\PYGZsh{} update}
\PYG{n}{premier}
\end{sphinxVerbatim}

\sphinxAtStartPar
La fonction intégrée \sphinxcode{\sphinxupquote{len()}} est aussi appliquable pour les dictionnaires. Si l’on désire déterminer le nombre d’elements:

\begin{sphinxVerbatim}[commandchars=\\\{\}]
\PYG{n+nb}{len}\PYG{p}{(}\PYG{n}{les\PYGZus{}jours}\PYG{p}{)}
\end{sphinxVerbatim}

\begin{sphinxVerbatim}[commandchars=\\\{\}]
7
\end{sphinxVerbatim}

\sphinxAtStartPar
La varaible \sphinxcode{\sphinxupquote{les\_jours}} contient 7 elements.

\sphinxAtStartPar
Si l’on veut tester l’appartenece d’un element a une liste/tuple on utilise l’operateur \sphinxcode{\sphinxupquote{in}}. L’expression est la suivante : \sphinxcode{\sphinxupquote{element in liste}}. Cela nous renvoi \sphinxcode{\sphinxupquote{True}} si \sphinxcode{\sphinxupquote{element}}est dans \sphinxcode{\sphinxupquote{liste}}, sinon \sphinxcode{\sphinxupquote{False}}. On peut aussi ecrire \sphinxcode{\sphinxupquote{element not in liste}} pour tester si l’element n’est pas dans \sphinxcode{\sphinxupquote{liste}} (\sphinxcode{\sphinxupquote{True}}) ou s’il est dans \sphinxcode{\sphinxupquote{liste}} (\sphinxcode{\sphinxupquote{False}}). Voici quelques exemples:

\begin{sphinxVerbatim}[commandchars=\\\{\}]
\PYG{n}{les\PYGZus{}jours}
\end{sphinxVerbatim}

\begin{sphinxVerbatim}[commandchars=\\\{\}]
\PYGZob{}1: \PYGZsq{}lundi\PYGZsq{},
 2: \PYGZsq{}mardi\PYGZsq{},
 3: \PYGZsq{}mercredi\PYGZsq{},
 4: \PYGZsq{}jeudi\PYGZsq{},
 5: \PYGZsq{}vendredi\PYGZsq{},
 6: \PYGZsq{}samedi\PYGZsq{},
 7: \PYGZsq{}dimanche\PYGZsq{}\PYGZcb{}
\end{sphinxVerbatim}

\begin{sphinxVerbatim}[commandchars=\\\{\}]
\PYG{n}{les\PYGZus{}jours}\PYG{o}{.}\PYG{n}{values}\PYG{p}{(}\PYG{p}{)}
\end{sphinxVerbatim}

\begin{sphinxVerbatim}[commandchars=\\\{\}]
dict\PYGZus{}values([\PYGZsq{}lundi\PYGZsq{}, \PYGZsq{}mardi\PYGZsq{}, \PYGZsq{}mercredi\PYGZsq{}, \PYGZsq{}jeudi\PYGZsq{}, \PYGZsq{}vendredi\PYGZsq{}, \PYGZsq{}samedi\PYGZsq{}, \PYGZsq{}dimanche\PYGZsq{}])
\end{sphinxVerbatim}

\begin{sphinxVerbatim}[commandchars=\\\{\}]
\PYG{n}{un\PYGZus{}tuple} \PYG{o}{=} \PYG{p}{(}\PYG{l+m+mi}{1}\PYG{p}{,} \PYG{l+s+s1}{\PYGZsq{}}\PYG{l+s+s1}{bonjour}\PYG{l+s+s1}{\PYGZsq{}}\PYG{p}{,} \PYG{l+m+mf}{4.5}\PYG{p}{,} \PYG{k+kc}{True}\PYG{p}{)}

\PYG{n+nb}{print}\PYG{p}{(}\PYG{l+m+mi}{1} \PYG{o+ow}{in} \PYG{n}{un\PYGZus{}tuple}\PYG{p}{)}

\PYG{n+nb}{print}\PYG{p}{(}\PYG{l+m+mi}{2} \PYG{o+ow}{in} \PYG{n}{un\PYGZus{}tuple}\PYG{p}{)}

\PYG{n+nb}{print}\PYG{p}{(}\PYG{l+s+s1}{\PYGZsq{}}\PYG{l+s+s1}{o}\PYG{l+s+s1}{\PYGZsq{}} \PYG{o+ow}{not} \PYG{o+ow}{in} \PYG{n}{un\PYGZus{}tuple}\PYG{p}{)}

\PYG{n+nb}{print}\PYG{p}{(}\PYG{l+s+s1}{\PYGZsq{}}\PYG{l+s+s1}{x}\PYG{l+s+s1}{\PYGZsq{}} \PYG{o+ow}{not} \PYG{o+ow}{in} \PYG{n}{un\PYGZus{}tuple}\PYG{p}{)}
\end{sphinxVerbatim}

\begin{sphinxVerbatim}[commandchars=\\\{\}]
True
False
True
True
\end{sphinxVerbatim}


\subsection{Quelle est la difference entre listes et tuples?}
\label{\detokenize{ch8:quelle-est-la-difference-entre-listes-et-tuples}}
\sphinxAtStartPar
Les listes et les tuples sont pareils dans la plupart des contextes. Cepandant, la difference primordiale entre les deux et que les listes sont des \sphinxstylestrong{objets mutables} (modifiables) alors que les tuples sont des \sphinxstylestrong{objets immuables} (ne sont pas modifiable). La question qui se pose est donc: qu’est\sphinxhyphen{}ce qu’un objet mutable et un objet immuable?
Parmi les objet immuable en python, on trouve:
\begin{itemize}
\item {} 
\sphinxAtStartPar
Les nombres entiers (int)

\item {} 
\sphinxAtStartPar
Les nombres décimaux (float)

\item {} 
\sphinxAtStartPar
Les chaînes de caractères (str)

\item {} 
\sphinxAtStartPar
Les booléens (bool)

\item {} 
\sphinxAtStartPar
Les tuples (tuple)
La plus part des autres objets que vous allez confronter en python sont mutables.

\end{itemize}

\sphinxAtStartPar
Nous allons illustre ca dans les exemples suivant:
\begin{enumerate}
\sphinxsetlistlabels{\arabic}{enumi}{enumii}{}{.}%
\item {} 
\sphinxAtStartPar
nous allons creer les variables suivantes:

\end{enumerate}
\begin{itemize}
\item {} 
\sphinxAtStartPar
\sphinxcode{\sphinxupquote{var\_chaine = "bonjour tout le monde"}},

\item {} 
\sphinxAtStartPar
\sphinxcode{\sphinxupquote{var\_liste = {[}1, 2, True, \textquotesingle{}bonjour\textquotesingle{}{]}}},

\item {} 
\sphinxAtStartPar
\sphinxcode{\sphinxupquote{var\_tuple = (1, 2, True, \textquotesingle{}bonjour\textquotesingle{})}}.

\end{itemize}
\begin{enumerate}
\sphinxsetlistlabels{\arabic}{enumi}{enumii}{}{.}%
\item {} 
\sphinxAtStartPar
nous qllons essayer de changer (par exemple) le premier element de chaque variable (par un autre element).

\end{enumerate}

\begin{sphinxVerbatim}[commandchars=\\\{\}]
\PYG{n}{var\PYGZus{}chaine} \PYG{o}{=} \PYG{l+s+s2}{\PYGZdq{}}\PYG{l+s+s2}{bonjour tout le monde}\PYG{l+s+s2}{\PYGZdq{}}
\PYG{n}{var\PYGZus{}liste} \PYG{o}{=} \PYG{p}{[}\PYG{l+m+mi}{1}\PYG{p}{,} \PYG{l+m+mi}{2}\PYG{p}{,} \PYG{k+kc}{True}\PYG{p}{,} \PYG{l+s+s1}{\PYGZsq{}}\PYG{l+s+s1}{bonjour}\PYG{l+s+s1}{\PYGZsq{}}\PYG{p}{]}
\PYG{n}{var\PYGZus{}tuple} \PYG{o}{=} \PYG{p}{(}\PYG{l+m+mi}{1}\PYG{p}{,} \PYG{l+m+mi}{2}\PYG{p}{,} \PYG{k+kc}{True}\PYG{p}{,} \PYG{l+s+s1}{\PYGZsq{}}\PYG{l+s+s1}{bonjour}\PYG{l+s+s1}{\PYGZsq{}}\PYG{p}{)}
\end{sphinxVerbatim}

\begin{sphinxVerbatim}[commandchars=\\\{\}]
\PYG{n}{var\PYGZus{}chaine}\PYG{p}{[}\PYG{l+m+mi}{0}\PYG{p}{]} \PYG{o}{=} \PYG{l+s+s1}{\PYGZsq{}}\PYG{l+s+s1}{B}\PYG{l+s+s1}{\PYGZsq{}}
\PYG{n+nb}{print}\PYG{p}{(}\PYG{n}{var\PYGZus{}chaine}\PYG{p}{)}
\end{sphinxVerbatim}

\begin{sphinxVerbatim}[commandchars=\\\{\}]
\PYG{g+gt}{\PYGZhy{}\PYGZhy{}\PYGZhy{}\PYGZhy{}\PYGZhy{}\PYGZhy{}\PYGZhy{}\PYGZhy{}\PYGZhy{}\PYGZhy{}\PYGZhy{}\PYGZhy{}\PYGZhy{}\PYGZhy{}\PYGZhy{}\PYGZhy{}\PYGZhy{}\PYGZhy{}\PYGZhy{}\PYGZhy{}\PYGZhy{}\PYGZhy{}\PYGZhy{}\PYGZhy{}\PYGZhy{}\PYGZhy{}\PYGZhy{}\PYGZhy{}\PYGZhy{}\PYGZhy{}\PYGZhy{}\PYGZhy{}\PYGZhy{}\PYGZhy{}\PYGZhy{}\PYGZhy{}\PYGZhy{}\PYGZhy{}\PYGZhy{}\PYGZhy{}\PYGZhy{}\PYGZhy{}\PYGZhy{}\PYGZhy{}\PYGZhy{}\PYGZhy{}\PYGZhy{}\PYGZhy{}\PYGZhy{}\PYGZhy{}\PYGZhy{}\PYGZhy{}\PYGZhy{}\PYGZhy{}\PYGZhy{}\PYGZhy{}\PYGZhy{}\PYGZhy{}\PYGZhy{}\PYGZhy{}\PYGZhy{}\PYGZhy{}\PYGZhy{}\PYGZhy{}\PYGZhy{}\PYGZhy{}\PYGZhy{}\PYGZhy{}\PYGZhy{}\PYGZhy{}\PYGZhy{}\PYGZhy{}\PYGZhy{}\PYGZhy{}\PYGZhy{}}
\PYG{n+ne}{TypeError}\PYG{g+gWhitespace}{                                 }Traceback (most recent call last)
\PYG{o}{\PYGZti{}}\PYGZbs{}\PYG{n}{AppData}\PYGZbs{}\PYG{n}{Local}\PYGZbs{}\PYG{n}{Temp}\PYGZbs{}\PYG{n}{ipykernel\PYGZus{}232}\PYGZbs{}\PYG{l+m+mf}{4274672008.}\PYG{n}{py} \PYG{o+ow}{in} \PYG{o}{\PYGZlt{}}\PYG{n}{module}\PYG{o}{\PYGZgt{}}
\PYG{n+ne}{\PYGZhy{}\PYGZhy{}\PYGZhy{}\PYGZhy{}\PYGZgt{} }\PYG{l+m+mi}{1} \PYG{n}{var\PYGZus{}chaine}\PYG{p}{[}\PYG{l+m+mi}{0}\PYG{p}{]} \PYG{o}{=} \PYG{l+s+s1}{\PYGZsq{}}\PYG{l+s+s1}{B}\PYG{l+s+s1}{\PYGZsq{}}
\PYG{g+gWhitespace}{      }\PYG{l+m+mi}{2} \PYG{n+nb}{print}\PYG{p}{(}\PYG{n}{var\PYGZus{}chaine}\PYG{p}{)}

\PYG{n+ne}{TypeError}: \PYGZsq{}str\PYGZsq{} object does not support item assignment
\end{sphinxVerbatim}

\begin{sphinxVerbatim}[commandchars=\\\{\}]
\PYG{n}{var\PYGZus{}liste}\PYG{p}{[}\PYG{l+m+mi}{0}\PYG{p}{]} \PYG{o}{=} \PYG{l+m+mi}{3333}
\PYG{n+nb}{print}\PYG{p}{(}\PYG{n}{var\PYGZus{}liste}\PYG{p}{)}
\end{sphinxVerbatim}

\begin{sphinxVerbatim}[commandchars=\\\{\}]
[3333, 2, True, \PYGZsq{}bonjour\PYGZsq{}]
\end{sphinxVerbatim}

\begin{sphinxVerbatim}[commandchars=\\\{\}]
\PYG{n}{var\PYGZus{}tuple}\PYG{p}{[}\PYG{l+m+mi}{0}\PYG{p}{]} \PYG{o}{=} \PYG{l+m+mi}{3333}
\PYG{n+nb}{print}\PYG{p}{(}\PYG{n}{var\PYGZus{}tuple}\PYG{p}{)}
\end{sphinxVerbatim}

\begin{sphinxVerbatim}[commandchars=\\\{\}]
\PYG{g+gt}{\PYGZhy{}\PYGZhy{}\PYGZhy{}\PYGZhy{}\PYGZhy{}\PYGZhy{}\PYGZhy{}\PYGZhy{}\PYGZhy{}\PYGZhy{}\PYGZhy{}\PYGZhy{}\PYGZhy{}\PYGZhy{}\PYGZhy{}\PYGZhy{}\PYGZhy{}\PYGZhy{}\PYGZhy{}\PYGZhy{}\PYGZhy{}\PYGZhy{}\PYGZhy{}\PYGZhy{}\PYGZhy{}\PYGZhy{}\PYGZhy{}\PYGZhy{}\PYGZhy{}\PYGZhy{}\PYGZhy{}\PYGZhy{}\PYGZhy{}\PYGZhy{}\PYGZhy{}\PYGZhy{}\PYGZhy{}\PYGZhy{}\PYGZhy{}\PYGZhy{}\PYGZhy{}\PYGZhy{}\PYGZhy{}\PYGZhy{}\PYGZhy{}\PYGZhy{}\PYGZhy{}\PYGZhy{}\PYGZhy{}\PYGZhy{}\PYGZhy{}\PYGZhy{}\PYGZhy{}\PYGZhy{}\PYGZhy{}\PYGZhy{}\PYGZhy{}\PYGZhy{}\PYGZhy{}\PYGZhy{}\PYGZhy{}\PYGZhy{}\PYGZhy{}\PYGZhy{}\PYGZhy{}\PYGZhy{}\PYGZhy{}\PYGZhy{}\PYGZhy{}\PYGZhy{}\PYGZhy{}\PYGZhy{}\PYGZhy{}\PYGZhy{}\PYGZhy{}}
\PYG{n+ne}{TypeError}\PYG{g+gWhitespace}{                                 }Traceback (most recent call last)
\PYG{o}{\PYGZti{}}\PYGZbs{}\PYG{n}{AppData}\PYGZbs{}\PYG{n}{Local}\PYGZbs{}\PYG{n}{Temp}\PYGZbs{}\PYG{n}{ipykernel\PYGZus{}232}\PYGZbs{}\PYG{l+m+mf}{613115800.}\PYG{n}{py} \PYG{o+ow}{in} \PYG{o}{\PYGZlt{}}\PYG{n}{module}\PYG{o}{\PYGZgt{}}
\PYG{n+ne}{\PYGZhy{}\PYGZhy{}\PYGZhy{}\PYGZhy{}\PYGZgt{} }\PYG{l+m+mi}{1} \PYG{n}{var\PYGZus{}tuple}\PYG{p}{[}\PYG{l+m+mi}{0}\PYG{p}{]} \PYG{o}{=} \PYG{l+m+mi}{3333}
\PYG{g+gWhitespace}{      }\PYG{l+m+mi}{2} \PYG{n+nb}{print}\PYG{p}{(}\PYG{n}{var\PYGZus{}tuple}\PYG{p}{)}

\PYG{n+ne}{TypeError}: \PYGZsq{}tuple\PYGZsq{} object does not support item assignment
\end{sphinxVerbatim}

\sphinxAtStartPar
Les listes ont une taille variable, les tuples et les chaines de caracteres ont une taille fixe. Enfin, les listes ont plus de fonctionnalités que les tuples. Cependant, c’est le contexte qui nous force a utiliser les listes ou les tuples. Nous allons rencontrer plusieurs contextes ou on est amener a choisir l’un des deux types.


\subsection{quelques methodes utiles pour les listes et les tuples}
\label{\detokenize{ch8:quelques-methodes-utiles-pour-les-listes-et-les-tuples}}
\sphinxAtStartPar
Dans cette sections nous allons voir qulques \sphinxcode{\sphinxupquote{methodes}}  pour les liste et/ou les tuples (les methodes communes et les methodes propres aux listes seulement). Les deux methodes suivantes sont communes aux listes et au tuples:
\begin{itemize}
\item {} 
\sphinxAtStartPar
count: cette methode est utlisee pour compter le nombre d’elements de la liste/tuple.

\item {} 
\sphinxAtStartPar
index: cette method est utilisee pour chercher une valeur spécifiée dans la liste/tuple et renvoie la position de l’endroit où il a été trouvé.

\end{itemize}

\begin{sphinxVerbatim}[commandchars=\\\{\}]
\PYG{n}{var\PYGZus{}liste} \PYG{o}{=} \PYG{p}{[}\PYG{l+m+mi}{1}\PYG{p}{,} \PYG{l+m+mi}{2}\PYG{p}{,} \PYG{l+m+mi}{2}\PYG{p}{,} \PYG{k+kc}{True}\PYG{p}{,} \PYG{l+s+s1}{\PYGZsq{}}\PYG{l+s+s1}{bonjour}\PYG{l+s+s1}{\PYGZsq{}}\PYG{p}{,} \PYG{l+m+mi}{2}\PYG{p}{]}
\PYG{n}{var\PYGZus{}tuple} \PYG{o}{=} \PYG{p}{(}\PYG{l+m+mi}{1}\PYG{p}{,} \PYG{l+m+mi}{2}\PYG{p}{,} \PYG{l+m+mi}{2}\PYG{p}{,} \PYG{k+kc}{True}\PYG{p}{,} \PYG{l+s+s1}{\PYGZsq{}}\PYG{l+s+s1}{bonjour}\PYG{l+s+s1}{\PYGZsq{}}\PYG{p}{,} \PYG{l+m+mi}{2}\PYG{p}{)}

\PYG{n+nb}{print}\PYG{p}{(}\PYG{n}{var\PYGZus{}liste}\PYG{o}{.}\PYG{n}{count}\PYG{p}{(}\PYG{l+m+mi}{2}\PYG{p}{)}\PYG{p}{)} \PYG{c+c1}{\PYGZsh{}\PYGZsh{} print(var\PYGZus{}tuple.count(2))}

\PYG{n+nb}{print}\PYG{p}{(}\PYG{n}{var\PYGZus{}tuple}\PYG{o}{.}\PYG{n}{index}\PYG{p}{(}\PYG{l+m+mi}{2}\PYG{p}{)}\PYG{p}{)}  \PYG{c+c1}{\PYGZsh{}\PYGZsh{} print(var\PYGZus{}tuple.index(2))}
\end{sphinxVerbatim}

\begin{sphinxVerbatim}[commandchars=\\\{\}]
3
\end{sphinxVerbatim}

\begin{sphinxVerbatim}[commandchars=\\\{\}]
\PYG{n}{var\PYGZus{}tuple}\PYG{o}{.}\PYG{n}{index}\PYG{p}{(}\PYG{l+m+mi}{44}\PYG{p}{)} \PYG{c+c1}{\PYGZsh{} var\PYGZus{}tuple.index(44)}
\end{sphinxVerbatim}

\begin{sphinxVerbatim}[commandchars=\\\{\}]
\PYG{g+gt}{\PYGZhy{}\PYGZhy{}\PYGZhy{}\PYGZhy{}\PYGZhy{}\PYGZhy{}\PYGZhy{}\PYGZhy{}\PYGZhy{}\PYGZhy{}\PYGZhy{}\PYGZhy{}\PYGZhy{}\PYGZhy{}\PYGZhy{}\PYGZhy{}\PYGZhy{}\PYGZhy{}\PYGZhy{}\PYGZhy{}\PYGZhy{}\PYGZhy{}\PYGZhy{}\PYGZhy{}\PYGZhy{}\PYGZhy{}\PYGZhy{}\PYGZhy{}\PYGZhy{}\PYGZhy{}\PYGZhy{}\PYGZhy{}\PYGZhy{}\PYGZhy{}\PYGZhy{}\PYGZhy{}\PYGZhy{}\PYGZhy{}\PYGZhy{}\PYGZhy{}\PYGZhy{}\PYGZhy{}\PYGZhy{}\PYGZhy{}\PYGZhy{}\PYGZhy{}\PYGZhy{}\PYGZhy{}\PYGZhy{}\PYGZhy{}\PYGZhy{}\PYGZhy{}\PYGZhy{}\PYGZhy{}\PYGZhy{}\PYGZhy{}\PYGZhy{}\PYGZhy{}\PYGZhy{}\PYGZhy{}\PYGZhy{}\PYGZhy{}\PYGZhy{}\PYGZhy{}\PYGZhy{}\PYGZhy{}\PYGZhy{}\PYGZhy{}\PYGZhy{}\PYGZhy{}\PYGZhy{}\PYGZhy{}\PYGZhy{}\PYGZhy{}\PYGZhy{}}
\PYG{n+ne}{ValueError}\PYG{g+gWhitespace}{                                }Traceback (most recent call last)
\PYG{o}{\PYGZti{}}\PYGZbs{}\PYG{n}{AppData}\PYGZbs{}\PYG{n}{Local}\PYGZbs{}\PYG{n}{Temp}\PYGZbs{}\PYG{n}{ipykernel\PYGZus{}232}\PYGZbs{}\PYG{l+m+mf}{1993419995.}\PYG{n}{py} \PYG{o+ow}{in} \PYG{o}{\PYGZlt{}}\PYG{n}{module}\PYG{o}{\PYGZgt{}}
\PYG{n+ne}{\PYGZhy{}\PYGZhy{}\PYGZhy{}\PYGZhy{}\PYGZgt{} }\PYG{l+m+mi}{1} \PYG{n}{var\PYGZus{}tuple}\PYG{o}{.}\PYG{n}{index}\PYG{p}{(}\PYG{l+m+mi}{44}\PYG{p}{)} \PYG{c+c1}{\PYGZsh{} var\PYGZus{}tuple.index(44)}

\PYG{n+ne}{ValueError}: tuple.index(x): x not in tuple
\end{sphinxVerbatim}

\sphinxAtStartPar
Les methodes decrites dans la table suivante sont appliquee au lites seulement:


\begin{savenotes}\sphinxattablestart
\centering
\begin{tabulary}{\linewidth}[t]{|T|T|}
\hline
\sphinxstyletheadfamily 
\sphinxAtStartPar
Méthode
&\sphinxstyletheadfamily 
\sphinxAtStartPar
Description
\\
\hline
\sphinxAtStartPar
append()
&
\sphinxAtStartPar
Ajoute un élément en fin de liste
\\
\hline
\sphinxAtStartPar
clear()
&
\sphinxAtStartPar
Supprime tous les éléments de la liste
\\
\hline
\sphinxAtStartPar
copy()
&
\sphinxAtStartPar
Renvoie une copie de la liste
\\
\hline
\sphinxAtStartPar
count()
&
\sphinxAtStartPar
Renvoie le nombre d’éléments avec la valeur spécifiée
\\
\hline
\sphinxAtStartPar
extend()
&
\sphinxAtStartPar
Ajouter les éléments d’une liste, à la fin de la liste actuelle
\\
\hline
\sphinxAtStartPar
index()
&
\sphinxAtStartPar
Renvoie l’indice du premier élément avec la valeur spécifiée
\\
\hline
\sphinxAtStartPar
insert()
&
\sphinxAtStartPar
Ajoute un élément à la position spécifiée
\\
\hline
\sphinxAtStartPar
pop()
&
\sphinxAtStartPar
Supprime l’élément à la position spécifiée
\\
\hline
\sphinxAtStartPar
remove()
&
\sphinxAtStartPar
Supprime le premier élément avec la valeur spécifiée
\\
\hline
\sphinxAtStartPar
reverse()
&
\sphinxAtStartPar
Inverse l’ordre de la liste
\\
\hline
\sphinxAtStartPar
sort()
&
\sphinxAtStartPar
Trie la liste
\\
\hline
\end{tabulary}
\par
\sphinxattableend\end{savenotes}


\subsection{sets}
\label{\detokenize{ch8:sets}}
\sphinxAtStartPar
On a vu que les chaines de caracteres, les liste et tuples sont des sequences ordonnees d’elements


\section{Essayez vous\sphinxhyphen{}meme!}
\label{\detokenize{exo5:essayez-vous-meme}}\label{\detokenize{exo5::doc}}



\subsection{Exercice 1.}
\label{\detokenize{exo5:exercice-1}}
\sphinxAtStartPar
Ordre des operations: quelle est selon vous le resulats de ces operations :
\begin{itemize}
\item {} 
\sphinxAtStartPar
\((4/2)^2\times 2 + 1\), \(4/2^{(2\times 2)} + 1\),

\item {} 
\sphinxAtStartPar
\(4/2^2\times (2 + 1)\),

\item {} 
\sphinxAtStartPar
\(4/2^2\times 2 + 1\)
verifier avec Python.

\end{itemize}

\begin{sphinxVerbatim}[commandchars=\\\{\}]
\PYG{c+c1}{\PYGZsh{}\PYGZsh{} votre code ici}
\end{sphinxVerbatim}




\subsection{Exercice 2.}
\label{\detokenize{exo5:exercice-2}}
\sphinxAtStartPar
Un pere a une somme d’argent de 1554 dh, il veut la partager sur ses 9 enfants de maniere equitable et s’il reste quelque dirham, il va acheter des chocolat a 1 dh l’unite. combien chanque enfant va recevoir? combien d’unite de chocolat peut\sphinxhyphen{}il acheter avec le reste?

\begin{sphinxVerbatim}[commandchars=\\\{\}]
\PYG{c+c1}{\PYGZsh{}\PYGZsh{} votre code ici}
\end{sphinxVerbatim}




\subsection{Exercice 3.}
\label{\detokenize{exo5:exercice-3}}
\sphinxAtStartPar
Quel est le type de donnees de valeurs suivantes:
\begin{itemize}
\item {} 
\sphinxAtStartPar
\sphinxcode{\sphinxupquote{1}}

\item {} 
\sphinxAtStartPar
\sphinxcode{\sphinxupquote{1.}}

\item {} 
\sphinxAtStartPar
\sphinxcode{\sphinxupquote{False}}

\item {} 
\sphinxAtStartPar
\sphinxcode{\sphinxupquote{"False"}}

\item {} 
\sphinxAtStartPar
\sphinxcode{\sphinxupquote{var1/var2}} avec \sphinxcode{\sphinxupquote{var1 = 1}} et \sphinxcode{\sphinxupquote{var2 = 2}}

\end{itemize}

\begin{sphinxVerbatim}[commandchars=\\\{\}]
\PYG{c+c1}{\PYGZsh{}\PYGZsh{} votre code ici}
\end{sphinxVerbatim}




\subsection{Exercice 4.}
\label{\detokenize{exo5:exercice-4}}
\sphinxAtStartPar
Quel est le type de donnees de valeurs suivantes:
\begin{itemize}
\item {} 
\sphinxAtStartPar
\sphinxcode{\sphinxupquote{1}}

\item {} 
\sphinxAtStartPar
\sphinxcode{\sphinxupquote{1.}}

\item {} 
\sphinxAtStartPar
\sphinxcode{\sphinxupquote{False}}

\item {} 
\sphinxAtStartPar
\sphinxcode{\sphinxupquote{"False"}}

\item {} 
\sphinxAtStartPar
\sphinxcode{\sphinxupquote{"5.4"}}

\item {} 
\sphinxAtStartPar
\sphinxcode{\sphinxupquote{var1/var2}} avec \sphinxcode{\sphinxupquote{var1 = 1}} et \sphinxcode{\sphinxupquote{var2 = 2}}

\item {} 
\sphinxAtStartPar
\sphinxcode{\sphinxupquote{list()}}

\item {} 
\sphinxAtStartPar
\sphinxcode{\sphinxupquote{None}}

\item {} 
\sphinxAtStartPar
\sphinxcode{\sphinxupquote{""}}

\end{itemize}

\begin{sphinxVerbatim}[commandchars=\\\{\}]
\PYG{c+c1}{\PYGZsh{}\PYGZsh{} votre code ici}
\end{sphinxVerbatim}




\subsection{Exercice 5.}
\label{\detokenize{exo5:exercice-5}}
\sphinxAtStartPar
Convertir, si c’est possible, de valeurs suivantes au types de donnees que nous avons vu. Expliciter les cas qui ne sont pas possible:
\begin{itemize}
\item {} 
\sphinxAtStartPar
\sphinxcode{\sphinxupquote{1}}

\item {} 
\sphinxAtStartPar
\sphinxcode{\sphinxupquote{1.}}

\item {} 
\sphinxAtStartPar
\sphinxcode{\sphinxupquote{False}}

\item {} 
\sphinxAtStartPar
\sphinxcode{\sphinxupquote{"False"}}

\item {} 
\sphinxAtStartPar
\sphinxcode{\sphinxupquote{"5.4"}}

\item {} 
\sphinxAtStartPar
\sphinxcode{\sphinxupquote{var1/var2}} avec \sphinxcode{\sphinxupquote{var1 = 1}} et \sphinxcode{\sphinxupquote{var2 = 2}}

\item {} 
\sphinxAtStartPar
\sphinxcode{\sphinxupquote{list()}}

\item {} 
\sphinxAtStartPar
\sphinxcode{\sphinxupquote{None}}

\item {} 
\sphinxAtStartPar
\sphinxcode{\sphinxupquote{""}}

\end{itemize}

\begin{sphinxVerbatim}[commandchars=\\\{\}]
\PYG{c+c1}{\PYGZsh{}\PYGZsh{} votre code ici}
\end{sphinxVerbatim}




\subsection{Exercice 6.}
\label{\detokenize{exo5:exercice-6}}
\sphinxAtStartPar
Soient \sphinxcode{\sphinxupquote{x = True}}, \sphinxcode{\sphinxupquote{y= 7\textless{}6}}, and \sphinxcode{\sphinxupquote{z= not y}}. Determinier la valeur logique de \sphinxcode{\sphinxupquote{x, y,}} et \sphinxcode{\sphinxupquote{z}} (\sphinxcode{\sphinxupquote{True}} ou \sphinxcode{\sphinxupquote{False}}) puis la valeur de chacun des expression suivantes:
\begin{itemize}
\item {} 
\sphinxAtStartPar
\sphinxcode{\sphinxupquote{x != False}}

\item {} 
\sphinxAtStartPar
\sphinxcode{\sphinxupquote{x and y}}

\item {} 
\sphinxAtStartPar
\sphinxcode{\sphinxupquote{x or y}}

\item {} 
\sphinxAtStartPar
\sphinxcode{\sphinxupquote{not y}}

\item {} 
\sphinxAtStartPar
\sphinxcode{\sphinxupquote{x and (y or z)}}

\item {} 
\sphinxAtStartPar
\sphinxcode{\sphinxupquote{(x and y) or z}}

\item {} 
\sphinxAtStartPar
\sphinxcode{\sphinxupquote{(not x or not y) and (not z)}}

\item {} 
\sphinxAtStartPar
\sphinxcode{\sphinxupquote{not ((x and y) or z)}}

\end{itemize}

\begin{sphinxVerbatim}[commandchars=\\\{\}]
\PYG{c+c1}{\PYGZsh{}\PYGZsh{} votre code ici}
\end{sphinxVerbatim}




\subsection{Exercice 7.}
\label{\detokenize{exo5:exercice-7}}
\sphinxAtStartPar
On veut recevoir ce message avec \sphinxcode{\sphinxupquote{print()}}:

\begin{sphinxVerbatim}[commandchars=\\\{\}]
\PYG{n}{la} \PYG{n}{valeur} \PYG{n}{de} \PYG{n}{x} \PYG{n}{est}\PYG{p}{:} \PYG{o}{@}\PYG{n+nd}{@True}\PYG{o}{@}\PYG{o}{@}
\PYG{o}{*}\PYG{o}{*}\PYG{o}{*}\PYG{o}{*}\PYG{o}{*}\PYG{o}{*}\PYG{o}{*}\PYG{o}{*}\PYG{o}{*}\PYG{o}{==}\PYG{o}{==}\PYG{o}{==}\PYG{o}{==}\PYG{o}{==}\PYG{o}{==}\PYG{o}{==}\PYG{o}{*}\PYG{o}{*}\PYG{o}{*}\PYG{o}{*}\PYG{o}{*}\PYG{o}{*}\PYG{o}{*}\PYG{o}{*}
\end{sphinxVerbatim}

\sphinxAtStartPar
En utilisant les argument suivants:
\begin{itemize}
\item {} 
\sphinxAtStartPar
\sphinxcode{\sphinxupquote{x = True}}

\item {} 
\sphinxAtStartPar
\sphinxcode{\sphinxupquote{a = "la valeur de x est:"}}

\end{itemize}

\sphinxAtStartPar
Toute modification devra etre faite au nivau de \sphinxcode{\sphinxupquote{sep=}}, et \sphinxcode{\sphinxupquote{end=}}.

\begin{sphinxVerbatim}[commandchars=\\\{\}]
\PYG{c+c1}{\PYGZsh{}\PYGZsh{} votre code ici}
\end{sphinxVerbatim}




\subsection{Exercice 8.}
\label{\detokenize{exo5:exercice-8}}
\sphinxAtStartPar
Ecrire un petit programe qui permet de demander a l’utilisatuer d’entrer son nom, son poids en kilograme (sans entrer l’unite), et sa taille en metre (sans entrer l’unite). puis il affiche l’indice du poids (Body mass index (BMI)):
\sphinxstyleemphasis{Formule du BMI}: \(BMI = \dfrac{poids}{taille^2}\)

\begin{sphinxVerbatim}[commandchars=\\\{\}]
\PYG{c+c1}{\PYGZsh{}\PYGZsh{} votre code ici}
\end{sphinxVerbatim}







\renewcommand{\indexname}{Index}
\printindex
\end{document}